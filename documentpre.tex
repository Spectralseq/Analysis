\usepackage[utf8]{inputenc}
\usepackage[colorlinks,linkcolor=blue,citecolor=blue,urlcolor=blue,hyperfootnotes=true,bookmarksopen=true]{hyperref}
\usepackage{footnotebackref}
\usepackage{graphicx}
\usepackage{multicol}
\usepackage{amsmath,amssymb,amsthm,amsfonts}
\usepackage[capitalize,nameinlink]{cleveref}%动态ref定理,推论,引理等
\usepackage{textgreek,bm,upgreek,mathrsfs}
\usepackage{arydshln}
%\usepackage{textcomp}
\usepackage{stmaryrd}
\expandafter\def\csname opt@stmaryrd.sty\endcsname
{only,shortleftarrow,shortrightarrow}
\usepackage{extpfeil}
\usepackage{extarrows}
\usepackage[bottom]{footmisc}
\usepackage{enumitem}
\usepackage{epigraph}
\usepackage[many]{tcolorbox}
\usepackage[english]{babel}
\usepackage[margin=1in]{geometry}
\usepackage{setspace}
\usepackage{microtype}
\usepackage{float}
\usepackage[framemethod=tikz]{mdframed} 
\usepackage[tikz]{bclogo}
\usepackage{wrapfig}
\usepackage{color}
\usepackage{epigraph}
\usepackage[many]{tcolorbox}
\usepackage{tikz-cd}
\usepackage{tikz}
\usepackage{fancyhdr}
\usepackage{titlesec}%定义section的格式
\usepackage{tocloft}
\usepackage{titletoc}%定义目录样式
\usepackage{sectsty}%定义section的格式
\usepackage{sectsty}%标题居中
\usepackage{booktabs}
\usepackage{easybmat}%定义大型分块矩阵
\usepackage{dsfont}%定义bold 1
\usepackage{calligra}

\usepackage[T1]{fontenc}
\PassOptionsToPackage{no-math}{fontspec}
%\usepackage{newtxmath}
%\usepackage{palatino}
\usepackage[noBBpl]{mathpazo}

%\usepackage{bbold}%定义bbold符号
\usepackage{l3draw,xparse}
\usepackage{accents}%创建新的widetilde
\usepackage{scalerel} %放大数学字体
\usepackage{calligra,mathrsfs}%for sheaf Hom
\usepackage{mathtools}%def new widetilde
\usepackage{calc}%定义箭头不同样式symbol
\usepackage[normalem]{ulem}%定义合适的下划线

%-----------------------------------------------------
%定义BOONDOX包的字体
\DeclareMathAlphabet{\mathbbb}{U}{bbold}{m}{n}
\DeclareMathAlphabet{\mathbb}{U}{msb}{m}{n}
%-----------------------------------------------------
\DeclareFontFamily{U}{BOONDOX-cal}{\skewchar\font=45}
\DeclareFontShape{U}{BOONDOX-cal}{m}{n}{
  <-> s*[1] BOONDOX-r-cal}{}
\DeclareMathAlphabet{\mathscr}{U}{BOONDOX-cal}{m}{n}
%-----------------------------------------------------
\DeclareFontFamily{U}{BOONDOX-frak}{\skewchar\font=45}
\DeclareFontShape{U}{BOONDOX-frak}{m}{n}{
  <-> s*[1]  BOONDOX-r-frak}{}
%\DeclareMathAlphabet{\mathfrak}{U}{BOONDOX-frak}{m}{n}

%----------------------------------------------------
%设置目录中number和文字的间距
\usepackage{tocloft}
%\setlength{\cftsecnumwidth}{25pt} % resets space for section number (usually 2.3em)
%\setlength{\cftsubsecnumwidth}{35pt} % resets space for subsection number (usually 3.2em)

%-----------------------------------------------------
%设置页面样式
%\geometry{top=3.5cm,bottom=3.5cm}
%\linespread{1.15}
\geometry{hcentering}
\pagestyle{fancy}
\setlength{\headheight}{15.0pt}
\fancyhf{}
\fancyhead[LE,RO]{\thepage}
\fancyhead[RE]{\bf\nouppercase{\rightmark}}
\fancyhead[LO]{\bf\nouppercase{\leftmark}}
\renewcommand{\headrulewidth}{0pt}
\renewcommand{\footrulewidth}{0pt}
%-------------------------------------------------------

%------------------------------------------------------
%重新定义autoref的section和subsection的名字
\renewcommand{\sectionautorefname}{\S\!}
\renewcommand{\subsectionautorefname}{\S\S\!}

%-------------------------------------------------------
%定义能编号的paragraph
\renewcommand{\theparagraph}{\thesubsection.\arabic{paragraph}}% How paragraphs are numbered
\setcounter{secnumdepth}{4}

%---------------------------------------------------------
%---控制enumerate编号正体
\usepackage{enumitem}
\setlist[enumerate,1]{font=\upshape}
\setlist[enumerate,2]{font=\upshape}

\makeatletter%define roman numbers
\newcommand{\rmnum}[1]{\textup{\romannumeral #1}}
\newcommand{\Rmnum}[1]{\textup{\expandafter\@slowromancap\romannumeral #1@}}
\makeatother

\chapterfont{\centering}%chapter居中
\sectionfont{\centering}%section居中

\usetikzlibrary{arrows.meta}
\usetikzlibrary{bending}
\tikzset{
  commutative diagrams/.cd,
  arrow style=tikz,
  diagrams={>={Computer Modern Rightarrow[length=5pt,width=4pt]}},
  symbol/.style={
    draw=none,
    every to/.append style={
      edge node={node [sloped, allow upside down, auto=false]{$#1$}}}
  },
  row sep/normal=8mm,
  column sep/normal=8mm
}

%------------------------------------------------
%引入mathabx包的widecheck
\DeclareFontFamily{U}{mathx}{}
\DeclareFontShape{U}{mathx}{m}{n}{ <-> mathx10 }{}
\DeclareSymbolFont{mathx}{U}{mathx}{m}{n}
\DeclareFontSubstitution{U}{mathx}{m}{n}
\DeclareMathAccent{\widecheck}{0}{mathx}{"71}
%\DeclareMathAccent{\widebar}{0}{mathx}{"73}
%------------------------------------------------

\def\sub{\subseteq}
\def\sups{\supseteq}
\def\emp{\varnothing}

\def\End{\mathrm{End}}
\def\Hom{\mathrm{Hom}}
\def\Aut{\mathrm{Aut}}
\def\im{\mathrm{im}\,}
\def\det{\mathrm{det}\,}
\def\tr{\mathrm{tr}}
\def\sgn{\mathrm{sgn}}
\def\Inn{\mathrm{Inn}}
\def\Sym{\mathrm{Sym}}
\def\Alt{\mathrm{Alt}}
\def\rank{\mathrm{rank}}
\def\id{\mathrm{id}}
\def\lcm{\mathrm{lcm}}
\def\dom{\mathrm{dom}}
\def\ran{\mathrm{ran}}
\def\Rad{\mathrm{Rad}}
\def\Poi{\mathrm{Poi}}
\def\Hyp{\mathrm{Hyp}}
\def\Cau{\mathrm{Cau}}
\def\Cov{\mathrm{Cov}}
\def\Var{\mathrm{Var}}
\def\bal{\mathrm{bal}}
\def\conv{\mathrm{conv}}
\def\balcore{\mathrm{balcore}}
\def\convbal{\mathrm{convbal}}
\def\Lat{\mathrm{Lat}}
\def\inv{\mathrm{inv}}

\def\Diff{\mathrm{Diff}}
\def\Int{\mathrm{Int}\,}
\def\Ext{\mathrm{Ext}\,}
\def\Iso{\mathrm{Iso}}
\def\diam{\mathrm{diam}}
\def\codim{\mathrm{codim}\,}
\def\Div{\mathrm{Div}}

\def\pr{\mathrm{pr}}
\def\supp{\mathrm{supp}}
\def\vol{\mathrm{vol}}

\def\GL{\mathrm{GL}}
\def\PGL{\mathrm{PGL}}
\def\PSL{\mathrm{PSL}}
\def\SL{\mathrm{SL}}
\def\O{\mathrm{O}}
\def\SO{\mathrm{SO}}
\def\U{\mathrm{U}}
\def\SU{\mathrm{SU}}

\def\p{\mathfrak{p}}
\def\z{\mathfrak{z}}
\def\g{\mathfrak{g}}
\def\h{\mathfrak{h}}
\def\m{\mathfrak{m}}
\def\n{\mathfrak{n}}
\def\k{\mathfrak{k}}
\def\gl{\mathfrak{gl}}
\def\sl{\mathfrak{sl}}
\def\o{\mathfrak{o}}
\def\so{\mathfrak{so}}
\def\u{\mathfrak{u}}
\def\su{\mathfrak{su}}
\def\Lie{\mathfrak{Lie}}
\def\Ad{\mathrm{Ad}}
\def\ad{\mathrm{ad}}
\def\Spec{\mathrm{Spec}}
\def\Max{\mathrm{Max}}
\def\grad{\mathrm{grad}\,}
\def\div{\mathrm{div}}
\def\curl{\mathrm{curl}\,}
\def\inj{\mathrm{inj}}
\def\Res{\mathrm{Res}}
\def\Gr{\mathrm{Gr}}
\def\C{\mathbb{C}}
\def\F{\mathbb{F}}
\def\K{\mathbb{K}}
\def\R{\mathbb{R}}
\def\B{\mathbb{B}}
\def\R{\mathbb{R}}
\def\Q{\mathbb{Q}}
\def\Z{\mathbb{Z}}
\def\T{\mathbb{T}}
\def\D{\mathbb{D}}
\def\N{\mathbb{N}}
\def\H{\mathbb{H}}
\def\X{\mathfrak{X}}
\def\J{\mathfrak{J}}
\def\RP{\mathbb{RP}}
\def\CP{\mathbb{CP}}
\def\P{\mathscr{P}}
\def\hash{\!\textit{\texttt{\#}}}

\def\dcheck{\check{ }\hspace{4pt}\check{}}
\def\eps{\varepsilon}
\def\And{\hspace{8pt}\text{and}\hspace{8pt}}
\def\for{\hspace{8pt}\text{for}\hspace{6pt}}
\def\llim{\varprojlim\limits}
\def\rlim{\varinjlim\limits}

\def\One{\mathrm{\Rmnum{1}}}
\def\Two{\mathrm{\Rmnum{2}}}
\def\Ric{\mathrm{Ric}}
\def\sec{\mathrm{sec}}
\def\loc{\mathrm{loc}}

\renewcommand{\Re}{\operatorname{Re}}
\renewcommand{\Im}{\operatorname{Im}}
\renewcommand{\i}{\mathrm{i}}

\tikzcdset{
    scale/.style={every label/.append style={scale=#1},
    cells={nodes={scale=#1}}}}
\newcommand\scalemath[2]{\scalebox{#1}{\mbox{\ensuremath{\displaystyle #2}}}}

\makeatletter
\newcommand{\nunder}[2][5]{\mathrlap{\mkern\the\numexpr#1/2mu\relax\underline{\phantom{\mathrm{#2}\mkern-#1mu}}}#2}
\renewcommand{\thefigure}{\arabic{chapter}.\arabic{figure}}
\renewcommand{\theequation}{\arabic{section}.\arabic{equation}}
\newcommand{\intprod}{\mathbin{\raisebox{\depth}{\scalebox{1}[-1]{$\lnot$}}}}%defining interior product
\newcommand{\bigw}{\scalebox{.95}[1]{\boldmath{$\bigwedge$}\hspace{-1pt}}}%define a nice wedge product

%调整图片和方程编号方式----------------------
%\makeatletter
%\@addtoreset{figure}{subsection}
%\makeatother
%\renewcommand{\thefigure}{\arabic{section}.\arabic{subsection}.\arabic{figure}}
\counterwithin*{figure}{subsection}

%\renewcommand{\theequation}{\arabic{section}.\arabic{subsection}.\arabic{equation}}
\numberwithin{equation}{subsection}
%-------------------------------------------

\def\llim{\varprojlim\limits}
\def\rlim{\varinjlim\limits}
\tikzcdset{row sep/normal=1cm}
\tikzcdset{column sep/normal=1cm}

\makeatletter

\def\upint{\mathchoice%
    {\mkern13mu\overline{\vphantom{\intop}\mkern7mu}\mkern-20mu}%
    {\mkern7mu\overline{\vphantom{\intop}\mkern7mu}\mkern-14mu}%
    {\mkern7mu\overline{\vphantom{\intop}\mkern7mu}\mkern-14mu}%
    {\mkern7mu\overline{\vphantom{\intop}\mkern7mu}\mkern-14mu}%
  \int}
\def\lowint{\mkern3mu\underline{\vphantom{\intop}\mkern7mu}\mkern-10mu\int}
\newcommand{\Log}{\mathop{}\!\mathrm{Log}\mathop{}\!}
\newcommand{\dif}{\mathop{}\!\mathrm{d}}
\renewcommand{\Re}{\operatorname{Re}}
\renewcommand{\Im}{\operatorname{Im}}

%--------------------------------------------------------
%定义widebar
\makeatletter
\let\save@mathaccent\mathaccent
\newcommand*\if@single[3]{%
  \setbox0\hbox{${\mathaccent"0362{#1}}^H$}%
  \setbox2\hbox{${\mathaccent"0362{\kern0pt#1}}^H$}%
  \ifdim\ht0=\ht2 #3\else #2\fi
  }
%The bar will be moved to the right by a half of \macc@kerna, which is computed by amsmath:
\newcommand*\rel@kern[1]{\kern#1\dimexpr\macc@kerna}
%If there's a superscript following the bar, then no negative kern may follow the bar;
%an additional {} makes sure that the superscript is high enough in this case:
\newcommand*\widebar[1]{\@ifnextchar^{{\wide@bar{#1}{0}}}{\wide@bar{#1}{1}}}
%Use a separate algorithm for single symbols:
\newcommand*\wide@bar[2]{\if@single{#1}{\wide@bar@{#1}{#2}{1}}{\wide@bar@{#1}{#2}{2}}}
\newcommand*\wide@bar@[3]{%
  \begingroup
  \def\mathaccent##1##2{%
%Enable nesting of accents:
    \let\mathaccent\save@mathaccent
%If there's more than a single symbol, use the first character instead (see below):
    \if#32 \let\macc@nucleus\first@char \fi
%Determine the italic correction:
    \setbox\z@\hbox{$\macc@style{\macc@nucleus}_{}$}%
    \setbox\tw@\hbox{$\macc@style{\macc@nucleus}{}_{}$}%
    \dimen@\wd\tw@
    \advance\dimen@-\wd\z@
%Now \dimen@ is the italic correction of the symbol.
    \divide\dimen@ 3
    \@tempdima\wd\tw@
    \advance\@tempdima-\scriptspace
%Now \@tempdima is the width of the symbol.
    \divide\@tempdima 10
    \advance\dimen@-\@tempdima
%Now \dimen@ = (italic correction / 3) - (Breite / 10)
    \ifdim\dimen@>\z@ \dimen@0pt\fi
%The bar will be shortened in the case \dimen@<0 !
    \rel@kern{0.6}\kern-\dimen@
    \if#31
      \overline{\rel@kern{-0.6}\kern\dimen@\macc@nucleus\rel@kern{0.4}\kern\dimen@}%
      \advance\dimen@0.4\dimexpr\macc@kerna
%Place the combined final kern (-\dimen@) if it is >0 or if a superscript follows:
      \let\final@kern#2%
      \ifdim\dimen@<\z@ \let\final@kern1\fi
      \if\final@kern1 \kern-\dimen@\fi
    \else
      \overline{\rel@kern{-0.6}\kern\dimen@#1}%
    \fi
  }%
  \macc@depth\@ne
  \let\math@bgroup\@empty \let\math@egroup\macc@set@skewchar
  \mathsurround\z@ \frozen@everymath{\mathgroup\macc@group\relax}%
  \macc@set@skewchar\relax
  \let\mathaccentV\macc@nested@a
%The following initialises \macc@kerna and calls \mathaccent:
  \if#31
    \macc@nested@a\relax111{#1}%
  \else
%If the argument consists of more than one symbol, and if the first token is
%a letter, use that letter for the computations:
    \def\gobble@till@marker##1\endmarker{}%
    \futurelet\first@char\gobble@till@marker#1\endmarker
    \ifcat\noexpand\first@char A\else
      \def\first@char{}%
    \fi
    \macc@nested@a\relax111{\first@char}%
  \fi
  \endgroup
}
\makeatother