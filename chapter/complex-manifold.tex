\chapter{Complex manifolds}
\section{Local theory of complex vector spaces}
\subsection{Complex and Hermitian Structures}
Let us now come back to an Euclidian vector space $(V,\langle\cdot\ ,\cdot\rangle)$ with a compatible almost complex structure $J$. The \textbf{fundamental form} associated to $(V,\langle\cdot\ ,\cdot\rangle,J)$ is the form $\omega$ defined by
\[\omega(v,w)=\langle Jv,w\rangle=-\langle v,Jw\rangle.\]
Since $\langle\cdot\ ,\cdot\rangle$ is compatible with $J$, it is easy to see $\omega$ is alternating:
\[\omega(v,w)=\langle Jv,w\rangle=\langle J^2v,Jw\rangle=-\langle v,Jw\rangle=-\omega(w,v).\]
Moreover, we note that $\omega$ is invariant under $J$:
\[(J\omega)(v,w)=\omega(Jv,Jw)=\langle J^2v,Jw\rangle=\langle Jv,w\rangle=\omega(v,w)\]
so that $\omega\in\bigw^{1,1}V^*$. We also note that two of the three structures $\{\langle\cdot\ ,\cdot\rangle,J,\omega\}$ determine the remaining
one.\par
Following a standard procedure, the scalar product and the fundamental form are encoded by a natural hermitian form.
\begin{proposition}\label{almost complex space induce Hermitian}
Let $(V,\langle\cdot\ ,\cdot\rangle)$ be an Euclidian vector space endowed with a compatible complex structure $J$. Then the form $(\cdot\ ,\cdot):=\langle\cdot\ ,\cdot\rangle-\i\omega$ is a positive Hermitian form on $(V,J)$.
\end{proposition}
\begin{proof}
The form $(\cdot\ ,\cdot)$ is clearly $\R$-linear and we have $(v,v)=\langle v,v\rangle>0$ for nonzero $v\in V$. Moreover, $(v,w)=\widebar{(w,v)}$ and
\begin{align*}
(Jv,w)&=\langle Jv,w\rangle-\i\omega(Jv,w)=\langle Jv,w\rangle+\i\langle v,w\rangle=\i(\langle v,w\rangle-\i\langle Jv,w\rangle)\\
&=\i(\langle v,w\rangle-\i\omega(v,w))=\i(v,w).
\end{align*}
This shows that $(v,w)$ is sesquilinear, so it is a Hermitian form.
\end{proof}
We can also consider the extension of the scalar product $\langle\cdot\ ,\cdot\rangle$ to a positive definite hermitian form $\langle\cdot\ ,\cdot\rangle_{\C}$ on $V_{\C}$. This is defined by
\[\langle v\otimes\lambda,w\otimes\mu\rangle_{\C}=\lambda\bar{\mu}\langle v,w\rangle\]
for $v,w\in V$ and $\lambda,\mu\in\C$.
\begin{proposition}\label{almost complex space decomposition of V_C}
Let $(V,\langle\cdot\ ,\cdot\rangle)$ be an Euclidian vector space endowed with a compatible complex structure $J$. Then $V_{\C}=V^{1,0}\oplus V^{0,1}$ is an orthogonal decomposition with respect to the Hermitian product $\langle\cdot\ ,\cdot\rangle_{\C}$.
\end{proposition}
\begin{proof}
Let $v-\i Jv\in V^{1,0}$ and $w+\i Jw\in V^{0,1}$ with $v,w\in V$. We then compute that
\begin{align*}
\langle v-\i Jv,w+\i Jw\rangle_{\C}&=\langle v,w\rangle_{\C}+\langle v,\i Jw\rangle_{\C}-\langle\i Jv,w\rangle_{\C}-\langle\i Jv,\i Jw\rangle_{\C}\\
&=\langle v,w\rangle-\i\langle v,Jw\rangle-\i\langle Jv,w\rangle-\langle Jv,Jw\rangle=0.
\end{align*}
This proves the claim, since $V^{1,0}$ is generated by $v-\i Jv$, and $V^{0,1}$ is generated by $w+\i Jw$.
\end{proof}
It turn out that $\langle\cdot\ ,\cdot\rangle_{\C}$ coincides with $(\cdot\ ,\cdot)$, as the following proposition shows.
\begin{proposition}\label{almost complex space Hermitian form relation}
Let $(V,\langle\cdot\ ,\cdot\rangle)$ be an Euclidian vector space endowed with a compatible complex structure $J$. Then under the canonical isomorphism $(V,J)\cong (V^{1,0},\i)$, we have $\frac{1}{2}(\cdot\ ,\cdot)=\langle\cdot\ ,\cdot\rangle_{\C}|_{V^{1,0}}$.
\end{proposition}
\begin{proof}
The natrual isomorphism is given by $v\mapsto\frac{1}{2}(v-\i Jv)$, so by the definition of $(\cdot\ ,\cdot)$, for $v,w\in V$ we have
\begin{align*}
\langle v-\i Jv,w-\i Jw\rangle_{\C}&=\langle v,w\rangle+\i\langle v,Jw\rangle-\i\langle Jv,w\rangle+\langle Jv,Jw\rangle\\
&=2\langle v,w\rangle+2\i\langle v,Jw\rangle=2(v,w).
\end{align*}
This proves the assertion.
\end{proof}
Often, it is useful to do calculations in coordinates. Let us see how the above products can be expressed explicitly once suitable basis have been chosen. Let $z_1,\dots,z_n$ be a $\C$-basis of $V^{1,0}$. Write $z_i=\frac{1}{2}(x_i-\i Jx_i)$ with $x_i\in V$, then we have seen that $x_1,y_1=Jx_1,\dots,x_n,y_n=Jx_n$ is a $\R$-basis of $V$ and $x_1,\dots,x_n$ is a $\C$-basis for $(V,J)$. The Hermitian form $\langle\cdot\ ,\cdot\rangle_{\C}$ on $V^{1,0}$ with respect to the basis $z_i$ is given by an Hermitian matrix, say $\frac{1}{2}(h_{ij})$. Concretely,
\[\langle\sum_{i=1}^{n}a_iz_i,\sum_{j=1}^{n}b_jz_j\rangle_{\C}=\frac{1}{2}\sum_{i,j=1}^{n}h_{ij}a_i\widebar{b}_j.\]
By Proposition~\ref{almost complex space Hermitian form relation}, the Hermitian form $(\cdot\ ,\cdot)$ on $(V,J)$ with respect to the basis $x_i$ is given by $(h_{ij})$. We want to compute the fundamental form $\omega$ in this case, with respect to the basis $x_i,y_i$.
\begin{proposition}\label{almost complex space fundamental class express}
The fundamental form $\omega$ on $(V,J)$ is given by
\begin{align}\label{almost complex space fundamental class express-1}
\omega=-\sum_{i<j}(\Im h_{ij})(x^i\wedge x^j+y^i\wedge y^j)+\sum_{i,j=1}^{n}(\Re h_{ij})x^i\wedge y^j
\end{align}
with respect to the basis $x_i,y_i$. Moreover, the extension of $\omega$ on $V_{\C}$ can be written as
\begin{align}\label{almost complex space fundamental class express-2}
\omega=\frac{\i}{2}\sum_{i,j=1}^{n}h_{ij}z^i\wedge\bar{z}^j.
\end{align}
\end{proposition}
\begin{proof}
By the definition of $(\cdot\ ,\cdot)$, we have $\omega=-\Im(\cdot\ ,\cdot)$ and $\langle\cdot\ ,\cdot\rangle=\Re(\cdot\ ,\cdot)$. Hence,
\[\omega(x_i,x_j)=\omega(y_i,y_j)=-\Im h_{ij},\quad \omega(x_i,y_j)=\Re h_{ij}.\]
This then gives the expression of $\omega$ as
\[\omega=-\sum_{i<j}\Im h_{ij}(x^i\wedge x^j+y^i\wedge y^j)+\sum_{i,j=1}^{n}\Re h_{ij}\,x^i\wedge y^j.\]
To see the expression (\ref{almost complex space fundamental class express-2}), we note that
\[z^i\wedge\bar{z}^j=(x^i+\i y^i)\wedge(x^j-\i y^j)=(x^i\wedge x^j+y^i\wedge y^j)-\i(x^i\wedge y^j+x^j\wedge y^i);\]
so (\ref{almost complex space fundamental class express-2}) is equal to
\begin{equation}\label{almost complex space fundamental class express-3}
\begin{aligned}
\frac{\i}{2}\sum_{i,j=1}^{n}h_{ij}z^i\wedge\bar{z}^j&=\sum_{i,j=1}^{n}\big[\frac{\i}{2}\Re h_{ij}-\frac{1}{2}\Im h_{ij}\big]\big[(x^i\wedge x^j+y^i\wedge y^j)-\i(x^i\wedge y^j+x^j\wedge y^i)\big]\\
&=\sum_{i,j}\frac{1}{2}\Re h_{ij}(x^i\wedge y^j+x^j\wedge y^i)-\sum_{i,j}\frac{1}{2}\Im h_{ij}(x^i\wedge x^j+y^i\wedge y^j)\\
&\ +\sum_{i,j}\frac{\i}{2}\Re h_{ij}(x^i\wedge x^j+y^i\wedge y^j)+\sum_{i,j}\frac{\i}{2}\Im h_{ij}(x^i\wedge y^j+x^j\wedge y^i).
\end{aligned}
\end{equation}
On the other hand, we note that since $(h_{ij})$ is a Hermitian matrix, we have
\[\Re h_{ij}=\Re h_{ji},\quad \Im h_{ij}=-\Im h_{ji}.\]
With these relations, we see the first two terms in the last summand of (\ref{almost complex space fundamental class express-3}) add up to $\omega$, and the last two terms are zero. This completes the proof.
\end{proof}
\begin{example}
If $x_i,y_i$ is an orthonormal basis for $V$ with respect to $\langle\cdot\ ,\cdot\rangle$, i.e.,
\[\langle\cdot\ ,\cdot\rangle=\sum_{i=1}^{n}(x^i\otimes x^i+y^i\otimes y^i)\]
then by Proposition~\ref{almost complex space fundamental class express}, we have
\[\omega=\frac{\i}{2}\sum_{i=1}^{n}z^i\wedge\bar{z}^i=\sum_{i=1}^{n}x^i\wedge y^i.\]
\end{example}
With the fundamental class $\omega$, we now define the Lefschetz operator $L:\bigw^*V^*_{\C}\to\bigw^*V^*_{\C}$ by $\alpha\mapsto\omega\wedge\alpha$. It is clear that $L$ is the $\C$-linear extension of the real operator $\bigw^*V\to\bigw^*V$ given by $\alpha\mapsto\omega\wedge\alpha$, and since $\omega$ is of bidegree $(1,1)$, $L$ is of bidegree $(1,1)$, i.e.,
\[L\Big(\bigw^{p,q}V^*\Big)\sub\bigw^{p+1,q+1}V^*.\]
We shall show that the power
\[L^k:\bigw^kV^*\to\bigw^{2n-k}V^*\]
is an isomorphism for all $k\leq n$, where we write $\dim_{\R}V=2n$. An elementary proof can be given by choosing a basis, but it is slightly cumbersome. A more elegant but less elementary argument, using $\sl(2,\C)$-representation theory, will be given below.\par
The Lefschetz operator comes along with its dual $\Lambda$. In order to define and to describe $\Lambda$ we need to recall the Hodge $\ast$-operator on a real vector space. Let $(V,\langle\cdot\ ,\cdot\rangle)$ be an oriented euclidian vector space of dimension $d$, then $\langle\cdot\ ,\cdot\rangle$ can be extended to all the exterior powers $\bigw^kV$. Explicitly, if $e_1,\dots,e_d\in V$ is an orthonormal basis of $V$, then $e^I\in\bigw^kV$ is an orthonormal basis of $\bigw^kV$, where $I=\{i_1<\cdots<i_k\}$ is a subset of $\{1,\dots,d\}$. Let $\mathrm{vol}\in\bigw^dV$ be the orientation of $V$ of norm $1$ given by $\mathrm{vol}=e_1\wedge\cdots\wedge e_d$. Then the Hodge $\ast$-operator is defined by
\[\alpha\wedge\ast\beta=\langle\alpha,\beta\rangle\cdot\mathrm{vol}\]
for $\alpha,\beta\in\bigw^*V$. This determines $\ast$ since the exterior product defines a nondegenerate pairing $\bigw^kV\times\bigw^{d-k}V=\mathrm{vol}\cdot\R$. One easily sees that $\ast:\bigw^kV\to\bigw^{d-k}V$. The most important properties of the Hodge $\ast$-operator are collected in the following proposition.
\begin{proposition}\label{almost complex space Hodge star operator prop}
Let $(V,\langle\cdot\ ,\cdot\rangle)$ be an oriented euclidian vector space of dimension $d$. Let $e_1,\dots,e_d$ be an orthonormal basis of $V$ and let $\mathrm{vol}\in\bigw^dV$ be the orientation of norm one given by $e_1\wedge\cdots\wedge e_d$. The Hodge $\ast$-operator associated to $(V,\langle\cdot\ ,\cdot\rangle,\mathrm{vol})$ satisfies the following conditions:
\begin{itemize}
\item[(a)] If $\{i_1,\dots,i_k\}$ and $\{j_1,\dots,j_{d-k}\}$ are complementary, then
\[\ast(e_{i_1}\wedge\cdots\wedge e_{i_k})=\eps\cdot e_{j_1}\wedge\cdots\wedge e_{j_{d-k}}\]
where $\eps=\sgn(i_1,\dots,i_k,j_1,\dots,j_{d-k})$. In particular, $\ast 1=\mathrm{vol}$. 
\item[(b)] The $\ast$-operator is self-adjoint up to a sign: for $\alpha\in\bigw^kV$ we have
\[\langle\alpha,\ast\beta\rangle=(-1)^{k(d-k)}\langle\ast\alpha,\beta\rangle.\] 
\item[(c)] The $\ast$-operator is involutive up to a sign:
\[(\ast|_{\bigw^kV})^2=(-1)^{k(d-k)}.\] 
\item[(d)] The $\ast$-operator is an isometry on $(\bigw^*V,\langle\cdot\ ,\cdot\rangle)$.
\end{itemize}
\end{proposition}
In our situation we will usually have $d=2n$ and $\ast$ and $\langle\cdot\ ,\cdot\rangle$ will be considered on the dual space $\bigw^*V^*$. Let us now come back to the situation considered before. Associated to $(V,\langle\cdot\ ,\cdot\rangle,J)$ we had introduced the Lefschetz operator $L:\bigw^kV^*\to\bigw^{k+2}V^*$. The \textbf{dual Lefschetz operator} $\Lambda$ is the operator $\Lambda:\bigw^*V^*\to\bigw^*V^*$ that is adjoint to $L$ with respect to $\langle\cdot\ ,\cdot\rangle$, i.e. $\Lambda\alpha$ is uniquely determined by the condition
\[\langle\Lambda\alpha,\beta\rangle=\langle\alpha,L\beta\rangle\]
for all $\beta\in\bigw^*V^*$. The $\C$-linear extension of the dual Lefschetz operator will also be denoted by $\Lambda$.
\begin{remark}
Recall that $J$ induces a natural orientation on $V$. Thus, the Hodge $\ast$-operator is well-defined. Using an orthonormal basis $x_i,y_i=Jx_i$ as above, a straightforward calculation yields
\[\omega^n=n!\cdot\mathrm{vol}\]
where $\omega$ is the associated fundamental form.
\end{remark}
\begin{proposition}\label{almost complex space Lefschetz dual operator expression}
The dual Lefschetz operator $\Lambda$ is of degree $-2$ and we have $\Lambda=\ast^{-1}\circ L\circ\ast$.
\end{proposition}
\begin{proof}
The first assertion follows from the fact that $L$ is of degree two and that the decomposition $\bigw^*V^*=\bigoplus_k\bigw^kV^*$ is orthogonal. Now by the definition of the Hodge $\ast$-operator we have
\begin{align*}
\langle L\beta,\alpha\rangle\cdot\mathrm{vol}&=L\beta\wedge\ast\alpha=\omega\wedge\beta\wedge\ast\alpha=\beta\wedge(\omega\wedge\ast\alpha)=\langle\beta,\ast^{-1}(L(\ast\alpha))\rangle\cdot\mathrm{vol}.
\end{align*}
Since $\alpha$ and $\beta$ are arbitrary, this shows $\Lambda=\ast^{-1}\circ L\circ\ast$, as desired.
\end{proof}
Recall that $\langle\cdot\ ,\cdot\rangle_{\C}$ had been defined as the Hermitian extension to $V^*_{\C}$ of the scalar product $\langle\cdot\ ,\cdot\rangle$ on $V^*$. It can further be extended to a positive definite hermitian form on $\bigw^*V_{\C}^*$. Equivalently, one could consider the extension of $(\cdot\ ,\cdot)$ on $\bigw^*V^*$ to an Hermitian form on $\bigw^*V^*_{\C}$. In any case, there is a natural positive Hermitian product on $\bigw^*V^*_{\C}$ which will also be called $\langle\cdot\ ,\cdot\rangle_{\C}$. The Hodge $\ast$-operator associated to $(V,\langle\cdot\ ,\cdot\rangle,\mathrm{vol})$ is extended $\C$-linearly to $\ast:\bigw^kV^*_{\C}\to\bigw^{2n-k}V_{\C}^*$. On $\bigw^*V^*_{\C}$ these two operators are now related by
\[\alpha\wedge\ast\widebar{\beta}=\langle\alpha,\beta\rangle_{\C}\cdot\mathrm{vol}.\]
Clearly, the Lefschetz operator $L$ and its dual $\Lambda$ on $\bigw^*V^*_{\C}$ are also formally adjoint to each other with respect to $\langle\cdot\ ,\cdot\rangle_{\C}$, and the equality $\Lambda=\ast^{-1}\circ L\circ\ast$ is still valid.
\begin{proposition}\label{almost complex space Lefschetz dual operator prop}
Let $(V,\langle\cdot\ ,\cdot\rangle,\mathrm{vol})$ be as above and $n=\dim_{\C}(V,J)$.
\begin{itemize}
\item[(a)] The decomposition $\bigw^kV^*_{\C}=\bigoplus\bigw^{p,q}V^*$ is orthogonal with respect to $\langle\cdot\ ,\cdot\rangle_{\C}$.
\item[(b)] The Hodge $\ast$-operator maps $\bigw^{p,q}V^*$ to $\bigw^{n-q,n-p}V^*$.
\item[(c)] The dual Lefschetz operator $\Lambda$ is of bidegree $(-1,-1)$.
\end{itemize}
\end{proposition}
We continue to assume that $\dim_{\R}V=2n$, and let $H:\bigw^*V\to\bigw^*V$ be the counting operator defined by
\[H=\sum_{k=0}^{2n}(n-k)\cdot\Pi^k.\]
It is clear that each $\bigw^kV^*$ is an eigenspace of $H$. With $H$, $L$, $\Lambda$, $\Pi$, etc., we dispose of a large number of linear operators on $\bigw^*V^*$ and one might wonder whether they commute. In fact, they do not, but their commutators can be computed.
\begin{theorem}\label{almost complex space sl(2,C) representation}
Let $(V,\langle\cdot\ ,\cdot\rangle)$ be an Euclidian vector space endowed with a compatible almost complex structure $J$. Consider the linear operators $H$, $L$, $\Lambda$, and $\Pi$.
\begin{itemize}
\item[(a)] $[L,J]=[\Lambda,J]=0$.
\item[(b)] $[H,L]=2L$, $[H,\Lambda]=-2\Lambda$, and $[\Lambda,L]=H$.
\end{itemize}
\end{theorem}
To prove Theorem~\ref{almost complex space sl(2,C) representation}, it is necessary to introduce some notation which will allow us to effectively work with the convectors in $\bigw^*V^*$. We consider multi-indices $I=(i_1,\dots,i_k)$, where $i_1<\cdots<i_k$ are distinct elements of $\{1,\dots,n\}$, and set $|I|=k$. If $I=\{i_1,\dots,i_k\}$, then we write
\[z^I=z^{i_1}\wedge\cdots\wedge z^{i_k},\quad x^I=x^{i_1}\wedge\cdots\wedge x^{i_k}\]
and so on. If $M$ is a multiindex, we let
\[\alpha^M=\prod_{i\in M}z^i\wedge\bar{z}^i=(-2\i )^{|M|}\prod_{i\in M}x^i\wedge y^i.\]
In this last product it is clear that the ordering of the factors is irrelevant, since the terms commute with one another, and we shall use the same symbol $M$ to denote the ordered $k$-tuple and its underlying set of elements, provided that this leads to no confusion. Any element of $\bigw^*V^*$ can be written in the form
\[\sum_{A,B,M}'c_{A,B,M}z^A\wedge\bar{z}^B\wedge\alpha^M\]
where $c_{A,B,M}\in\C$, and $A$, $B$, and $M$ are (for a given term) mutually disjoint multiindices, and, as before, the prime on the summation sign indicates that the sum is taken over multiindices whose elements are strictly increasing sequences (what we shall call an increasing multiindex).\par
We have the following fundamental and elementary lemma which shows the interaction between the $\ast$-operator (defined in terms of the real structure) and the bigrading on $\bigw^*V^*$ (defined in terms of the almost complex structure).
\begin{lemma}\label{almost complex space Hodge star on standard form lemma}
Suppose that $A$, $B$, and $M$ are mutually disjoint increasing multiindices. Then
\[\ast(z^A\wedge\bar{z}^B\wedge\alpha^M)=\gamma(a,b,m)z^A\wedge\bar{z}^B\wedge\alpha^N\]
for a nonvanishing constant $\gamma(a,b,m)$, where $a=|A|$, $b=|B|$, $m=|M|$, and $N$ is the complement of $A\cup B\cup M$. Moreover,
\[\gamma(a,b,m)=\i^{a-b}(-1)^{\frac{p(p+1)}{2}+m}(-2\i)^{p-n}\]
where $p=a+b+2m$ is the total degree of $z^A\wedge z^B\wedge\alpha^M$.
\end{lemma}
\begin{proof}
Let $\beta=z^A\wedge\bar{z}^B\wedge\alpha^M$. If $A=A_1\cup A_2$ for some multiindex $A$, let
\[\epsilon_A^{A_1A_2}=\begin{cases}
0&\text{if $A_1\cap A_2\neq\emp$}\\
1&\text{$A_1A_2$ is an even permutation of $A$}\\
-1&\text{$A_1A_2$ is an odd permutation of $A$}
\end{cases}\]
Using this notation it is easy to see that
\[z^A=\sum_{A=A_1\cup A_2}'\epsilon_A^{A_1A_2}\cdot\i^{a_2}x^{A_1}\wedge y^{A_2},\]
where the sum runs over all decompositions of $A$ into increasing multiindices, and we set $a_1=|A_1|$, $a_2=|A_2|$, etc. We then obtain 
\[\beta=(-2\i)^m\sum'_{\substack{A=A_1\cup A_2\\B=B_1\cup B_2}}\epsilon_A^{A_1A_2}\epsilon_B^{B_1B_2}\cdot\i^{a_2-b_2}x^{A_1}\wedge y^{A_2}\wedge x^{B_1}\wedge y^{B_2}\wedge\prod_{\mu\in M}x^\mu\wedge y^\mu.\]
We want to compute $\ast\beta$, having expressed $\beta$ in terms of a real basis, and we shall do this term by term and then sum the result. To simplify the notation, consider the case where $B=\emp$. In this case,
\begin{align}\label{almost complex space Hodge star on standard form lemma-1}
\ast(z^A\wedge\alpha^M)=(-2\i)^m\sum_{A=A_1\cup A_2}\epsilon_A^{A_1A_2}\i^{a_2}\ast\Big\{x^{A_1}\wedge y^{A_2}\wedge\prod_{\mu\in M}x^\mu\wedge y^\mu\Big\}.
\end{align}
It is clear that the result of $\ast$ acting on the bracketed expression is of the form
\begin{align}\label{almost complex space Hodge star on standard form lemma-2}
\pm x^{A_2}\wedge y^{A_1}\wedge\prod_{\mu\in N}x^\mu\wedge y^\mu
\end{align}
where $N=(A\cup M)^c$. The only problem is to determine the sign. To do this it suffices (because of the commutativity of $\prod_{\mu\in M}x^\mu\wedge y^\mu)$ to consider the product
\[x^{A_1}\wedge y^{A_2}\wedge x^{A_2}\wedge y^{A_1}=(-1)^{a_2^2}x^{A_1}\wedge y^{A_1}\wedge x^{A_2}\wedge y^{A_2}.\]
In general, we have
\[x^C\wedge y^C=(-1)^{\frac{|C|(|C|-1)}{2}}x^{\mu_1}\wedge y^{\mu_2}\wedge\cdots\wedge x^{\mu_{|C|}}\wedge y^{\mu_{|C|}}\]
and applying this to our problem above, we see immediately that the sign in (\ref{almost complex space Hodge star on standard form lemma-2}) is of the form
\[(-1)^{a_2^2+\frac{a_1(a_1-1)}{2}+\frac{a_2(a_2-1)}{2}}=:(-1)^r\]
From this and (\ref{almost complex space Hodge star on standard form lemma-1}), we conclude that
\begin{align}\label{almost complex space Hodge star on standard form lemma-3}
\ast(z^A\wedge\alpha^M)=(-2\i)^m\sum'_{A=A_1\cup A_2}\epsilon_A^{A_1A_2}\i^{a_2}(-1)^rx^{A_2}\wedge y^{A_1}\wedge\prod_{\mu\in N}x^\mu\wedge y^\mu.
\end{align}
The idea now is to change variables in the summation. We write
\[\epsilon_A^{A_1A_2}=(-1)^{a_1a_2}\epsilon_A^{A_2A_1},\quad \i^{a_2}=\i^a(-1)^{a_1}\i^{a_1},\]
and by substituting this in (\ref{almost complex space Hodge star on standard form lemma-3}), we obtain that
\begin{align*}
\ast(z^A\wedge\alpha^M)=\i^a(-2\i)^m\sum'_{A=A_1\cup A_2}\epsilon_A^{A_2A_1}\i^{a_1}\cdot\{(-1)^{r+a_1+a_1a_2}\}\cdot x^{A_2}\wedge y^{A_1}\wedge\prod_{\mu\in N}x^\mu\wedge y^\mu,
\end{align*}
which is, modulo the bracketed term, of the right form to be $(z^A\wedge\alpha^M)$. A priori, this term depends on the decompositions $A=A_1\cup A_2$; however, one can verify that in fact
\[(-1)^{r+a_1+a_1a_2}=(-1)^{\frac{a(a+1)}{2}}=(-1)^{\frac{p(p+1)}{2}+m}\]
and the bracketed constant pulls out in front the summation, so
\[\ast(z^A\wedge\alpha^M)=\i^a(-1)^{\frac{p(p+1)}{2}+m}(-2\i)^{p-n}z^A\wedge\alpha^N.\]
The general case can be treated similarly.
\end{proof}
\begin{proof}[\textbf{Proof of Theorem~\ref{almost complex space sl(2,C) representation}}]
The first assertion follows from the fact that $L$ and $\Lambda$ are homogeneous operators and are real. We now prove the second one. It is immediate that, for $\alpha\in\bigw^kV^*$, we have
\begin{align*}
&[H,L](\alpha)=(n-k-2)(\omega\wedge\alpha)-\omega\wedge((n-k)\alpha)=-2\omega\wedge\alpha,\\
&[H,\Lambda](\alpha)=(n-k+2)(\Lambda\alpha)-\Lambda((n-k)\alpha)=-2\Lambda\alpha.
\end{align*}
This proves the first two equalities. Now using the notation in Lemma~\ref{almost complex space Hodge star on standard form lemma}, we observe that
\begin{align*}
L(z^A\wedge\bar{z}^B\wedge\alpha^M)=\frac{\i}{2}\Big(\sum_{i=1}^{n}z^i\wedge\bar{z}^i\Big)\wedge z^A\wedge\bar{z}^B\wedge\alpha^M=\frac{\i}{2}z^A\wedge\bar{z}^B\wedge\Big(\sum_{i\notin A\cup B\cup M}\alpha^{M+\{i\}}\Big)
\end{align*}
On the other hand, we see that, using Lemma~\ref{almost complex space Hodge star on standard form lemma} and the definition of $\Lambda$,
\begin{align*}
\Lambda(z^A\wedge\bar{z}^B\wedge\alpha^M)=\frac{2}{\i}z^A\wedge\bar{z}^B\wedge\Big(\sum_{i\in M}\alpha^{M-\{i\}}\Big).
\end{align*}
Using these formulas, one obtains easily, assuming that $z^A\wedge\bar{z}^B\wedge\alpha^M$ has total degree $k$,
\[(\Lambda L-L\Lambda)(z^A\wedge\bar{z}^B\wedge\alpha^M)=(n-k)z^A\wedge\bar{z}^N\wedge\alpha^M,\]
and part (c) of Theorem~\ref{almost complex space sl(2,C) representation} follows immediately.
\end{proof}
\begin{corollary}
Let $(V,\langle\cdot\ ,\cdot\rangle)$ be an Euclidian vector space endowed with a compatible almost complex structure $J$. The triple $(\Lambda,H,L)$ defines a natural $\sl(2,\C)$-representation on $\bigw^*V^*$.
\end{corollary}
\begin{corollary}\label{almost complex space Lefschetz commutator with dual}
For any $\alpha\in\bigw^kV^*$, we have
\[[L^i,\Lambda]\alpha=i(k-n+i-1)L^{i-1}(\alpha).\]
\end{corollary}
\begin{proof}
This can be proved by induction on $i$, using the relation $[\Lambda,L]=H$.
\end{proof}
Following the terminology of representations of $\sl(2,\C)$, an element $\alpha\in\bigw^kV^*$ is called \textbf{primitive} if $\Lambda\alpha=0$. The linear subspace of all primitive elements $\alpha\in\bigw^kV^*$ is denoted by $P^k\sub\bigw^kV^*$. Accordingly, an element $\alpha\in\bigw^kV^*_{\C}$ is called \textbf{primitive} if $\Lambda\alpha=0$. Clearly, the subspace of those is just the complexification of $P^k$.
\begin{proposition}\label{almost complex space Lefschetz decomposition}
Let $(V,\langle\cdot\ ,\cdot\rangle)$ be an Euclidian vector space of dimension $2n$ with a compatible almost complex structure $J$ and let $L$ and $\Lambda$ be the associated Lefschetz operators.
\begin{itemize}
\item[(a)] There is a \textbf{Lefschetz decomposition} of the form
\[\bigw^kV^*=\bigoplus_{k\geq 0} L^i(P^{k-2i})\]
which is orthogonal with respect to $\langle\cdot\ ,\cdot\rangle$.
\item[(b)] If $k>n$, then $P^k=0$, and $P^k=\{\alpha\in\bigw^kV^*:L^{n-k+1}\alpha=0\}$ for $k\leq n$.
\item[(c)] The map $L^{n-k}:P^k\to\bigw^{2n-k}V^*$ is injective for $k\leq n$.
\item[(d)] The map $L^{n-k}:\bigw^kV^*\to\bigw^{2n-k}V^*$ is bijective for $k\leq n$.
\end{itemize}
\end{proposition}
\begin{remark}
Since $L$, $\Lambda$, and $H$ are of pure type $(1,1)$, $(-1,-1)$ and $(0,0)$, respectively, the Lefschetz decomposition is compatible with the bidegree decomposition. Therefore we conclude
\[P^k=\bigoplus_{p+q=k}P^{p,q}\]
where $P^{p,q}=P^k_{\C}\cap\bigw^{p,q}V^*$. Since $\Lambda$ and $L$ are real, we also have $\widebar{P^{p,q}}=P^{q,p}$.\par
In particular, we have
\[\bigw^0V^*_{\C}=P^{0,0}=P^0_{\C}=\C,\quad \bigw^1V^*_{\C}=P^{1,0}\oplus P^{0,1},\]
and
\[\bigw^2V^*_{\C}=\bigw^{2,0}V^*\oplus\bigw^{1,1}V^*\oplus\bigw^{0,2}V^*=P^{2,0}\oplus(P^{1,1}\oplus\omega\C)\oplus P^{0,2}.\]
\end{remark}
Roughly, the Lefschetz operators and its dual $\Lambda$ induce a reflection of $\bigw^*V^*$ in the middle exterior product $\bigw^nV^*$. But there is another operator with this property, namely the Hodge $\ast$-operator. We now want to prove some fundamental results concerning the relationship between the operators $\ast$, $L$, and $\Lambda$ which are important in the theory of K\"ahler manifolds. The development we give here is due to Hecht and differs from the more traditional viewpoint of Weil in that a global representation of both $\SL(2,\C)$ and $\sl(2,\C)$ on the Hermitian exterior algebra is utilized, leading to some simple ordinary differential equations which simplifies some of the combinatorial arguments found by Weil.\par
Let $\{z_i=\frac{1}{2}(x_i-Jx_i)\}$ be an orthonormal basis for $V^{1,0}$, so that $\{x_i,y_i=Jx_i\}$ is an orthonormal real basis for $V$. With this, the fundamental form $\omega$ has the form
\[\omega=\frac{\i}{2}\sum_{i=1}^{n}z^i\wedge\bar{z}^i=\sum_{i=1}^{n}x^i\wedge y^i.\]
Now if $\eta$ is any $k$-form in $\bigw^*V^*_{\C}$, we let
\[e(\eta)(\alpha):=\eta\wedge\alpha\]
be the operator acting on $\bigw^*V^*_{\C}$ given by wedging with $\eta$. We note that
\begin{align}\label{almost complex space interoir prod into L,Lambda}
L=e(\omega)=\frac{\i}{2}\sum_{i=1}^{n}e(z^i)e(\bar{z}^i),\quad \Lambda=e^*(\omega)=-\frac{\i}{2}\sum_{i=1}^{n}e^*(\bar{z}^i)e^*(z^i).
\end{align}
It is clear that $[L,e(\eta)]=0$ since $\omega$ is a $2$-form. As for $[\Lambda,e(\eta)]$, we have the following proposition.
\begin{proposition}\label{almost complex space interoir prod commutator with L,Lambda}
Let $\omega$ be the fundamental form of $(V,\langle\cdot\ ,\cdot\rangle)$ and $\eta$ be a real $1$-form. Then
\begin{align}\label{almost complex space interoir prod commutator with L,Lambda-2}
[\Lambda,e(\eta)]=-Je^*(\eta)J^{-1}.
\end{align}
In particular, if $\eta$ is a $(1,0)$-form, we have
\begin{align}\label{almost complex space interoir prod commutator with L,Lambda-1}
[\Lambda,e(\eta)]=-\i e^*(\bar{\eta}),\quad [\Lambda,e(\bar{\eta})]=\i e^*(\eta).
\end{align}
\end{proposition}
\begin{proof}
If $\eta$ is a real $1$-form, then we claim that
\[e(\eta)^*=\ast e(\eta)\ast.\]
To see this, we note that for any $\alpha,\beta\in\bigw^*V^*_{\C}$ with degrees $k-1$ and $k$, we have (note that $\eta\wedge\ast\beta$ is of degree $2n-k+1$)
\begin{align*}
\langle e(\eta)\alpha,\beta\rangle\cdot\mathrm{vol}=\eta\wedge\alpha\wedge\ast\beta=(-1)^{k-1}\alpha\wedge(\eta\wedge\ast\beta)=\alpha\wedge\ast\ast(\eta\wedge\ast\beta)=\langle\alpha,\ast e(\eta)\ast\beta\rangle\cdot\mathrm{vol}.
\end{align*}
Now by computing the right side $\ast e(\eta)\ast$, we see
\begin{equation}\label{almost complex space interoir prod commutator with L,Lambda-3}
e^*(z^j)(z^I\wedge\bar{z}^K)=\begin{cases}
0,&j\notin I,\\
2\eps(j,I)z^{I-\{j\}}\wedge\bar{z}^K,&j\in I,
\end{cases}
\end{equation}
\begin{equation}\label{almost complex space interoir prod commutator with L,Lambda-4}
e^*(\bar{z}^k)(z^I\wedge\bar{z}^K)=\begin{cases}
0,&k\notin K,\\
2(-1)^{|I|+n(k,K)}z^I\wedge\bar{z}^{K-\{k\}},&k\in K.
\end{cases}
\end{equation}
where $\eta(j,I)$ is the number of indices in $I$ that is strictly smaller than $j$, and similar for $\eta(k,K)$. Now by (\ref{almost complex space interoir prod into L,Lambda}), we then have
\begin{align*}
[\Lambda,e(z^j)]&=-\frac{\i}{2}\Big[\sum_{i=1}^{n}e^*(\bar{z}^i)e^*(z^i)e(z^j)-\sum_{i=1}^{n}e(z^j)e^*(\bar{z}^i)e^*(z^i)\Big]\\
&=-\frac{\i}{2}[e^*(\bar{z}^j)e^*(z^j)e(z^j)-e(z^j)e^*(\bar{z}^j)e^*(z^j)]
\end{align*}
since $e(z^j)$ commutes with $e^*(z^i)$ and $e^*(\bar{z}^i)$ for $i\neq j$, which follows from (\ref{almost complex space interoir prod commutator with L,Lambda-3}) and (\ref{almost complex space interoir prod commutator with L,Lambda-4}). We now consider the action of $[\Lambda,e(z^j)]$ on $z^I\wedge\bar{z}^K$, which is distinguished into two cases: if $j\in I$, then $e(z^j)(z^I\wedge\bar{z}^K)=0$, so
\[[\Lambda,e(z^j)](z^I\wedge\bar{z}^K)=\frac{\i}{2}(-1)^{n(j,I)}e(z^j)e^*(\bar{z}^j)(z^{I-\{j\}}\wedge\bar{z}^K).\]
Again, if $j\notin K$ then the right side is identically zero, and otherwise, we have
\begin{align*}
[\Lambda,e(z^j)](z^I\wedge\bar{z}^K)&=\i\cdot(-1)^{n(j,I)+\eta(j,K)+|I|-1}e(z^j)(z^{I-\{j\}}\wedge\bar{z}_{K-\{j\}})\\
&=-\i\cdot(-1)^{|I|+\eta(j,K)}z^I\wedge\bar{z}_{K-\{j\}}.
\end{align*}
On the other hand, if $j\notin I$, then $e^*(z^j)(z^I\wedge\bar{z}^K)=0$, so
\[[\Lambda,e(z^j)](z^I\wedge\bar{z}^K)=-\frac{\i}{2}e^*(\bar{z}^j)e^*(z^j)(z^j\wedge z^I\wedge\bar{z}^K).\]
Again, this is zero if $j\notin K$, and if $j\in K$ we have
\begin{align*}
[\Lambda,e(z^j)](z^I\wedge\bar{z}^K)=-\i\cdot(-1)^{|I|+n(j,K)}z^I\wedge\bar{z}^{K-\{j\}}.
\end{align*}
In both cases, we have $[\Lambda,e(z^j)](z^I\wedge\bar{z}^K)=-\i e^*(\bar{z}^j)(z^I\wedge\bar{z}^K)$, so the first equality of (\ref{almost complex space interoir prod commutator with L,Lambda-1}) follows. The second one can be proved similarly. Finally, to see (\ref{almost complex space interoir prod commutator with L,Lambda-2}), we simply note that any real $1$-form can be written in the form $\eta=\alpha+\beta$, where $\alpha$ is of type $(1,0)$ and $\beta$ is of type $(0,1)$. Then we can check that
\[-\i e^*(\alpha)=-Je^*(\bar{\alpha})J^{-1},\quad \i e^*(\beta)=-Je^*(\beta)J^{-1}.\]
Combine this with (\ref{almost complex space interoir prod commutator with L,Lambda-1}), we see (\ref{almost complex space interoir prod commutator with L,Lambda-2}) follows.
\end{proof}
With these preparations made, we now want to prove two basic lemmas due to Hecht. We introduce the following operator on $\bigw^*V^*_{\C}$ induced by the action of $\SL(2,\C)$ on $\bigw^*V^*_{\C}$ by the representation of $\sl(2,\C)$. Recall that the Weyl element $\theta$ in $\SL(2,\C)$ is given by
\[\theta=\exp\Big(\frac{\i\pi}{2}(e+f)\Big)=\begin{pmatrix}
0&i\\
i&0
\end{pmatrix}.\]
The point is that conjugation by $\theta$ in $\sl(2,\C)$ gives rise to a reflection with respect to basis $(e,h,f)$ (the Weyl group reflection). Namely,
\[\theta h\theta^{-1}=-h,\quad \theta e\theta^{-1}=f,\quad \theta f\theta^{-1}=e.\]
Now we let $\hash$ be the action of $\theta$ on $\bigw^*V^*_{\C}$, that is,
\[\hash=\theta_{\bigw^*V^*_{\C}}=\exp\Big(\frac{\pi\i}{2}(e+f)\Big)_{\bigw^*V^*_{\C}}=\exp\Big(\frac{\pi\i}{2}(\Lambda+L)\Big).\]
\begin{lemma}\label{almost complex space action by hash}
Let $(V,\rho)$ be a representation of $\sl(2,\C)$ and $v\in V$ be a primitive element of weight $n$. Let $\pi$ be the induced representation of $\SL(2,\C)$ on $V$, then for $0\leq k\leq n-1$,
\begin{align}\label{almost complex space action by hash-1}
\pi(\theta)\rho(f)^k(v)=\i^n\frac{k!}{(n-k)!}\rho(f)^{n-k}v.
\end{align}
\end{lemma}
\begin{proof}
Let $\SL(2,\C)$ act on $\C^2$ by left matrix multiplication, and we consider $\mathrm{Sym}^n(\C^2)$, the $n$-fold symmetric tensor product of $\C^2$ with itself. Then $V$ is isomorphic to this representation, so we consider $V=\mathrm{Sym}^n(\C^2)$. If $n=1$, we let
\[v=\begin{pmatrix}
1\\
0
\end{pmatrix}\quad w=\begin{pmatrix}
0\\
1
\end{pmatrix}\]
be a basis of $\C^2$. Then we have $h\cdot v=v$, $e\cdot v=0$, and $f\cdot v=w$, so $v$ is a primitive vector. We also note that $\theta\cdot v=\i w=\i fv$, so (\ref{almost complex space action by hash-1}) holds in this case.\par
Now the general case is obtained by taking symmetric power on $\C^2$. That is, if we set
\[\omega_i=\binom{n}{i}v^{n-i}w^i\]
then $\omega_0=v^n$ is a primitive vector of $V$, and we have
\[\rho(f)^k\omega_0=\frac{n!}{(n-k)!}v^{n-k}w^k=\frac{1}{k!}\omega_k.\]
Therefore, the action of $\theta$ on $\rho(f)^k\omega_0$ is given by
\[\pi(\theta)\rho(f)^k\omega_0=\frac{n!}{(n-k)!}\pi(\theta)(v^{n-k}w^k)=\i^n\frac{k!}{(n-k)!}\frac{n!}{k!}v^kw^{n-k}=\i^n\frac{k!}{(n-k)!}\rho(f)^{n-k}\omega_0.\]
Since every irreducible representation of $V$ is isomorphic to some $\mathrm{Sym}^n(\C^2)$, we see the claim follows.
\end{proof}
\begin{lemma}\label{almost complex space conjugate by hash}
Let $\eta$ be a real $1$-form. Then
\begin{align}\label{almost complex space conjugate by hash-1}
\hash e(\eta)\hash^{-1}=-\i Je^*(\eta)J^{-1}.
\end{align}
\end{lemma}
\begin{proof}
For simplicity we write $e=e(\eta)$. Now for $t\in\C$, we consider the operator
\[e_t=\exp(\i t(\Lambda+L))\cdot e\cdot \exp(-\i t(\Lambda+L)).\]
We note that $e_{\pi/2}=\hash e\hash^{-1}$. We will see that $e_t$ satisfies a simple differential equation with initial condition $e_0=e$, which can be easily solved, and evaluating the solution at $t=1/2\pi$ will give the desired result. We first note that by the equality $\exp(\ad(A))=\Ad(\exp(A))$, we have
\begin{align}\label{almost complex space conjugate by hash-2}
e_t=\exp(\ad(\i t(\Lambda+L)))(e)=\sum_{k=0}^{\infty}\frac{\ad^k(\i t(\Lambda+L))}{k!}(e).
\end{align}
Now $\ad^k(\Lambda+L)$ is a sum of monomials in $\ad(\Lambda)$ and $\ad(L)$. Since $\Lambda L=L\Lambda+H$, $\ad(L)(e)=0$, and $\ad(H)(e)=-e$ (since $\eta$ is of degree $1$), we see that $e_t$ can be expressed in the form
\[e_t=\sum_{k=0}^{\infty}a_k(t)\ad^k(\Lambda)(e),\]
where $a_k(t)$ are real-analytic functions in $t$. Now (\ref{almost complex space interoir prod commutator with L,Lambda-2}) implies that $\ad(\Lambda)(e)=0$ for $k\geq 2$, since $\Lambda$ commutes with $J$ and $e^*$. Thus
\begin{align}\label{almost complex space conjugate by hash-3}
e_t=a_0(t)e+a_1(t)\ad(\Lambda)(e).
\end{align}
We now consider differentiating with respect to $t$. From (\ref{almost complex space conjugate by hash-2}), we see $e_t$ satisfies the differential equation
\[\begin{cases}
e'_t=\i(\ad(\Lambda)+\ad(L))(e_t),\\
e_0=e.
\end{cases}\]
We can solve this using (\ref{almost complex space conjugate by hash-3}). Namely, we have
\begin{align}\label{almost complex space conjugate by hash-4}
e_t=a_0'(t)e+a_1'(t)\ad(\Lambda)(e),
\end{align}
which must equal to the product
\begin{align*}
\i(\ad(\Lambda+L))[a_0(t)e+a_1(t)\ad(\Lambda)(e)]=\i a_0(t)\ad(\Lambda)(e)+\i a_1(t)\ad(L)\ad(\Lambda)(e),
\end{align*}
using the fact that $\ad^2(\Lambda)(e)=0$ and $\ad(L)(e)=0$. But
\[\ad(L)\ad(\Lambda)(e)=\ad([L,\Lambda])(e)+\ad(\Lambda)\ad(L)(e)=\ad(-H)(e)=e,\]
so the right side of (\ref{almost complex space conjugate by hash-4}) must equal to $\i a_0(t)\ad(\Lambda)(e)+\i a_1(t)e$. Comparing the coefficients, we conclude that
\[\begin{cases}
a_0'(t)=\i a_1(t),\\
a_1'(t)=\i a_0(t).
\end{cases}\]
Then by letting $a_0(t)=\cos t$ and $a_1(t)=\i\sin t$, we find that
\[e_t=\cos t\cdot e+\i\sin t\cdot\ad(\Lambda)(e).\]
Now set $t=\pi/2$, we then find that $e_{\pi/2}=\i[\Lambda,e]$, which proves the lemma.
\end{proof}
\begin{lemma}\label{almost complex space hash and Hodge star}
For any $\alpha\in\bigw^kV^*$, we have
\begin{align}\label{almost complex space hash and Hodge star-1}
\ast\alpha=\i^{k^2-n}J^{-1}\hash\alpha.
\end{align}
\end{lemma}
\begin{proof}
We first note that the $\ast$-operator is characterized by
\begin{align}\label{almost complex space hash and Hodge star-2}
\ast 1=\mathrm{vol}=\frac{1}{n!}L^n(1),\quad  \ast e(\eta)=(-1)^ke^*(\eta)\ast,
\end{align}
as an operator on $\bigw^kV^*_{\C}$ for any real $1$-form $\eta$, as the forms obtained from $1$ by repeated application of $e(\eta)$ span $\bigw^*V^*_{\C}$. The first equation is clear, and the second one follows from Proposition~\ref{almost complex space Hodge star operator prop}(c): for a $k$-form $\alpha$, we have
\[\ast e(\eta)\alpha=\ast e(\eta)\ast\ast^{-1}\alpha=(-1)^ke^*(\eta)\ast\alpha.\]
Now consider the operator
\[\tilde{\ast}=\i^{k^2-n}J\hash\]
defined on $\bigw^kV^*_{\C}$. We recall that by (\ref{almost complex space action by hash-1}),
\[\hash L^k\alpha=\i^n\frac{k!}{(n-k)!}L^{n-k}\alpha\]
if $\alpha$ is primitive of weight $n$. But $1$ is just a primitive form of weight $n$, so apply this equality we have
\[\hash 1=\frac{\i^n}{n!}L^{n}(1).\]
We then conclude that
\[\widetilde{\ast}1=\i^{-n}\cdot\frac{\i^n}{n!}L^n(1)=\frac{L^n(1)}{n!}=\mathrm{vol}.\]
Similarly, if $\eta$ is a real $1$-form and $\alpha$ is a $k$-form, using (\ref{almost complex space conjugate by hash-1}) we see 
\begin{align*}
\tilde{\ast}e(\eta)\alpha&=\i^{(k+1)^2-n}J^{-1}\hash e(\eta)\alpha=\i^{k^2-n}(-1)^k\i J^{-1}\hash e(\eta)\hash^{-1}\hash\alpha\\
&=\i^{k^2-n}(-1)^ke^*(\eta)J^{-1}\hash\alpha=(-1)^ke^*(\eta)\tilde{\ast}\alpha.
\end{align*}
This verifies (\ref{almost complex space hash and Hodge star-2}) for $\tilde{\ast}$, so $\ast=\tilde{\ast}$.
\end{proof}
We now prove the following mysterious but extremely useful result about the interplay between $L$ and $\ast$.
\begin{proposition}\label{almost complex space star and L power}
Let $\alpha$ be a primitive $k$-form in $\bigw^*V^*$. Then for $0\leq j\leq n-k$, we have
\begin{align}
\ast L^j\alpha=(-1)^{\frac{k(k+1)}{2}}\frac{j!}{(n-k-j)!}L^{n-k-j}J\alpha
\end{align}
\end{proposition}
\begin{proof}
Let $V^*_\alpha$ be the subspace of $\bigw^*V^*$ generated by $\{L^j\alpha:0\leq j\leq n-k\}$. Then $V^*_\alpha$ is an irreducible representation of $\sl(2,\C)$, and by (\ref{almost complex space action by hash-1}) we have
\[\hash L^j\alpha=\i^{n-k}\frac{j!}{(n-k-j)!}L^{n-k-j}\alpha.\]
Hence, by Lemma~\ref{almost complex space hash and Hodge star}, if $\alpha\in P^k$ we have (recall that $J^2=\sum_k(-1)^k\Pi^k$)
\begin{align*}
\ast L^j\alpha&=\i^{(k+2j)^2-n}J^{-1}\hash L^j\alpha=\i^{k^2-n}J^{-1}\i^{n-k}\frac{j!}{(n-k-j)!}L^{n-k-j}\alpha\\
&=\i^{k^2-k}(J^{-1})^2\frac{j!}{(n-k-j)!}L^{n-k-j}J\alpha\\
&=\i^{k^2-k}(-1)^{p-q}\frac{j!}{(n-k-j)!}L^{n-k-j}\alpha\\
&=\i^{k^2-k}(-1)^{k}\frac{j!}{(n-k-j)!}L^{n-k-j}\alpha\\
&=(-1)^{\frac{k(k+1)}{2}}\frac{j!}{(n-k-j)!}L^{n-k-j}\alpha.\qedhere
\end{align*}
\end{proof}
\begin{example}
Here are a few instructive special cases. Let $j=k=0$ and $\alpha=1$, then we obtain 
\[\ast 1=\frac{1}{n!}L^n1=\frac{\omega^n}{n!}.\]
Thus, $\mathrm{vol}=\omega^n/n!$ as was claimed before. Also, for $k=0$, $j=1$, and $\alpha=1$, the proposition yields 
\[\ast\omega=\frac{\omega^{n-1}}{(n-1)!}\]
Finally, if $\alpha$ is a primitive $(1,1)$-form, then
\[\ast\alpha=\frac{-1}{(n-2)!}\omega^{n-2}\wedge\alpha.\]
\end{example}
Let $(V,\langle\cdot\ ,\cdot\rangle,J)$ be as before and let $\omega$ be the associated fundamental form. The \textbf{Hodge-Riemann pairing} of $V$ is the bilinear form
\[Q:\bigw^kV^*\times\bigw^kV^*\to\R,\quad (\alpha,\beta)\mapsto(-1)^{\frac{k(k-1)}{2}}\alpha\wedge\beta\wedge\omega^{n-k}\]
where $\bigw^{2n}V^*$ is identified with $\R$ via the volume form $\mathrm{vol}$. By definition $Q=0$ on $\bigw^kV^*$ for $k>n$. We will also denote by $Q$ the $\C$-linear extension of the Hodge-Riemann pairing to $\bigw^*V^*_{\C}$. Our final result will be an orthogonal condition on $Q$:
\begin{proposition}[\textbf{Hodge-Riemann bilinear relation}]\label{almost complex space Hodge-Riemann bilinear relation}
Let $(V,\langle\cdot\ ,\cdot\rangle,J)$ be an Euclidian vector space endowed with a compatible almost complex structure. Then the associated Hodge-Riemann pairing $Q$ satisfies
\[Q(\bigw^{p,q}V^*,\bigw^{p',q'}V^*)=0\]
for $(p,q)\neq(q',p')$ and
\[\i^{p-q}Q(\alpha,\bar{\alpha})=(n-(p+q))!\cdot\langle\alpha,\alpha\rangle_{\C}>0\]
for nonzero $\alpha\in P^{p,q}$ with $p+q\leq n$.
\end{proposition}
\begin{proof}
From the definition of $Q$ and the fact that $\omega$ is of type $(1,1)$, the first assertion follows immediately. We only need to verify the second assertion. By definition,
\begin{align*}
Q(\alpha,\bar{\alpha})\cdot\mathrm{vol}&=(-1)^{\frac{k(k-1)}{2}}\alpha\wedge\bar{\alpha}\wedge\omega^{n-k}=(-1)^{\frac{k(k-1)}{2}}\alpha\wedge L^{n-k}\bar{\alpha}\\
&=(-1)^{\frac{k(k-1)}{2}}\langle\alpha,\beta\rangle_{\C}\cdot\mathrm{vol},
\end{align*}
where $k=p+q$ and $\beta$ is the $p$-form given by $\ast\bar{\beta}=L^{n-k}\bar{\alpha}$. Hence $\ast^2\bar{\beta}=(-1)^k\bar{\beta}$ and, on the other hand, by Proposition~\ref{almost complex space star and L power} we have
\[\ast^2\bar{\beta}=\ast L^{n-k}\bar{\alpha}=(-1)^{\frac{k(k+1)}{2}}(n-k)!\i^{p-q}\bar{\alpha}.\]
Thus $\beta=(-1)^{\frac{k(k+1)}{2}+k}(n-k)!\i^{p-q}\alpha$ and we then see
\[Q(\alpha,\bar{\alpha})=(-1)^{\frac{k(k+1)}{2}+\frac{k(k-1)}{2}+k}(n-k)!\i^{p-q}\langle\alpha,\alpha\rangle_{\C}.\]
This yields $\i^{p-q}Q(\alpha,\bar{\alpha})=(n-k)!\cdot\langle\alpha,\alpha\rangle_{\C}>0$ for nonzero $\alpha\in P^{p,q}$.
\end{proof}
\begin{example}
Suppose that $n\geq 2$ and consider the decomposition $(\bigw^{1,1}V^*)_{\R}=\omega\R\oplus P^{1,1}_{\R}$, where $(-)_{\R}$ denotes the intersection with $\bigw^2V^*$. This decomposition is $Q$-orthogonal, because $(\alpha\wedge\omega)\wedge\omega^{n-2}=\alpha\wedge\omega^{n-1}=L^{n-1}\alpha=0$ for any $\alpha\in P^2$ (Proposition~\ref{almost complex space Lefschetz decomposition}(b)). Moreover, $Q$ is a positive definite symmetric bilinear form on $\omega\R$ and a negative definite symmetric bilinear from on $P^{1,1}_{\R}$.
\end{example}
\subsection{Tangent space and differential forms on \texorpdfstring{$\C^n$}{C}}
A real manifold $M$ is studied by means of its tangent bundle $TM$, the collection of all tangent spaces $T_xM$ for $x\in M$, and its $k$-form bundle $\bigw^kT^*M$. In this subsection we will apply the linear algebra developed previously to the form bundles of an open subset $M=U\sub\C^n$. The bidegree decomposition induces a decomposition of the exterior differential $d$ which is well suited for the study of holomorphic functions on $U$. We conclude by a local characterization of so called K\"ahler metrics which will be of central interest in the global setting.\par
Let $U\sub\C^n$ be an open subset. Then $U$ can in particular be considered as a $2n$-dimensional real manifold. For $x\in U$ we have its real tangent space $T_xU$ at the point $x$ which is of real dimension $2n$. A canonical basis of $T_xU$ is given by the tangent vectors
\[\frac{\partial}{\partial x^1},\dots,\frac{\partial}{\partial x^n},\frac{\partial}{\partial y^1},\dots,\frac{\partial}{\partial y^n}.\]
where $z^i=x^i+\i y^i$ are the standard coordinates on $\C^n$. Moreover, the vectors $\frac{\partial}{\partial x^1},\dots,\frac{\partial}{\partial y^n}$ are global trivializing sections of $TU$.\par
Each tangent space $T_xU$ admits a natural almost complex structure defined by
\[J:T_xU\to T_xU,\quad\frac{\partial}{\partial x^i}\mapsto\frac{\partial}{\partial y^i},\frac{\partial}{\partial y^i}\mapsto-\frac{\partial}{\partial x^i}.\]
which is compatible with the global trivialization. We shall regard $J$ as a vector bundle endomorphism of the real vector bundle $TU$ over $U$. The dual basis of $(T_xU)^*$ is denoted by $dx^1,\dots,dx^n,dy^1,\dots,dy^n$. Recall that the induced almost complex structure on $T_xU$ in terms of this dual basis is described by $J(dx^i)=-dy^i$, $J(dy^i)=dx^i$.\par
The general theory developed in the previous subsections applies to this almost complex structure and yields the following result.
\begin{proposition}
The complexified tangent bundle $TU_{\C}:=TU\otimes_{\R}\C$ decomposes as a direct sum of complex vector bundles
\[T_{\C}U=T^{1,0}U\oplus T^{0,1}U,\]
such that the complex linear extension of $J$ satisfies
\[J|_{T^{1,0}U}=\i\cdot\id,\quad J|_{T^{0,1}U}=-\i\cdot\id.\]
The vector bundles $T^{1,0}U$ and $T^{0,1}U$ are trivialized by the sections
\[\frac{\partial}{\partial z^i}=\frac{1}{2}\Big(\frac{\partial}{\partial x^i}-\i\frac{\partial}{\partial y^i}\Big)\And\frac{\partial}{\partial\bar{z}^i}=\frac{1}{2}\Big(\frac{\partial}{\partial x^i}+\i\frac{\partial}{\partial y^i}\Big)\]
for $i=1,\dots,n$, respectively.
\end{proposition}
The complexified cotangent bundle $T^*_{\C}U:=T^*U\otimes_{\R}\C$ admits an analogous decomposition $T^*U_{\C}=(T^*U)^{1,0}\oplus(T^*U)^{0,1}$, where $(T^*U)^{1,0}$ and $(T^*U)^{0,1}$ are trivialized by the dual basis $dz^i:=dx^i+\i dy^i$ and $d\bar{z}^i:=dx^i-\i dy^i$, for $i=1,\dots,n$, respectively. Note that these decompositions are compatible with restriction to smaller open subsets $U'\sub U$.
\begin{proposition}
Let $f:U\to V$ be a holomorphic map between open subsets $U\sub\C^m$ and $V\sub\C^n$. Then the $\C$-linear extension of the differential $df:T_xU\to T_{f(x)}V$ respects the above decomposition, i.e., $df(T_x^{0,1}U)\sub T^{0,1}_{f(x)}V$ and $df(T_x^{1,0}U)\sub T^{1,0}_{f(x)}V$.
\end{proposition}
\begin{proof}
Since $f$ is holomorphic, the extension of $df$ is given by the Jacobian matrix
\[\begin{pmatrix}
J(f)&0\\
0&\widebar{J(f)}
\end{pmatrix}\]
It then follows that $df$ maps $T^{1,0}U$ into $T^{1,0}V$ and $T^{0,1}U$ into $T^{0,1}V$.
\end{proof}
In a similar fashion, we can use the previous results in order to decompose the bundles of $k$-forms. Let $U\sub\C^n$ be an open subset. Over $U$ one defines the complex vector bundles
\[\bigw^{p,q}T^*U=\bigw^p(T^*U)^{1,0}\otimes\bigw^q(T^*U)^{0,1}.\]
By $\mathcal{A}^k_{\C}(U)$ and $\mathcal{A}^{p,q}(U)$ we denote the spaces of sections of $\bigw^kT^*_{\C}U$ and $\bigw^{p,q}T^*U$, respectively. Now by Proposition~\ref{}, we have the following decomposition for $\bigw^kT^*_{\C}U$:
\begin{proposition}
There are natrual decompositions
\[\bigw^k_{\C}T^*U=\bigoplus_{p+q=k}\bigw^{p,q}T^*U,\quad \mathcal{A}_{\C}^k(U)=\bigoplus_{p+q=k}\mathcal{A}^{p,q}(U).\]
\end{proposition}
We note that the restriction map $\mathcal{A}^k(U)\to\mathcal{A}^{k}(U')$ for an open subset $U'\sub U$ respects this decomposition. As before, the projection operators $\bigw^k_{\C}T^*U\to\bigw^{p,q}T^*U$ and $\mathcal{A}^k_{\C}(U)\to\bigw^{p,q}(U)$ will be denoted by $\Pi^{p,q}$.\par
We now consider the complex extension $d:\mathcal{A}^k_{\C}(U)\to\mathcal{A}^{k+1}_{\C}(U)$ of the usual exterior differential $d$ on $k$-forms. Then
\[\partial:\mathcal{A}^{p,q}(U)\to\mathcal{A}^{p+1,q}(U),\quad\bar{\partial}:\mathcal{A}^{p,q}(U)\to\mathcal{A}^{p,q+1}(U)\]
are defined by $\partial:=\Pi^{p+1,q}\circ d$ and $\bar{\partial}:=\Pi^{p,q+1}\circ d$. For any local function $f$ we have
\begin{align}\label{manifold complex differential expression on C^n}
df=\sum_i\frac{\partial f}{\partial x^i}dx^i+\sum_i\frac{\partial f}{\partial y^i}dy^i=\sum_i\frac{\partial f}{\partial z^i}dz^i+\sum_i\frac{\partial f}{\partial\bar{z}^i}dz^i.
\end{align}
Thus, $f$ is holomorphic if and only if $\bar{\partial}=0$. Moreover, using (\ref{manifold complex differential expression on C^n}), the operators $\partial$ and $\bar{\partial}$ can be explicitly expressed as
\begin{align*}
&\partial(f\,dz^{i_1}\wedge\cdots\wedge dz^{i_p}\wedge d\bar{z}^{j_1}\wedge\cdots\wedge d\bar{z}^{j_q})=\sum_{k=1}^{n}\frac{\partial f}{\partial z^k}dz^k\wedge dz^{i_1}\wedge\cdots\wedge dz^{i_p}\wedge d\bar{z}^{j_1}\wedge\cdots\wedge d\bar{z}^{j_q},\\
&\bar{\partial}(f\,dz^{i_1}\wedge\cdots\wedge dz^{i_p}\wedge d\bar{z}^{j_1}\wedge\cdots\wedge d\bar{z}^{j_q})=\sum_{k=1}^{n}\frac{\partial f}{\partial \bar{z}^k}d\bar{z}^k\wedge dz^{i_1}\wedge\cdots\wedge dz^{i_p}\wedge d\bar{z}^{j_1}\wedge\cdots\wedge d\bar{z}^{j_q},
\end{align*}
\begin{proposition}\label{complex manifold partial bar-partial on C^n prop}
For the differentials $\partial$ and $\bar{\partial}$, we have
\begin{itemize}
\item[(a)] $d=\partial+\bar{\partial}$.
\item[(b)] $\partial^2=\bar{\partial}^2=0$ and $\partial\bar{\partial}+\bar{\partial}\partial=0$.
\item[(c)] The operators $\partial$ and $\bar{\partial}$ satisfy the Leibniz rule, i.e.,
\begin{align*}
\partial(\alpha\wedge\beta)&=\partial(\alpha)\wedge\beta+(-1)^{p+q}\alpha\wedge\partial(\beta),\\
\partial(\alpha\wedge\beta)&=\partial(\alpha)\wedge\beta+(-1)^{p+q}\alpha\wedge\partial(\beta),
\end{align*} 
for $\alpha\in\mathcal{A}^{p,q}(U)$ and $\beta\in\mathcal{A}^{r,s}(U)$.
\end{itemize}
\end{proposition}
\begin{proof}
The first assertion follows from the description of $\partial$ and $\bar{\partial}$ given above, and the second one from the expansion $0=d^2=(\partial+\bar{\partial})^2$. For the third one, we recall that
\[d(\alpha\wedge\beta)=d\alpha\wedge\beta+(-1)^{p+q}\alpha\wedge d\beta.\]
Taking the $(p+r+1,q+s)$-parts on both sides one obtains the Leibniz rule for $\partial$. Similarly, taking $(p+r,q+s+1)$-parts proves the assertion for $\bar{\partial}$.
\end{proof}
Since $\partial$ and $\bar{\partial}$ share the usual properties of the exterior differential $d$ and reflect the holomorphicity of functions, it seems natural to build up a holomorphic analogue of the de Rham complex. As we work here exclusively in the local context, only the local aspects will be discussed. Of course, locally the de Rham complex is exact due to the standard Poincar\'e lemma. We will show that this still holds true for $\bar{\partial}$ (and $\partial$).
\begin{proposition}[\textbf{$\bm{\bar{\partial}}$-Poinca\'e Lemma on $\bm{\C}$}]
Consider an open neighbourhood of the closure of a bounded one-dimensional disc $B_\eps\sub\widebar{B}_\eps\sub U\sub\C$. For $\alpha=f\,d\bar{z}\in\mathcal{A}^{0,1}(U)$ the function
\[g(z)=\frac{1}{2\pi i}\int_{B_\eps}\frac{f(w)}{w-z}dw\wedge d\bar{w}\]
on $B_\eps$ satisfies $\alpha=\bar{\partial}g$.
\end{proposition}
The following proposition and its corollary are known as the Grothendieck-Poincar\'e lemma. The first proof of it is due to Grothendieck and was presented by Serre in the Seminaire Cartan in 1958.
\begin{proposition}[\textbf{$\bm{\bar{\partial}}$-Poinca\'e Lemma on $\bm{\C^n}$}]
Let $U$ be an open neighbourhood of the closure of a bounded poly disc $B_\eps\sub\widebar{B}_\eps$ in $\C^n$. If $\alpha\in\mathcal{A}^{p,q}(U)$ is $\bar{\partial}$-closed and $q>0$, then there exists a form $\beta\in\mathcal{A}^{p,q-1}(B_\eps)$ with $\alpha=\bar{\partial}\beta$ on $B_\eps$.
\end{proposition}
\begin{corollary}[\textbf{$\bm{\bar{\partial}}$-Poinca\'e Lemma on Polydisc}]
Let $B$ be a polydisc in $\C^n$ which can be unbounded. If $\alpha\in\mathcal{A}^{p,q}(B)$ is $\bar{\partial}$-closed and $q>0$, then there exists $\beta\in\mathcal{A}^{p,q-1}(B)$ with $\alpha=\bar{\partial}\beta$.
\end{corollary}
\begin{remark}
It is easy to see that $\widebar{\partial\alpha}=\bar{\partial}\bar{\alpha}$ for any $\alpha\in\mathcal{A}^{p,q}(U)$, so by conjugation we can also form a $\partial$-Poincar\'e lemma for open disk $U$ in $\C^n$.
\end{remark}
\begin{corollary}[\textbf{$\bm{\bar{\partial}}\partial$-Poinca\'e Lemma on Polydisc}]
Let $B$ be a polydisc in $\C^n$. If $\alpha\in\mathcal{A}^{p,q}(B)$ is $d$-closed, then for any point $x\in B$ there is a neighbourhood $U$ of $x$ and $\beta\in\mathcal{A}^{p-1,q-1}(U)$ such that $\partial\bar{\partial}\beta=\alpha$ in $U$. 
\end{corollary}
\begin{proof}
The proof consists of an application of the Poincar\'e lemmas for the operators $d$, $\partial$, and $\bar{\partial}$. Namely, since $d\alpha=0$, for each $x\in B$ there is a neighbourhood $U$ of $x$ and $\eta\in\mathcal{A}^{k-1}(U)$ such that $d\eta=\alpha$, where $k=p+q$ is the total degree of $\alpha$. Thus we see that if we write $\eta=\sum_{i,j}\eta^{i,j}$, we have
\[d\eta=\bar{\partial}\eta^{p,q-1}+\partial\eta^{p-1,q},\quad \bar{\partial}\eta^{p-1,q}=\partial\eta^{p,q-1}=0.\]
and by shrinking $U$ there exists (by the $\bar{\partial}$ and $\partial$ Poincar\'e lemmas) forms $\beta_1\in\mathcal{A}^{p-1,q-1}(U)$ and $\beta_2\in\mathcal{A}^{p-1,q-1}(U)$ so that
\[\partial\beta_1=\eta^{p,q-1},\quad \bar{\partial}\beta_2=\eta^{p-1,q}.\]
Now we then see
\[\alpha=d\beta=\bar{\partial}\beta_1+\partial\bar{\partial}\beta_2=\partial\bar{\partial}(\beta_2-\beta_1).\]
This completes the proof of the corollary.
\end{proof}
So far, only the consequences of the existence of a natural (almost) complex structure on each $T_xU$ have been discussed. Following the presentation of the last part, we shall conclude by combining this with certain metric aspects of the manifold $U$. Let $U\sub\C^n$ be an open subset and consider a Riemannian metric $g$ on $U$. For what follows we may always assume that $U$ is a polydisc. The metric $g$ is compatible with the natural (almost) complex structure on $U$ if for any $x\in U$ the induced scalar product $g_x$ on $T_xU$ is compatible with the induced almost complex structure $J$, i.e. $g_x(v,w)=g_x(Jv,Jw)$ for all $v,w\in T_xU$. Recall that in this situation a natural $(1,1)$-for $\omega\in\mathcal{A}^{1,1}(U)\cap\mathcal{A}^2(U)$ defined by
\[\omega=g(J(-),(-)),\]
which is called the fundamental form of $g$. Moreover, $h:=g-\i\omega$ defines a positive Hermitian form on the complex vector spaces $(T_xU,g_x)$ for any $x\in U$.
\begin{example}
Let $g$ be the constant standard metric such that
\[\frac{\partial}{\partial x^1},\dots,\frac{\partial}{\partial x^n},\frac{\partial}{\partial y^1},\dots,\frac{\partial}{\partial y^n}\]
is an orthonormal basis for any $T_xU$. Clearly, complex structure and $g$ are compatible. The form $\omega$ in this is case is
\[\omega=\frac{\i}{2}\sum_{i=1}^{n}dz^i\wedge d\bar{z}^i.\]
\end{example}
An arbitrary metric $g$ on $U$, if compatible with the almost complex structure, is uniquely determined by the matrix $h_{ij}(z):=h(\partial/\partial x^i,\partial/\partial x^j)$. The fundamental form can then be written as (Proposition~\ref{almost complex space fundamental class express})
\begin{align}\label{complex manifold fundamental on C^n express}
\omega=\frac{\i}{2}\sum_{i,j=1}^{n}h_{ij}\,dz^i\wedge d\bar{z}^j.
\end{align}
Even if $g$ is not the standard metric, one might try to change the complex coordinates such that it becomes the standard metric with respect to the new coordinates. Of course, this cannot always be achieved, but a reasonable class of metrics can be then defined: we say the metric $g$ \textbf{osculates in the origin to order two} to the standard metric if $(h_{ij})=\delta_{ij}+O(|z|^2)$. Explicitly, the condition means 
\[\frac{\partial h_{ij}}{\partial z^k}(0)=\frac{\partial h_{ij}}{\partial\bar{z}^k}(0)=0\]
for all $i,j,k$. In other words, the power series expansion of $(h_{ij})$ differs from the constant matrix $(\delta_{ij})$ by terms of order at least two, thus terms of the form $a_{ij}z^k+b_{ij}\bar{z}^k$ do not occur.\par
Osculating metrics will provide the local models of K\"ahler metrics which will be extensively studied in the later chapters. Here is the crucial fact:
\begin{proposition}\label{complex manifold metric on C^n Kahler iff two order}
Let $g$ be a compatible metric on $U$ and let $\omega$ be the associated fundamental form. Then $d\omega=0$ if and only if for any point $x\in U$ there exist a neighbourhood $V$ of $0\in\C^n$ and a local biholomorphic map $f:V\cong f(V)\sub U$ with $f(0) =x$ and such that $f^*g$ osculates in the origin to order two to the standard metric.
\end{proposition}
\begin{proof}
First note that for any local biholomorphic map $f$ the pull-back $f^*\omega$ is the associated fundamental form to $f^*g$. In particular, $\omega$ is closed on $f(V)$ if and only if $f^*w$ is closed. Thus, in order to show that $d\omega=0$ one can assume that the metric $g$ osculates to order two to the standard metric and then one verifies that $d\omega$ vanishes in the origin. But the latter follows immediately from
\[\frac{\partial h_{ij}}{\partial z^k}(0)=\frac{\partial h_{ij}}{\partial\bar{z}^k}(0)=0.\]

For the other direction let us assume that $d\omega=0$. We fix a point $x\in U$. After translating we may assume that $x=0$. By a linear coordinate change we may furthermore assume that $(h_{ij})(0)=(\delta_{ij})$. Thus
\[h_{ij}=\delta_{ij}+\sum_ka_{ijk}z^k+\sum_kb_{ijk}\bar{z}^k+O(|z|^2).\]
Thus, we have $a_{ijk}=\frac{\partial h_{ij}}{\partial z^k}(0)$ and $b_{ijk}=\frac{\partial h_{ij}}{\partial z^k}(0)$. The assumption $d\omega(0)=0$ toghether with (\ref{complex manifold fundamental on C^n express}) implies $a_{ijk}=a_{kji}$ and $b_{ijk}=b_{ikj}$. Furthermore, since $\omega$ is real, $h_{ij}=\bar{h}_{ji}$ and thus $b_{ijk}=\bar{a}_{jik}$. New holomorphic coordinates in a neighbourhood of the origin can now be defined by
\[w^j=z^j+\frac{1}{2}\sum_{i,k=1}^{n}a_{ijk}z^iz^k.\]
Then we have
\[dw^j=dz^j+\frac{1}{2}\sum_{i,k=1}^{n}a_{ijk}(dz^i)z^k+\frac{1}{2}\sum_{i,k=1}^{n}a_{ijk}z^i(dz^k)=dz^j+\sum_{i,k=1}^{n}a_{ijk}z^k\,dz^i\]
and similarly
\[d\bar{w}^j=d\bar{z}^j+\sum_{i,k=1}^{n}b_{ijk}\bar{z}^k\,d\bar{z}^i.\]
Therefore, up to terms of order at least two, we find that
\begin{align*}
\frac{\i}{2}\sum_{j=1}^{n}dw^j\wedge d\bar{w}^j&=\frac{\i}{2}\sum_{j=1}^{n}\Big[dz^j\wedge d\bar{z}^j+\big(\sum_{i,k=1}^{n}a_{ijk}z^k\,dz^i\Big)\wedge d\bar{z}^j+dz^j\wedge\Big(\sum_{i,k=1}^{n}b_{jik}\bar{z}^k\,d\bar{z}^i\Big)\Big]\\
&=\frac{\i}{2}\Big[\sum_{j=1}^{n}dz^j\wedge d\bar{z}^j+\sum_{i,j=1}^{n}\Big(\sum_{k=1}^{n}a_{ijk}z_k\Big)dz^i\wedge d\bar{z}^j+\sum_{i,j=1}^{n}\Big(\sum_{k=1}^{n}b_{jik}\bar{z}^k\Big)dz^j\wedge d\bar{z}^i\Big]\\
&=\omega.
\end{align*}
This completes the proof of the proposition.
\end{proof}
\begin{example}
Any compatible metric on $U\sub\C$ satisfies the above condition. Clearly, the three-form $d\omega$ vanishes for dimension reasons.
\end{example}
\section{Complex manifolds}