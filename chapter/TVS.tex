\chapter{Topological vector spaces}
\section{Topological vector spaces}
A topological vector space is a pair $(X,\mathcal{T})$ where $X$ is a real vector space and $\mathcal{T}$ is a topology such that the structure maps
\[X\times X\to X:(x,y)\mapsto x+y,\quad \K\times X\to X:(a,x)\mapsto a x\]
are continuous with respect to the product topologies on $X\times X$ and $\K\times X$. The topology on $X$ is said to be a \textbf{linear topology} or \textbf{vector topology}. A linear map of topological vector spaces which is also a homeomorphism is called a \textbf{linear homeomorphism}. If there is a linear homeomorphism mapping from $X$ to $Y$, then $X$ is said to be \textbf{linearly homeomorphic} to $Y$. If there is a linear homeomorphism into $Y$, we sometimes say $X$ is \textbf{embedded} in $Y$.\par
Note that a topological vector space is also a topological group with addition. Thus the properties of topological groups all apply.
\begin{proposition}
Let $X$ be a topological vector space.
\begin{itemize}
\item[(a)] Translations, inversion, and multiplier maps are continuous
\item[(b)] $X$ is a homogeneous topological space.
\item[(c)] If $\mathcal{B}$ is a neighborhood base at $0$ in $X$, then $x+\mathcal{B}$ is a neighborhood base at $x$ in $X$.
\item[(d)] For any neighborhood $V$ of $0$ there exists a neighborhood $U$ of zero such that $U+U\sub V$.
\item[(e)] There is a basis of symmetric neighborhoods of $0$.  
\item[(f)] $X$ is regular---every closed subset $F$ of $X$ and a point $x$ not contained in $F$ admit non-overlapping open neighborhoods.
\item[(g)] $X$ is $T_0$ iff $X$ is Hausdorff, iff $\{0\}$ is closed in $X$. 
\end{itemize}
\end{proposition}
Using part (a), we have the following interesting consequence.
\begin{corollary}
Let $X$ be a topological vector space and $A\sub X$. For any $x\in X$ and any scalar $a\in\K$, we have
\[\Int(x+A)=x+\Int A,\quad \Int(aA)=a(\Int A),\quad \widebar{x+A}=x+\bar{A},\quad\widebar{aA}=a\bar{A}.\]
\end{corollary}
\begin{proposition}\label{TVS interior and closure}
Let $X$ be a topological vector space.
\begin{itemize}
\item[(a)] The sum of an open set and a subset is open.
\item[(b)] The sum of a compact set and a closed set is closed.
\item[(c)] For any subsets $A,B$ of $X$, we have $\bar{A}+\widebar{B}\sub\widebar{A+B}$.
\item[(d)] The closure of a subspace is a subsapce.
\item[(e)] If $\mathcal{B}$ is a base for the filter of the neighborhood of $0$, then for any subset $E$ we have
\[\widebar{E}=\bigcap_{B\in\mathcal{B}}(E+B).\]
\item[(f)] For any subsets $A,B$ of $X$, we have
\[\Int A+\Int B\sub A+\Int B\sub\Int(A+B).\] 
\end{itemize}
\end{proposition}
\begin{proof}
Suppose that $x\in A:=\bigcap_{B\in\mathcal{B}}(E+B)$. Let $B$ be a symmetric neighborhood of $0$ in $X$ and choose $B'\in\mathcal{B}$ such that $B'\sub B$. Since $x\in E+B'$, there exists $e\in E$ such that $x\in e+B'$, and hence $e\in x-B'\sub x-B=x+B$. It follows that $x\in\widebar{E}$. Conversely, if $x\in\widebar{E}$; choose $B\in\mathcal{B}$ and a symmetric neighborhood $B'$ of $0$ contained in $B$. Then $(x+B')\cap E\neq\emp$, which implies that $x\in E-B'=E+B'\sub E+B$. Hence $x\in A$.\par
It is clear that $\Int A+\Int B\sub A+\Int B$. Moreover, $A+\Int B$ is an open set contained in $A+B$. Hence is contained in $\Int(A+B)$.
\end{proof}
\begin{proposition}
Every topological vector space is connected.
\end{proposition}
\begin{proof}
Let $X$ be a topological vector space. Then $X=\bigcup_{x\in X}\K x$. Since each set $\K x$ is connected and $\bigcap_{x\in X}\K x=\{0\}$, it follows that $X$ is connected.
\end{proof}
It is now easy to give examples of topologized linear spaces that are not topological vector spaces.
\begin{example}
Let $X$ be a nontrivial vector space.
\begin{itemize}
\item[(a)] The discrete topology is not a vector topology. On the other hand, the trivial topology $\{\emp,X\}$ is.
\item[(b)] Since $X\neq\{0\}$, $X$ has an infinite number of elements. The cofinite topology is not a vector topology because is is not regular: open sets are infinite and the closure of any infinite subset is $X$ (because the only infinite closed set is $X$).
\end{itemize}
\end{example}
\section{Absorbent, balanced, and convex sets}
\begin{definition}
Let $E$ be a subset of a topological vector space $X$.
\begin{itemize}
\item $E$ is \textbf{balanced} (or \textbf{circled}) if $ax\in E$ for all $|a|\leq 1$ and $x\in E$.
\item $E$ is \textbf{absorbent} (or \textbf{absorbing}) if for any $x\in X$ there exists a positive number $r$ such that $a x\in E$ for all $|a|\leq r$.
\end{itemize}
\end{definition}
Clearly, $0$ must be contained in any balanced set or absorbent set. Note that the underlying field can make a difference. For example, the closed interval $[-1,1]$ is balanced in the real space $\R^2$, but it is not a balanced subset in $\C^2$ because $i\notin[-1,1]$.\par
First let's consider balanced and absorbent sets in $\mathfrak{U}(0)$. We have the following result.
\begin{proposition}\label{TVS nhbd base of 0 balanced and absorbent}
Let $X$ be a topological vector space and $\mathfrak{U}(0)$ be its neighborhood base at $0$.
\begin{itemize}
\item[(a)] Any neighborhood of $0$ is absorbent.
\item[(b)] Any neighborhood of $0$ contains a balanced neighborhood of $0$.
\item[(c)] There is a neighborhood base at $0$ consists of closed, balanced neighborhoods.
\end{itemize}
\end{proposition}
\begin{proof}
Let $U$ be a neighborhood of $0$, and let $x\in X$. Since $0x=0$, by the continuity of scalar multiplication, there exist neighborhoods $B_r(0)\sub\K$ and $V$ of $x$ such that $B_r(0)V\sub U$. Thus $B_r(0)x\sub U$, and the absorbency of $U$ follows.\par
Consider the continuous map $f:\K\times X\to X,(a,x)\mapsto ax$. Given a neighborhood $U$ of $0$ in $X$, then $f^{-1}(U)$ is a neighborhood of $(0,0)$ in $\K\times X$. Consequently, $f^{-1}(U)$ must contain a set of the form $B_r(0)\times V$ for some $r>0$ and $V\in\mathfrak{U}(0)$, i.e., $B_r(0)V\sub U$. As $B_r(0)V=\bigcup_{|a|\leq r}aV$, it is a neighborhood of $0$ in $X$. It is easy to see $B_r(0)V$ is balanced.\par
First note that the closure of a balanced set is balanced. To prove (c), let $W$ be a neighborhood of $0$ in $X$. By the regularity of topological vector spaces, $W$ contains a closed neighborhood $V$ of $0$. By part (b), $W$ contains a balanced neighborhood $U$ of $0$. Now $\widebar{U}$ is closed and balanced, and is contained in $W$.
\end{proof}
\begin{example}[\textbf{Absorbent and Balanced Sets}]
\mbox{}
\begin{itemize}
\item[(a)] In $\R^3$ the unit ball $B_1(0)$ is absorbent and balanced. The disk $B_1(0)\cap\R^2$ is balanced but not absorbent in $\R^3$.
\item[(b)] The disk $B_{1}(1/2)$, centered at $1/2$, is absorbent and balanced in $\R^2$. Its boundary $\partial B_1(1/2)$ is absorbent but not balanced.
\item[(c)] The collection $P$ of polynomials is not an absorbent subset of the linear space $C([0,1])$. But $P$ is clearly balanced.
\item[(d)] Clearly any subspace $V$ of $X$ is balanced. If $E\sub X$ is absorbent, then its linar span $\mathrm{span}(E)$ must be $X$. Thus, proper linear subspaces are never absorbent.
\item[(e)] Arbitrary unions and intersections of balanced sets are balanced.
\item[(f)] Arbitrary unions of absorbent sets are absorbent. But infinite intersection of absorbent sets need not be absorbent. For example, consider a decreasing sequence of circles in $\R^2$ whose diameters shrink to $0$. Absorbency survives finite intersections, however.
\end{itemize}
\end{example}
Let $X$ be a topological vector space and $E\sub X$ be a subset. Since arbitrary unions and intersections of balanced sets are balanced, we make the following definitions. The \textbf{balanced hull} $\bal(E)$ of $E$ is defined to be the smallest balanced subset of $X$ containing $E$, and the \textbf{balanced core} $\balcore(E)$ of $E$ is defined to be the biggest balanced subset of $E$. Since $X$ is balanced, it is clear that $\bal(E)$ makes sense and is nonempty. However, since $E$ may contain no balanced subsets, the set $\balcore(E)$ may be empty.\par
We note that a subset $E\sub X$ is balanced iff $aE\sub E$ for any $|a|\leq 1$, iff $E\sub aE$ for any $|a|\geq 1$. This leads to the following characterization.
\begin{proposition}\label{TVS balanced hull and core char}
Let $X$ be a topological vector space and $E\sub X$ be a subset. Then
\[\bal(E)=\bigcup_{|a|\leq 1}aE,\quad\balcore(E)=\bigcap_{|a|\geq 1}aE.\]
\end{proposition}
\begin{proof}
It is clear that the $\bigcup_{|a|\leq 1}aE$ and $\bigcap_{|a|\geq 1}aE$ are balanced and we have 
\[\bigcap_{|a|\geq 1}aE\sub E\sub\bigcup_{|a|\leq 1}aE.\]
Let $B$ be a balanced subset in $X$. If $E\sub B$, then $aE\sub aB\sub B$ for any $|a|\leq 1$, and thus $\bigcup_{|a|\leq 1}aE\sub B$. Similarly, if $B\sub E$, then $B\sub aB\sub aE$ for any $|a|\geq 1$, which means $B\sub\bigcup_{|a|\geq 1}aE$. The claim now follows.
\end{proof}
\begin{corollary}\label{TVS balanced hull and core of closed open}
Let $X$ be a topological vector space and $E\sub X$.
\begin{itemize}
\item[(a)] If $E$ is open, then so is $\bal(E)$.
\item[(b)] If $E$ is closed, then so is $\balcore(E)$.
\item[(c)] If $E$ is absorbent, then so are $\bal(E)$ and $\balcore(E)$.
\end{itemize}
\end{corollary}
\begin{proof}
Let $E$ be absorbent. Then for every $x\in X$, there exists $r>0$ such that $ax\in E$ for $|a|\leq r$. Note that the set $\{ax:|a|\leq r\}$ is balanced, so by definition is contained in $\balcore(E)$. Therefore $\balcore(E)$ is also absorbent.
\end{proof}
Now we turn to the properties of balanced or absorbent subsets.
\begin{proposition}\label{TVS balanced set prop}
Let $X$ be a topological vector space and $E\sub X$ be balanced.
\begin{itemize}
\item[(a)] $aE=|a|E$ for any scalar $a\in\K$.
\item[(b)] $aE\sub bE$ for any scalars $|a|\leq|b|$.
\item[(c)] $E$ is absorbent if and only if for every $x\in X$ there exists $r>0$ such that $x\in rE$.
\end{itemize}
\end{proposition}
\begin{proof}
Part (a) comes from the observation
\[aE=|a|\cdot\frac{a}{|a|}E\sub|a|E\]
since $E$ is balanced. Part (b) follows from the fact $(a/b)\cdot E\sub E$, since $|a|/|b|\leq 1$.\par
For (c), if $E$ is absorbent then there exists $r>0$ such that $ax\in E$ for $|a|\leq r$. In particular, $rx\in E$, and so $x\in r^{-1}E$. Conversely, if there exists $r>0$ such that $x\in rE$, then $r^{-1}x\in E$ and $ax\in E$ for $|a|\leq r^{-1}$, since $E$ is balanced.
\end{proof}
\begin{proposition}\label{TVS balanced and absorbent under linear map}
Let $X$ and $Y$ be topological vector spaces and $A:X\to Y$ be a linear map.
\begin{itemize}
\item[(a)] If $E\sub X$ is balanced, then $A(E)$ is balanced.
\item[(b)] If $E\sub X$ is absorbent and $A$ is surjective, then $A(E)$ is absorbent.
\item[(c)] Inverse image under $A$ of balanced or absorbent subsets of $Y$ are balanced or absorbent, respectively. 
\end{itemize}
\end{proposition}
\begin{proof}
If $E$ is balanced, then $aA(x)=A(ax)\in A(E)$ provided $|a|\leq 1$. Thus $A(E)$ is also balanced. Similarly, if $E$ is absorbent and $A$ is surjective, then for any $y\in X$, there exists $x\in X$ and $r>0$ such that $y=A(x)$ and $ax\in E$ for $|a|\leq r$. This implies $ay=aA(x)=A(ax)\in E$ for $|a|\leq r$, so $A(E)$ is also absorbent.\par
Now let $S\sub Y$ be balanced, and $x\in A^{-1}(S)$. Then for $|a|\leq 1$, we have $A(ax)=aA(x)\in S$, and so $ax\in A^{-1}(S)$. This implies $A^{-1}(S)$ is balanced. If $S$ is absorbent. Let $x\in X$. Then there exists $r>0$ such that $A(ax)=aA(x)\in S$ for $|a|\leq r$. Then $ax\in A^{-1}(S)$ for $|a|\leq r$, and so $A^{-1}(S)$ is absorbent.
\end{proof}
\begin{corollary}\label{TVS balanced and absorbent linear combination}
Let $X$ be a topological vector space.
\begin{itemize}
\item[(a)] If $E$ is balanced or absorbent, then nonzero multiples of $E$ are also balanced or absorbent, respectively.
\item[(b)] If $S$ and $T$ are balanced, then $aS+bT$ is balanced for any $a,b\in\K$.
\item[(c)] If $S$ and $T$ are absorbent, then $aS+bT$ is absorbent provided $a$ and $b$ are not both zero.
\end{itemize}
\end{corollary}
The notation of absorbency may be generalized as follows. If $A$ and $B$ are subsets of a topological vector space $X$, we say $A$ \textbf{absorbs} $B$ if $A$ can be sufficiently inflated to cover $B$. In other words, there exists $r>0$ such that $B\sub aA$ for $|a|\geq r$, or equivalently, $aB\sub A$ for $|a|\leq r$.\par
Now we come to the most important concept of this part, namely the convexity. Let $X$ be a topological vector space and $x,y\in X$. The \textbf{line segement} between $x$ and $y$ is defined to be $[x,y]=\{(1-a)x+ay:a\in[0,1]\}$. Sets such as $[x,y):=[x,y]\setminus\{y\}$ are called intervals. A subset $E$ of $X$ is called \textbf{convex} if it contains the line segement joining any two points of $E$. equivalently, $E$ is convex if and only if $(1-a)E+E\sub E$ for any $a\in[0,1]$.\par
Linear combinations $\sum_ia_ix_i$ in which the $a_i$'s are nonnegative and add up to $1$ are called \textbf{convex combinations} of the $x_i$'s. In this terminology, a subset $E$ is convex if and only if it contains all its convex combinations.\par
Unions of convex sets are not convex in general, but intersections clearly are. Thus we can define the \textbf{convex hull}, $\conv(E)$, as the smallest convex set containing $E$. Like the balanced hull, convex hull of a given subset has a nice characterization, as we will now show.
\begin{proposition}\label{TVS convex hull char}
Let $X$ be a topological vector space.
\begin{itemize}
\item[(a)] The convex hull of a subset $E$ is the collection of all convex combinations of elements in $E$.
\item[(b)] Let $\{E_i\}$ be a family of convex subsets. Then the convex hull of $\bigcup_iE_i$ is the collection of all convex combinations $\sum_ia_ix_i$ of $x_i$'s, where $x_i\in E_i$. 
\end{itemize}
\end{proposition}
\begin{proof}
It is clear that every convex set containing $E$ contains the convex combinations $\sum_ia_ix_i$. To see this collection is convex, we just note that
\[a\sum_{i=1}^{n}a_ix_i+b\sum_{i=1}^{n}b_ix_i=\sum_{i=1}^{n}(aa_i+bb_i)x_i.\]
If $a+b=1$, then $\sum_i(aa_i+bb_i)=a+b=1$. This proves (a).\par
For (b), we just need to show the claimed set $S$ is convex. Let $x=\sum_{i=1}^{n}a_ix_i$ and $y=\sum_{i=1}^{n}b_iy_i$ be points of $S$. Then consider
\[z=ax+by=a\sum_{i=1}^{n}a_ix_i+b\sum_{i=1}^{n}b_iy_i=\sum_{i=1}^{n}(aa_i+bb_i)\Big[\frac{aa_ix_i}{aa_i+bb_i}+\frac{bb_iy_i}{aa_i+bb_i}\Big].\]
Since $E_i$'s are convex, the element in the parenthesis belongs to $E_i$, for each $i$. Since $\sum_i(aa_i+bb_i)=1$, $z\in S$ and the claim follows.
\end{proof}
Using this proposition, we can prove an interesting property of the convex hull.
\begin{proposition}\label{TVS convex hull of finite union of compact}
Let $X$ be a topological vector spaces. Then the convex hull of a finite union of convex compact sets is compact.
\end{proposition}
\begin{proof}
Let $K_1,\dots,K_n$ be convex and compact. Then the convex hull of $\bigcup_iK_i$ is given by the collection of all convex combinations $\sum_ia_ix_i$, where $x_i\in K_i$ (Proposition~\ref{TVS convex hull char}). Now consider the set
\[S=\{(a_1,\dots,a_n)\in[0,1]^n:\sum_ia_i=1\}.\]
It is clear that $S$ is closed in $[0,1]^n$, so it is compact. Since $K_i$ are compact, the set $\prod_iK_i$ is also compact. The map
\[S\times\prod_iK_i\to X,\quad ((a_i),(x_i))\mapsto\sum_ia_ix_i\]
is continuous since it is the composition of continuous maps, and has range $\conv(\bigcup_iK_i)$. This implies $\conv(\bigcup_iK_i)$ is compact.
\end{proof}
\begin{example}
The convex hull of an infinite union of compact sets need not be compact. For example, let $K_n=[-n,n]\sub\R$. Then $\bigcup_nK_n=\R$ has convex hull $\R$, which is not compact.
\end{example}
Now we concentrate on the properties of convex sets. The first result is immediate.
\begin{proposition}[\textbf{Algebraic Propoties of Convex Sets}]\label{TVS convex set prop}
Let $X$ be a topological vector space.
\begin{itemize}
\item[(a)] If $E_1$ and $E_2$ are convex then so is $aE_1+bE_2$ for any scalars $a$ and $b$.
\item[(b)] Image and inverse images of a convex set under linear maps are convex.
\item[(c)] If $a$ and $b$ are positive scalars and $E$ is convex, then $(a+b)E=aE+bE$.
\item[(d)] Let $\{X_s:s\in S\}$ be a family of vector spaces. If $E_s\sub X_s$ is convex for each $s$, then $\prod_sE_s$ is convex. 
\end{itemize}
\end{proposition}
\begin{proof}
Part (b) and (d) are clear. If $E_1$ and $E_2$ are convex, then any convex combination $\sum_ic_i(ax_i+by_i)$ can be written as
\[\sum_ic_i(ax_i+by_i)=a\sum_ic_ix_i+b\sum_ic_iy_i\]
and hence is in $aE_1+bE_2$.\par
It is clear that $(a+b)E\sub aE+bE$, whether $E$ is convex or not. Let $w\in aE+bE$, then $w=ax+by$ with $x,y\in E$. We can write $w$ as
\[w=(a+b)\Big[\frac{ax}{a+b}+\frac{by}{a+b}\Big].\]
Since $E$ is convex, the right side is in $E$, this implies $w\in(a+b)E$, hence $aE+bE=(a+b)E$.
\end{proof}
Let $X$ be a topological vector space. A subset $D$ is called \textbf{disked} or \textbf{absolutely convex} if it is convex and balanced. A typical example of disked set is a disk in $X$. If $f$ is a linear functional on $X$, then the set $\{x:|f(x)|\leq a\}$ is disked, too.
\begin{proposition}\label{TVS disked set char}
Let $X$ be a topological vector space and $D$ be a subset. Then $D$ is disked if and only if $\sum_ia_iD\sub D$ for any scalars $a_i$ such that $\sum_i|a_i|\leq 1$.
\end{proposition}
\begin{proof}
If the condition holds, $D$ is clearly balanced. If $a_i$ are nonnegative scalars which add up to $1$, then we get $\sum_ia_iD\sub D$, so $D$ is also convex.\par
Conversely, assume that $D$ is disked and $a_i$ are scalars such that $\sum_i|a_i|\leq 1$. We observe
\[\sum_ia_iD=\sum_i|a_i|D\sub\sum_i\frac{|a_i|}{\sum_k|a_k|}D=D.\]
Since $D$ is convex and balanced. This proves the claim.
\end{proof}
\begin{corollary}\label{TVS disked set distributive prop}
Let $D$ be a disked set and $a_i$ be scalars such that $\sum_i|a_i|=1$. Then $D=\sum_ia_iD$.
\end{corollary}
\begin{proof}
We just note that if $\sum_i|a_i|=1$, then $\sum_ia_iD=\sum_i|a_i|D=D$, by the convexity of $D$ and Proposition~\ref{TVS balanced set prop}.
\end{proof}
Note that Corollary~\ref{TVS disked set distributive prop} can be viewed as a stronger version of Proposition~\ref{TVS convex set prop}: namely, for any disked set we have $\sum_ia_iD=(\sum_{i}|a_i|)D$.\par
Since arbitrary unions of disked sets are disked, we can define the \textbf{disked hull} or \textbf{absolutely convex hull} of a subset $E$ to be the smallest disked set containing $E$. We denote the disked hull of $E$ by $\convbal(E)$. This notation is justified by the following proposition.
\begin{proposition}\label{TVS disked hull char}
Let $X$ be a topological vector space and $E\sub X$. Then the disked hull of $E$ is given by the convex hull of the balanced hull of $E$:
\[\convbal(E)=\conv(\bal(E))=\{\sum_ia_ix_i:x_i\in E,\sum_i|a_i|\leq 1\}.\]
\end{proposition}
\begin{proof}
It is easy to verify the second equality, using Proposition~\ref{TVS balanced hull and core char} and Proposition~\ref{TVS convex hull char}. In particular, this shows $\conv(\bal(E))$ is balanced, and so $\convbal(E)\sub\conv(\bal(E))$. On the other hand, since $\convbal(E)$ is balanced, it contains $\bal(E)$, and hence contains $\conv(\bal(E))$ since it is also convex. This proves the first equality, hence the claim.
\end{proof}
One should note that, on the other hand, the set $\bal(\conv(E))$ need not to be convex. Here is an example.
\begin{example}
In $\R^2$, let $E$ be the convex hull of the set $V:=\{(0,0),(0,1),(1,1)\}$. Then $E$ is the triangle in $\R^2$ with vertices given by $V$, and $\bal(E)$ is given by $E$ with its reflection about the origin. It is clear that $\bal(E)$ is not convex.
\end{example}
Now we the durability of convexity and balancedness with respect to closures, ect.
\begin{theorem}\label{TVS bal conv absor int and closure}
Let $X$ be a topological vector space and $E\sub X$.
\begin{itemize}
\item[(a)] If $E$ is balanced, then $\widebar{E}$ is also balanced; if $\Int E$ contains $0$, then it is also balanced.
\item[(b)] The interior of an absorbent set is not generally absorbent; any superset of an absorbent set is absorbent; hence the closure of an absorbent set is absorbent.
\item[(c)] If $E$ is open then so is $\conv(E)$.
\item[(d)] If $E$ is convex then so is $\Int E$ and $\widebar{E}$.   
\end{itemize}
\end{theorem}
\begin{proof}
Suppose $E$ is balanced and consider a scalar $a$ such that $0<|a|\leq 1$. For such $a$, we have
\[a\widebar{E}=\widebar{aE}\sub\widebar{E},\quad a(\Int E)\sub aE\sub E\]
Since $a\neq 0$, $a(\Int E)$ is open in $X$, and so $a(\Int E)\sub\Int E$. Thus $\widebar{E}$ is balanced, and $\Int E$ is balanced if it contains $0$.\par
Suppose that $E$ is open. Then for any nonzero scalars $a_1,\dots,a_n$, the set $\sum_{i=1}^{n}a_iE$ is open. If the $a_i$'s are positive and add up to $1$, then $\sum_{i=1}^{n}a_iE\sub\conv(E)$. Since any $x$ in $\conv(E)$ belongs to such a linear combination, every $x$ is an interior point of $\conv(E)$. Thus $\conv(E)$ is open.\par
Suppose that $E$ is convex and $a,b$ are scalars adding up to $1$. Then by Proposition~\ref{TVS interior and closure}(f),
\[a(\Int E)+b(\Int E)=\Int(aE)+\Int(bE)\sub\Int(aE+bE)\sub\Int E.\]
Thus $\Int E$ is convex. Similarly, we can prove that $\widebar{E}$ is also convex.
\end{proof}
As intersections of closed convex sets are closed and convex, we can define the \textbf{closed convex hull} and the \textbf{closed disked hull} of a subset $E$ to be the smallest closed convex set or closed disked set containing $E$, which we denote by $\widebar{\conv}(E)$ and $\widebar{\convbal}(E)$. These notations are justified in the following proposition.
\begin{proposition}\label{TVS closed convex hull char}
Let $X$ be a topological vector space and $E\sub X$. Then
\[\widebar{\conv}(E)=\widebar{\conv(E)},\quad \widebar{\convbal}(E)=\widebar{\convbal(E)}.\]
\end{proposition}
\begin{proof}
By Theorem~\ref{TVS bal conv absor int and closure} the set $\widebar{\conv(E)}$ is closed and convex, so it contains $\widebar{\conv}(E)$. On the other hand, since $\widebar{\conv}(E)$ is a convex set containing $E$, we have $\conv(E)\sub\widebar{\conv}(E)$, and hence $\widebar{\conv(E)}\sub\widebar{\conv}(E)$. This proves the the first part, the second part follows similarly.
\end{proof}
Now we define locally convex spaces. Let $X$ be a topological vector space. $X$ is called \textbf{locally convex} if there is a neighborhood base at $0$ of convex sets. The topology for such a space is called a \textbf{locally convex topology}. If $X$ is Hausdorff and locally convex, we abbreviate this as "$X$ is a locally convex Hausdorff space".\par
Since balls, closed or open, determined by seminorms are convex, any normed linear space is locally convex. There is an intimate connection with locally convex spaces and seminorms. We will examine locally convex spaces in next section. For now we only the following property.
\begin{proposition}\label{LCS nbhd of disked sets}
Let $X$ be a locally convex space. Then there are neighborhood base at $0$ consists of closed or open disked sets.
\end{proposition}
\begin{proof}
If $V$ is a neighborhood of $0$, then $V$ contains a convex neighborhood $B$ of $0$. By Proposition~\ref{TVS nhbd base of 0 balanced and absorbent}, $B$ must contain a balanced neighborhood $U$ of $0$. Since $B$ is convex, it contains the convex hull $\conv(U)$ of $U$. Since $U$ is balanced, by Proposition~\ref{TVS disked set char} we have $\convbal(U)=\conv(U)$, thus $\conv(U)$ is also balanced.\par
Let $\mathcal{B}$ be a base of disked neighborhood of $0$. Then each $B\in\mathcal{B}$ contains a closed neighborhood $F$ of $0$ ($X$ is regular) and $F$ must contains some $B'\in\mathcal{B}$. The closure of $B'$ is also disked by Proposition~\ref{TVS bal conv absor int and closure}, so the first claim follows. The collection $\{\Int B:B\in\mathcal{B}\}$ is a base of open disked neighborhoods at $0$ by Proposition~\ref{TVS bal conv absor int and closure}, this proves the second claim.
\end{proof}
\section{Generating the vector topology}
We now consider some conditions which, if satisfied by a filter of subsets of a vector space $X$, generates a vector topology on $X$.
\begin{theorem}\label{TVS generating topology}
Let $X$ be a vector space over $\K$. Consider a filter base $\mathcal{B}$ of subsets of $X$ for which
\begin{itemize}
\item[(B1)] Each $B\in\mathcal{B}$ is balanced and absorbent. 
\item[(B2)] For each $B\in\mathcal{B}$, there exists $U\in\mathcal{B}$ such that $U+U\sub B$.
\end{itemize}
Then $\mathcal{B}$ is a neighborhood base at $0$ for a vector topology for $X$.
\end{theorem}
\begin{proof}
Since each $B\in\mathcal{B}$ is balanced, it is also symmetric. Thus the basis conditions of Proposition~\ref{topological group nbhd filter iff} is satisfied, and $\mathcal{B}$ is a neighborhood base at $0$ of a group topology for $X$. Let $\mathfrak{U}(0)$ be the filter generated by $\mathcal{B}$ so that the family of the neighborhoods of any $x$ is given by $x+\mathfrak{U}(0)$. To prove the continuity of scalar multiplication, we first show that
\begin{align}\label{TVS generating topology-1}
\text{Given $B\in\mathcal{B}$ and $a\neq 0$, there exists $V\in\mathcal{B}$ such that $aV\sub B$.}
\end{align}
Given $B\in\mathcal{B}$, there exists $V\in\mathcal{B}$ such that $2V\sub V+V\sub B$ by (\rmnum{2}). Hence for each $n\in\N$, there is a $V\in\mathcal{B}$ such that $2^nV\sub B$. For $a\neq 0$, choose $n$ such that $|a|\leq 2^n$. Since $V\in\mathcal{B}$, $V$ is balanced and it follows that $aV\sub 2^nV\sub B$ and $(\ref{TVS generating topology-1})$ is verified.\par
For $a_0\in\K$, $x_0\in X$, and $B\in\mathcal{B}$, consider the neighborhood $a_0x_0+B$ of $a_0x_0$. By (\rmnum{2}), there exists $W\in\mathcal{B}$ such that $W+W+W\sub B$. Suppose that $a_0=0$. Since $W$ is absorbent, there exists $r>0$ such that $B_r(0)x_0\sub W\sub B$. Since $W$ is also balanced, there exists $r\leq 1$ such that
\[B_r(0)(x_0+W)=B_r(0)x_0+B_r(0)W\sub W+W\sub B=0x_0+B.\]
Now suppose that $a_0\neq 0$ and let $W$ be as above. By $(\ref{TVS generating topology-1})$ there exists $W'\sub B$ such that $a_0W'\sub W$. Since $\mathcal{B}$ is a filterbase, there is $V\in\mathcal{B}$ such that $V\in W\cap W'$. Since $V$ is absorbent, there is $r\leq 1$ such that $B_r(0)x_0\in V$. Now if $a\in B_r(a_0)$ and $x\in x_0+V$, then
\[(a-a_0)x_0\in V\sub W,\quad (a-a_0)(x-x_0)\in B_r(0)W\sub W,\quad a_0(x-x_0)\in a_0V\sub a_0W'\sub W.\]
Combining these, we get
\[ax-a_0x_0=(a-a_0)x_0+(a-a_0)(x-x_0)+a_0(x-x_0)\sub W+W+W\sub B.\]
This means $B_r(a_0)(x_0+V)\sub a_0x_0+B$, as required. 
\end{proof}
Next we look at basis condition for a locally convex topology. Another way, using seminorms, will be discussed in next section.
\begin{theorem}\label{LCS generating topology}
Let $X$ be a vector space over $\K$. Consider a filter base $\mathcal{B}$ of subsets of $X$ for which
\begin{itemize}
\item[(C1)] Each $B\in\mathcal{B}$ is absorbent and disked. 
\item[(C2)] For each $B\in\mathcal{B}$, there exists $a\in(0,1/2]$ such that $aB\in\mathcal{B}$.
\end{itemize}
Then $\mathcal{B}$ is a neighborhood base at $0$ for a locally convex topology for $X$.
\end{theorem}
\begin{proof}
By (C1), if $\mathcal{B}$ is a base for a vector topology, that topology will be locally convex. Thus we only need to show (B2) is satisfied. Let $B\in\mathcal{B}$, by (C2) there is $a\in(0,1/2]$ such that $aB+B$. Since $aB$ is convex, $aB+aB=2aB$ by Proposition~\ref{TVS convex set prop}; since $B$ is balanced and $2a\leq 1$, $2aB\sub B$.
\end{proof}
Note that if the collection $\mathcal{B}$ does not satisfy (C2), then the collection $\{aB:a>0,B\in\mathcal{B}\}$ satisfies both (C1) and (C2).
\begin{theorem}\label{TVS subbase generate topo}
Let $\mathcal{S}$ be a nonempty collection of subsets of a vector space $X$.
\begin{itemize}
\item[(a)] If $\mathcal{S}$ satisfies (B1) and (B2), then it is a neighborhood subbase at $0$ for a vector topology for $X$.
\item[(b)] If $\mathcal{S}$ consists of absorbent disked sets then a neighborhood base of $0$ for a locally convex topology is given by the collection $\mathcal{B}$ of positive multiples of finite intersections of sets from $\mathcal{S}$.
\end{itemize}
\end{theorem}
\begin{proof}
If $\mathcal{S}$ satisfies (B1) and (B2) then so do the collection $\mathcal{B}$ of finite intersections of sets from $\mathcal{S}$ and the result follows from Theorem~\ref{TVS generating topology}.\par
If $\mathcal{S}$ consists absorbent disked sets then so does $\mathcal{B}$ of positive multiples of finite intersections of sets from $\mathcal{S}$. Moreover, $\mathcal{B}$ is a filterbase, for given $0<a<b$ and elements $a\bigcap_{i=1}^{n}S_i$ and $b\bigcap_{j=1}^{m}S'_j$ of $\mathcal{B}$ then (Proposition~\ref{TVS balanced set prop})
\[a\bigcap_{i,j}(S_i\cap S_j')\sub\Big(a\bigcap_{i=1}^{n}S_i\Big)\cap\Big(b\bigcap_{i=1}^{m}S_j'\Big).\]
Cleraly for each $a\in(0,1/2]$ and $B\in\mathcal{B}$ we have $aB\in\mathcal{B}$, so the result follows from Theorem~\ref{LCS generating topology}. 
\end{proof}
\begin{example}[\textbf{Projective Topology}]\label{TVS initial topo}
Let $X$ be a vector space and $\{X_s:s\in S\}$ be a family of topological vector spaces such that for each $s\in S$ there is a linear map $A_s:X\to X_s$. The \textbf{projective topology} or \textbf{initial topology} $\mathcal{T}$ for $X$ determined by the family $\{A_s\}$ is the coarsest topology with respect to which each $A_s$ is continuous. For each $s\in S$, let $\mathfrak{U}_s(0)$ denote the filterbase of balanced neighborhoods of $0$ in $X_s$. A base $\mathfrak{U}_{\mathcal{T}}(0)$ at $0$ for the initial topology $\mathcal{T}$ is given by the intersections of the form $\bigcap_{k\in K}A_k^{-1}(V_k)$ where $K$ is a finite set of $S$ and $V_k\in\mathfrak{U}_k(0)$ for each $k\in K$. Then $\mathcal{T}$ is a group topology on $X$, so (B2) of Theorem~\ref{TVS generating topology} is satisfied. As for (B1), it is clear that each set $\bigcap_{k\in K}A_k^{-1}(V_k)$ is balanced and absorbent by Proposition~\ref{TVS balanced and absorbent under linear map}, therefore Theorem~\ref{TVS generating topology} implies that $\mathcal{T}$ is a vector topology. We also say that $\mathcal{T}$ is the \textbf{projective limit topology} of $\{X_s\}$ with respect to $\{A_s\}$. If each $X_s$ is locally convex, then it is easy to see $\mathcal{T}$ is locally convex, using Proposition~\ref{TVS convex set prop}.
\end{example}
\begin{example}
In particular, consider the product $X=\prod_sX_s$ of $X_s$. Then $X$ is a vector space with respect to pointwise addition and scalar multiplication. Since the product topology on $X$ is defined to be the initial topology on $X$ with respect to the projections $\{\pi_s:X\to X_s\}$ and each $\pi_s$ is linear, it follows from the argument above that $X$ is a topological vector space with the product topology. From now on, when we say a product of topological vector spaces, we always refer to the product topology and the pointwise operations. As is generally ture for products, $X=\prod_sX_s$ is Hausdorff iff each $X_s$ is.
\end{example}
\begin{example}[\textbf{Supremum Topology}]\label{TVS supremum topo}
Let $X$ be a linear space and let $\{\mathcal{T}_s:s\in S\}$ be a family of vector topologies for $X$. By the \textbf{supremum topology} $\mathcal{T}$ for $X$, we mean the topology generated by the sets $\bigcup_{s\in S}\mathcal{T}_s$ (as a subbase) and we write $\mathcal{T}=\sup\{\mathcal{T}_s:s\in S\}$. Let $X_s$ be $X$ topologized by $T_s$, and let $I_s$ denote the identify map from $X$ into $X_i$. $\mathcal{T}$ is readily identified as the weakest topology for $X$ with respect to which each $I_i$ is continuous, i.e., the initial topology for $X$ determined by $\{I_s:s\in S\}$. As such, $\mathcal{T}$ is a vector topology with neighborhood base at $0$ for $\mathcal{T}$ given by sets of the form $\bigcap_{k\in K}V_k$ where $K$ is a finite subset of $S$ and each $V_k$ is a $\mathcal{T}_k$-neighborhood of $0$. If each $\mathcal{T}_s$ is locally convex then so is $\mathcal{T}$.
\end{example}
\begin{example}[\textbf{Inductive Topology}]\label{TVS final topo}
Let $X$ be a vector space and let $\{X_s:s\in S\}$ be a family of topological vector spaces such that for each $s\in S$ there is a linear map $A_s:X_s\to X$. The \textbf{inductive topology} or \textbf{final topology} on $X$ with respect to the family $\{A_s:s\in S\}$ is the finest vector topology for which each of the mappings $A_s$ is continuous. To see that this topology is well defined, we note that the class $\mathfrak{T}$ of such topologies is not empty. By the result of Example~\ref{TVS supremum topo}, $\mathcal{T}=\sup\mathfrak{T}$ is a vector topology for $X$. A neighborhood base at $0$ for $\mathcal{T}$ is given by finite intersections $\bigcap_{i=1}^{n}V_{i}$, where each $V_i$ is a neighborhood of $0$ in a topology for $X$ with respect to which each $A_s$ is continuous. Since, for each $s\in S$,
\[A_s^{-1}\Big(\bigcap_{i=1}^{n}V_i\Big)=\bigcap_{i=1}^{n}A_s^{-1}(V_i)\]
is a neighborhood of $0$ in $X_s$, each $A_s$ is $\mathcal{T}$-continuous. Therefore $\mathcal{T}\in\mathfrak{T}$ and it follows that $\mathcal{T}$ is the finest vector topology for $X$ with respect to which each $A_s$ is continuous.\par
With the same setting, we can consider the collection $\mathfrak{T}_c$ of all locally convex topologies for $X$ with respect to which each $A_s$ is continuous. The statements above retain their validity. The significant difference is that $\mathcal{T}=\sup\mathfrak{T}_c$ is a locally convex topology for $X$ by Example~\ref{TVS initial topo}. It is the finest locally convex topology for $X$ with respect to which each $A_s$ is continuous and $\mathcal{T}$ is called the \textbf{final locally convex topology} or \textbf{inductive limit topology} determined by the family $\{A_s:s\in S\}$.
\end{example}
\begin{example}
In particular, if $M$ is a subspace of a topological vector space $X$, then the quotient topology on $X/M$ is the final topology induced by the quotient map $\pi:X\to X/M$. Note that $\pi$ is an open map with this topology, thus the open sets in $X/M$ are of the form $\pi(U)$, where $U$ is open in $X$. If $\mathcal{B}$ is the filterbase of balanced neighborhoods of $0$ in $X$, then $\pi(\mathcal{B})$ is a neighborhood base of $0$ for the quotient topology. whenever we speak the a quotient space of a topological vector space, we always endow it with the quotient topology.
\end{example}
\begin{example}
If $\{X_s:s\in S\}$ is a family of vector spaces over $\K$, the algebraic direct sum $\bigoplus_sX_s$ is defined to be the subspace of $\prod_sX_s$ for whose elements $x$ all but a finite number of the projections $x_s$ are $0$. Denote by $I_s$ the injection map (or canonical imbedding) $X_s\to\bigoplus_sX_s$. The \textbf{direct sum} of the family $\{X_s:s\in S\}$ of topological vector spaces is defined to be $\bigoplus_sX_s$ under the inductive topology with respect to the family $\{I_s:s\in S\}$. If each $X_s$ is locally convex then the direct sum endowed with the final locally convex topology determined by the canonical injections $I_s$ is called the \textbf{locally convex direct sum} of the $X_s$. If we identify each $X_s$ with its canonical image $I_s(X_s)$ in $\bigoplus_sX_s$ then a base at $0$ for $\bigoplus_sX_s$ is given by the set of all absorbent disks $V$ in $\bigoplus_sX_s$ such that $V\cap X_s$ is a neighborhood of $0$ in $X_s$ for each $s$ in $S$.
\end{example}
\begin{proposition}\label{TVS quotient topo prop}
Let $M$ be a subspace of a topological vector space $X$ and $\pi:X\to X/M$ be the quotient map. Then
\begin{itemize}
\item[(a)] $\pi$ is open and continuous.
\item[(b)] $X/M$ is Hausdorff iff $M$ is closed in $X$.
\item[(c)] If $X$ is locally convex, so is $X/M$.
\end{itemize}
\end{proposition}
\begin{proof}
Part (a) comes from Proposition~\ref{topological group quotient map open}, part (b) follows from Theorem~\ref{topological group quotient group Hausdorff or discrete iff}, and (c) follows from (a) and Proposition~\ref{TVS convex set prop}.
\end{proof}
\begin{proposition}\label{TVS first isomorphism thm}
Let $A:X\to Y$ be a linear map between topological vector spaces. Let $N=N(A)$ and consider the following decomposition of $A$:
\[\begin{tikzcd}
X\ar[r,"\pi"]&X/N\ar[r,"\bar{A}"]&R(A)\ar[r,"\iota"]&Y
\end{tikzcd}\]
Then $\iota$ is a linear embedding and $\bar{A}$ is a vector space isomorphism. The map $A$ is continuous if and only if $\bar{A}$ is, and $A$ is relatively open if and only if $\bar{A}$ is open.
\end{proposition}
\section{Metrizability and completion}
We discuss pseudometrizability and completion of topological vector spaces here.
\begin{definition}\label{F-seminorm def}
Let $X$ be a vector space. An \textbf{$\bm{F}$-seminorm} is a map $p:X\to\R_{\geq 0}$ such that for any $a\in\K$, $x,y\in X$,
\begin{itemize}
\item[(a)] $p(ax)\leq p(x)$ for $|a|\leq 1$.
\item[(b)] $\lim_np(x/n)=0$.
\item[(c)] $p(x+y)\leq p(x)+p(y)$.
\end{itemize}
If $p(x)=0$ iff $x=0$, then $p$ is called an \textbf{$\bm{F}$-norm}.
\end{definition}
A seminorm is clearly an $F$-seminorm, and a norm is an $F$-norm. However, the converse of these is not ture. The point of $F$-seminorms is that they always induce vector topologies.
\begin{theorem}
$F$-seminorms induce pseudometrizable vector topologies.
\end{theorem}
\begin{proof}
Let $p$ be an $F$-seminorm on a vector space $X$. By definition $p(-x)\leq p(x)$ for any $x\in X$, and hence $p(x)=p(-x)$. Clearly $p(0)\geq 0$ and $p(0)\leq p(x/n)+p(-x/n)=2p(x/n)\to 0$, so $p(0)=0$. This shows $d(x,y)=p(x-y)$ is an invariant pseudometric which defines a group topology $\mathcal{T}$ on $X$ which therefore satisfies (B2) of Theorem~\ref{TVS generating topology}. Since $p(x/n)\to 0$ and $p(ax)\leq p(x)$ for $|a|\leq 1$, each closed ball $B_r(0)$ is absorbent and balanced. Therefore $\mathcal{T}$ satisfies (B1) of Theorem~\ref{TVS generating topology} and is a vector topology on $X$ thereby. Since $\{B_{1/n}(0):n\in\N\}$ is a base at $0$ for $\mathcal{T}$, it follows that $X$ is first countable and so is pseudometrizable by Theorem~\ref{topological group pseudometrizable iff}.
\end{proof}
\begin{theorem}\label{TVS pseudometrizable}
A topological vector space is pseudometrizable iff it is first countable; in this case its topology is induced by an $F$-seminorm. A pseudometrizable topological vector space is metrizable iff it is Hausdorff, in which case its topology is induced by an $F$-norm.
\end{theorem}
\begin{proof}
The pseudometrizability assertion follows from Theorem~\ref{topological group pseudometrizable iff}. Now suppose that a topological vector space $X$ is pseudometrizable so that is has a countable base $\{U_n\}$ of balanced neighborhoods at $0$. By Theorem~\ref{topological group pseudometrizable iff} we know that there exists an invariant pseudometric $d$ which induced the group topology. Since each $U_n$ is balanced, the function $f$ of Theorem~\ref{topological group pseudometrizable iff} is such that $f(tx,0)\leq f(x,0)$ for $|t|\leq 1$. This means $d(tx,0)\leq d(x,0)$ for $|t|\leq 1$, and therefore $p$ satisfies (a) of Definition~\ref{F-seminorm def}. Since each $U_n$ is absorbent, $p$ satisfies (b) of Definition~\ref{F-seminorm def}; $p$ satisfies the triangle inequality since $d$ does. The ramaining parts are clear.
\end{proof}
\begin{corollary}
Let $M$ be a subspace of a topological vector space $X$. If $X$ is pseudometrizable then so is $X/M$. If $p$ is an $F$-seminorm on $X$ then the quotient $F$-seminorm $\bar{p}(\bar{x})=\inf\{p(x+m):m\in M\}$ determines the quotient topology; if $p$ is an $F$-norm and $M$ is closed, then $X/M$ is metrizable. 
\end{corollary}
\begin{proof}
If $p$ is a $F$-seminorm, then for $|a|\leq 1$,
\begin{equation*}\small
\begin{aligned}
\bar{p}(a\bar{x})=\inf\{p(ax+m):m\in M\}=\inf\{p(a(x+m)):m\in M\}\leq\inf\{p(x+m):m\in M\}=\bar{p}(\bar{x}).
\end{aligned}
\end{equation*}
Moreover, since $p(x/n)\to 0$ and $0\in M$, we have
\[\bar{p}(\bar{x}/n)=\inf\{p(x/n+m):m\in M\}\leq p(x/n)\to 0.\]
Finally, the triangle inequality follows easily:
\begin{equation*}\small
\begin{aligned}
\bar{p}(\bar{x}+\bar{y})&=\inf\{p(x+y+m):m\in M\}=\inf\{p(x+y+m+n):m,n\in M\}\\
&\leq\inf\{p(x+m):m\in M\}+\inf\{p(y+n):n\in M\}=\bar{p}(\bar{x})+\bar{p}(\bar{y}).
\end{aligned}
\end{equation*}
Thus $\bar{p}$ is also an $F$-seminorm. If $p$ is an $F$-norm and $M$ is closed, then it is not hard to see $\bar{p}$ is an $F$-norm, and thus $X/M$ is metrizable.
\end{proof}
A topological vector space is complete if it is complete as a topological group. A complete metrizable locally convex space is called a \textbf{Fr\'echet space}. We will also use \textbf{$\bm{F}$-spaces} to refer topological vector spaces which are complete and metrizable.
\begin{definition}
Let $X$ be a topological vector space. A complete topological vector space $\widehat{X}$ which contains a linearly homeomorphic image of $X$ as a dense subset is called a \textbf{completion} of $X$.
\end{definition}
\begin{proposition}
Let $(X,d)$ be a pesudometrizable topological vector space. Then $X$ possesses a completion.
\end{proposition}
\begin{proof}
Given a pseudometrizable topological vector space $(X,d)$ ($d$ invariant), let $\widehat{X}$ denote the collection of Cauchy sequences from $X$. Given $(x_n),(y_n)\in\widehat{X}$ and a scalar $a$, define $a(x_n)=(ax_n)$ and $(x_n)+(y_n)=(x_n+y_n)$. With respect to these definitions, $\widehat{X}$ is a vector space; moreover, $d((x_n),(y_n))=\lim_nd(x_n,y_n)$ defines an invariant pseudometric on $\widehat{X}$ with respect to which $X$ is complete. The map $f:X\to\widehat{X},x\mapsto(x)$, where $(x)$ denotes the sequence consisting solely of $x$'s, is an isometric linear isomorphism and $f(X)$ is dense in $\widehat{X}$, i.e., $(\widehat{X},\hat{d})$ is a completion of $(X,d)$.
\end{proof}
Since a topological vector space is just a topological $\K$-module, we have
\begin{theorem}
Every topological vector space possesses a completion.
\end{theorem}
Also, it follows from Proposition~\ref{topological group metric quotient is complete}, on completeness of quotients of a topological group, that:
\begin{theorem}\label{TVS quotient is complete if}
If $X$ is a complete pseudometrizable topological vector space and $M$ is a closed subsapce then $X/M$ is complete.
\end{theorem}
With completion avaliable, we can now return to the compactness of the convex hull of a compact set.
\begin{example}[\textbf{Convex Hull of Compact Need Not Be Compact}]
Consider the subspace $c_{00}$ of $\ell^2$ consists of finite sequences; that is, sequences which has only finitely many nonzero component. For each $n\in\N$, let $\zeta_n$ be the sequence whose $n$-th entry is $1/n$ and $0$ everywhere else. Then $\zeta_n\to 0$ in $\ell^2$, so the set $K:=\{\zeta_n\}\cup\{0\}$ is compact in $\ell^2$, hence in $c_{00}$. Set
\[s_n=\sum_{j=1}^{n}2^{-j}\zeta_j,\quad a_n=\sum_{j=1}^{n}2^{-j}.\]
Then $s_n/a_n$ belongs to the convex hull of $K$. Let $s$ be the limit of $s_n$ in $\ell^2$, then $s_n/a_n\to s$. But $s$ is not an element of $X$, hence not of the closure of $\conv(K)$ in $X$. It follows that $\conv(K)$ is not compact, since it is not complete.
\end{example}
As the space $c_{00}\sub\ell^2$ is not complete, (the sequence $x_n=(1,1/2,\dots,1/n,0,\dots)$ is Cauchy but has no limit in $X$), maybe some sort of completeness condition provides a framework in which the convex hull of a compact set is compact. We now show that the convex hull of a totally bounded subset of a locally convex space is totally bounded; hence, if $X$ is a complete locally convex space then the closed convex hull of a compact set is compact.
\begin{theorem}[\textbf{Hulls of Totally Bounded and Compact Sets}]\label{TVS hull of totally bounded or compact sets}
\mbox{}
\begin{itemize}
\item[(a)] In any topological vector space the balanced hull of a totally bounded or compact set is totally hounded or compact, respectively.
\item[(b)] If $K$ is a totally bounded subset of a locally convex space, then so is its convex hull, and therefore its disked hull.
\item[(c)] If $K$ is a compact subset of a locally convex space, then its convex hull $\conv(K)$ and disked hull $\convbal(K)$ are compact iff $\conv(K)$ and $\conv(\bal(K))$ are complete, respectively. Thus, if $X$ is complete, $\widebar{\conv}(K)$ and $\widebar{\convbal}(K)$ are compact.
\end{itemize}
\end{theorem}
\begin{proof}
Let $X$ be a topological vector space and $\widehat{X}$ be a completion of $X$. If $K\sub X$ is totally bounded, so is its closure $\mathrm{cl}_{\widehat{X}}K$ in $\widehat{X}$; $\mathrm{cl}_{\widehat{X}}K$ is therefore compact by Theorem~\ref{metric space compact iff complete totally bounded}.\par
The closed unit disk $D$ is compact in $\K$, hence $D\times \mathrm{cl}_{\widehat{X}}K$ is compact in the product topology of $\K\times\widehat{X}$. By the continuity of scalar multiplication, the algebraic product $D(\mathrm{cl}_{\widehat{X}}K)$ is therefore compact. Since the balanced hull $\bal(K)=DK\sub D(\mathrm{cl}_{\widehat{X}}K)$, it is also totally bounded. If $K$ is compact, then clarly $\bal(K)=DK$ is compact. This proves (a).\par
Suppose that $K$ is a totally bounded subset of the locally convex space $X$. Given a neighborhood $U$ of $0$, choose a disked neighborhood $V$ of $0$ such that $V+V\sub U$. Since $K$ is totally bounded, there exist $x_1,\dots,x_n$ in $K$ such that $K\sub\bigcup_{i=1}^{n}(x_i+V)\sub\conv(x_1,\dots,x_n)+V$. As this last set is convex by Proposition~\ref{TVS convex set prop}(a), $\conv(K)\sub\conv(x_1,\dots,x_n)+V$. Since $\conv(x_1,\dots,x_n)$ is compact by Proposition~\ref{TVS convex hull of finite union of compact}, there exist $y_1,\dots,y_k\in\conv(x_1,\dots,x_n)$ such that $\conv(x_1,\dots,x_n)\sub\bigcup_{j=1}^{k}(y_j+V)$. Thus 
\[\conv(K)\sub\bigcup_{j=1}^{k}(y_j+V)+V=\bigcup_{j=1}^{k}(y_j+V+V)\sub\bigcup_{j=1}^{k}(y_j+U),\]
which proves $\conv(K)$ is totally bounded.\par
By Proposition~\ref{uniform space precompact iff totally bounded}, completeness is equivalent to compactness for totally bounded sets, so it follows from (b) that $\conv(K)$ and $\convbal(K)$ are compact iff they are complete. If $X$ is complete, then any closed subset is complete, so (c) follows.
\end{proof}
\section{Topological complements}
Two subspaces $M$ and $N$ of a vector space $X$ are algebraic complements if $M\cap N=\{0\}$ and $X=M+N$. Under these circumstances each vector $x$ in $X$ has a representation of the form $m+n$ for a unique $m$ in $N$ and a unique $n$ in $N$; we write $X=M\oplus N$ and say $X$ is the \textbf{algebraic direct sum} of $M$ and $N$ and that $M$ and $N$ are an \textbf{algebraic direct sum decomposition} of $X$ and that $M$ and $N$ arc \textbf{algebraic complements} of each other. The space $X$ has been split iuto two parts and recovered by "adding" them back together again. In this part we investigate analogous considerations for topological vector space $X$---reconstitute $X$ topologically as well as algebraically when $M$ and $N$ carry their subspace topologies and $M\times N$ the product topology. We have to consider the question: If $X=M\oplus N$, under what circumstances is the product topology on $M\times N$ equal to the original on $X$? In other words, when is the map $S$ ("$S$" for "sum")
\[S:M\times N\to X,\quad (m,n)\mapsto m+n\]
a homeomorphism? $S$ is always linear, bijective, and continuous (by continuity of addition), so the question reduces to: When is $S$ an open map?, i.e., when does $m+n\to 0$ imply that $m\to 0$ and $n\to 0$?\par
When $S$ is a homeomorphism, we say that $M$ and $N$ are \textbf{topological complements} (or \textbf{supplements}) and $M$ is \textbf{(topologically) complemented}, that $X$ is the \textbf{topological direct sum} of $M$ and $N$. The common usage is to say that $M$ is complemented, rather than topologically complemented. Note that "is a topological complement of" is a symmetric relation.
\begin{definition}
Let $X$ be a vector space. A linear map $P:X\to X$ such that $P^2=P\circ P=P$ (i.e., such that $P$ is idempotent) is called a \textbf{projection}.
\end{definition}
The range $P(X)$ of a projection $P$ and its null space are algebraically complementary: Any $x\in X$ can be uniquely represented as $x=Px+(x-Px)$. Conversely, if $M$ and $N$ are algebraic complements in $X$, the map sending $x=m+n\in M+N$ into $m$ is a projection called the \textbf{projection on $\bm{M}$ along (or parallel to) $\bm{N}$}.
\begin{theorem}\label{TVS topological complement iff}
Let $M$ be a linear subspace of a topological vector space $X$. An algebraic complement $N$ of $M$ is a topological complement of $M$ iff either of the following conditions holds:
\begin{itemize}
\item[(a)] The projection $P_M$ on $M$ along $N$ is continuous. Hence $M$ is complemented in $X$ iff there is a continuous projection of $X$ onto $M$.
\item[(b)] The canonical isomorphism $N$ onto $X/M$ is bicontinuous. Hence any topological complement of $M$ is linearly homeomorphic to $X/M$.
\end{itemize}
\end{theorem}
\begin{proof}
Let $M$ have a topological complement $N$ and let $P$ be the projection on $M$ along $N$. Suppose that the net $m_\alpha+n_\alpha\to 0$, $m_\alpha\in M$, $n_\alpha\in N$; then, since the summation map $S$ is a homeomorphism, $m_\alpha\to 0$ and $n_\alpha\to 0$. Hence $m_\alpha+n_\alpha\to 0$ implies that $m_\alpha=P(m_\alpha+n_\alpha)\to 0$ and $P_M$ is continuous.\par
Conversely, suppose that $P$ is a continuous projection with range $M$. It is elementary to show that the null space $N$ of $P$ is an algebraic complement of $M$. To prove that $N$ is a topological complement, suppose that $m_\alpha+n_\alpha\to 0$, where $(m_\alpha)$ and $(n_\alpha)$ are nets from $M$ and $N$, respectively. By the continuity of $P$, $m_\alpha=P(m_\alpha+n_\alpha)\to 0$ and $n_\alpha=(m_\alpha+n_\alpha)-P(m_\alpha+n_\alpha)\to 0$ and it follows that $N$ is a topological complement of $M$.\par
If $N$ is an algebraic complement of $M$ then we may decompose the projection $P$ on $N$ along $M$ as follows:
\[\begin{tikzcd}
X=M\oplus N\ar[r,"\pi"]&X/M\ar[r,"\iota"]&N
\end{tikzcd}\]
($X/M$ carries its quotient topology, and $N$ its subspace topology.) $N$ is a topological complement of $M$ iff $P$ is continuous by (a). However, $P=\iota\circ\pi$ is continuous iff $\iota$ is continuous. Since $\iota$ is open, the continuity of $\iota$ is equivalent to the bicontinuity of the canonical isomorphism $\iota^{-1}$.
\end{proof}
As an immediate consequence of Theorem~\ref{TVS topological complement iff}, we show next that a complemented subspace of a Hausdorff topological vector space is closed.
\begin{corollary}
A topologically complemented subspace $M$ of a Hausdorff topological vector space is closed.
\end{corollary}
\begin{proof}
Let $N$ be a topological complement of $M$ of a Hausdorff topological vector space and let $P_N$ be the (continuous) projection on $N$ along $M$ so that $M=P_N^{-1}(0)$.
\end{proof}
A subspace $M$ of a vector space $X$ is \textbf{maximal} if there exists no proper subspace $N$ of $X$ (i.e., $N\neq X$) which contains $M$ properly. If $X$ is a topological vector space then $\widebar{M}$ is a subspace, so a maximal subspace of a topological vector space must be closed or dense in $X$. A \textbf{hyperplane} is a translate of a maximal subspace.
\begin{proposition}\label{TVS maximal subspace iff}
Let $X$ be a topological vector space.
\begin{itemize}
\item[(a)] $M$ is maximal iff $\dim X/M=1$.
\item[(b)] $M$ is a (closed) maximal subspace of $X$ iff $M$ is the null space of a nontrivial (continuous) linear functional on $X$.
\item[(c)] $H$ is a (closed) hyperplane in $X$ iff $H=\{x\in X:f(x)=a\}$ for some nontrivial (continuous) linear functional $f$ on $X$ and $a\in\K$.
\end{itemize}
\end{proposition}
\begin{proof}
If $\dim X/M\geq 2$, there are $x,y\in X$ such that $x+M$ and $y+M$ are linearly independent in $X/M$. Moreover, we have the proper inclusions $M\subsetneq\mathrm{span}(M,x)\subsetneq\mathrm{span}(M,x,y)$ and $M$ is not maximal. Conversely, if $M$ is not maximal, there is a proper subspace $N$ of $X$ which contains $M$ properly; hence $\dim X/M>\dim N/M>1$.\par
Let $f$ be a linear functional and let $M=N(f)$. Then $X/M$ is isomorphic to $\K$, so $\dim X/M=1$; hence $M$ is maximal by (a). Conversely, if $M$ is maximal and $x\notin M$, then $M\oplus\K x=X$. Defining $f(y)=f(m+ax)=a$ for each $y\in X$, $f$ is a linear functional with null space $M$.\par
If $H$ is a hyperplane, then $H=x+M$ for some $x\in X$ and maximal subspace $M\sub X$. Since $M$ is maximal, there is a nontrivial linear functional $f$ on $X$ such that $M=N(f)$ by (b). Let $f(x)=a$. Clearly $H\sub f^{-1}(a)$. As to the reverse inclusion, if $f(y)=a$ then $y-x\in M$ and $y\in H$, i.e., $H=\{y\in X:f(y)=a\}$. The converse is proved similarly.
\end{proof}
We now show that a closed maximal subspace of a topological vector space always has a topological complement.
\begin{proposition}\label{TVS closed maximal subspace complemented}
If $M$ is a closed maximal subspace of a topological vector space $X$ then any algebraic complement $N$ of $M$ is a topological complement.
\end{proposition}
\begin{proof}
Let $M$ be a closed maximal subspace of a topological vector space $X$. Since $M$ is maximal, $\dim X/M=1$; hence if $N$ is an algebraic complement of $M$, $\dim N=1$. Therefore there must be some $x\neq 0$ such that $N=\K x$. Consequently $X=M\oplus\K x$ and each vector $y$ has a unique representation in the form $y=m+tx$, $m\in M$, $t\in\K$. To show that $\K x$ is a topological complement of $M$, we use the criterion of Theorem~\ref{TVS topological complement iff}(a): We show that the projection $P$ on $\K x$ along $M$, $tx+m\mapsto tx$, is continuous.\par
To this end note that $P^{-1}(0)=M$. As $M$ is closed and $x\notin M$, there exists a balanced neighborhood $U$ of $0$ in $X$ such that $(x+U)\cap M=\emp$. We claim that if $m+tx\in U$ then $|t|\leq 1$. If $|t|\geq 1$ then, as $U$ as balanced, $(1/t)(m+tx)=m/t+x\in U$, which contradicts $(x+U)\cap M=\emp$. Hence if $0<r<1$ and $m+tx\in rU$ then $|t|\leq r$.\par
To establish the continuity of $P$ at $0$, suppose that the net $m_\alpha+t_\alpha x\to 0$. As such, for any $0<r<1$, $m_\alpha+t_\alpha x\in rU$ eventually. Therefore $|t_r|\leq r$ eventually. In other words, $t_\alpha\to 0$, which implies that $t_\alpha x=P(m_\alpha+t_\alpha x)\to 0$, and proves the continuity of $P$.
\end{proof}
\begin{example}\label{TVS uncomplemented eg}
Let $\ell^\infty$ be the Banach space of bounded real sequences and $c_0$ the closed subspace of null sequences. To show that $c_0$ is uncomplemented, we show the criterion of Proposition~\ref{TVS topological complement iff}(a) to be violated, that there is no continuous projection of $\ell^\infty$ onto $c_0$. What moves the proof is the following ingenious observation:
\begin{itemize}
\item Any denumerable set $I$ has an uncountable family $\{U_s:s\in S\}$ of infinite subsets, each of whose pairwise intersections is finite. (We may suppose that $I$ is the set of rationals in $(0,1)$. Let $S$ denote the irrationals in $(0,1)$ and for each $s$ in $S$ let $U_s$ consist of the elements of a sequence of rationals in $(0,1)$ which converges to $s$.)
\item $\ell^\infty/c_0$ in its quotient topology is a normed space. The canonical norm of the coset $x+c_0$ is given by $\|x+c_0\|=\inf_{y\in c_0}\|x+y\|$.
\item If $h$ is a continuous linear functional on a normed linear space then $h$ is bounded on the unit ball $B_1(0)$.
\end{itemize}
A collection $\mathscr{T}$ of linear functionals on a vector space $X$ is called \textbf{total} if the only vector on which each member of $T$ vanishes is $0$. The coefficient functionals $\xi_k:\ell^\infty\to\R,(a_n)\mapsto a_k$, for example, are a total set of continuous linear functionals on $\ell^\infty$. Moreover, if $X$ is linearly isomorphic to $Y$ via a map $A$, and $T$ is a total set of linear functionals on $X$ then $\{f\circ A^{-1}:f\in\mathscr{T}\}$ is a total set of linear functionals for $Y$. If $P$ is a continuous projection off􀁣 onto Co then the closed subspace $M=P^{-1}(0)$ is a topological complement of $c_0$, Since the continuous dual of $\ell^\infty$ has $\{\xi_k:k\in\N\}$ as a countable total subset, so does $M$ and therefore so does $\ell^\infty/c_0$ since $\ell^\infty/c_0$ is linearly homeomorphic to $M$ (Proposition~\ref{TVS topological complement iff}(b)). We now seek to contradict this fact about $\ell^\infty/c_0$.\par
Let $\{U_s:s\in S\}$ be an uncountable family of infinite subsets of $\N$ with finite pairwise intersections. For each $s\in S$ let $\chi_s$ be the characteristic function of $U_s$; note that each $\chi_s\in\ell^\infty$. Let $\bar{\chi}_s$ denote the coset $\chi_s+c_0$. For a nontrivial continuous linear functional $f$ on $\ell^\infty/c_0$ and $n\in\N$, let $B_n(f)=\{\bar{\chi}_s:|f(\bar{\chi}_s)|\geq n^{-1}\}$. We show that
\[\{\bar{\chi}_s:f(\bar{\chi}_s)\neq 0\}=\bigcup_{n}B_n(f).\]
is countable by showing that each $B_n(f)$ is finite.\par
Suppose that $\bar{\chi}_{s_1},\dots,\bar{\chi}_{s_m}\in B_n(f)$ and let $b_i=\sgn(f(\bar{\chi}_{s_i}))$. Let $y=\sum_{i=1}^{m}b_i\chi_{s_i}$ and $\bar{y}=\sum_{i=1}^{m}b_i\bar{\chi}_{s_i}$. Since the intersection of any $U_{s_i}$ and $U_{s_j}$ is finite, $y$ can take values other than $0,\pm 1$ with only finitely many times. Define $w:\N\to\R$ by
\[w(j)=\begin{cases}
-y(j)&|y(j)|>1,\\
0&|y(j)|\leq 1.
\end{cases}\]
Then $w\in c_0$ and $\|y+w\|_\infty=1$, and so $\|\bar{y}\|\leq 1$. Now, since $\bar{\chi}_{s_1},\dots,\bar{\chi}_{s_m}\in B_n(f)$,
\[\|f\|\geq\|f\|\|\bar{y}\|\geq|f(\bar{y})|=\Big|\sum_{i=1}^{m}\sgn(f(\bar{\chi}_{s_i}))f(\bar{\chi}_{s_i})\Big|=\sum_{i=1}^{m}|f(\bar{\chi}_{s_i})|\geq m/n.\]
Thmefore $m\leq\|f\|n$ and the finiteness of $B_n(f)$ is established.\par
Let $\{f_n:n\in\N\}$ be a countable collection of continuous linear functionals on $\ell^\infty/c_0$. By the previous argument, any particular $f_n$ can be nonzero on only countably many $\bar{\chi}_s$. Therefore, since $S$ is uncountable, there exists some $s\in S$ such that $f_n(\bar{\chi}_s)=0$ for each $n$. Thus $\{f_n:n\in\N\}$ is not a total set of continuous linear functionals on $\ell^\infty/c_0$ and there can be no countable total set of continuous linear functionals on $\ell^\infty/c_0$. The proof is now complete.
\end{example}
\section{Finite-dimensional and locally compact spaces}
There are two points to this part. One is that there is only one $n$-dimeusional Hausdorff topological vector space for each $n$ up to linear homeomorphism. The other is that local compactness is an overwhelming hypothesis on a topological vector space: it is equivalent to finite-dimensionality. We begin with two lemmas.
\begin{lemma}\label{TVS dim 1 homeomorphic to K}
If $X$ is a one-dimensional Hausdorff topological vector space over $\K$ then for any nonzero vector $x$, the map $\K\to X,a\mapsto ax$ is a linear homeomorphism of $\K$ onto $X$.
\end{lemma}
\begin{proof}
Let $f$ denote the map. Clearly $f$ establishes a surjective linear isomorphism between $X$ and $\K x$; continuity of $f$ follows from the continuity of scalar multiplication. It remains to show that $f$ is open. To this end, let $r$ be positive. We show that $f$ is open by showing that there is a neighborhood $V$ of $0$ in $X$ such that if $y=bx\in V$ then $|b|\leq r$; in other words, $V\sub f(B_r(0))$. Let $a\in B_r(0)$ and $a\neq 0$. Since $X$ is Hausdorff and $ax\neq 0$, there is a balanced neighborhood $V$ of $0$ in $X$ such then $ax\notin V$. Now choose $y=bx\in V$ for some scalar $b$. If $|b|\geq r\geq|a|$ then $|a/b|\leq 1$. Since $V$ is balanced and $bx\in V$, this implies $ax\in V$, which contradicts the choice. We conclude that $|b|\leq r$ and therefore that $V\sub B_r(0)x$. 
\end{proof}
\begin{lemma}\label{TVS finite dim homeomorphic to K^n}
For any $n\in\N$, if $X$ is an $n$-dimensional topological vector space, then every surjective linear isomorphism $\K^n\to X$ is bicontinuous.
\end{lemma}
\begin{proof}
Let $X$ be an $n$-dimensional Hausdorff topological vector space over $\K$ and let $A:\K^n\to X$ be a surjective linear isomorphism. We proceed by induction on $\dim X$. The case $\dim X=1$ is handled by Lemma~\ref{TVS dim 1 homeomorphic to K}. Suppose that $\dim X=n$ and the theorem holds for dimensions $\leq n-1$. Let $\{e_i\}$ be the standard basis of $\K^n$, and $Ae_i=x_i$. Let $M=\mathrm{span}(x_1,\dots,x_{n-1})$. Then with the relative topology, $M$ is a Huasdorff topological vector space. By induction hypothesis, the linear isomorphism $(a_1,\dots,a_{n-1})\mapsto\sum_{i=1}^{n-1}a_ix_i$ is bicontinuous. Since $\K^{n-1}$ is complete, $M$ is also complete by Proposition~\ref{topological group homomorphism is uniformly continuous}, and so is closed since $X$ is Hausdorff. Since $\dim X/M=1$, $M$ is a maximal subspace of $X$. The subspace $N=\K x_n$ is clearly an algebraic complement of $M$, and so is a topological complement of $M$, by Proposition~\ref{TVS closed maximal subspace complemented}; hence the linear isomorphism of $M\times N$ to $X$,
\[M\times N\to X,\quad (\sum_{i=1}^{n-1}a_ix_i,a_nx_n)\mapsto\sum_{i=1}^{n}a_ix_i\]
is bicontinuous. By the induction hypothesis, the linear isomorphisms $(a_1,\dots,a_{n-1})\mapsto\sum_{i=1}^{n-1}a_ix_i$ is bicontinuous, so the linear isomorphism
\[\K^{n-1}\times\K\to M\times N\to M\times N,\quad ((a_1,\dots,a_{n-1}),a_n)\mapsto (\sum_{i=1}^{n-1}a_ix_i,a_nx_n)\]
is also bicontinuous. Combined with the previous observations, it follows that the linear isomorphism $((a_1,\dots,a_{n-1}),a_n)\mapsto\sum_{i=1}^{n}a_ix_i$ is bicontinuous. To complete the proof it only remains to observe that the map
\[\K^n\to\K^{n-1}\times\K,\quad ((a_1,\dots,a_{n-1}),a_n)\mapsto(a_1,\dots,a_n)\]
is a bicontinuous linear isomorphism.
\end{proof}
As a consequence of Lemma~\ref{TVS finite dim homeomorphic to K^n}, finite-dimensional topological vector spaces have the following properties.
\begin{theorem}\label{TVS finite dim prop}
Let $X$ and $Y$ be topological vector spaces. Let $X$ be $n$-dimensional with basis $\{x_1,\dots,x_n\}$. Then
\begin{itemize}
\item[(a)] If $X$ and $Y$ are Hausdorff and are of dimension $n$ then every linear isomorphism of $X$ onto $Y$ is bicontinuous.
\item[(b)] If $\mathcal{T}_1$ and $\mathcal{T}_2$ are Hausdorff vector topologies on $X$, then $\mathcal{T}_1=\mathcal{T}_2$. In particular, all norms are equivariant on finite-dimensional Hausdorff topological vector spaces.
\item[(c)] If $X$ is Hausdorff then the topology on $X$ is generated by a norm which makes $X$ a Banach space.
\item[(d)] If $X$ is a subspace of a Hausdorff topological vector space, then $X$ is closed.
\item[(e)] If $M$ is a closed subspace of $Y$ and $N$ is a finite-dimensional subspace of $Y$, then $M+N$ is closed.
\item[(f)] If $A:X\to Y$ is a linear map and $X$ and $Y$ are Hausdorff, then $A$ is continuous. In particular, any linear functional space $X$ is continuous, provided $X$ is Hausdorff.
\item[(g)] If $A:Y\to X$ is a continuous surjective linear map and $X$ is Hausdorff, then $A$ is open. Furthermore, $X$ is linearly homeomorphic to $Y/N(A)$. In particular, any nontrivial linear functional must be open.
\end{itemize}
\end{theorem}
\begin{proof}
For (a), let $A$ be a linear isomorphism from $X$ to $Y$. Consider the linear isomorphisms $L_1:X\to\K^n$ and $L_2:\K^n\to Y$. By Lemma~\ref{TVS finite dim homeomorphic to K^n} they are bicontinuous, and so is $A=L_2\circ L_1$. Now (b) follows from (a), since the identity from $(X,\mathcal{T}_1)$ to $(X,\mathcal{T}_2)$ is bicontinuous.\par
Norm $X$ by $\|\sum_ia_ix_i\|=\sum_i|a_i|$. If $X$ is Hausdorff, the norm topology must coincide with the original one. Then $X$ is linearly homeomorphic to $\K^n$ and the completeness follows from Proposition~\ref{topological group homomorphism is uniformly continuous}. Part (d) follows from (c), since $X$ is a Hausdorff complete topological vector space in its relative topology.\par
Let $M$ be a closed subspace of $Y$ and $N$ be a finite-dimensional subspace of $Y$. Then $Y/M$ is Hausdorff by Proposition~\ref{TVS quotient topo prop}. Let $\pi:Y\to Y/M$ be the quotient map. Since $\pi$ is linear, $\pi(N)$ is a finite-dimensional subspace of $Y/M$, and so is closed by (d). Since $\pi$ is continuous, $M+N=\pi^{-1}(\pi(N))$ is closed.\par
If $A:X\to Y$ is a linear map and $X$ and $Y$ are Hausdorff, then $R(A)$ is finite-dimensional, and so a Hausdorff topological vector space in its relative topology. Let $N=N(A)$ and decompose $A$ as in Proposition~\ref{TVS first isomorphism thm}:
\[\begin{tikzcd}
X\ar[r,"\pi"]&X/N\ar[r,"\bar{A}"]&R(A)
\end{tikzcd}\]
Since $N$ is finite-dimensional, $N$ is closed by (d), so $X/N$ is Hausdorff and therefore the isomorphism $\bar{A}$ is bicontinuous by (a). Since $\pi$ is continuous, the continuity of $A$ follows.\par
Let $A:Y\to X$ be a continuous surjective linear map. If $X$ is Hausdorff, then $\{0\}$ is closed, so $N=A^{-1}(0)$ is closed in $Y$ and $Y/N$ is Hausdorff. Since the map $\bar{A}:Y/N\to X$ is a linear isomorphism, it is bicontinuous by (a) and the openness of $A$ follows from Proposition~\ref{TVS first isomorphism thm}.
\end{proof}
\begin{theorem}\label{TVS locally compact}
A Hausdorff topological vector space $X$ is locally compact if and only if $X$ is finite-dimensional.
\end{theorem}
\begin{proof}
Let $X$ be a Hausdorff topological vector space of dimension $n\in\N$. Since $X$ is linearly homeomorphic to $\K^n$ by Theorem~\ref{TVS finite dim prop}, $X$ is locally compact. To prove the converse we use the fact that if a subspace $M$ contains a neighborhood $V$ of $0$ then, since $V$ is absorbent, $M=X$.\par
Assume that the Hausdorff topological vector space $X$ is locally compact and let $V$ be a balanced compact neighborhood of $0$. We show that the collection $\{2^{-n}V:n\in\N\}$ forms a neighborhood base at $0$. Let $B$ be any neighborhood of $0$ and choose a balanced neighborhood $U$ of $0$ such that $U+U\sub B$. Since $V$ is compact, it is totally bounded; hence there exists a finite subset $S$ of $X$ such that $V\sub S+U$. Since $U$ is absorbent and balanced, there exists a $r\geq 1$ such that $S\sub tU$ and $U\sub tU$ for $|t|\geq r$. Therefore, for $|t|\geq r$,
\[V\sub S+U\sub tU+U\sub tU+tU\sub tB.\]
and it follows that the sets $2^{-n}V,n\in\N$, form a base at $0$.\par
Since $V$ is totally bounded, there is a finite subset $D=\{x_1,\dots,x_n\}\sub X$ such that $V\sub D+(1/2)V$. We show that $\dim X\leq n$. The linear span $M$ of $D$ is closed by Theorem~\ref{TVS finite dim prop}(d). Since $V\sub M+(1/2)V$ and $tM=M$ for all nonzero scalars $t$, $(1/2)V\sub M+(1/4)V$. Hence we observe that
\[V\sub M+(1/2)V\sub M+(M+(1/4)V)=M+(1/4)V\sub\cdots\sub M+2^{-n}V\sub\cdots\]
By induction, $V\sub\bigcap_{n\in\N}(M+2^{-n}V)$. This last set is the closure of $M$ by Theorem~\ref{TVS interior and closure}(e) which is just $M$. Thus, $M$ contains a neighborhood of $0$ and therefore $M=X$.
\end{proof}
Finally, we establish an interesting result for convex hulls in finite-dimensional spaces.
\begin{lemma}[\textbf{Carath\'eodory's theorem}]\label{Caratheodory theorem}
Let $E$ be a subset of $\R^n$, then for every point $x\in\conv(E)$, there are $n+1$ points $x_0,\dots,x_{n}\in E$ with $x\in\conv(x_0,\dots,x_n)$.
\end{lemma}
\begin{proof}
Let $x$ be a point in the convex hull of $E$. Then, $x$ is a convex combination of a finite number of points in $E$:
\[x=\sum_{i=1}^{k}a_ix_i.\]
Suppose $k>n+1$ (otherwise, there is nothing to prove). Then, the vectors $x_2-x_1,\dots,x_k-x_1$ are linearly dependent, so there are real scalars $\lambda_2,\dots,\lambda_k$, not all zero, such that
\[\sum_{i=2}^{k}\lambda_i(x_i-x_1)=0.\]
If $\lambda_1:=-\sum_{j=2}^{k}\lambda_i$, then $\sum_{i=1}^{k}\lambda_ix_i=0$, $\sum_{i=1}^{k}\lambda_i=0$, and not all of the $\lambda_j$ are equal to zero. Therefore, for any $\alpha>0$ we have
\[x=\sum_{i=1}^{k}a_ix_i-\alpha\sum_{i=1}^{k}\lambda_ix_i=\sum_{i=1}^{k}(a_i-\alpha\lambda_i)x_i.\]
In particular, the equality will hold if $\alpha$ is defined as
\[\alpha=\min_{1\leq i\leq k}\{a_i/\lambda_i:\lambda_i>0\}=a_{i_0}/\lambda_{i_0}.\]
With this choice of $\alpha$, each coefficient $a_i-\alpha\lambda_i$ is nonnegative and at least one is zero. Since $\sum_i(a_i-\alpha\lambda_i)=0$, this shows $x$ is represented as a convex combination of at most $k-1$ points of $E$. This process can be repeated until $x$ is represented as a convex combination of at most $n+1$ points in $E$.
\end{proof}
\begin{proposition}
Let $X$ be an $n$-dimensional space topological vector space. If $K\sub X$ is compact, then its convex hull $\conv(K)$ is compact.
\end{proposition}
\begin{proof}
We may assume that $X=\R^n$. By Lemma~\ref{Caratheodory theorem}, $\conv(K)$ consists of all convex combinations of the form $\sum_{i=1}^{n+1}a_ix_i$. Then $\conv(K)$ is the image of the set $S\times E$, where $S\sub[0,1]^{n+1}$ is defined by
\[S=\{(a_0,\dots,a_n):\sum_{i=0}^{n}a_i=1\}.\]
Since $S$ is compact, $\conv(K)$ is seen to be compact.
\end{proof}
\section{Extreme points and faces}
We recall our conventions about lines and line segments in vector spaces.
\begin{definition}
Let $x$ and $y$ be two points in a vector space $X$ over $\K$. Then:
\begin{itemize}
\item The \textbf{closed line segment} (or interval) $[x,y]$ with endpoints $x$ and $y$ or line joining $x$ and $y$ is the set $\{tx+(1-t)y:t\in[0,1]\}$ of convex combinations of $x$ and $y$. It is a proper line segment if $x\neq y$.
\item The \textbf{open line segment} (interval) with endpoints $x$ and $y$ is $(x,y)=\{tx+(1-t)y:t\in(0,1)\}$. lf $x=y$, $(x,y)=\emp$. lf $x\neq y$ and $z\in (x,y)$, then we say that $z$ is \textbf{between $\bm{x}$ and $\bm{y}$}. The point $(x+y)/2$ is called the \textbf{midpoint} of $[x,y]$ and $(x,y)$ and is between $x$ and $y$ for $x\neq y$.
\item The \textbf{line through $\bm{x}$ and $\bm{y}$} is $L(x,y)=\{tx+(1-t)y:t\in\R\}=y+\R(y-x)$.
\end{itemize}
\end{definition}
\begin{definition}
A map $A$ of a vector space $X$ into a vector space $Y$ is said to be affine if $A(tx+(1-t)y)=tAx+(1-t)Ay$ for all $x,y\in X$ and $t\in[0,1]$, in other words, if $A([x,y])=[Ax,Ay]$ for all $x,y\in X$.
\end{definition}
The following result for isometric maps will be used later.
\begin{proposition}
Let $X$ and $Y$ be normed spaces over $\K$ and let $A:X\to Y$ be a surjective isometry such that $A0=0$. Then $A$ maps midpoints into midpoints and is $\R$-linear.
\end{proposition}
\begin{proof}
With notation as above, for any $x,y\in X$, it is routine to verify that $m:=(x+y)/2$ is equidistant from $x$ and $y$, namely that $\|x-m\|=\|y-m\|=\|x-y\|/2$. There can be other points $w$ equidistant from $x$ and $y$, however. We denote the set of such equidistant points by $E_1(x,y)=\{w\in X:\|w-x\|=\|w-y\|=\|x-y\|/2\}$ and let $d(E_1(x,y))$ denote the diameter of $E_1(x,y)$. Now compress $E_1(x,y)$: for $n\geq 2$, let
\[E_n(x,y)=\{z\in E_{n-1}(x,y):\|z-w\|\leq (1/2)d(E_{n-1}(x,y))\text{ for each $w\in E_{n-1}(x,y)$}\}.\]
It is easy to see $d(E_n(x,y))\leq 2^{-n+1}d(E_1(x,y))$. Since $\{E_n\}$ is decreasing, if $\bigcap_nE_n(X,y)\neq\emp$, then it must be a singleton. We now show that $\bigcap_nE_n(x,y)=\{m\}=\{(x+y)/2\}$.\par
First, we need the following result: for each $n\in\N$, $z\in E_n(x,y)$ implies that $\bar{z}=x+y+z\in E_n(x,y)$. We argue by induction. For $n=1$ and $z\in E_1(x,y)$,
\[\|\bar{z}-x\|=\|y-z\|=\|x-z\|=\|\bar{z}-y\|.\]
Thus $\bar{z}\in E_1(x,y)$. Suppose that $n\geq 2$ and the claim is ture for $k=1,\dots,n-1$. Now let $z\in E_n(x,y)$ and $\bar{z}=x+y-z$. Then for $w\in E_{n-1}(x,y)$, $\bar{w}=x+y-w\in E_{n-1}(X,y)$, so
\[\|\bar{z}-w\|=\|x+y-z-w\|=\|\bar{w}-z\|\leq(1/2)d(E_{n-1}(x,y))\]
thus $\bar{z}\in E_n(x,y)$.\par
We now show that $m\in E_n(x,y)$ for all $n$. Clearly $m\in E_1(x,y)$. Now suppose that $n\geq 2$ and $z\in E_{n-1}(x,y)$. Then
\[2\|m-z\|=\|x+y-2z\|=\|x+y-z-z\|=\|\bar{z}-z\|\leq d(E_{n-1}(x,y))\]
Thus $\|m-z\|\leq (1/2)d(E_{n-1}(x,y))$ for all $z\in E_{n-1}(x,y)$; therefore $m=(x+y)/2\in E_n(x,y)$. Since $d(E_n(x,y))\to 0$, we conclude that $\bigcap_nE_n(x,y)=\{m\}$. By the same argument, $\bigcap_nE_n(Ax,Ay)=\{(Ax+Ay)/2\}$.\par
We now show that $A(E_n(x,y))=E_n(Ax,Ay)$ for all $n$. Since $A$ is an isometry, it is not hard to see $A(E_1(x,y))=E_1(Ax,Ay)$. Now suppose $n\geq 2$. Since $A$ is a an isometry, $d(E_{n-1}(Ax,Ay))=d(E_{n-1}(x,y))$; hence, since $A$ is surjective,
\begin{align*}
v=Az\in E_n(Ax,Ay)&\iff\forall w\in E_n(Ax,Ay),\|Az-Aw\|\leq d(E_{n-1}(Ax,Ay))\\
&\iff\forall w\in E_n(x,y),\|z-w\|\leq d(E_{n-1}(x,y))\iff z\in E_n(x,y).
\end{align*}
Thus $A(E_n(x,y))=E_n(Ax,Ay)$. With this, we then get
\[A(\{(x+y)/2\})=\bigcap_{n}A(E_n(x,y))=\bigcap_nE_n(Ax,Ay)=\{(Ax+Ay)/2\}.\]
Thus $Am=(Ax+Ay)/2$. This yields the $\Q$-additivity of $A$ and the $\R$-linearity follows from the continuity.
\end{proof}
\begin{proposition}
Any real linear functional is monotone (hence one-to-one) or constant on lines and line segments.
\end{proposition}
\begin{proof}
Let $f$ be a real linear functional on a vector space and let $x$ and $y$ be distinct vectors. Since $x\neq y$, the map $t\mapsto y+t(x-y)$ establishes a one-to-one correspondence between $\R$ and the line through $x$ and $y$ and we order $L(x,y)$ accordingly: $y+s(x-y)\leq y+t(x-y)$ iff $s\leq t$. Now the restriction $f|_{L(x,y)}$ sends $y+t(x-y)$ into $f(y)+tf (x -y)$. Identifying $\R$ and $L(x,y)$, the derivative of $f$ with respect to $t$ is $f(x-y)$, so $f$ is increasing if $f(x-y)>0$, constant if $f(x)=f(y)$, and decreasing if $f(x-y)<0$.
\end{proof}
\begin{definition}
A point $x$ of a convex set $K$ is an \textbf{extreme point} of $K$ if $x$ is not between any two distinct points of $K$. The set of extreme points of $K$ is denoted $\mathcal{E}(K)$.
\end{definition}
The following result is immediate and we omit its proof.
\begin{proposition}
Let $K$ be convex and $x\in K$. Then $x\in\mathcal{E}(K)$ iff any of the following conditions hold.
\begin{itemize}
\item[(\rmnum{1})] $K\setminus\{x\}$ is convex.
\item[(\rmnum{2})] $x$ is not the midpoint of any proper line segment of $K$.
\item[(\rmnum{3})] If $y,z\in K$ are such that $x\in[y,z]$ then $x=y=z$.
\item[(\rmnum{4})] If $x+y,x-y\in K$, then $y=0$.
\item[(\rmnum{5})] $x$ can only be written as a trivial convex combination of any points in $K$. 
\end{itemize}
\end{proposition}
The linear image of an extreme point of a convex set need not be an extreme point of the image. Consider, for example, an equilateral triangle whose base is on the $x$-axis. The projection onto the $x$-axis maps the apex of the triangle into an interior point of the image. However, injective linear maps---hence linear isometries---carry extreme points into extreme points.
\begin{proposition}\label{extreme point affine injective map}
Let $X$ and $Y$ be linear spaces, $K$ a convex subset of $X$, and $A$ an affine map of $X$ into $Y$. If $A$ is injective, then $\mathcal{E}(A(K))=A(\mathcal{E}(K))$. Hence injective linear maps take extreme points into extreme points.
\end{proposition}
\begin{proof}
With notation as above, if $A$ is injective, then $Ax\in(Ay,Az)$ implies that $x\in(y,z)$. Thus if $x\in\mathcal{E}(K)$, then $Ax$ must be an extreme point of $A(K)$.
\end{proof}
Now we investigate the connection between the purely algebraic notion of extreme point of a convex subset $K$ of a topological vector space and the topological boundary of $K$. Comparing boundary points and extreme points, we first note that an extreme point must be in the set, but a boundary point need not. Extreme points, however, are necessarily boundary points.
\begin{proposition}\label{TVS extreme point in boundary}
Let $K$ be a convex subset of a topological vector space $X$. If $x\in\Int K$, then $x$ is not an extreme point of $K$.
\end{proposition}
\begin{proof}
Let $K$ be a convex subset of $X$ and suppose $x\in\Int K$. Since the map $f:\R\to X,t\mapsto tx$ is continuous, there must be some $r>0$ such that the segment $f((1-r,1+r))=((1-r)x,(1+r)x)\sub\Int K\sub K$. Since $x$ is contained in this interval, $x\notin\mathcal{E}(K)$.
\end{proof}
\begin{example}
In any inner product space $X$, $\mathcal{E}(B_X)=S_X$. To see this, let $x,y\in B_X$ be such that $x\neq y$ and $t\in[0,1]$. Then
\[\|tx+(1-t)y\|^2=t^2\|x\|^2+(1-t)^2\|y\|^2+2t(1-t)\Re\langle x,y\rangle\leq t^2+(1-t)^2+2t(1-t)=1\]
where equality holds iff $\|x\|=\|y\|=1$ and $x=\lambda y$, which means $x=y$. Thus any point in $S_X$ is an extreme point.
\end{example}
\begin{example}[\textbf{Extreme Points of $\bm{L^p}$ Spaces}]
\mbox{}
\begin{itemize}
\item[(a)] Let $(X,\mathcal{A},\mu)$ be a measure space and consider $L^p(\mu)$. Let $x,y\in B_{L^p(\mu)}$ and consider $z=(x+y)/2$. If $\|z\|=1$, then $2=\|x+y\|\leq\|x\|+\|y\|$, so $\|x\|=\|y\|=1$ and $\|x+y\|=\|x\|+\|y\|$. For $p>1$, equality holds in the Minkowski inequality iff one vector is a scalar multiple of the other; thus $x=ay$. Since $\|x+y\|=\|(1+a)y\|=2$ and $\|x\|=\|ay\|=1$, we get $a=1$, thus $x=y=z$. This implies $\mathcal{E}(B_{L^p(\mu)})=S_{L^p(\mu)}$.
\item[(b)] Let $(c_0,\|\cdot\|_\infty)$ denote the Banach space of all complex null sequences. By Proposition~\ref{TVS extreme point in boundary}, $\mathcal{E}(B_{c_0})\sub S_{c_0}$. We show next that $\mathcal{E}(B_{c_0})=\emp$. Choose $n\in\N$ such that $|x_n|<\|x\|_\infty=1$ (since $x\in c_0$, such $n$ exists). For sufficiently small positive $r$, replace $x_n$ by $x_n-r$ and $x_n+r$ to create vectors $y$ and $z$ in $S_{c_0}$ such that $x=(y+z)/2$.
\item[(c)] Let $\ell^1$ be the Banach space of all absolutely summable complex sequences $x=(x_i)$. The extreme points of $B_{\ell^1}$ are unit multiples of the standard basis vectors $e_n$. To see this, suppose that $|a|=1$ and $ae_n=(x+y)/2$ for $x=(x_i)$ and $y=(y_i)$ in $B_{\ell^1}$. This immediately implies that $a=(x_n+y_n)/2$ and so $x_n=y_n=a$. With this and the condition $\|x\|=\|y\|=1$, we see $x=y=ae_n$, so $ae_n$ is an extreme point. On the other hand, suppose that $x=(x_n)\in S_{\ell^1}$ has two nonzero entries $x_i$ and $x_j$ with $i<j$. Then consider
\[p(t)=\sum_{k\notin\{i,j\}}x_ke_k+(x_i-\sgn(x_i)t)e_i+(x_j+\sgn(x_j)t)e_j.\]
Let $\delta=\min\{|x_i|,|x_j|\}$. Then $\|p(t)\|=1$ for $t\in[-\delta,\delta]$, $p(0)=x$. Thus $x=(p(\delta)+p(-\delta))/2$ and $x$ is not an extreme point.
\item[(d)] Let $L^1(\R)$ be the Banach space of (absolutely) Lebesgue-integrable real- or complex-valued functions $f$ on $\R$ normed by $\|f\|=\int_\R|f(x)|\,dx$. Let $f\in L^1(\R)$ be such that $\|f\|=1$. Then we can find $a\in\R$ such that $\|f\chi_{(-\infty,a)}\|=\|f\chi_{[a,+\infty)}\|=1/2$. Then we see $f=(2f\chi_{(-\infty,a)}+2f\chi_{[a,+\infty)})/2$. Since $\|2f\chi_{(-\infty,a)}\|=\|2f\chi_{[a,+\infty)}\|=1$, it follows that $f\notin\mathcal{E}(B_{L^1(\R)})$. Thus $\mathcal{E}(B_{L^1(\R)})=\emp$.
\end{itemize}
\end{example}
\begin{example}\label{extreme point of C(X)}
Let $\widetilde{X}=C(X,\K,\|\cdot\|_\infty)$ denotes the Banach space of all continuous maps of the compact Hausdorff space $X$ into $\K$. By Proposition~\ref{TVS extreme point in boundary} we know that if $f\in\mathcal{E}(B_{\widetilde{X}})$ then $\|f\|_\infty=1$. Moreover, as in (b), if $|f(x)|<1$ for some $x$ in $X$, by the complete regularity of $X$, there must be some $g\in C(X,\R)$, which vanishes outside a neighborhood of $x$, of sufficiently small norm that $f+g$ and $f-g$ belong to $B_{\widetilde{X}}$. Hence $f=((f+g)+(f-g))/2$ and $f$ is not an extreme point of $B_{\widetilde{X}}$. On the other hand, if $|f|\equiv 1$ then for each $x\in X$, $f(x)$ is on the circumference of the closed unit disk $\D$ of $\K$. In other words, $f(x)$ is an extreme point of $\D$ for each $x\in X$. From this, it is now easy to see $f$ is an extreme point of $B_{\widetilde{X}}$. Thus $\mathcal{E}(B_{\widetilde{X}})=\{f\in C(X,\K):|f|\equiv 1\}$.\par
Let $BC(X,\R,\|\cdot\|)$ denotes the Banach space of all bounded continuous maps of the completely regular Hausdorff space $X$ into $\R$. Let $\beta X$ be the Stone-Cech compactification of $X$. Then $BC(X,\R)$ is linearly isometric to $C(\beta X,\R)$ under the mapping 
\[A:BC(X,\R)\to C(\beta X,\R),\quad f\mapsto \beta f,\]
the continuous extension of $f$ to $\beta X$. By (d), $\mathcal{E}(B_{C(\beta X,\R)})=\{f\in C(\beta X,\R):|f|\equiv 1\}$. Since $A$ is an isometry, by Proposition~\ref{extreme point affine injective map} we get
\[\mathcal{E}(B_{BC(X,\R)})=A^{-1}(\mathcal{E}(B_{C(\beta X,\R)}))=\{f\in BC(X,\R):|f|\equiv 1\}.\]
an observation first made by Krein and Milman.
\end{example}
\begin{example}
Consider the extreme point of the unit ball of $M(X)=C(X)^*$ where $X$ is a compact Hausdorff space. First, let $|a|=1$, $x\in X$ and consider the measure $a\delta_x$. It is easy to see $a\delta_x$ is an extreme point, using the fact that $a$ is an extreme point of $\D$. Now, suppose that $\mu$ is an extreme point of ball $M(X)$ and let $K$ be the support of $\mu$. We show that $K$ is a singleton. Assume that there are distinct points $x,y\in K$. Let $U$ and $V$ be open subsets of $X$ such that $x\in U$, $y\in V$, and $\widebar{U}\cap\widebar{V}=\emp$. By Urysohn's Lemma there is an $f$ in $C(X,[0,1])$ such that $f(\widebar{U})=\{1\}$ and $f(\widebar{V})=\{0\}$. Consider the measures 
\[d\mu_1=f\,d\mu,\quad \mu_2=(1-f)\,d\mu\]
and put $\alpha=\|\mu_1\|=\int f\,d|\mu|$. Note that $\alpha\leq\|\mu\|=1$ and $\alpha=\int f\,d|\mu|>0$ since $U$ is open and $U\cap K\neq\emp$. Also, $1-\alpha=1-\int f\,d|\mu|=\int(1-f)\,d|\mu|=\|\mu_2\|$. But then $\mu_1/\alpha$ and $\mu_2/(1-\alpha)$ are in $B_{M(X)}$ and we have
\[\mu=\alpha\Big[\frac{\mu_1}{\alpha}\Big]+(1-\alpha)\Big[\frac{\mu_2}{1-\alpha}\Big].\]
Since $\mu_1$ is not supported at $y$ and $\mu_2$ is not supported at $x$, we have $\mu_1\neq\mu_2$. This is a contradiction since $\mu$ is an extreme poitn. Thus $K=\{x_0\}$ and $\mathcal{E}(B_{M(X)})=\{a\delta_x:|a|=1,x\in X\}$.
\end{example}
\begin{definition}
Let $K$ be a nonempty convex set. A nonempty subset $F$ of $K$ is a \textbf{face} (\textbf{extremal subset}) of $K$ if for any $x$ in $F$, $x\in(y,z)$ for $y,z\in K$ implies $y,z\in F$.
\end{definition}
\begin{example}[\textbf{Examples of Faces}]\label{TVS face eg}
\mbox{}
\begin{itemize}
\item[(a)] The perimeter of any convex polygon in $\R^2$ is a face of the polygon.
\item[(b)] Any convex subset of a vector space is a face of itself.
\item[(c)] A singleton $\{x\}$ is a face of a convex set $K$ iff $x\in\mathcal{E}(K)$.
\item[(d)] "Is a face of" is a transitive relation: For convex sets $F$ and $K$, if $A$ is a face of $F$ and $F$ a face of $K$, then $A$ is a face of $K$. This follows by using the face condition twice.
\item[(e)] If $F$ is a convex face of the convex set $K$ then $\mathcal{E}(F)\sub\mathcal{E}(K)$. For if $x\in\mathcal{E}(F)$ and $y,z\in K$ are such that $x\in (y,z)$ then $y,z\in F$. Since $x\in\mathcal{E}(F)$, this is impossible.  
\end{itemize}
\end{example}
\begin{example}
Let $X$ be a compact Hausdorff space and let $\mathcal{P}(X)$ be the collection of probability measures on $X$. Then since $1$ is an extreme point of $\D$, $B_{\mathcal{P}(X)}$ is a fact of $B_{M(X)}$. Thus by Example~\ref{TVS face eg}, the extreme points of $B_{\mathcal{P}(X)}$ are $\delta_x$, where $x\in x$.
\end{example}
\begin{proposition}\label{LCS convex face iff boundary}
Let $K$ be a convex subset of the locally convex space $X$ with boundary $\partial K$. If $F:=K\cap\partial K\neq\emp$ then $F$ is a face of $K$. Conversely, if $x$ belongs to a proper face $F$ of $K$ (i.e., $F$ is a proper subset of $K$), then $x$ belongs to the boundary of $K$.
\end{proposition}
\begin{proof}
Suppose $F=K\cap\partial K\neq\emp$ and $v\in F$. If $v\in\mathcal{E}(K)$ then $v$ is not between distinct points of $K$, so suppose $v\notin\mathcal{E}(K)$. Assume that $v\in(x,y)$ with at least one of $x$ and $y$ is not in the boundary of $K$, say $x$. Then $x$ is an interior point of $K$. We write
\[v=tx+(1-t)y,\quad t\in(0,1).\]
Since $x\in\Int K$, there is a seminorm $p$ on $X$ such that $x+B_p\sub\Int K$. Choose $0<d<t$, since $v$ is a boundary point of $K$, there must be some $z\notin K$ such that $p(z-v)<d$. Now let $w\in X$ be such that
\[z=tw+(1-t)y.\]
Then $(z-v)=t(w-x)$, so that
\[p(w-x)=p(z-v)/t<d/t<1,\]
which implies $w\in x+B_p\sub\Int K$. Since $K$ is convex, this implies that $z=tx+(1-t)w\in K$ which is contradictory.\par
Conversely, suppose that $v\in\Int K$ is in a proper subset $F$ of $K$. We show that $F$ cannot be a face of $K$. Since $F$ is proper, there is some $x\in K\setminus F$. Since $v\in\Int K$, there is a continuous seminorm $p$ on $X$ such that $v+B_p\sub K$. Let $t>0$ be such that $y:=(v-tx)/(1-t)\in v+B_p$, so that $v=tx+(1-t)y$. Then $v$ is between $x$ and $y$ and $x\notin F$. Thus $F$ is not a fact of $K$. 
\end{proof}
\begin{proposition}\label{TVS face induced by inf of functional}
Let $K$ be a convex subset of a vector space $X$ and let $f$ be a real linear functional on $X$ which is bounded below on $K$. If $F=\{x\in K:f(x)=\inf f(K)\}$ then $F$ is a convex face of $K$. The analogous statement holds for suprema.
\end{proposition}
\begin{proof}
Since the hyperplane $f^{-1}(t)$ is convex for any $t\in\R$, it follows that $F$ is convex. With $a=\inf f(K)$, since the linear image of a convex set is convex, $f(K)=[a,b]$ or $[a,b)$ for some $b\in\R$ or $+\infty$. Suppose $v=tx+(1-t)y$ for some $x,y\in K$ and $t\in(0,1)$ and assume that $f(v)=a$. Since $a$ is an extreme point of $[a,b]$ or $[a,+\infty)$ and $f(v)=a=tf(x)+(1-t)f(y)$ implies $f(x)=f(y)=a$, i.e., that $x,y\in F$.
\end{proof}
\begin{proposition}\label{NVS face induced by adjoint}
Let $X$ and $Y$ be normed spaces and let their continuous duals $X^*$ and $Y^*$ carry their norm topologies. Suppose $A:X\to Y$ is a continuous linear map with adjoint $A^*$. If $\|A\|\leq 1$, $f\in\mathcal{E}(B_{X^*})$, and $(A^*)^{-1}(f)\cap B_{Y^*}\neq\emp$, then $(A^*)^{-1}(f)\cap B_{Y^*}$ is a convex face of $B_{Y^*}$.
\end{proposition}
\begin{proof}
Suppose that $g\in(A^*)^{-1}(f)\cap B_{Y^*}$ and $g\in(\phi,\psi)$, thus $g=t\phi+(1-t)\psi$ with $t\in(0,1)$. Applying $A^*$ gives
\[f=A^*g=tA^*\phi+(1-t)A^*\psi.\]
Since $\|A\|=\|A^*\|\leq 1$, $A^*(B_{Y^*})\sub B_{X^*}$, so that $A^*\phi,A^*\psi\in B_{Y^*}$. Since $f\in\mathcal{E}(B_{X^*})$, by the equality above we then conclude that $A^*\phi=A^*\psi=f$, which proves the claim.
\end{proof}
\section{Krein-Milman theorems}
Any closed half plane of $\R^2$ is a closed convex set with no extreme points. We have already seen that there are bounded closed convex sets that are devoid of extreme points as well, namely, the closed unit balls of $c_0$ and $L^1(\R)$. If one thinks of a compact convex set as the generalization of an interval in $\R$, then here is a genus of set that should have extreme points.
\begin{theorem}[\textbf{Krein-Milman I}]\label{Krein-Milman compact convex has extreme point}
If $K$ is a nonempty compact convex subset of a locally convex Hausdorff space $X$ then $\mathcal{E}(K)\neq\emp$.
\end{theorem}
\begin{proof}
Let $K$ be as above. Using Zorn's Lemma we get a minimal face $F$ of $K$ and then show that $F$ is a singleton.\par
Let $f$ be a real continuous linear functional on $X$. Since $K$ is compact, the closed set $F=\{x\in K:f(x)=\min f(K)\}$ is nonempty. Hence $F$ is a convex face of $K$ by Proposition~\ref{TVS face induced by inf of functional}. It follows that the class $\mathcal{F}$ of nonempty closed convex faces of $K$ is nonempty. Order $\mathcal{F}$ by inclusion: $F\preceq G$ iff $F\sub G$ for $F,G$ in $\mathcal{F}$. If $\{F_s:s\in S\}$ is a totally ordered subset of $\mathcal{F}$, it must satisfy the finite intersection condition; hence, the compactness of $K$ implies that $E=\bigcap_{s\in S}F_s$ is nonempty and closed; $E$ is a convex face of $K$ because each $F_s$ is. Thus, $E$ is a lower bound for $\{F_s\}$ and Zorn's lemma implies the existence of a minimal element $F\in\mathcal{F}$, i.e., a closed convex face of $K$ which is minimal with respect to set inclusion. If $F$ is a singleton $\{x\}$, then the theorem is proved because, as observed in Example~\ref{TVS face eg}, a singleton $\{x\}$ is a face of $K$ iff $x$ is an extreme point of $K$. Suppose that $x$ and $y$ are distinct points of $F$. Since $X$ is locally convex and Hausdorff, there is a real continuous linear functional $f$ on $X$ such that $f(x)<f(y)$ by Proposition~\ref{LCHS linear functional prop}. Since $F$ is compact, $f$ is bounded below on $F$ so, by Proposition~\ref{TVS face induced by inf of functional}, the set $E=\{w\in F:f(w)=\min f(F)\}$ is a closed convex face of $F$, hence also of $K$. Since $f(x)<f(y)$, only one of $x,y$ can belong to $E$ which means that $E$ is a proper subset of $F$. As this contradicts the minimality of $F$, we conclude that $F$ is a singleton.
\end{proof}
\begin{example}\label{NVS dual has extreme point}
Let $X$ be a normed space and let $X^*$ be its continuous dual. By the Alaoglu theorem, $B_{X^*}$ is $\sigma(X^*,X)$-compact. Since $B_{X^*}$ is convex, it must have extreme points by Theorem~\ref{Krein-Milman compact convex has extreme point}. In other words, the unit ball of the dual of any normed space must have extreme points. As the unit balls of $c_0$ and $L^1(\R)$, neither of them can be the dual of a normed space---in particular, they cannot be reflexive.
\end{example}
\begin{example}\label{LCHS closed convex face extreme point}
If $F$ is a closed convex face of a compact convex subset $K$ of a locally convex Hausdorff space, then $F$ has an extreme point $x$ by Theorem~\ref{Krein-Milman compact convex has extreme point}; $x$ is an extreme point of $K$ by Example~\ref{TVS face eg}.
\end{example}
\begin{example}\label{LCHS real functional minimal on extreme point}
If $K$ is a compact convex subset of a locally convex Hausdorff space $X$ and $f$ a real continuous linear functional on $X$, then $f$ assumes its minimum at an extreme point of $K$: since $K$ is compact, $F=\{x\in K:f(x)=\min f(K)\}$ is nonempty and is therefore a closed convex face of $K$. Consequently, an extreme point of $K$ is in $F$ by Example~\ref{LCHS closed convex face extreme point}. (Note the this does not mean that $f$ attains its maximum and minimum \textit{only} at extreme points.) Thus, if $f=0$ on $\mathcal{E}(K)$ then $f=0$.
\end{example}
\begin{theorem}[\textbf{Krein-Milman II}]\label{Krein-Milman closed convex hull equal iff linear functional}
Let $B$ be a subset of a compact convex set $K$ in a locally convex Hausdorff space $X$. Then:
\begin{itemize}
\item[(a)] $\inf f(B)=\inf f(\widebar{\conv}(B))$ for any real continuous linear functional $f$ on $X$;
\item[(b)] $\widebar{\conv}(B)=K$ iff for all real continuous linear functionals $f$ on $X$, $\inf f(B)=\min f(K)$.  
\end{itemize}
\end{theorem}
\begin{proof}
Let $f$ be a real continuous linear functional on $X$. Since $K$ is compact , $f(K)$ is bounded below; hence, since $B\sub\widebar{\conv}(B)$, $a=\inf f(B)\geq\inf f(\widebar{\conv}(B))$. To reverse the inequality, let $x\in\conv(B)$. Then $x=\sum_{i=1}^{n}t_ix_i$ for $x_i\in B$ and $\sum_it_i=1$. Then $f(x)=\sum_it_if(x_i)\geq \inf f(B)$, so $\inf f(\conv(B))\geq\inf f(B)$. Since $f$ is continuous, $\inf f(\conv(B))=\inf f(\widebar{\conv}(B))$, so the claim follows.\par
If $K=\widebar{\conv}(B)$, then $\min f(K)=\inf f(\widebar{\conv}(B))=\inf f(B)$ by (a). Conversely, suppose that for all real continuous linear functionals $f$ on $X$, $\inf f(B)=\min f(K)$. Clearly, $\widebar{\conv}(B)\sub K$. To reverse the inclusion, suppose that $x\in K$. If $x\notin $, there is, by Proposition~\ref{LCHS linear functional prop}, some real-valued $f\in X^*$ such that $f(x)<\inf f(\widebar{\conv}(B))=\inf f(B)=\min f(K)$ which is contradictory.
\end{proof}
\begin{example}
Consider the line segment $[v,w]$ connecting $v=(0, 0,1),w=(0,0,-1)$ in $\R^3$. Let $D$ be the closed disk in the $xy$-plane of radius $1$ centered at $(1,0,0)$. Let $K$ be the convex hull of $L\cup D$. The extreme points of the compact set $K$ are $v,w$ and all points on the circumference of $D$ except $(0,0,0)$. In particular, $\mathcal{E}(K)$ is not closed.
\end{example}
\begin{lemma}\label{LCHS extreme point of closed convex hull}
If $B$ is a subset of a locally convex Hausdorff space whose closed convex hull $\widebar{\conv}(B)$ is compact, then $\mathcal{E}(\widebar{\conv}(B))\sub\widebar{B}$.
\end{lemma}
\begin{proof}
Let $B$ be a subset of $X$ such that $\widebar{\conv}(B)$ is compact and let $x\in\mathcal{E}(\widebar{\conv}(B))$. Let $V$ be a closed disked neighborhood of $0$. We show that $(x+V)\cap B\neq\emp$, or equivalently $x\in B+V$.\par
Since $\widebar{\conv}(B)$ is compact, $B$ is totally bounded so there exist $x_1,\dots,x_n\in B$ such that $B\sub\bigcup_{i=1}^{n}(x_i+V)$. Each set $K_i:=\widebar{\conv}[(x_i+V)\cap B]$ is convex and compact, so $\conv(\bigcup_iK_i)$ is a compact convex set by Proposition~\ref{TVS convex hull of finite union of compact}. Since $B\sub\bigcup_iK_i$, it follows that $\widebar{\conv}(B)\sub\conv(\bigcup_iK_i)$. Since $K_i\sub\widebar{\conv}(B)$ holds for each $i$, we then get $\conv(\bigcup_iK_i)=\widebar{\conv}(B)$. Thus any $x\in\mathcal{E}(\widebar{\conv}(B))$ must be a convex combination $\sum_it_iy_i$, where $y_i\in K_i$ (Proposition~\ref{TVS convex hull char}). As $x$ is an extreme point of $\widebar{\conv}(B)$, this convex combination must be trivial: for some $j$ we must have $x=y_j\in K_j=\widebar{\conv}[(x_j+V)\cap B]\sub B+V$.
\end{proof}
If $X$ is a complete locally convex Hausdorff space, it follows from Theorem~\ref{TVS hull of totally bounded or compact sets} that if $K\sub X$ is compact then $\widebar{\conv}(K)$ is compact; hence, by Lemma~\ref{LCHS extreme point of closed convex hull}, $\mathcal{E}(\widebar{\conv}(K))\sub K$. The result below subsumes the fact that for compact convex $K$, any subset of $K$ whose closure contains the extreme points of $K$ can be used to reconstitute $K$.
\begin{theorem}[\textbf{Krein-Milman III}]\label{Krein-Milman closed convex hull equal iff extreme point}
Let $B$ be a subset of the compact convex set $K$ in a locally convex Hausdorff space $X$. Then $K=\widebar{\conv}(B)$ iff $\mathcal{E}(K)\sub\widebar{B}$. In particular, $K=\widebar{\conv}(\mathcal{E}(K))$.
\end{theorem}
\begin{proof}
If $K=\widebar{\conv}(B)$, then, by Lemma~\ref{LCHS extreme point of closed convex hull}, $\mathcal{E}(K)=\mathcal{E}(\widebar{\conv}(B))\sub\widebar{B}$. Conversely, suppose that $\mathcal{E}(K)\sub\widebar{B}$. Then by Example~\ref{LCHS real functional minimal on extreme point}, for any real functional $f$ on $X$ we have $\min f(K)=\inf f(B)=\inf f(\widebar{B})$. Thus Theorem~\ref{Krein-Milman closed convex hull equal iff linear functional} implies $K=\widebar{\conv}(B)$.
\end{proof}
Given any distinct points $x$ and $y$ in a locally convex Hausdorff space $X$, there is an $f\in X^*$ such that $f(x)\neq f(y)$. In particular, this is so for any normed space $(X,\|\cdot\|)$. We can sharpen this to the assertion that the unit ball $B_{X^*}$ distinguishes the points of $X$ (just divide $f$ by $\|f\|$). The Krein-Milman theorem permits the following refinement.
\begin{proposition}\label{NVS extreme point distinguish point}
For any normed space $X$ and any two distinct points $x$ and $y$ of $X$ there is an extreme point $f$ of the unit ball $B_{X^*}$ of $X^*$ such that $f(x)\neq f(y)$.
\end{proposition}
\begin{proof}
With notation as above, by the Alaoglu theorem, $B_{X^*}$ is $\sigma(X^*,X)$-compact, so $B_{X^*}=\mathrm{cl}_{\sigma(X^*,X)}\conv(\mathcal{E}(B_{X^*}))$. It suffices to show that there exists $f\in\mathcal{E}(B_{X^*})$ that separates $x\neq 0$ from $0$, i.e., that $f(x)\neq 0$. Given $x\neq 0$, there exists $g\in B_{X^*}$ such that $g(x)\neq 0$. Since $B_{X^*}=\mathrm{cl}_{\sigma(X^*,X)}\conv(\mathcal{E}(B_{X^*}))$, we may approximate $g$ by a convex combination of extreme points $f_1,\dots,f_n$ of $B_{X^*}$ to any degree of closeness in the $\sigma(X^*,X)$ topology. Hence for $0<r<|g(x)|$ there must be a convex combination $\sum_{i=1}^{n}a_if_i$ of elements $f_i\in\mathcal{E}(B_{X^*})$ such that $|g(x)-\sum_{i=1}^{n}a_if_i(x)|<r$; therefore not all the $f_i$ can vanish on $x$. It follows that there must be an $f\in\mathcal{E}(B_{X^*})$ such that $f(x)\neq 0$.
\end{proof}
\begin{example}[\textbf{Abundance of Extreme Points}]
\mbox{}
\begin{itemize}
\item[(a)] If the Banach space $X=C([0,1],\R,\|\cdot\|)$ of continuous real-valued functions on $[0,1]$ were the dual of a normed space $Y$, then the unit ball $B_X$ of $X$ would be $\sigma(X,Y)$-compact. Hence $B_X=\mathrm{cl}_{\sigma(X,Y)}\conv(\mathcal{E}(B_X))$. Since $[0,1]$ is connected, the extreme points of $B_X$ are just $f(x)\equiv 1$ and $g(x)\equiv -1$ by Example~\ref{extreme point of C(X)}. The convex hull of $\{f,g\}$, the line segment $[x,y]$, is closed, so this would say that any continuous function on $[0,1]$ of norm one could be written as a convex combination of the constants $f$ and $g$ which is ridiculous. It also follows that $C([0,1])$ is not reflexive.
\item[(b)] More generally, no infinite-dimensional normed space $X$ whose unit ball $B_X$ has only a finite number of extreme points can be the continuous dual of any normed space $Y$. If $X=Y^*$ and $\mathcal{E}(B_X)=\{x_1,\dots,x_n\}$ then $B_X$ is $\sigma(X,Y)$-compact and by Theorem~\ref{Krein-Milman closed convex hull equal iff extreme point}, $B_X=\mathrm{cl}_{\sigma(X,Y)}\conv(\{x_1,\dots,x_n\})=\conv(\{x_1,\dots,x_n\})$, which implies that $X$ is finite-dimensional. Consequently, the unit ball of any infinite-dimensional reflexive normed space must have infinitely many extreme points.
\end{itemize}
\end{example}
\newpage
\chapter{Locally convex spaces and seminorms}
\section{Seminorms and guages}
Recall that a seminorm is a map $p:X\to\R_{\geq 0}$ of a real or complex vector space $X$ which satisfies
\begin{itemize}
\item $p(ax)=|a|p(x)$ for $a\in\K$.
\item $p(x+y)\leq p(x)+p(y)$.
\end{itemize}
Seminorms arise naturally in many ways in analysis, through integrals, evaluation at a point, and suprema of functions on sets. Also any norm is a seminorm. Here are some others.
\begin{example}[\textbf{Examples of Seminorms}]\label{seminorm eg}
\mbox{}
\begin{itemize}
\item[(a)] If $A$ is a linear map of a vector space $X$ over $\K$ into a seminormed space $(Y,p)$ then $p\circ A$ is a seminorm on $X$. In particular, if $f$ is a linear functional on $X$ then $p(x)=|f(x)|$ is a seminorrn on $X$.
\item[(b)] If $a_1,\dots,a_n$ are nonnegative scalars and $p_1,\dots,p_n$ are seminorms then $\max_ia_ip_i$ and $\sum_ia_ip_i$ are seminorms.
\item[(c)] If $X$ is a linear space of integrable $\K$-valued functions on some set $T$ then $p(x)=|\int_Tf|$ is a seminorm on $X$.
\item[(d)] If $X$ is a vector space of $\K$-valued functions on a set $T$ then, for any $t_0\in T$, the map $x\mapsto|x(t_0)|$ is a seminorm on $X$.
\item[(e)] Let $C(X)$ be the linear space of all continuous maps of the topological space $X$ into $\K$ and let $K$ be a compact subset of $X$. Then $p_K(x)=\sup_{x\in K}|f(x)|$ is a seminorm.
\end{itemize}
\end{example}
If $p$ is a seminorm on $X$, the open unit ball determined by $p$ is denoted by $B_p$, and the closed unit ball is denoted by $\widebar{B}_p$. Some properties of the balls $B_p$ and $\widebar{B}_p$ follow.
\begin{proposition}\label{seminorm prop}
Let $p$ be a seminorm on a vector space $X$. Then:
\begin{itemize}
\item[(a)] $B_p$ is absorbent and disked;
\item[(b)] if $q$ is a seminorm on $X$ then $p\leq q$ iff $B_q\sub B_p$;
\item[(c)] for any linear map $A$ into $X$, $B_{p\circ A}=A^{-1}(B_p)$;
\item[(d)] If $X$ is a topological vector space and $p$ is continuous then the closure of $B_p$ in $X$ is $\widebar{B}_p$.
\end{itemize}
\end{proposition}
\begin{proof}
Note that the analogs of (a)--(c) hold for $\widebar{B}_p$. As (a)--(c) require only routine verification, we prove only (d). To that end, suppose $p\neq 0$ is a continuous seminorm. For any $x$ in the closure of $B_p$, consider a net $x_\alpha$ in $B_p$ that converges to $x$. Then $p(x_\alpha)\to p(x)$ and $p(x_\alpha)<1$ for each $\alpha$ imply that $p(x)\leq 1$. Hence the closure of $B_p$ is contained in $\widebar{B}_p$. Conversely, suppose that $x\in\widebar{B}_p$. If $p(x)<1$ then $x\in B_p$ so suppose $p(x)=1$. Let $t_n=1-1/n$ so that $t_xn\in B_p$ for every $n$. Since $t_nx\to x$, it follows that $x$ is in the closure of $B_p$.
\end{proof}
An immediate conseqmmce of (a) is that $p=q$ iff $B_q=B_p$. In regard to (e), note that if $(X,d)$ is a metric space and $x$ a point of $X$ then the closure of the open ball $B_r(x)$ is not necessarily equal to the closed ball $\widebar{B}_r(x)$. If $d$ is the trivial metric, for example (the distance between distinct points is $1$, otherwise $0$) then the closure of $B_1(x)$ is $\{x\}$, yet $\widebar{B}_r(x)=X$.
\begin{proposition}\label{seminorm continuous iff}
If $p$ is a seminorm on the topological vector space $X$ with filter $\mathfrak{U}(0)$ of neighborhoods of $0$ then the following are equivalent:
\begin{itemize}
\item[(a)] $p$ is uniformly continuous.
\item[(b)] $B_p$ is an open set.
\item[(c)] $\widebar{B}_p$ is a neighborhood of $0$.
\item[(d)] $p$ is continuous at $0$.
\item[(e)] There is a continuous seminorm $q$ such that $p\leq q$.
\end{itemize}
\end{proposition}
\begin{proof}
It is clear that $(a)\Rightarrow(c)$. To see that (c) implies (d), suppose that $\widebar{B}_p\in\mathfrak{U}(0)$. Since, for any nonzero $a$, the map $x\mapsto ax$ is a homeomorphism of $X$ onto $X$, it follows that $r\widebar{B}_p$ is a neighborhood of $0$ for any positive $r$. Consequently, if the net $x_\alpha\to 0$ in $X$, then $x_\alpha\in rB_p$ eventually or, equivalently, $p(x_\alpha)<r$ eventually. In other words, $p(x_\alpha)\to 0$.\par
Continuity of $p$ at $0$ means that for any $r>0$ there is a neighborhood $V$ of $0$ such that $p(V)\sub[0,r)$. By Proposition~\ref{TVS nhbd base of 0 balanced and absorbent} there is a balanced neighborhood $U$ of $0$ such that $U-U\sub V$. For $x,y\in U$ then $x-y\in V$ so $p(x-y)<r$. The triangle inequality yields the uniform continuity of $p$. This proves $(d)\Rightarrow(a)$.\par
To capture the equivalence of (d) and (e), note that if $p$ is continuous at $0$ then (e) holds trivially: $p\leq p$. Conversely, note that for any two seminorms $p$ and $q$, $p\leq q$ iff $B_q\sub B_p$. Thus, if $q$ is continuous, $B_q$ is a neighborhood of $0$ and therefore so is $B_p$· The desired result now follows from (c).
\end{proof}
\begin{proposition}\label{seminorm continuous on a seminormed space iff}
For a seminorm $q$ to be continuous on a seminormed space $(X,p)$ it is necessary and sufficient for $q$ to be dominated by a positive multiple of $p$.
\end{proposition}
\begin{proof}
If $q$ is a continuous seminorm on $(X,p)$ then for some $r>0$, $q(r\widebar{B}_p)\leq 1$, i.e., $r\widebar{B}_p\sub\widebar{B}_q$, which is equivalent to $q(x)\leq(1/r)p(x)$ for every $x\in X$.
\end{proof}
\begin{example}\label{seminorm space continuity eg}
\mbox{}
\begin{itemize}
\item[(a)] For $1\leq p<+\infty$ the \textbf{coefficient functionals} are defined by
\[\Lambda_k:\ell^\infty\to\K,\quad (a_n)\mapsto a_k\]
They are clearly linear and (therefore) $p_k=|\Lambda_k|$ is a seminorm. To prove continuity of $p_k$, suppose that $x=(a_n)\in\ell^\infty$. For any $k\in\N$, $|a_k|^p\leq\|x\|_p^p$, hence $|\Lambda_k(x)|\leq\|x\|_p$. The continuity of $p_k$ then follows from Proposition\ref{seminorm continuous on a seminormed space iff}. Thus $p_k$ is also continuous.
\item[(b)] For any set $A$ and any $a\in A$, the map
\[e_a:\ell^\infty(A)\to\K,\quad x\mapsto x(a)\]
is called the \textbf{evaluation map determined by $\bm{a}$}. Since $|e_a(x)|\leq\|x\|_\infty$, the continuity of $e_a$ follows from Proposition~\ref{seminorm continuous on a seminormed space iff}. Thus the seminorm defined by $p_a=|e_a|$ is also Continuous.
\end{itemize}
\end{example}
We noted that if $p$ is a seminorm then $B_p$ and $\widebar{B}_p$ are absorbent disks. There is a closer connection between absorbent disks and seminorms. If $D$ is an absorbent disked set, there is a seminorm $p$ such that $B_p\sub V\sub\widebar{B}_p$, as we show in the next two results.
\begin{theorem}\label{Minkowski functional}
If $D$ is an absorbent disk in the linear space $X$ then the \textbf{Minkowski functional} (or \textbf{gauge}) of $D$, defined at each $x$ in $X$ by
\[p_D(x)=\inf\{r>0:x\in rD\}\]
is a seminorm on $X$.
\end{theorem}
\begin{proof}
Let $D$ be an absorbent disk. It is clear that $p_D(0)=0$. Since $D$ is absorbent, $\{r>0:x\in rD\}$ is nonempty. Since $D$ is convex then $aD+bD=(a+b)D$ for any positive $a$ and $b$ (Proposition~\ref{TVS convex set prop}). Thus, if $x\in aD$ and $y\in bD$ then $x+y\in (a+b)D$. Consequently, $p_D(x+y)\leq a+b$. Since $a$ and $b$ are arbitrary, it follows that $p_D(x+y)\leq p_D(x)+p_D(y)$.\par
As to the absolute homogeneity of $p_D$, consider a nonzero scalar $b$. For $a>0$, $bx\in aD$ if and only if $x\in (a/b)D$. Since $|a/b|=|a|/|b|$ and $D$ is balanced then, by Proposition~\ref{TVS balanced set prop}, $(a/b)D=(a/|b|)D$. Hence
\[p_D(bx)=\inf\{a>0:bx\in aD\}=\inf\{a>0:x\in\frac{a}{|b|}D\}=|b|\inf\{\frac{a}{|b|}:x\in\frac{a}{|b|}D\}=|b|p_D(x).\]
Thus the claim follows.
\end{proof}
\begin{proposition}\label{TVS absorbent disk inclusion}
If $D$ is an absorbent disk in the topological vector space $X$ then we have
\[\Int D\sub B_{p_D}\sub D\sub\widebar{B}_{p_D}\sub\widebar{D}.\]
\end{proposition}
\begin{proof}
Let $D$ be an absorbent disk in the $X$. If $x\in\Int D$ then $D$ is a neighborhood of $x$. Since $(1+1/n)x\to x$, $(1+1/n)x\in\Int D$ eventually; since $(1+1/n)^{-1}<1$, this implies that $p_D(x)<1$ and $\Int D\sub B_{p_D}$. For $x\in B_{p_D}$, there exists $a\in[0,1)$ such that $x\in aD$. Since $D$ is balanced, $aD\sub D$ and therefore $B_{p_D}\sub D$. Clearly, $x\in D$ implies that $p_D(x)\leq 1$, so $D\sub\widebar{B}_{p_D}$.\par
To prove that $\widebar{B}_{p_D}\sub\widebar{D}$, first consider $x\in X$ such that $p_D(x)<1$. There must exist $a\in[0,1)$ such that $x\in aD\sub D\sub\widebar{D}$. If $p_D(x)=1$ then for any $0<a<1$, $ax\in B_{p_D}\sub\widebar{D}$. By the continuity of scalar multiplication, for any neighborhood $V$ of $x$, there is some open ball $B_r(1)$, $0<r<1$, such that $B_r(1)x\sub V$. It follows that $V$ contains points of $D$. Hence $x\in\widebar{D}$.
\end{proof}
The criteria for continuity of seminorms of Proposition~\ref{seminorm continuous iff} apply, of course, to gauges. For gauges, we have:
\begin{proposition}\label{TVS absorbent disk guage prop}
Let $p_D$ denote the gauge of an absorbent disk $D$ of the topological vector space $X$. Then:
\begin{itemize}
\item[(a)] If $D$ is open then $D=B_{p_D}$.
\item[(b)] $p_D$ is continuous if and only if $D$ is a neighborhood of $0$.
\item[(c)] if $p_D$ is continuous then $B_{p_D}=\Int D$ and $\widebar{B}_{p_D}=\widebar{D}$.
\end{itemize}
\end{proposition}
\begin{proof}
If $D$ is open then $D=\Int D$ and the fact that $D=B_{p_D}$ follows immediately from Proposition~\ref{TVS absorbent disk inclusion}. If $p_D$ is continuous, it is clear that $B_{p_D}$ is an open neighborhood of $0$. Conversely, suppose that $D$ is a neighborhood of $0$. Since $D\sub\widebar{B}_{p_D}$, $\widebar{B}_{p_D}$ is a neighborhood of $0$ and the continuity of $p$ follows from Proposition~\ref{seminorm continuous iff}(c).\par
If $p_D$ is continuous then $B_{p_D}$ is open and $\widebar{B}_{p_D}$ is closed. It follows from Proposition~\ref{TVS absorbent disk inclusion} that $B_{p_D}=\Int D$ and $\widebar{D}=\widebar{B}_{p_D}$.
\end{proof}
\section{Seminorm topologies}
Let $X$ be a vector space. Let $p$ be a seminorm on $X$. Then each $rB_p$, $r>0$, is an absorbent disk. Thus, condition (C1) of Proposition~\ref{LCS generating topology} is satisfied. Clearly $\mathcal{B}_p=\{rB_p:r>0\}$ also satisfies condition (C2) of Proposition~\ref{LCS generating topology}. Hence, seminormed spaces are locally convex spaces.\par
More generally, let $\mathscr{P}$ be a family of seminorms on $X$. Since $\mathcal{S}=\{B_p:r>0,p\in\mathscr{P}\}$ consists of absorbent disks, the collection of positive multiples of finite intersections of sets from $\mathcal{S}$ is a base at $0$ for a locally convex topology $\mathcal{T}$ for $X$ (Proposition~\ref{TVS subbase generate topo}). It is called the topology determined or generated by $\mathscr{P}$; it is clearly the weakest topology with respect to which each $p\in\mathscr{P}$ is continuous.
\begin{proposition}\label{seminorm topo Hausdorff iff}
Let $\mathscr{P}$ be a family of seminorms on the vector space $X$. Let $\mathfrak{U}_{\mathcal{P}}(0)$ denote the filter of neighborhoods of $0$ for the seminorm topology $\mathcal{T}_{\mathscr{P}}$. Then:
\begin{itemize}
\item[(a)] $\mathcal{T}_{\mathscr{P}}$ is Hausdorff if and only if for each nonzero $x\in X$ there is a $p\in\mathscr{P}$ such that $p(x)\neq 0$.
\item[(b)] An open base for $\mathfrak{U}_{\mathscr{P}}(0)$ is given by positive multiples of finite intersections of open balls $B_p$, $p\in\mathscr{P}$.
\end{itemize}
\end{proposition}
\begin{proof}
If $\mathcal{T}_{\mathscr{P}}$ is Hausdorff then for any vector $x\neq 0$, there is a basic neighborhood $r\bigcap_{i=1}^{n}B_{p_i}$ of $0$ to which $x$ does not belong. Therefore, $p_i(x)\neq 0$ for some $1\leq i\leq n$. Conversely, if $p(x)\neq 0$ for some $p$ in $\mathscr{P}$ then $x\notin p(x)B_{p}$, so $\mathcal{T}_{\mathscr{P}}$ is Hausdorff if (a) is satisfied.\par
Suppose that $p_1,\dots,p_n\in\mathscr{P}$ and $x\in r\bigcap_{i=1}^{n}B_{p_i}$ with $r>0$. For $a>0$ such that $a+\max_ip_i(x)<r$, we have $x+a\bigcap_{i=1}^{n}B_{p_i}\sub r\bigcap_{i=1}^{n}B_{p_i}$. Hence each $r\bigcap_{i=1}^{n}B_{p_i}$ is open.
\end{proof}
Having seen that families of seminorrns generate locally convex topologies, we show next that this is the only way---that any locally convex topology is determined by a family of seminorms.
\begin{proposition}\label{LCS iff generated by seminorms}
A topological vector space $X$ is locally convex if and only if its topology is generated by a family of seminorms. In particular, the topology is determined by the gauges $p_U$ of all open disks $U$ in $X$.
\end{proposition}
\begin{proof}
Let $X$ be a locally convex space, let $\mathfrak{U}(0)$ denote the neighborhood filter at $0$ in $X$, and let $\mathscr{P}$ denote the collection of all continuous seminorms on $X$. Certainly, $\mathscr{P}$ is not empty, for there must be disked neighborhoods of $0$ and the gauges of sueh sets populate $\mathscr{P}$. Let $\mathfrak{U}_{\mathscr{P}}(0)$ denote the neighborhood filter at $0$ in the seminorm topology determined by $\mathscr{P}$. Since eaeh $p\in\mathscr{P}$ is continuous, $\mathfrak{U}_{\mathscr{P}}(0)\sub\mathfrak{U}(0)$. Conversely, if $V\in\mathfrak{U}(0)$, there must be some open disk $U$ such that $U\sub V$. With $p_U$ denoting the gauge of $U$, we have $B_{p_U}\sub U\sub V$, since $\B_{p_U}\in\mathfrak{U}_{\mathscr{P}}(0)$, $V\in\mathfrak{U}_{\mathscr{P}}(0)$ and $\mathfrak{U}_{\mathscr{P}}(0)=\mathfrak{U}(0)$.
\end{proof}
\begin{definition}
Let $\mathscr{P}$ be the class of all continuous seminorms on the topological vector space $X$. A subset $\mathscr{Q}$ of $\mathscr{P}$ is said to be a \textbf{base of continuous seminorms} if, for any $p\in\mathscr{P}$, there is a $q\in\mathscr{Q}$ and a constant $r>0$ such that $p\leq rq$.
\end{definition}
If $\mathscr{Q}$ is a base of continuous seminorms then sets of the form $aB_q$ where $a>0$ and $q\in\mathscr{Q}$ are a neighborhood base at $0$. It follows from Proposition~\ref{seminorm continuous on a seminormed space iff} that the singleton $\{p\}$ is a base for the continuous seminorms on a seminormed space $(X,p)$. If $X$ carries a topology $\mathcal{T}_{\mathscr{P}}$ generated by a family $\mathscr{P}$ of seminorms and $\mathscr{Q}$ is a base of continuous seminorms then the topology generated by $\mathscr{Q}$ is a base for $\mathcal{T}_{\mathscr{P}}$, as follows from the fact that $p\leq rq$ iff $(1/r)B_q=B_{rq}\sub B_p$.\par
If $p_1,\dots,p_n$ are seminorms then so is $\max_ia_ip_i$ and $\sum_ia_ip_i$ for nonnegative scalars $a_i$. This motivates us to consider the following notion.
\begin{definition}
A family $\mathscr{P}$ of seminorms is \textbf{saturated} if $\max_ip_i\in\mathscr{P}$ for any $p_1,\dots,p_n\in\mathscr{P}$.
\end{definition}
For saturated families $\mathscr{P}$ of seminorms, a typical basic neighborhood of $0$ is of the form $rB_p$ where $r>0$ and $p\in\mathscr{P}$---no intersections needed.
\begin{example}[\textbf{Weak Topology and Weak$^*$ Topology}]\label{weak topo wea-star topo def}
Let $X^*$ denote the \textbf{continuous dual} of $X$, the linear space of all continuous linear functionals on the topological vector space $X$.
\begin{itemize}
\item[(a)] Consider the family $\mathscr{P}$ of seminorms $p_f(x)=|f(x)|$, $f\in X^*$. The topology $\sigma(X,X^*)$ generated by $\mathscr{P}$ on $X$ is called the \textbf{weak topology}. This locally convex topology is clearly weaker than the original topology on $X$.
\item[(b)] The topology $\sigma(X^*,X)$ generated by the seminorms $p_x(f)=|f(x)|$ is called the \textbf{weak$^*$ topology}.
\end{itemize}
\end{example}
To conclude this part, we address the determing seminorms for product and quotient topologies.
\begin{proposition}\label{LCS initial topo seminorm}
Let $X$ be a vector space, $\{X_s:s\in S\}$ be a family of locally convex spaces with a linear map $A_s:X\to X_s$ for each $s\in S$. Let $\mathcal{T}$ denote the initial topology determined by $\{A_s:s\in S\}$. For each $s$, let $\mathscr{P}_s$ be a family of seminorms that generates the topology on $X_s$. Then $\mathcal{T}$ is determined by the seminorms $\mathscr{P}=\{A_s\circ p:p\in\mathscr{P}_s,s\in S\}$.
\end{proposition}
\begin{proof}
For each index $s$, we may take $\mathcal{B}_s$ to be the filterbase $\{B_p:p\in\mathscr{P}_s\}$ as a base at $0$ in $X_s$. The subbasic neighborhoods of the initial topology are $U_s=A_s^{-1}(B_p)=B_{p\circ A_s}$. Thus the result follows.
\end{proof}
\begin{example}\label{TVS box product}
Let $\{X_t:t\in T\}$ be an infinite family of nontrivial Hausdorff topological vector spaces and let $X$ be the linear space $\prod_tX_t$. The box topology $\mathcal{T}$ for $X$ is that topology which has as a base at $0$ sets of the form $\prod_tU_t$ where each $U_t$ is a neighborhood of $0$ in $X_t$. This topology is easily seen to be compactible with the additive group structure of $X$, by using Proposition~\ref{topological group nbhd filter iff}. However, $\mathcal{T}$ is not a vector topology on $X$: Let $x\in X$ be such that $x(t)\neq 0$ for all $t$. Then since each $X_t$ is Hausdorff, we can choose a balanced neighborhood $V_t$ of $0$ for each $t$ such that $x(t)\notin V_t$. Let $H=\{t_n\}$ be a denumerable subset of $T$ and define
\[V=\prod_{n\in\N}(1/n)V_{t_n}\times\prod_{t\notin H}V_t.\]
We claim that there does not exists $\delta>0$ such that $B_\delta(0)x\sub x+V$, for this implies $(\delta n-1)x(t_n)\in V_{t_n}$ for all $n$, and hence $x(t_n)\in V_{t_n}$ for $n$ sufficiently large since $V_{t_n}$ is balanced, which is a contradiction. This means the map $\K\to X,a\to ax$ is not continuous, thus $\mathcal{T}$ is not a vector topology.\par
However, thing becomes different if we consider the subspace $E=\bigoplus_tX_t$. Let $\mathcal{T}'$ denote the subspace topology induced by $\mathcal{T}$. Since each element $x\in E$ has only finitely many nonzero components, it is easy to see a product of absorbent neighborhods absorbs elements in $E$ (This does not hold in $X$). Thus the filterbase $\mathcal{B}$ consists of $E\cap\prod_tV_t$ where $V_t$ is an absorbent and balanced neighborhood of $0$ in $X_t$ satisfies Theorem~\ref{TVS generating topology}, so $\mathcal{T}'$ is a vector topology. If each $X_t$ is locally convex then so is $E$. Moreover, $E$ is a closed subspace of $X$: if $(x_\alpha)$ is a net in $E$ converging to $x\in X$ and $x(t)\neq 0$ for $t$ in an infinite subset of $T$. Then for each $t\in H$, we can find neighborhoods $V_t$ of $x(t)$ such that $0\notin V_t$. Now consider the neighborhod
\[V=\prod_{t\in H}V_t\times\prod_{t\notin H}X_t.\]
For any $x_\alpha$, since $x_\alpha$ is almostly zero, $x_\alpha\notin V$. Thus $x_\alpha$ does not converges to $x$, which is a contradiction.\par
Now assume that each $X_t$ is complete. Let $(x_\alpha)$ be a Cauchy net in $(X,\mathcal{T})$. Then for each $t\in T$, $(x_\alpha(t))$ is Cauchy in $X_t$, so it converges to $x$. We now prove that $(x_\alpha)\to x$ in $\mathcal{T}$. Let $V=\prod_tV_t$ a neighbourhood of $0$. For each $t$, choose a closed neighbourhood $U_t$ of $0$ in $X_t$ such that $\widebar{U}_t\sub V_t$. Let $U=\prod_tU_t$. Since $(x_\alpha)$ is a Cauchy net, there is an index $\gamma$ such that for $\alpha,\beta\succeq\gamma$ we have $x_\alpha-x_\beta\in U$, so that $x_\alpha(t)-x_\beta(t)\in U_t$ for all $t$. Since $x_\beta(t)\to x(t)$, it follows that $x_\alpha(t)-x(t)\in\widebar{U}_t\sub V_t$ for all $t$ and all $\alpha\succeq\gamma$. But that means $x_\alpha-x\in V$ for all $\alpha\succeq\gamma$, so $x_\alpha\to x$. Thus $(X,\mathcal{T})$ is complete, and so is $(E,\mathcal{T}')$.
\end{example}
\begin{proposition}\label{LCS quotient generating seminorm}
Let $\mathcal{D}$ denote the family of open disks in the locally convex space $X$. Let $M$ be a subspace of $X$ and let $\pi:X\to X/M$ denote the quotient map. The open disks of $X/M$ are $\pi(\mathcal{D})$. For $D\in\mathcal{D}$, let $\widebar{D}=\pi(D)$ with gauge $p_{\widebar{D}}$. The quotient topology is generated by $\{p_{\widebar{D}}:D\in\mathcal{D}\}$ and for $D\in\mathcal{D}$,
\begin{align}\label{LCS quotient generating seminorm-1}
p_{\widebar{D}}(\bar{x})=\inf_{m\in M}p_D(x+m).
\end{align}
\end{proposition}
\begin{proof}
By Proposition~\ref{TVS quotient topo prop}(a), for any open disk $D$ of $X$, $\pi(D)=\widebar{D}$ is an open disk in $X/M$. Conversely, if $W$ is an open disk in $X/M$, then $\pi^{-1}(W)=D$ is an open disk in $X$ and $W=\pi(D)=\widebar{D}$. Consequently the gauges $\{p_{\widebar{D}}:D\in\mathcal{D}\}$ generate the quotient topology of $X/M$. For any $x\in X$ and any open disk $D$ of $X$,
\[p_{\widebar{D}}(\bar{x})=\inf\{r>0:\bar{x}\in r\widebar{D}\}\]
For $r>0$ and $x\in X$, $\bar{x}\in r\widebar{D}$ iff there is some $m\in M$ such that $x+m\in rD$. Hence, $r\geq p_D(x+m)\geq\inf_{m\in M}p_D(x+m)$. Since $r$ is arbitrary, it follows that $p_{\widebar{D}}(\bar{x})\geq\inf_{m\in M}p_D(x+M)$. To reverse the inequality, let $m\in M$ and suppose that $r>0$ is such that $x+m\in rD$. It follows that $r\geq p_{\widebar{D}}(\bar{x})$ and, since $r$ is arbitrary, that $p_D(x+m)\geq p_{\widebar{D}}(\bar{X})$. Formula $(\ref{LCS quotient generating seminorm-1})$ now follows from the arbitrariness of $m$.
\end{proof}
\begin{proposition}
If $\mathscr{P}$ is a base of continuous seminorms for the locally convex space $X$ and $M$ is a subspace of $X$ then a base $\widebar{\mathscr{P}}$ of continuous seminorms for $X/M$ is given by seminorms of the form $\bar{p}$ where $p\in\mathscr{P}$ and $\bar{p}(\bar{x})=\inf_{m\in M}p(x+m)$.
\end{proposition}
It follows that if $M$ is a subspace of a seminormed space $(X,p)$ then the quotient topology on $X/M$ is determined by $\bar{p}$. In particular, if $M$ is a closed subspace of a normed space $X$, we always norm $X/M$ by taking $\|x+M\|:=\inf_{m\in M}\|x+m\|$.\par
If $p$ is a seminorm on a vector space $X$ then clearly the null space (or
kernel) $N_p=p^{-1}(0)$ of $p$ is a subspace of $X$. Moreover, for any $y\in N_p$, $p(x+y)\leq p(x)+p(y)=p(x)$, so $p(x+N_p)=p(x)$ for any $x\in X$. This implies that $p$ is actually a norm on $X/N_p$: $p(x+N_p)=0$ if and only if $p(x)=0$ if and only if $x\in N_p$. We call $p$ the \textbf{factor norm} on $X/N_p$.
\section{Metrizability and completion}
Suppose that $X$ is a locally convex space whose topology is determined by a countable family $\{p_n\}$ of seminorms. There is no loss of generality in assuming that the $p_n$ are increasing for we may replace each $p_n$ by $q_n=\max\{p_i:1\leq i\leq n\}$ and still get the same the topology. Since $\{p_n\}$ is increasing, the balls $B_{p_n}$ are decreasing. Moreover, as $\{(1/j)B_{p_n}:j,n\in\N\}$ is a countable base at $0$, $X$ is pseudometrizable by Proposition~\ref{TVS pseudometrizable}. Now we exhibit an $F$-seminorm that generates the topology.
\begin{theorem}\label{LCS metrizable iff}
A locally convex space $X$ is pseudometrizable if and only if its topology is generated by an increasing sequence $\{p_n\}$ of continuous seminorms. In this case the topology is generated by the $F$-seminorm $p$ defined by 
\[p(x)=\sum_{n=1}^{\infty}2^{-n}\frac{p_n(x)}{1+p_n(x)}.\]
If $X$ is metrizable, then $p$ is an $F$-norm.
\end{theorem}
\begin{proof}
The assertion about pseudometrizability of a locally convex space $X$ being implied by the presence of an increasing sequence $\{p_n\}$ of seminorms follows from the discussion above. Conversely, if $X$ is pseudometrizable, it has a countable base $\{V_n:n\in\N\}$ of neighborhoods of $0$ which we may assume to be decreasing. By the local convexity, each $V_n$ contains a disked neighborhood $U_n$ of $0$ and we can assume that the $U_n$ are decreasing as well. Consequently the gauges $p_n$ of the $U_n$ are an increasing family of seminorms which generate the topology. We show now that the $p$ of the statement is an $F$-seminorm. The series by which $p$ is defined is seen to converge by comparison with $\sum_n2^{-n}$ and obviously $p\geq 0$. The function $h(t)=t/(1+t)$ is increasing for $t\geq 0$ and, for $|t|\leq 1$ and $x\in X$, $p_n(tx)=|t|p_n(x)\leq p_n(x)$. Therefore, for each $n\in\N$, $p_n(tx)/(1+p_n(tx))\leq p_n(x)/(1+p_n(x))$; it follows that $p(tx)\leq p(x)$ for $|t|\leq 1$. Note that, for every $n\in\N$ and $x,y\in X$,
\[\frac{p_n(x+y)}{1+p_n(x+y)}\leq\frac{p_n(x)+p_n(y)}{1+p_n(x)+p_n(y)}\leq\frac{p_n(x)}{1+p_n(x)}+\frac{p_n(y)}{1+p_n(y)}.\]
Thus the triangle inequality follows.\par
To see that $p(x/n)\to 0$, let $x\in X$ and $r>0$ be given. Choose $k\in\N$ such that $\sum_{n>k}2^{-n}<r/2$. By the continuity of each $p_n$ and $h$ at $0$, there exists $a>0$ such that
\[\frac{p_n(x/a)}{2^n(1+p_n(x/a))}<\frac{r}{2k}\for n=1,\dots,k.\]
It follows that $p(x/a)<r$ and that $p$ is an $F$-seminorm.\par
To see that the topologies determined by $p$ and $\{p_n:n\in\N\}$ coincide, let $r>0$ be given and choose $k$ such that $\sum_{n>k}2^{-n}<r/2$. Since $s/(1+s)\to 0$ as $s\to 0$, we may choose $t>0$ such that $tr/(1+tr)<r/2$. For any $x\in X$ such that $p_k(x)<rt$, since the $p_n$ are increasing,
\[\sum_{n=1}^{k}2^{-n}\frac{p_n(x)}{1+p_n(x)}<\sum_{n=1}^{k}2^{-n}\frac{rt}{1+rt}<\frac{r}{2}.\]
Since $\sum_{n>k}2^{-n}<r/2$ and $p_n(x)/[1+p_n(x)]<1$ for all $n$, it follows that $p(x)<r$; therefore $rtB_{p_k}\sub B_p$ and the topology determined by $p$ is seen to be weaker than that determined by $p_k$, hence weaker than that determined by $\{p_n:n\in\N\}$.\par
Conversely, consider the basic neighborhood $rB_{p_n}$ of $0$ in the original topology. Consider $t>0$ such that $t<r/[2^n(1+r)]$. For $x\in X$ such that $p(x)<t$ then, by the way $p$ is defined,
\[\frac{p_n(x)}{2^n[1+p_n(x)]}\leq p(x)<t<\frac{r}{2^n(1+r)}.\]
Since $s/(1+s)$ is increasing, this implies that $p_n(x)<r$ and therefore that $tB_{p}\sub rB_{p}$. Hence the topology determined by $\{p_n:n\in\N\}$ is coarser than that determined by $p$ and the two are seen to be equal.
\end{proof}
To give an example of a locally convex space that is not pseudometrizable, we need the following lemma.
\begin{lemma}\label{TVS metrizable infinite dim noncontinuous dual}
Let $X$ be a infinite-dimensional pseudometrizable topological vector space. Then $X^*$ is a proper subspace of $X^\star$ (the algebraic dual of $X$).
\end{lemma}
\begin{proof}
If $\dim X<+\infty$ then by Proposition~\ref{TVS finite dim prop} any element in $X^{\star}$ is continuous. If $\dim X=+\infty$, let $\{x_n\}$ be a denumerable linearly independent subset of $X$ and let $\{U_n\}$ be a decreasing sequence of neighborhoods of $0$ which are a base at $0$. Choose positive numbers $t_n$ for each $n\in\N$ such that $t_nx_n\in U_n$. Clearly, $t_nx_n\to 0$. Extend $\{x_n\}$ to a Hamel base $B$. A discontinuous linear functional $f$ is defined on $X$ by taking $f(x_n)=1/t_n$ for each $n\in\N$ and $0$ on the elements of the set difference $B\setminus\{x_n\}$.
\end{proof}
\begin{corollary}\label{NVS finite dim iff X^star=X^*}
A normed linear space $X$ is finite-dimensional iff every linear functional on $X$ is continuous.
\end{corollary}
\begin{proof}
If the normed space $X$ is finite-dimensional, the continuity follows from Lemma~\ref{TVS finite dim prop}. By Proposition~\ref{TVS metrizable infinite dim noncontinuous dual}, if all linear functionals are continuous on $X$, then $X$ must be finite-dimensional.
\end{proof}
\begin{example}[\textbf{Finest Locally Convex Topology}]\label{LCS finest LC topo nonmetrizable}
Let $X$ be a vector space. If $\mathscr{P}$ is the family of all seminorms on $X$, the topology $\mathcal{T}_{lc}$ determined by $\mathscr{P}$ is called the \textbf{finest locally convex topology}. Because of the correspondence between gauges of absorbent disks and seminorms, This may also be described as the topology having the family of all absorbent disks as a base at $0$. If $D$ is the unit disk in the scalar field $\K$ and $f$ is any linear functional on $X$ then $f^{-1}(D)$ is an absorbent disk in $X$, hence a neighborhood of $0$. Therefore every linear functional on $X$ is continuous. Since there are no discontinuous linear functionals on $(X,\mathcal{T}_{lc})$, $\mathcal{T}_{lc}$ is not pesudometrizable on any infinite-dimensional space by Lemma~\ref{TVS metrizable infinite dim noncontinuous dual}.
\end{example}
\section{Continuity of linear maps}
From Proposition~\ref{topological group homomorphism continuous iff at one point} and \ref{topological group homomorphism is uniformly continuous} on topological groups, it is clear that:
\begin{proposition}
If $X$ and $Y$ are topological vector spaces over the same topological field and $A:X\to Y$ is a linear map, then
\begin{itemize}
\item[(a)] $A$ is continuous iff it is continuous at one point of $X$.
\item[(b)] If $A$ is continuous then it is uniformly continuous.
\end{itemize}
\end{proposition}
We use the following convergence result for nets in locally convex spaces to characterize continuity of certain linear maps.
\begin{proposition}\label{LCS Cauchy net under seminorm}
Let $\mathscr{P}$ be a base of continuous seminorms in the locally convex space $X$. For each net $(x_\alpha)$ of points of $X$:
\begin{itemize}
\item[(a)] $x_\alpha\to x$ if and only if $p(x_\alpha-x)\to 0$ for each $p\in\mathscr{P}$.
\item[(b)] If $(x_\alpha)$ is Cauchy then so is $(p(x_\alpha))$ for each $p\in\mathscr{P}$.
\end{itemize}
\end{proposition}
\begin{proof}
Clearly, if $x_\alpha\to x$ then $x_\alpha-x\to 0$ and $p(x_\alpha-x)\to 0$ for any continuous $p\in\mathscr{P}$. Conversely, suppose that $p(x_s-x)\to 0$ for each $p$ in $\mathscr{P}$. Since $X$ is locally convex, it suffices to show that $x_\alpha-x$ eventually belongs to any open disk $D$. The gauge $p_D$ of $D$ is a continuous seminorm by Proposition~\ref{seminorm continuous iff}. Since $\mathscr{P}$ is a base, for some $r>0$ and $p\in\mathscr{P}$, $p_D\leq rp$. By hypothesis, $rp(x_\alpha-x)<1$ eventually, so $x_\alpha-x\in B_{p_D}$ eventually. Since $D$ is open, $B_{p_D}=D$ by Proposition~\ref{TVS absorbent disk guage prop}(a).\par
Any continuous seminorm is uniformly continuous by Proposition~\ref{seminorm continuous iff}(a) and the uniformly continuous image of a Cauchy net is Cauchy, so if $(x_\alpha)$ is Cauchy, so is $(p(x_\alpha))$ for any $p\in\mathscr{P}$.
\end{proof}
\begin{proposition}\label{LCS as image continuity iff}
Let $X$ ba a topological vector space and $Y$ be a locally convex space. Then a linear map $A:X\to Y$ is continuous if and only if for each continuous seminorm $q$ on $Y$, there is a continuous seminorm $p$ on $X$ such that $q\circ A\leq p$.
\end{proposition}
\begin{proof}
If $A$ is continuous and $q$ is a continuous seminorm on $Y$, then $q\circ A$ is a continuous seminorm on $X$ which satisfies $q\circ A\leq q\circ A$. If $q$ is a seminorm on $Y$, $q\circ A$ is a seminorm on $X$. If the condition holds then, for each continuous seminorm $q$ on $Y$, there is a continuous seminorm $p$ on $X$ which dominates $q\circ A$. Therefore, $q\circ A$ is continuous by Proposition~\ref{seminorm continuous iff}. To prove that $A$ is continuous suppose that $x_\alpha\to 0$ in $X$. If so, then $p(x_\alpha)\to 0$ which implies that $q(Ax_\alpha)\to 0$. Since $q$ is arbitrary, $Ax_\alpha\to 0$ by Proposition~\ref{LCS Cauchy net under seminorm}(a).
\end{proof}
\begin{corollary}\label{TVS linear functional iff seminorm}
For a topological vector space $X$, a linear functional $f$ is continuous if and only if there is a continuous seminorm $p$ on $X$ such that $|f|\leq p$.
\end{corollary}
\begin{proposition}\label{LCS by seminorm continuous iff}
Suppose $X$ and $Y$ are locally convex spaces with topologies defined, respectively, by the families $\mathscr{P}$ and $\mathscr{Q}$ of seminorms, and $A:X\to Y$ is a linear map. Then $A$ is continuous if and only if for each $q\in\mathscr{Q}$ there exist $p_1,\dots,p_n\in\mathscr{P}$ and $C>0$ such that $q(Ax)\leq C\sum_{i=1}^{n}p_i(x)$ for all $x\in X$.
\end{proposition}
\begin{proof}
If the latter condition holds and $(x_i)$ is a net converging to $x\in X$, by Proposition~\ref{LCS Cauchy net under seminorm} we have $p(x_i-x)\to 0$ for all $p\in\mathscr{P}$, hence $q(Ax_i-Ax)\to 0$ for all $q\in\mathscr{Q}$, hence $Ax_i\to Ax$. By Proposition~\ref{continuous iff net}, $A$ is continuous. Conversely, if $A$ is continuous, for every $q\in\mathscr{Q}$ there is a neighborhood $U$ of $0$ in $X$ such that $q(Ax)<1$ for $x\in U$. We may assume that $U=\bigcap_{i=1}^{n}\eps_i(x+B_{p_i})$. Let $\eps=\min\{\eps_1,\dots,\eps_n\}$; then $q(Ax)<1$ whenever $p_i(x)<\eps$ for all $i$. Now, given $x\in X$, there are two possibilities. If $p_i(x)>0$ for some $i$, let $y=\eps x/\sum_{i=1}^{n}p_i(x)$. Then $p_i(y)<\eps$ for all $i$, so
\[q(Ax)=\sum_{i=1}^{n}\eps^{-1}p_i(x)q(Ay)\leq\eps^{-1}\sum_{i=1}^{n}p_i(x).\]
On the other hand, if $p_i(x)=0$ for all $i$, then $p_i(rx)=0$ for all $i$ and all $r>0$, hence $rq(Ax)=q((A(rx))<1$ for all $r>0$, hence $q(Ax)=0$. Ahus $q(Ax)<\eps^{-1}\sum_{i=1}^{n}p_i(x)$ in this case too, and we are done.
\end{proof}
We now generalize these connections between continuity of linear maps and domination by seminorms. We say that a subset $B$ of a seminormed space $(X,p)$ is \textbf{bounded} if $p(B)$ is a bounded set of scalars. If $(X,p)$ and $(Y,q)$ are seminormed spaces and $f:X\to Y$ maps bounded sets into bounded sets, we say that $f$ is \textbf{bounded}. We deviate from this convention for linear maps on seminormed spaces for historical reasons and often say "bounded" linear map instead of bounded linear map.
\begin{theorem}\label{seminormed space continuous iff}
If $(X,p)$ and $(Y,q)$ are seminormed spaces and $A:X\to Y$ is linear then the following statements are equivalent:
\begin{itemize}
\item[(\rmnum{1})] $A$ is continuous.
\item[(\rmnum{2})] $A$ is bounded on $\widebar{B}_p$.
\item[(\rmnum{3})] $A$ is bounded.
\end{itemize}
\end{theorem}
\begin{proof}
Since $q$ is a seminorm on $Y$, $q\circ A$ is a seminorm on $X$. Since $q$ is continuous, $q\circ A$ is continuous iff $A$ is. By Proposition~\ref{seminorm continuous on a seminormed space iff}, $q\circ A$ is continuous iff there is some positive constant $r$ such that $q\circ A\leq rp$, i.e., continuity of $A$ is equivalent to the statement that for any $x\in\widebar{B}_p$, $(q\circ A)(x)\leq r$.\par
Since $\widebar{B}_p$ is bounded, (\rmnum{3}) implies (\rmnum{2}). To show that (\rmnum{2}) implies (\rmnum{3}), let $r>0$ be such that $(q\circ A)(\widebar{B}_p)\sub[0,r]$ and let $B$ be a bounded subset of $X$. As such, there is some positive $s$ such that $B\sub s\widebar{B}_p$. Hence $A(B)\sub sA(\widebar{B}_p)$ and $(q\circ A)(B)\sub[0,sr]$.
\end{proof}
For linear functionals there is a useful characterization for continuity.
\begin{lemma}\label{TVS linear functional balanced nbhd lemma}
Let $f$ be a nontrivial linear functional on a vector space $X$. If $f(x)=1$ and $U\sub X$ is balanced then $(x+U)\cap N(f)=\emp$ iff $|f(u)|<1$ for all $u$ in $U$.
\end{lemma}
\begin{proof}
Suppose $f(x)=1$. Clearly, if $|f(u)|<1$ for each $u$ in $U$ then for any $u\in U$, $f(x+u)=f(x)+f(u)=1+f(u)\neq 0$; hence $(x+U)\cap N(f)=0$. Conversely, if $|f(u)|\geq 1$ for some $u\in U$ then $-u/f(u)\in U$ and $x-u/f(u)\in(x+U)\cap N(f)$.
\end{proof}
\begin{proposition}\label{TVS linear functional continuous iff kernel closed}
Let $f$ be a linear functional on a topological vector space $X$. Then $f$ is continuous if and only if $N(f)$ is closed.
\end{proposition}
\begin{proof}
If $f$ is continuous then clearly $N(f)$ is closed. Suppose conversely that $N(f)$ is closed and $f(x_0)=1$. Since $N(f)$ is closed, there is a balanced neighborhood $U$ of $0$ such that $(x_0+U)\cap N(f)=\emp$. By Lemma~\ref{TVS linear functional balanced nbhd lemma}, therefore, $\{x\in X:|f(x)|<1\}$ contains $U$ and so is a neighborhood of $0$. Since $f$ is linear, for each $r>0$, $\{x\in X:|f(x)|<r\}=r\{x\in X:|f(x)|<1\}$ and it follows that $f$ is continuous at $0$.
\end{proof}
Now we consider completion of locally convex space.
\begin{proposition}\label{seminormed space completion}
A seminormed space $(X,p)$ has a completion $(\widehat{X},\hat{p})$ as a seminormed space. If $p$ is a norm then $(X,p)$ has a completion as a Banach space.
\end{proposition}
\begin{proof}
A seminorm $p$ on a vector space $X$ determines an invariant pseudometric $d$ when we take $d(x,y)=p(x-y)$. Consequently, a seminormed space $(X,p)$ possesses a completion $(\widehat{X},\hat{d})$ as a complete pseudometrizable group where $\hat{d}$ extends $d$. $\widehat{X}$ consists of Cauchy sequences $(x_n)$ from $X$ and is a topological group with respect to pointwise addition. With respect to scalar multiplication defined as $a(x_n)=(ax_n)$, $X$ becomes a topological vector space. We can extend $p$ to $\widehat{X}$ by taking $p((x_n))$ to be the limit of the Cauchy sequence $p(x_n)$ (Proposotion~\ref{LCS Cauchy net under seminorm}) and $p$ is a seminorm on $\widehat{X}$.
\end{proof}
We make use of some of these notions now in obtaining the characterization
of locally convex Hausdorff spaces.
\begin{proposition}\label{LCS subspace of product}
Let $X$ be a locally convex space whose topology is generated by a family $\mathscr{P}$ of seminorms.
\begin{itemize}
\item[(a)] $X$ is linearly homeomorphic to a subspace of a product of seminormed spaces.
\item[(b)] $X$ is Hausdorff if and only if $X$ is linearly homeomorphic to a subspace of a product of Banach spaces.
\end{itemize}
\end{proposition}
\begin{proof}
For each $p\in\mathscr{P}$, let $X_p$ denote the seminormed space $(X,p)$ and $A_p$ the continuous linear map $x\mapsto x$ from $X$ onto $X_p$. Now consider the injective map
\[A:X\to\prod_pX_p,\quad x\mapsto(A_px)_{p\in\mathscr{P}}.\]
By considering subbasic neighborhoods $rB_q\prod_{p\neq q}X_p$ of $0$ in the product, we see that the the initial topology induced by $A$ is just $X$'s original topology. Consequently, $A$ is a continuous open map of $X$ onto $R(A)$. That $A$ is a linear isomorphism is clear.\par
Suppose that $X$ is Hausdorff. For $p\in\mathscr{P}$, let $N_p$ denote the closed subspace $p^{-1}(0)$ and, for any $x\in X$, $\bar{p}(\bar{x})=\inf_{n\in N_p}p(x+n)$. Let $X_p$ denote the normed space $(X/N_p,\bar{p})$. For each $p\in P$, let $Y_p$ denote a completion of $X_p$ as a Banach space. Let $\pi_p$ denote the canonical map $x\mapsto\bar{x}$ from $X$ onto $X_p$ and let $B_p$ denote the open unit ball determined by $p$ and note that $B_p=\pi_p^{-1}(B_{\bar{p}})$. Since a base $\mathcal{B}$ at $0$ for the topology for $X$ is given by finite intersections of the unit balls $B_p$, $\mathcal{B}$ is a base at $0$ for the initial topology for $X$ determined by the maps $\pi_p$, or equivalently as the initial topology for $X$ determined by the map
\[A:X\to R(A)\sub\prod_pY_p,\quad x\mapsto(\pi_p(x)).\]
That $A$ is injective follows immediately from the fact that $X$ is Hausdorff. The openness of $A$ follows from the definition of initial topology and the fact that $A$ is onto $R(A)$.
\end{proof}
\begin{proposition}\label{LCS completion}
Every locally convex space $X$ possesses a completion which is a locally convex space.
\end{proposition}
\begin{proof}
Let $X$ be a locally convex space and let $\mathscr{P}$ be the collection of continuous seminorms on $X$. Each seminormed space $(X,p)$, $p\in\mathscr{P}$, has a completion $Y_p$ which is also a seminormed space by Proposition~\ref{seminormed space completion}. By Proposition~\ref{uniform space completion of initial topology}, the product $\prod_{p\in\mathscr{P}}Y_p$ is complete. It is locally convex. As in Proposition~\ref{LCS subspace of product}, the map $A:X\to Y_p,x\mapsto(x_p)$ where $x_p=x$ for each $p\in\mathscr{P}$ is a linear homeomorphism. The closure of $R(A)$ in $\prod_pY_p$ is therefore the desired completion of $X$.
\end{proof}
\section{Strict inductive limits}
Let $\{X_s:s\in S\}$ be a family of locally convex space and $X$ a vector space. Suppose that for each $s\in S$ that $A_s:X_s\to X$ is a surjective linear map and that the linear span of $\bigcup_{s\in S}A_s(X_s)$ is $X$. In this case, the finest locally convex topology for $X$ with respect to which each $A_s$ is continuous, the final locally convex topology, is called the inductive limit topology for $X$. We write $X=\rlim X_s$ and say that $X$ is the inductive limit of $\{X_s:s\in S\}$.
A base at $0$ for the inductive limit topology is given by the collection of disks $D$ in $X$ such that $A_s^{-1}(D)$ is a neighborhood of $0$ in $X_s$ for each $s\in S$. The requirement that $\bigcup_sA_s(X_s)$ span $X$ is not essential. Its loss entails taking absorbent disks $D$ in $X$ such that each $A_s^{-1}(D)$ is a neighborhood of $0$ in each $X_s$ as a base at $0$ for the inductive limit topology.
\begin{proposition}\label{LCS inductive limit topo continuous iff}
Let $X$ carry the inductive limit topology determined by the maps $A_s:X_s\to X$ and let $A$ be a linear map of $X$ into a locally convex space $Y$. Then $A$ is continuous iff for each $s\in S$, $A\circ A_s$ is continuous.
\end{proposition}
\begin{proof}
The map $A$ is continuous iff for each disked neighborhood $D$ of $0$ in $Y$, $A^{-1}(D)$ is a neighborhood of $0$ in $X$, iff $A_s^{-1}(A^{-1}(D))$ is a neighborhood of $0$ in $X_s$ for each $s\in S$, iff $A\circ A_s$ is continuous for each $s$.
\end{proof}
\begin{proposition}\label{LCS inductive limit is subspace of direct sum}
If $X=\rlim X_s$ determined by the maps $A_s:X_s\to X$ then $X$ is linearly homeomorphic to a quotient by a closed subspace of the locally convex direct sum $\bigoplus_sX_s$.
\end{proposition}
\begin{proof}
Let $X=\rlim X_s$ and consider the map
\[A:\bigoplus_sX_s\to X,\quad (x_s)\mapsto\sum_{s\in S}A_sx_s.\]
For each $s\in S$, let $I_s$ denote the canonical injection of $X_s$ into $\bigoplus_sX_s$ and note that $\bigoplus_sX_s$ carries the inductive limit topology determined by the canonical injection $\{I_s:s\in S\}$. Note also that $A\circ I_s=A_s$ for each $s\in S$. By Proposition~\ref{LCS inductive limit topo continuous iff}, the continuity of $A$ is equivalent to that of $A\circ I_s$ for each $s\in S$. Thus, since each $A_s$ is continuous, $A$ is continuous. Since $\bigcup_sA_s(X_s)$ spans $X$, $A$ is onto. Let $N$ denote the null space of $A$ and define $\bar{A}(x+N)=Ax$.
\[\begin{tikzcd}
X_s\ar[r,"A_s"]\ar[rd,swap,"I_s"]&X\ar[r,"\bar{A}^{-1}"]&(\bigoplus_sX_s)/N\\
&\bigoplus_sX_s\ar[ru,swap,"\pi"]\ar[u,"A"]&
\end{tikzcd}\]
To show that $X$ is linearly homeomorphic to $(\bigoplus_sX_s)/N$, by Proposition~\ref{TVS first isomorphism thm} it only remains to show that $\bar{A}^{-1}$ is continuous. Let $\pi$ denote the canonical map of $\bigoplus_sX_s$ onto $\bigoplus_sX_s/N$. Then $\bar{A}^{-1}\circ A_s=\pi\circ I_s$. Since $\pi\circ I_s$ is continuous for each $s\in S$, the continuity of $\bar{A}^{-1}$ follows by Proposition~\ref{LCS inductive limit topo continuous iff}.
\end{proof}
Now let $\{X_n\}$ is an increasing sequence of locally convex spaces such that each $X_n$ is a subspace of $X_{n+1}$ and $X=\bigcup_nX_n$ is a vector space. For each $n\in\N$, let $I_n:X_n\to X$ be the canonical inclusion of $X_n$ into $X$. The final locally convex topology $\mathcal{T}$ for $X$ induced by $\{I_n:n\in\N\}$ is the finest locally convex topology for $X$ with respect to which each $I_n$ is continuous; it has as a base at $0$ the class of all absorbent disks $D\sub X$ such that $D\cap X_n$ is a neighborhood of $0$ in $X_n$ for each $n\in\N$. In the situation under consideration, we can omit "absorbent"--each disk $D$ such that $D\cap X_n$ is a neighborhood of $0$ in $X_n$ for each $n\in\N$ is absorbent because any $x\in X$ must belong to some $X_n$ and since $D\cap X_n$ is a neighborhood of $0$ in $X_n$, there is some $t>0$ such that $x\in t(D\cap X_n)\sub tD$. The collection $\mathcal{D}$ of all disks $D$ in $X$ such that $D\cap X_n$ is a neighborhood of $0$ in $X_n$ for each $n\in\N$ forms a base at $0$ for the direct image locally convex topology $\mathcal{T}_i$ on $X$ which is called the \textbf{strict inductive limit topology} for $X$ in this context. We say that $(X,\mathcal{T}_i)$ is the strict inductive limit of $\{X_n\}$ and write $X=\rlim X_n$. The word "strict" is reserved for denumerable increasing collections $\{X_n\}$ of locally convex spaces.
\begin{proposition}\label{LCS inductive limit base at 0}
For $X=\rlim X_n$, the disked hulls $\convbal(\bigcup_nU_n)$ in $X$ of sets of the form $\bigcup_nU_n$ where each $U_n$ is a neighborhood of $0$ in $X_n$ form a base at $0$ for the strict inductive limit topology on $X$.
\end{proposition}
\begin{proof}
For each $n\in\N$, let $U_n$ be a neighborhood of $0$ in $X_n$. Clearly $D=\convbal(\bigcup_nU_n)$ is an absorbent disk in $X$ whose intersection with any particular $X_n$ contains $U_n$ so $D\in\mathcal{D}$. Conversely, for any $D\in\mathcal{D}$ and any $n\in\N$, $D\cap X_n\sub D$; hence $\bigcup_n(D\cap X_n)\sub D$. Since $D$ is a disk, $\convbal(\bigcup_n(D\cap X_n))\sub D$.
\end{proof}
\begin{proposition}\label{LCS disked nbhd in subspace}
Let $M$ be a subspace of the locally convex space $X$. If $U$ is a disked neighborhood of $0$ in $M$ then there is a disked neighborhood $V$ of $0$ in $X$ such that $U=V\cap M$. If $M$ is closed and $x\notin M$, then $V$ can be chosen such that $x\notin V$.
\end{proposition}
\begin{proof}
If $U$ is a disked neighborhood of $0$ in $M$, there is some neighborhood $W$ of $0$ in $X$ such that $U=W\cap M$ but $W$ need not be convex. As $X$ is locally convex, $W$ must contain some disked neighborhood $W_1$ of $0$; thus $W_1\cap M\sub U$. Let $V=\conv(W_1\cup U)$. With this convex binding, we show that $U=V\cap M$. Clearly, $U\sub V\cap M$. If, conversely, $z\in V\cap M$, since $W_1$ and $U$ are disks, there exist $w\in W_1$ and $u\in U$ such that $z=tw+(1-t)u$ for some $t\in[0,1]$. If $t=0$, then $z\in U$ and we are done. If $t>0$, then $w=(1/t)[z-(1-t)u]\in W_1\cap M\sub U$. Since $U$ is convex, $z\in U$.\par
Suppose that $M$ is closed and $x\notin U$. If $x\in M$ then $x\notin V$. If $x\notin M$, then there exists a disked neighborhood $N$ of $0$ in the locally convex space $X$ such that $(x+N)\cap M=0$. Let $W_2=W_1\cap N$ so $(x+W_2)\cap M=\emp$. Now let $V=\conv(W_2\cup U)$. Could $x\in V$? If so, then there exists $w\in W_2$, $u\in U$, and $t\in[0,1]$ such that $x=tw+(1-t)u$ which implies that $x-tw=(1-t)u\in(x+W_2)\cap M$ which is a contradiction. Finally, since $W_2\sub W_1$, $\conv(W_2\cup U)\cap M\sub U$. The reverse inclusion follows from the fact that $U\sub M$.
\end{proof}
Now we use Proposition~\ref{LCS disked nbhd in subspace} to prove some properties of the strict inductive limit topology.
\begin{proposition}\label{LCS str inductive limit topo}
If $X=\rlim X_n$, then
\begin{itemize}
\item[(a)] for each $n\in\N$, the subspace topology on $X_n$ coincides with the original topology on $X_n$;
\item[(b)] if each $X_n$ is Hausdorff, then so is $X$;
\item[(c)] if each $X_n$ is closed in $X_{n+1}$ then each $X_n$ is closed in $X$.
\end{itemize}
\end{proposition}
\begin{proof}
We show that any disked neighborhood $U_n$ of $0$ in $X_n$ may he written in the form $U\cap X_n$ for some disked neighborhood $U$ of $0$ in $X$. We argue by induction. For $n\in\N$, let $U_n$ be a disked neighborhood of $0$ in $X_n$. By Proposition~\ref{LCS disked nbhd in subspace} there is a disked neighborhood $U_{n+1}$ of $0$ in $X_{n+1}$ such that $U_{n+1}\cap X_n=U_n$. Likewise, for each $k\in\N$, there is a disked neighborhood $U_{n+k}$ of $0$ in $X_{n+k}$ such that
\[U_{n+k}\cap X_{n+k-1}=U_{n+k-1}.\]
Clearly $U=\bigcup_{k\in\N}U_{n+k}$ is a disk in $X$ and $U\cap X_{n+k}=U_{n+k}$. For $m<n$, $U\cap X_m=U\cap(X_n\cap X_m)=U_n\cap x_m$ which is a neighborhood of $0$ in $X_m$. Thus $U\cap X_m$ is a neighborhood of $0$ in $X_m$ for each $m\in\N$. Therefore, $U$ is a neighborhood of $0$ in the strict inductive limit topology and (a) is established.\par
For $x\in X$, if $x\neq 0$, there must be some $n\in\N$ and neighborhood $V_n$ of $0$ in $X_n$ such that $x\in X_n$ and $x\notin V_n$. By (a) there is a neighborhood $V$ of $0$ in $X$ such that $V_n=V\cap X_n$. Clearly, $x$ cannot belong to $V$.\par
Given $n\in\N$, suppose that $x\notin X_n$. There is some $m>n$ such that $x\in X_m$. As $X_n$ must be closed in $X_m$ (by induction, from the hypothesis) there exists a neighborhood $U_m$ of $x$ in $X_m$ such that $U_m\cap X_n=\emp$. By (a) there is a neighborhood $U$ of $x$ in $X$ such that $U\cap X_m=U_m$. Since $U\cap X_n=U_m\cap X_n=\emp$, it follows that $X_n$ is closed in $X$.
\end{proof}
The most important kind of strict inductive limit is that which arises when each $X_n$ is a Fr\'echet space. Then $X=\rlim X_n$ is called an \textbf{LF-space}. LF-spaces are generally meager since the closed proper linear subspaces $X_n$ must have empty interior. Note also that since each $X_n$ is complete and $X$ is Hausdorff, it is a closed subspace of $X_{n+1}$ for every $n$. We consider two important LF-spaces next.
\begin{example}
Let $\Omega$ be an open subset of $\R^n$. Let $\mathscr{D}(\Omega)$ denote the linear space (with pointwise operations) of infinitely differentiable functions on $\Omega$ with compact support. We topologize $\mathscr{D}(\R^n)$ as a strict inductive limit by considering an exhaustion by compact sets $\{K_m\}$ of $\Omega$. Let $\mathscr{D}_{K_m}$ be the subspace of $\mathscr{D}(\Omega)$ consisting of those functions whose support is in $K_m$. $\mathscr{D}_{K_m}$ is then topologized by the family of seminorms $\{p_N:N\in\N\}$ where, for $f\in\mathscr{D}_{K_m}$,
\[p_N(f)=\sup\{\|\partial^jf\|_{K_m}:0\leq j\leq N\}.\]
Then each $\mathscr{D}_{K_m}$ is a Fr\'echet space and is a closed subspace of $\mathscr{D}_{K_{m+1}}$. Clearly, $\mathscr{D}(\Omega)=\bigcup_m\mathscr{D}_{K_m}$. When endowed with the strict inductive limit topology determined by the $\mathscr{D}_{K_m}$, $\mathscr{D}(\Omega)$ is called a space of \textbf{test functions} on $\Omega$. Its continuous dual is called a space of \textbf{distributions} on $\Omega$.
\end{example}
\begin{proposition}\label{LCS str inductive limit bounded compact}
For $X=\rlim X_n$, suppose that each $X_n$ is a closed subspace of $X_{n+1}$. Then:
\begin{itemize}
\item[(a)] A subset $B$ of $X$ is bounded or totally bounded iff there exists $m\in\N$ such that $B\sub X_m$ and $B$ is bounded or totally bounded, respectively, in $X_m$.
\item[(b)] A sequence $x_i$ in $X$ is Cauchy iff for some $m\in\N$, $x_i\in X_m$ and $x_i$ is Cauchy in $X_m$.
\item[(c)] A sequence $x_i\to x$ in $X$ iff $x_i\in X_m$ for some $m$ and $x_i\to x$ in $X_m$.
\item[(d)] A subset $B$ of $X$ is compact iff there is some $m\in\N$ such that $B$ is a compact subset of $X_m$.
\end{itemize}
\end{proposition}
\begin{proof}
If $B$ is bounded in some $X_m$ and $U$ is a neighborhood of $0$ in $X$, then $B$ is absorbed by $U\cap X_m$, hence by $U$. To prove the converse, we show that if $B$ is not contained in any $X_n$, then $B$ is unbounded in $X$. To accomplish this, we use the criterion of Proposition~\ref{TVS bounded set iff} and show that such a $B$ contains a sequence $\{x_n\}$ such that $x_n/n\to 0$.\par
Since $B$ is not contained in any $X_n$, there exists $x_1\in B$ such that $x_1\notin X_1$. Let $k_2$ be the least positive integer such that $x_1\in X_{k_2}$. Choose $x_2\in B\setminus X_{k_2}$ and let $k_3$ be the least positive integer such that $x_2\in X_{k_3}$. Thus, by induction, there exists a strictly increasing sequence $\{k_i\}$ of positive integers and elements $x_i\in B$ such that $x_i\notin X_{k_i}$ but $x_i\in X_{k_{i+1}}$.\par
For any disked neighborhood $W_1$ of $0$ in $X_1$, since $x_1\notin X_1$, $x_1\notin W_1$. Consequently, by Proposition~\ref{LCS disked nbhd in subspace}, there exists a disked neighborhood $W_2$ of $0$ in $X_{k_2}$ such that $W_1=W_2\cap X_1$ and $x_1\notin W_2$. Since neither $x_1$ nor $x_2/2$ belong to $W_2$, there exist a disked neighborhood $W_3$ of $0$ in $X_{k_3}$ such that $x_1\notin W_3$, $x_2/2\notin W_3$, and $W_2=W_3\cap X_{k_2}$. Continuing in this fashion, we see that there exists an increasing sequence of disked neighborhoods $W_i$ of $0$ in $X_{k_i}$ such that $x_i/i\notin W_j$ for any $i,j\in\N$ and that $W_{i}=W_{i+1}\cap X_{k_i}$. Consider the disk $W=\bigcup_{i}W_i$ in $X$. Since each $W_i$ is a disk neighborhood in $X_{k_i}$, $W$ is a disk neighborhood of $0$ in $X$. With the choice of $W_i$, we see $\{x_i/i:i\in\N\}\not\subset W$, so $x_i/i\not\to 0$. This shows $B$ is not bounded.\par
If $B$ is totally bounded in $X_m$ and $V$ is a neighborhood of $0$ in $X$ then there exist a finite subset $F$ in $X_m$ such that $B\sub F+(V\cap X_m)\sub F+V$. Conversely, if $B$ is totally bounded in $X$, it is bounded. Consequently, there exists $m\in\N$ such that $B\sub X_m$. To show that it is totally bounded in $X_m$, let $(x_\alpha)$ be a net in $B$. Then, since $B$ is totally bounded in $X$, there exists a subnet of $(x_{\alpha_\beta})$ of $(x_\alpha)$ that is Cauchy in $X$ (Proposition~\ref{uniform space precompact iff totally bounded}). Let $V_m$ be a neighborhood of $0$ in $X_m$ and $V$ be a neighborhood of $0$ in $X$ such that $V_m=V\cap X_m$. As $(x_{\alpha_\beta})$ is Cauchy in $X$, there exists an index $\beta_0$ such that $x_{\alpha_{\beta}}-x_{\alpha_{\beta'}}\in V$ for $\beta\succeq\beta_0$. Since $X_m$ is a subspace and $x_{\alpha_{\beta}},x_{\alpha_{\beta'}}\in X_m$, it follows that $x_{\alpha_{\beta}}-x_{\alpha_{\beta'}}\in V\cap X_m=V_m$. Thus $(x_{\alpha_\beta})$ is Cauchy in $X_m$ and the claim follows by Proposition~\ref{uniform space precompact iff totally bounded}.\par
For (b), since Cauchy sequences are bounded, the claim follows from (a). We prove (c). The condition clearly suffices. The converse follows from the fact that if $x_i\to x$ then $\{x_n:n\in\N\}\cup\{x\}$ is compact, hence bounded1 and therefore contained in some $X_m$ by (a).\par
If $B\sub X_m$ is compact in $X_m$, it is clear compact in $X$. If $B\sub X$ is compact, it is bounded and therefore contained in some $X_m$. If $\{U_s\}$ is an open cover of $B$ from $X_m$, there exist open subsets $W_s$ of $X$ such that $U_s=W_s\cap X_m$ for each index $s$. The desired result now follows from the fact that a finite number of the $W$'s must cover $B$.
\end{proof}
\begin{proposition}\label{LCS LF-space nonmetrizable}
If $X=\rlim X_n$ and each $X_n$ is a closed proper subspace of $X_{n+1}$ then $X$ is not pseudometrizable. Hence LF-spares are not pseudometrizable.
\end{proposition}
\begin{proof}
To show that $X$ is not pseudometrizable, we show that no countable collection $\{U_n\}$ of balanced neighborhoods of $0$ in $X$ can be a base at $0$. If $\{U_n\}$ is a base at $0$, we may suppose that it is decreasing. Note that for each $n\in\N$, $U_n$ cannot be a subset of $X_n$ because each $U_n$ is absorbent in $X$ and $X_n\neq X$. Thus, for every $n$, we may choose an element $x_n$ in $U_n$ not in $X_n$. Since the $x_n$'s cannot be in any one $X_m$, they comprise an unbounded set. However, since the $U_n$'s are decreasing, for any for $j\in\N$, $x_n\in U_n\sub U_m$ for $n\geq m$; while $U_m$ must absorb $x_1,\dots,x_{m-1}$---in other words, any $U_m$ absorbs the $x_n$'s. Thus, if $\{U_n\}$ were a base at $0$, $\{x_n:n\in\N\}$ would be a bounded set which is a contradiction.
\end{proof}
\begin{proposition}\label{LCS strict inductive limit topo complete}
If each $X_n$ is complete then $X=\rlim X_n$ is complete; hence LF-spaces are complete.
\end{proposition}
\begin{proof}
Let $\mathcal{B}$ be a Cauchy filetr on $X$. Let $\mathfrak{U}(0)$ be the filter of neighbourhoods of $0$ in $X$. Then $\mathcal{B}+\mathfrak{U}(0)$ is a Cauchy filterbase which converges if and only if $\mathcal{B}$ converges. We show that there must exist an $n_0\in\N$ for which the trace of $\mathcal{B}+\mathfrak{U}(0)$ on $X_{n_0}$ is a filter base; if so, this trace converges in $X_{n_0}$, since $X_{n_0}$ is complete, and hence $\mathcal{B}$ converges in $X$.\par
Assume the converse. Then there exists a sequence $B_n\in\mathcal{B}$ and a decreasing sequence of disked neighborhoods $W_n$ of $0$ in $X$ such that $(B_n+W_n)\cap X_n=\emp$ for each $n\in\N$. Now $U=\convbal\bigcup_n(W_n\cap X_n)$ is a disked neighborhood of $0$ in $X$; we show that $(B_n+U)\cap X_n=\emp$ for each $n\in\N$.\par
Let $y\in(B_n+U)\cap X_n$; then $y=y_n+\sum_ia_ix_i$, where $\sum_i|a_i|\leq 1$, $x_i\in W_i\cap X_i$ for each $i$ and $y_n\in B_n$, hence
\[y-\sum_{i\leq n}a_ix_i=y_n+\sum_{i>n}a_ix_i.\]
Since $W_i\sub W_n$ for $i>n$ and $W_n$ is disked, the right-hand member
of the last equality is in $B_n+W_n$, while the left-hand member is in $X_n$, which is impossible; thus $(B_n+U)\cap X_n=\emp$ for all $n$.\par
Since $\mathcal{B}$ is a Cauchy filter, there exists $B\in\mathcal{B}$ such that $B-B\sub U$. Let $x\in B$, then $x\in X_n$ for some $n\in\N$. Let $y\in B_n\cap B$, then
\[x=y+(x-y)\in y+B-B\sub B_n+U,\]
which is contradictory.
\end{proof}
\begin{example}
Consider the normed space $c_{00}$ of all finite sequences of real scalars. Under the obvious identification, $\K^n$ is an increasing sequence of subspaces whose union is $c_{00}$ and each $\K^n$ is a closed subspace of $\K^{n+1}$.\par
How does the strict inductive limit topology $\mathcal{T}_i$ compare to the $\|\cdot\|_\infty$-topology $\mathcal{T}$ on $c_{00}$? Recall that the finest locally convex topology $\mathcal{T}_{lc}$ for $c_{00}$ is that which has the filterbase of all absorbent disks as a base of neighborhoods of $0$. We show that $\mathcal{T}_{lc}=\mathcal{T}_\infty$ by the following argument. Since $\mathcal{T}_i$ is locally convex, $\mathcal{T}_{i}\sub\mathcal{T}_{lc}$. Conversely, if $D$ is an absorbent disk in $c_{00}$, then $D\cap\K^n$ is a disk in $\K^n$; it is absorbent because $D$ is. As an absorbent disk in $\K^n$, $D\cap\K^n$ is a basic neighborhood of $0$ in the finest locally convex topology for $\K^n$. But as all Hausdorff linear topologies coincide on a finite-dimensional space, so $D\cap\K^n$ is a $\mathcal{T}_\infty$-neighborhood of $0$ in $\K^n$. Thus any absorbent disk in $c_{00}$ induces a $\mathcal{T}_\infty$-neighborhood of $0$ in $\K^n$, so $\mathcal{T}_{lc}\sub\mathcal{T}_i$.\par
As $\mathcal{T}_\infty$ is a locally convex topology for $c_{00}$, $\mathcal{T}_{\infty}\sub\mathcal{T}_{lc}$. Since $\mathcal{T}_\infty$ is metrizable and $\mathcal{T}_i$ is not, it follows that $\mathcal{T}_\infty$ is strictly finer than $\mathcal{T}_i$. Yet each induces the same topology on each of the subspaces $\K^n$.
\end{example}
\begin{proposition}\label{TVS disk closure}
If $D$ is a disked neighborhood of $0$ in a topological vector space then $\widebar{D}\sub 2D$.
\end{proposition}
\begin{proof}
If $D$ is any convex set then $D+D=2D$. Suppose that $D$ is a disked neighborhood of $0$ and that $y\in\widebar{D}$. Since $D$ is a neighborhood of $0$, then there exists $z\in(y+D)\cap D$. Since $z\in y+D$, $z-y\in D$. Since $z\in D$ and $D$ is a disk, $y\in D+D=2D$. Therefore $\widebar{D}\sub 2D$.
\end{proof}
\begin{proposition}\label{LCHS barreled is strict inductive limit}
If $(X,\mathcal{T})$ is a barreled Hausdorff space and $\{X_n\}$ an increasing sequence of subspaces such that $X=\bigcup_nX_n$ then $X$ is the strict inductive limit of the $X_n$, i.e., $\mathcal{T}$ is the strict inductive limit topology $\mathcal{T}_i$.
\end{proposition}
\begin{proof}
In the notation of the statement, since $X$ is a locally convex space, there is a base of disked $\mathcal{T}$-neighborhoods $D$ of $0$ in $X$. Since the $X_n$ are subspaces of $X$, each such $D$ meets each $X_n$ in a neighborhood of $0$ so $\mathcal{T}\sub\mathcal{T}_i$. A disk $D$ which meets each $X_n$ in a neighborhood of $0$ must be absorbent in $X=\bigcup_nX_n$; hence $\widebar{D}$ is a barrel in $X$. In the remainder of the proof we show that $(1/8)\widebar{D}\sub D$ from which it follows that $D$ is a $\mathcal{T}$-neighborhood of $0$.\par
Let $x\in(1/2)\widebar{D}$ so that there exists a Cauchy filterbase $\mathcal{B}$ in $(1/2)D$ such that $\mathcal{B}\to x$. Since $X$ is barreled and $D=\bigcup_n(D\cap X_n)$ is absorbent, it follows that given any $y\in X$ there is some $t_y>0$ such that $y\in t_yD$ and therefore that for some $N\in\N$ we have $y\in t_y(D\cap X_n)$ provided $n\geq N$. Hence, for $n\geq N$ and $f_n\in(D\cap X_n)^\circ$, $|f_n(y)|\leq t_y$; therefore, for any sequence $f_n\in (D\cap X_n)^\circ$, there is some $K_y>0$ such that $|f_n(y)|\leq K_y$ for all $n\in\N$, i.e., $\{f_n:n\in\N\}$ is pointwise bounded on $X$. Since $X$ is barreled, it follows from the Banach-Steinhaus theorem that $\{f_n:n\in\N\}$ is equicontinuous on $X$, i.e., satisfies the condition of Proposition~\ref{LCS closure of increasing disk}. Therefore, with $r=1$ and $\mathfrak{U}(0)$ denoting the filter of neighborhoods of $0$ in $X$, there is some $m\in\N$ such that $\mathcal{B}+\mathfrak{U}(0)$ induces a Cauchy filterbase on $2(D\cap X_m)$. Since the disks are increasing, we may suppose that $m$ is big enough so that $x\in X_m$. Since $\mathcal{B}\to x$, it is straightforward to verify that $B+\mathfrak{U}(0)\to x$ as well; therefore $x\in\mathrm{cl}_{X_m}(D\cap X_m)$. By Proposition~\ref{TVS disk closure}, $\mathrm{cl}_{X_m}(D\cap X_m)\sub 2(D\cap X_m)$. Therefore $x\in 4(D\cap X_m)$ and $(1/2)\widebar{D}\sub 4\bigcup_n(D\cap X_n)=4D$. As $(1/8)\widebar{D}$ is a barrel, the desired result follows.
\end{proof}
\section{Bounded sets}
A subset $B$ of a normed space $X$ is bounded if it is contained in a sufficiently large ball $B_r(0)$ of radius $r>0$ about $0$. In other words, $B$ is bounded if it is contained in a sufficiently large multiple of the unit ball. If $A$ and $B$ are subsets of a topological vector space, we say that $A$ \textbf{absorbs} $B$ if $B\sub aA$ for all scalars $a$ of sufficiently large magnitude. Thus, a subset $B$ of a normed space $X$ is bounded if it is absorbed by the unit ball.
\begin{definition}
A subset $B$ of a topological vector space is said to be \textbf{bounded} if $B$ is contained in all sufficiently large multiples of any neighborhood $V$ of $0$; thus, given $V$, there must be some $r>0$ such that whenever $|a|\geq r$ we have $B\sub aV$. Equivalently, for all scalars $b$ of sufficiently small magnitude, $bB\sub V$.
\end{definition}
For $B$ to be bounded, it suffices that it be absorbed by each neighborhood of a basic system of neighborhoods of $0$. As there is a basis of balanced neighborhoods of $0$ in any topological vector space, we can say that $B$ is bounded if for every balanced neighborhood $V$ of $0$, there exists $a\in\K$ such that $B\sub aV$. The finer the topology, the fewer bounded sets there will be; if a set is bounded in one topology, it remains so in all coarser topologies. Finite sets are obviously bounded.\par
Is there a base of bounded neighborhoods at $0$ in the general topological vector spaces? More basically: When is a neighborhood of $0$ bounded? As it turns out, the existence of so much as one bounded neighborhood of $0$ implies that the space is pseudometrizable. There can only be a bounded convex neighborhood of $0$ in a seminormed space.\par
A type of set frequently classified as "small" is the compact set. Our next result shows that compact sets are bounded. Recall that a subset $E$ of an abelian topological group $X$ is totally bounded if, for all neighborhoods $V$ of $0$, there exists a finite number of elements $x_1,\dots,x_n$ from $X$ such that the sets $x_1+V,\dots,x_n+V$ cover $E$. As observed before, relatively compact sets are totally bounded and complete totally bounded sets are relatively compact.
\begin{proposition}\label{TVS totally bounded is bounded}
In any topological vector space $X$ totally bounded sets are bounded.
\end{proposition}
\begin{proof}
Let $B$ be totally bounded and let $V$ be a neighborhood of $0$. Choose a balanced neighborhood $U$ of $0$ such that $U+U\sub V$. Since $B$ is totally bounded, there is a finite subset $S$ of $X$ such that $B\sub S+U$; since $U$ is absorbent and balanced there is an $r\geq 1$ such that $S\sub rU$. Thus, by Proposition~\ref{TVS balanced set prop}, for $|a|\geq r$ we have
\[B\sub aU+U\sub aU+aU\sub aV,\]
and it follows that $B$ is bounded.
\end{proof}
If $(x_n)$ is a Cauchy sequence in a topological vector space then $\{x_n\}$ is Cauchy totally bounded by Proposition~\ref{uniform space precompact iff totally bounded}. Therefore any Cauchy sequence (strictly speaking, the set of points which comprise the sequence) is bounded.\par
We mentioned above that boundedness and norm boundedness are identical notions in a normed space. In metrizable topological vector spaces this coincidence does not take place.
\begin{example}\label{TVS bounded and metric bounded}
In a pseudometric space $(X,d)$, a set $B\sub X$ is \textbf{metrically bounded} or $d$-bounded if $\diam(B)<+\infty$. If $(X,d)$ is a pseudometrizable topological vector space, there is an $F$-seminorm $p$ which generates the topology on $X$ by Proposition~\ref{TVS pseudometrizable}. By the triangle inequality, $p(kx)\leq kp(x)$ for any positive integer $k$. Hence $kB_r(0)\sub B_{kr}(0)$. If $B\sub X$ is bounded then, for some $k\in\N$, $B\sub kB_1(0)\sub B_k(0)$. Hence $B$ is metrically bounded.
\end{example}
Komolgorov defined a subset $B$ of a topological vector space $X$ to be bounded if, for any sequence $(a_n)$ of real numbers with $a_n\to 0$ and any sequence $(x_n)$ of vectors, $a_nx_n\to 0$. Komolgorov immediately put the notion to use in proving that a topological vector space is seminormable iff it has a bounded convex neighborhood of $0$. The version of boundedness we use, introduced by von Neumann, is equivalent to Komolgorov's as we now show.
\begin{proposition}\label{TVS bounded set iff}
Let $B$ be a subset of a topological vector space $X$.
\begin{itemize}
\item[(a)] $B$ is bounded if and only if for any sequence $(x_n)$ from $B$ and any sequence $(a_n)$ of scalars which tends to $0$, $a_nx_n\to 0$.
\item[(b)] For $B$ to be bounded, it suffices that all denumerable subsets be bounded.
\end{itemize}
\end{proposition}
\begin{proof}
Suppose that $B$ is bounded and let $V$ be a neighborhood of $0$. Since $B$ is bounded, there is some $r>0$ such that $aB\sub V$ for all scalars $a$ such that $|a|\leq r$. Hence, if $a_n\to 0$, $a_nB\sub V$ eventually; it follows that, for any sequence $(x_n)$ from $B$, $a_nx_n\in V$ eventually. Conversely, if $B$ is not bounded, there must be some balanced neighborhood $U$ of $0$ such that, for each $n\in\N$, there is some $x_n\in B$ such that $x_n\notin nU$. Thus $(1/n)x_n\not\to 0$, and (a) follows. As the set $\{x_n:n\in\N\}$ is an unbounded denumerable subset of $B$, (b) follows.
\end{proof}
If $p$ is a seminorm on $X$ and $B_p=\{x:p(x)<1\}$, then $B\sub aB_p$ iff $p(B)\sub[0,|a|)$. As a result of this observation, it is clear that we may characterize bounded subsets of locally convex spaces as follows.
\begin{proposition}\label{LCS bounded set iff}
Let $X$ be a locally convex space $X$ whose topology is generated by a family of seminorms $\mathscr{P}$. Let $B$ be a subset $B$. Then the following are equivalent.
\begin{itemize}
\item[(\rmnum{1})] $B$ is bounded.
\item[(\rmnum{2})] $p(B)$ is bounded for each $p$ in $\mathscr{P}$.
\item[(\rmnum{3})] For any denumerable subset $D$ of $B$, $p(D)$ is bounded for each $p$ in $\mathscr{P}$.
\end{itemize}
\end{proposition}
\begin{proof}
Since $\mathcal{B}=\{r\bigcap_{i=1}^{n}B_{p_i}:r>0,p_1,\dots,p_n\in\mathscr{P}\}$ is a base for $\mathfrak{U}(0)$, $B$ is bounded iff $B\sub r\bigcap_{i=1}^{n}B_{p_i}$ for any $r>0$ and $p_1,\dots,p_n\in\mathscr{P}$, iff $p(B)$ is bounded. The equivalence of (\rmnum{1}) and (\rmnum{3}) follows from Proposition~\ref{TVS bounded set iff}.
\end{proof}
As mentioned in Example~\ref{weak topo wea-star topo def}, the weak topology $\sigma(X,X^*)$ on a topological vector space $X$ is that generated by the seminorms $\{|f(\cdot)|:f\in X^*\}$. Hence $B\sub X$ is $\sigma(X,X^*)$-boundeed (i.e., weakly hounded) iff $f(B)$ is bounded for each $f\in X^*$.\par
Generally subspaces of a topological vector space are not bounded because they contain lines $\K x$. Our next result characterizes bounded subspaces. Note that among its consequences is the fact that no nontrivial subspace of a Hausdorff topological vector space can he bounded.
\begin{proposition}\label{TVS bounded subspace iff}
A subspace $M$ of a topological vector space $X$ is bounded if and only if $M\sub\widebar{\{0\}}=\bigcap\mathfrak{U}(0)$.
\end{proposition}
\begin{proof}
If $B\sub\widebar{\{0\}}=\bigcap\mathfrak{U}(0)$ then $B$ is a subset of any neighborhood $V$ of $0$; therefore any subset of $\widebar{\{0\}}$ is bounded. Conversely, suppose that $M$ is a bounded subspace and that $x$ is an element of $M$ which is not in $\widebar{\{0\}}$. Since $nx\in M$ for any positive integer $n$ and $M$ is bounded and $n^{-1}\to 0$, Proposition~\ref{TVS bounded set iff} implies that $\lim_n(1/n)nx=0$. It follows that $x\in\widebar{\{0\}}$ which is contradictory. (Since $X$ is not necessarily Hausdorff, sequences may have more than one limit.)
\end{proof}
We now use Proposition~\ref{TVS bounded subspace iff} to exhibit some topological vector spaces which have unbounded basic neighborhoods of $0$.
\begin{example}[\textbf{Unbounded Neighborhoods}]\label{TVS no bounded nbhd eg}
\mbox{}
\begin{itemize}
\item[(a)] Let $X$ be a topological vector space whose continuous dual $X^*$, the linear space of all continuous linear forms on $X$, is infinite-dimensional. The weak topology $\sigma(X,X^*)$ for $X$ has basic neighborhoods of $0$ of the form
\[V_{0,r}(f_1,\dots,f_n)=\{x\in X:|f_i(x)|<r,i=1,\dots,n\}\]
for $f_1,\dots,f_n\in X^*$ and $r>0$. Let $N=\bigcap_{i=1}^{n}N(f_i)$. Since $X^*$ is infinite-dimensional, there exists $g\in X^*$ linearly independent of $f_1,\dots,f_n$, consequently, there is a nonzero $x\in N$ such that $g(x)=1$. Therefore $\K x\sub V_{0,r}(f_1,\dots,f_n)$. If $\K x$ is bounded, then $\K x\sub\bigcap\mathfrak{U}(0)$ by Proposition~\ref{TVS bounded subspace iff}. But $\K x\notin V_{0,1}(g)$ since $g(2x)=2$, which is a contradiction. Thus $\sigma(X,X^*)$ has no bouned neighborhood of $0$.
\item[(b)] Let $C(\R)$ be the locally convex space of all continuous functions on $\R$ (Example~\ref{seminorm eg}), with topology induced by the seminorms $p_n(f)=\sup_{|x|\leq n}|f(x)|$. By Proposition~\ref{LCS bounded set iff}, a subset $B$ in $C(\R)$ is bounded iff $B$ is uniformly bounded on the compact sets $[-n,n]$, for each $n\in\N$. For any positive integer $n$, $B_{p_n}$ contains the nontrivial subspace $M_n=\{f\in C(\R):f|_{[-n,n]}=0\}$. It follows from Proposition~\ref{TVS bounded subspace iff} that no $B_{p_n}$ is bounded and therefore that no neighborhood of $0$ in $C(\R)$ is bounded.
\end{itemize}
\end{example}
Now we consider the consequences of there being a bounded neighborhood of $0$ in a topological vector space $X$. If there is a bounded neighborhood of $0$ then $X$ is pseudometrizable; $X$ is seminormable iff it possesses a bounded convex neighborhood of $0$. We have already established a result like this: A Hausdorff topological vector space $X$ has a compact neighborhood of $0$ iff it is finite-dimensional and therefore normable by Theorem~\ref{TVS finite dim prop}.
\begin{proposition}\label{TVS bounded nhbd consequence}
Let $X$ be a topological vector space.
\begin{itemize}
\item[(a)] If $X$ has a bounded neighborhood of $0$ then $X$ is pseudometrizable.
\item[(b)] $X$ has a bounded convex neighborhood of $0$ if and only if it is seminormable.
\end{itemize}
If $X$ is a Hausdorff space then the "pseudo" and the "semi" may be omitted.
\end{proposition}
\begin{proof}
If $B\sub X$ is a bounded neighborhood of $0$ and $V$ any neighborhood of $0$ then $B\sub kV$ for some $k$ in $\N$. Hence $\{(1/n)B:n\in\N\}$ is a base at $0$ in $X$ and $X$ is pseudometrizable by Proposition~\ref{TVS pseudometrizable}.\par
If $(X,p)$ is a seminormed space then positive multiples $rB_p$ of the open unit ball $B_p$ determined by $p$ are a base at $0$. In particular, since any neighborhood $V$ of $0$ must contain some $rB_p$, i.e., $B_p\sub(1/r)V$ , $B_p$ is a bounded convex neighborhood of $0$. Conversely, suppose that $B$ is a bounded convex neighborhood of $0$. Since $B$ must contain a balanced neighborhood of $0$ and since the convex hull of a balanced set is balanced, we may suppose that $B$ is a disk. We now show that the gauge $p_B$ of $B$ determines the topology on $X$. Since $B$ is bounded, if $V$ is any neighborhood of $0$, there is some positive $r$ such that $rB\sub V$. Hence, letting $B_{p_B}$ denote the open unit ball determined by $p$, $rB_{p_B}\sub rB\sub V$ (Proposition~\ref{TVS absorbent disk guage prop}). The positive multiples of $B_p$ are therefore a base at $0$ for the topology on $X$. The last statement follows directly from Proposition~\ref{TVS pseudometrizable}.
\end{proof}
Part (a) of Proposition~\ref{TVS bounded nhbd consequence}, however, is not necessary, as the following examples show.
\begin{example}
We consider four metrizable topological vector spaces below that are not normable:
\begin{itemize}
\item[(a)] The space $C(\R)$ is locally convex and Huasdorff, but does not have bounded neighborhoods (Example~\ref{TVS no bounded nbhd eg}). Hence it is not normable by Proposition~\ref{TVS bounded nhbd consequence}. However, $C(\R)$ is metrizable by Proposition~\ref{LCS metrizable iff}. 
\item[(b)] Let $C^\infty([0,2\pi])$ be the locally convex Hausdorff space of smooth real-valued functions on $[0,2\pi]$ with topology generated by the increasing sequence of seminorms
\[p_n(f)=\sup\{\|f^{(i)}\|_\infty:1\leq i\leq n\}.\] 
$C^\infty([0,2\pi])$ is metrizable since its topology is defined by a countable family of seminorms. As $\{p_n\}$ is a saturated family of seminorms, to see that $C^\infty([0,2\pi])$ has no bounded neighborhood of $0$, it suffices to show that none of the unit balls $B_{p_n}$ is bounded. As $p_1(\sin nt)=n$ and $\sin nt\in B_{p_0}$, it follows that $B_{p_0}$ is not bounded (Proposition~\ref{LCS bounded set iff}). Similar considerations show that none of the $B_{p_n}$, $n\in\N$, is bounded.
\end{itemize}
\end{example}
We examine the stability of boundedness in regard to the formation of Cartesian products and balanced and convex hulls.
\begin{proposition}\label{TVS bounded set union sum product}
Let $X$ be a topological vector space.
\begin{itemize}
\item[(a)] Subsets of bounded sets (hence intersections) are bounded.
\item[(b)] Finite unions of bounded sets are bounded.
\item[(c)] Finite sums and scalar multiples of bounded sets are bounded.
\item[(d)] The quotient of a bounded set is bounded in a quotient topology.
\item[(e)] If $\{X_s:s\in S\}$ are topological vector spaces and $B_s$ is a bounded subset of $X_s$ for each $s$, then $\prod_sB_s$ is a bounded subset of $\prod_sX_s$. 
\end{itemize}
\end{proposition}
\begin{proof}
Parts (a) and (b) are clear, as is the boundedness of scalar multiples of bounded sets. To prove that $B_1+\cdots+B_n$ is bounded if each $B_i$ is, let $U\in\mathfrak{U}(0)$ and choose $V\in\mathfrak{U}(0)$ such that $V+\cdots+V\sub U$ ($n$ times). Since each $B_i$ is bounded, there is an $r>0$ such that
\[B_1+\cdots+B_n\sub rV+\cdots+rV\sub rU,\]
so $B_1+\cdots+B_n$ is bounded.\par
By Proposition~\ref{TVS quotient topo prop}, with $\pi$ denoting the canonical map of $X$ into the quotient, if $B$ is the filterbase of balanced neighborhoods of $0$ in $X$ then $\pi(B)$ is a neighborhood base at $0$ for the quotient topology. Hence, if $B\sub rV$ then $\pi(B)\sub r\pi(V)$. To prove (e), it suffices to show that $\prod_sB_s$ is absorbed by each basic neighborhood $\prod_sV_s$ of $0$ in $\prod_sX_s$ where the neighborhoods $V_s=X_s$ for each $s\in S$ except $s_1,\dots,s_n$. As there is some $r>0$ such that $B_s\sub rV_s$ for $i=1,\dots,n$, it follows that $\prod_sB_s\sub r\prod_sV_s$.
\end{proof}
\begin{proposition}\label{TVS bounded set closure bal}
Let $X$ be a topological vector space.
\begin{itemize}
\item[(a)] The closure (and interior) of a bounded set is bounded.
\item[(b)] The balanced hull of a bounded set is bounded.
\end{itemize}
\end{proposition}
\begin{proof}
Let $B$ be a bounded subset of the topological vector space $X$. By Proposition~\ref{TVS nhbd base of 0 balanced and absorbent}, there is a base of closed neighborhoods $V$ of $0$ so that $B\sub rV\Rightarrow\widebar{B}\sub rV$ and (a) follows. The proof of (b) is quite similar, only here we use the fact that there is a base of balanced neighborhoods of $0$.
\end{proof}
In Proposition~\ref{TVS hull of totally bounded or compact sets} we showed that the convex hull of a totally bounded subset in a locally convex space is totally bounded. A similar result holds for bounded sets. Namely,
\begin{proposition}\label{LCS convex hull of bounded}
In a locally convex space, the convex hull of a bounded set $B$ is bounded.
\end{proposition}
\begin{proof}
Let $B$ be bounded and suppose that $U$ is a convex neighborhood of $0$. Let $r>0$ be such that $B\sub rU$. The desired result now follows from the
fact that $\conv(B)\sub\conv(rU)=rU$.
\end{proof}
A base $\mathcal{B}$ for a neighborhood filter $\mathfrak{V}(x)$ is a collection of neighborhoods of $x$ such that each $V\in\mathfrak{U}(x)$ contains some $B\in\mathcal{B}$ as a subset. A notion of "base" for the system of bounded sets in a topological vector space is defined next.
\begin{definition}
In a topological vector space $X$ a collection $\mathcal{B}$ of bounded sets is a \textbf{base} (\textbf{fundamental system}) of bounded sets if for any bounded subset $E$ of $X$ there is a $B\in\mathcal{B}$ such that $E\sub B$.
\end{definition}
The closed intervals $[-n,n]$ ($n\in\N$) are a base of bounded sets in $\R$, as are the closed balls $B_n(0)$ about $0$ of radius $n\in\N$ in any seminormed space.
\begin{proposition}\label{LCS bounded base}
In any locally convex space the closed bounded disks are a base of bounded sets.
\end{proposition}
\begin{proof}
Let $X$ be a locally convex space, suppose that $B\sub X$ is bounded and let $C$ be the closure of the disked hull $\convbal(B)$ of $B$. Since $B$ is bounded, $\bal(B)$ is bounded by Proposition~\ref{TVS bounded set closure bal}. Since $X$ is locally convex, $\convbal(B)$ is bounded by Proposition~\ref{LCS convex hull of bounded}; therefore, so is $C$, by Proposition~\ref{TVS bounded set closure bal}.
\end{proof}
\section{Continuity and boundedness}
Recall that for topological vector spaces $X$ and $Y$, we say that $f:X\to Y$ is \textit{bounded} if $f$ maps bounded sets into bounded sets. Having seen that boundedness and continuity of a linear map between seminormed spaces are equivalent (Proposition~\ref{seminormed space continuous iff}), we investigate this connection in more general situations here. What emerges is that continuity of a linear map always implies boundedness while the converse holds in a wide variety of special cases.
\begin{proposition}\label{TVS homogeneous image of bounded is bounded}
Let $X$ and $Y$ be topological vector spaces, and suppose that $f:X\to Y$ is continuous and such that $f(ax)=a^rf(x)$ for some $r>0$ for all $a>0$ and any $x\in X$. If $B\sub X$ is bounded then $f(B)$ is bounded. In particular, continuous linear maps are bounded.
\end{proposition}
\begin{proof}
With notation as above, let $\mathfrak{U}_X(0)$ and $\mathfrak{U}_Y(0)$ denote the neighborhood filters of $0$ in $X$ and $Y$, respectively. Since $f(ax)=a^rf(x)$, it follows that $f(0)=0$. Since $f$ is continuous, for any $U\in\mathfrak{U}_Y(0)$, $V=f^{-1}(U)\in\mathfrak{U}_X(0)$. Since $B$ is bounded in $X$, there is some positive number $a$ such that $B\sub aV$. Hence $f(B)\sub a^rf(V)\sub a^rU$.
\end{proof}
It follows from Proposition~\ref{TVS homogeneous image of bounded is bounded} that if a subset $B$ of an arbitrary product $\prod_sX_s$ of topological vector spaces is bounded then so are all its projections (continuous linear maps). Conversely, if all projections $\pi_s(B)$ of $B\sub \prod_sX_s$ are bounded, so is $B$ since $B\sub\prod_s\pi_s(B)$ and $\prod_s\pi_s(B)$ is bounded by Proposition~\ref{TVS bounded set union sum product}(e). We summarize this now for future reference.
\begin{proposition}\label{TVS product space bounded iff projection bounded}
A subset of a product of topological vector spaces is bounded iff all its projections are.
\end{proposition}
Now we use Proposition~\ref{TVS product space bounded iff projection bounded} to prove some interesting results.
\begin{proposition}\label{seminormed space product not seminormable}
Let $\{X_s\}$ be an infinite family of nontrivial seminormed spaces. The product $\prod_sX_s$ is not seminormable.
\end{proposition}
\begin{proof}
With notation as above, if $\prod_sX_s$ were seminormable it would have a bounded convex neighborhood of $0$ by Proposition~\ref{TVS bounded nhbd consequence}. Hence it would have a bounded basic neighborhood of $0$ in the product topology. We now show that no basic neighborhood of $0$ is bounded. To this end, let $\prod_sV_s$ be a basic neighborhood of $0$: $V_s=X_s$ for all $s$ except $s_1,\dots,s_n$. For any $s\notin\{s_1,\dots,s_n\}$, $\pi_s(V)=X_s$, which is not bounded. Consequently, $V$ is not bounded by Proposition~\ref{TVS product space bounded iff projection bounded}.
\end{proof}
\begin{proposition}\label{TVS initial topo bounded iff}
Let $\{X_s:s\in S\}$ be a family of topological vector spaces. For each $s\in S$, let $A_s:X\to X_s$ be a linear map. Endow $X$ with the initial topology, then $B\sub X$ is bounded iff $A_s(B)$ is bounded for each $s\in S$.
\end{proposition}
\begin{proof}
The initial topology on $X$ is induced by the single map $A:X\to\prod_sX_s$. Since $A$ is continuous, by Proposition~\ref{TVS homogeneous image of bounded is bounded}, if $B\sub X$ is bounded then $A(B)$ is bounded. Hence the projection $\pi_s(A(B))=A_s(B)$ is bounded for each $s$ in $S$. Conversely, if each $A_s(B)$ is bounded, so is $\prod_sA_s(B)$ by Proposition~\ref{TVS product space bounded iff projection bounded}. Since $A(B)\sub\prod_sA_s(B)$, it is also bounded. Let $U\in\mathfrak{U}_X(0)$, then there exists $V\in\mathfrak{U}_{\prod_sX_s}(0)$ such that $U=A^{-1}(V)$. Since $A(B)$ is bounded, there exists $r>0$ such that $A(B)\sub rV$ and
\[B\sub A^{-1}(A(B))\sub A^{-1}(rV)=rU,\]
thus $B$ is bounded.
\end{proof}
We have seen that continuous linear maps are bounded. The converse is false but there are many important special cases in which boundedness implies continuity. We now develop a criterion for continuity and then show that bounded maps with pseudometrizable domains must be continuous.
\begin{proposition}\label{TVS bounded on a nbhd is continuous}
If a linear map $A:X\to Y$ takes a neighborhood $V$ of $0$ in the topological vector space $X$ into a bounded subset of the topological vector space $Y$ then $A$ is continuous.
\end{proposition}
\begin{proof}
With notation as above, since $A(V)$ is bounded, for any neighborhood $U$ of $0$ in $Y$ there is some positive number $r$ such that $rA(V)=A(rV)\sub U$. The continuity of $A$ follows from the fact that $rV$ is a neighborhood of $0$ in $X$.
\end{proof}
\begin{proposition}\label{TVS pseudometrizable linear map continuous iff bounded}
Suppose that $A$ is a linear map taking a pseudometrizable topological vector space $X$ into a topological vector space $Y$. Then if $A$ is bounded, it is continuous.
\end{proposition}
\begin{proof}
With notation as above, let $\{U_n\}$ be a countable neighborhood base at $0$ in $X$. By taking some intersections, it is easy to see that we may assume that $U_n$'s are decreasing. If $A$ is not continuous, there is a balanced neighborhood $V$ of $0$ in $Y$ such that $A^{-1}(V)$ is not a neighborhood of $0$. Thus, for every $n$, $(1/n)U_n\nsubseteq A^{-1}(V)$, so there is a sequence of elements $u_n\in (1/n)U_n$ such that for each $n$, $Au_n\notin V$. Since $\{U_n\}$ is decreasing, it follows that $nu_n\to 0$. Hence $\{nu_n:n\in\N\}$ is relatively compact and therefore bounded. Since $A$ is a bounded map, $\{nAu_n:n\in\N\}$ is bounded; consequently, there is some $r>0$ such that $\{nAu_n:n\in\N\}\sub rV$. Since $V$ is balanced, for $n>r$ we have $Au_n\in(r/n)V\sub V$ which contradicts the way in which the $u_n$ were chosen.
\end{proof}
\chapter{Hahn-Banach theorem}
\begin{definition}
Let $X$ be a vector space. A function $p:X\to\R$ is called \textbf{sublinear} if
\begin{itemize}
\item[(a)] $p(x+y)\leq p(x)+p(y)$ for $x,y\in X$.
\item[(b)] $p(rx)=rp(x)$ for $r\geq 0$.
\end{itemize}
If $p(x)\geq 0$ for all $x\in X$ then p is called a \textbf{positive sublinear functional}.
\end{definition}
Two principal versions of the Hahn-Banach theorem are as a continuous extension theorem (analytic form) and as a separation theorem (geometric form) about inserting a hyperplane between open convex sets.\par
\textbf{Dominated extension}. Let $f$ be a linear functional defined on a subspace $M$ of a real vector space (no norm) $X$, a sublinear functional defined on $X$ and $f\leq p$ on $M$ (dominated by $p$), $f$ can be extended to a linear functional $F$ defined on $X$ with $F\leq p$. For complex spaces, we mainly need some absolute values: If $X$ is complex, and $p$ a seminorm such that $|f|\leq p$ on $M$ then $|F|\leq p$.\par
\textbf{Continuous extension}. If $X$ is normed space over $\K$ and $f:M\to\K$ is a continuous linear functional then there exists a continuous linear functional $\K$ extending $f$ defined on all of $X$; in particular, there is a continuous extension $F$ such that $\|F\|=\|f\|$.\par
If a ball $B$ lies to one side of a line $L$ in $\R^3$ then there is a plane containing $L$ that lies to one side of $B$. The plane is not unique unless the line is tangent to $B$. The geometric form of the Hahn-Banach theorem generalizes this idea: Let $M$ be a linear subspace of a real or complex topological vector space $X$. If the linear variety $x+M$ does not meet the open convex set $G$ then there exists a closed hyperplane $H$ containing $x+M$ that does not meet $G$.
\section{Linear functional}
We collect some basic facts about linear functionals here. The subspace $N(f)=f^{-1}(0)$ of vectors on which a linear functional $f$ vanishes is called the \textbf{null space} or \textbf{kernel} of $f$.
\begin{proposition}\label{linear functional prop}
Let $X$ be a vector space over $\K$ and let $f$ be a nontrivial linear functional on $X$ with null space $N$. Then:
\begin{itemize}
\item[(a)] $f$ is surjective;
\item[(b)] $X/N$ is linearly isomorphic to $\K$; hence $\dim X/N=1$;
\item[(c)] for any $x\notin N$, $X=N\oplus\K x$;
\item[(d)] for any scalar $a$ and any $x\in H=f^{-1}(a)$, $N=H-x$; 
\item[(e)] Let $f$ be a linear functional on $X$. If $g_1,\dots,g_n$ are nontrivial linear functionals on $X$, then $\bigcap_{i=1}^{n}N(g_i)\sub N(f)$ iff $f\in\mathrm{span}(g_1,\dots,g_n)$.
\end{itemize}
\end{proposition}
\begin{proof}
We only prove (e). We first prove the case $n=1$. Suppose $N(g)\sub N(f)$. If $N(g)$ is a proper subset of $N(f)$, there is some $x\in N(f)$ such that $x\notin N(g)$. Then $X=N(g)+\K x\sub N(f)$ and $f$ would be zero. Hence we may assume that $N=N(f)=N(g)$. For $x\notin N$ we have $X=N+\K x$, so any $y\in X$ can be written as $y=n+ax$, where $n\in N$ and $a\in\K$; hence $f(y)=af(x)$ and $g(y)=ag(x)$. Then $f=(f(x)/g(x))g$.\par
We onw prove the claim by induction. The result for $n=1$ is proved so we now assume that the result holds for $n-1$ and suppose that $\bigcap_{i=1}^{n}N(g_i)\sub N(f)$. There is no loss of generality in assuming that $g_1,\dots,g_n$ are linearly independent so we assume that no $g_k$ can be written as a linear combination of the remaining ones; hence, by the induction hypothesis, for each $1\leq k\leq n$, $\bigcap_{i\neq k}N(g_i)\nsubseteq N(g_k)$, and there exist $x_k\in X$ such that $g_j(x_k)=\delta_{jk}$. For any $x\in X$, each $g_i$ vanishes on $x-\sum_{i=1}^{n}g_i(x)x_i$ and hence $x-\sum_{i=1}^{n}g_i(x)x_i\in\bigcap_{i=1}^{n}N(g_i)\sub N(f)$. Therefore $f=\sum_{i=1}^{n}f(x_i)g_i$.
\end{proof}
On any complex vector space $X$, there is an intimate connection between the real and imaginary parts of a linear functional $f$ on $X$, namely that
\[\Re(f(x))=\Im(f(ix)).\]
Any complex vector space $X$ can, of course, be viewed as a real one. A linear map of $X$ into $\R$ is then called a \textbf{real linear functional} on $X$. For emphasis we often refer to linear functionals $f$ on $X$ as complex linear functionals.
\begin{proposition}\label{linear functional real and complex}
Let $X$ be a vector space over $\C$.
\begin{itemize}
\item[(a)] If $f$ is a complex linear functional on $X$, then $\phi:=\Re(f)$ is a real linear functional and for all $x$ we have $f(x)=\phi(x)-i\phi(ix)$.
\item[(b)] If $\phi$ is a real linear functional on $X$ and $f:X\to\C$ is defiend by $f(x)=\phi(x)-i\phi(ix)$, then $f$ is complex linear. If $X$ is normed, we also have $\|f\|=\|\phi\|$.
\end{itemize}
\end{proposition}
\begin{proof}
If $f$ is complex linear and $\phi=\Re(f)$, then $\phi$ is clearly real linear and
\[\Im(f(x))=-\Re(if(x))=-\phi(ix).\]
On the other hand, if $\phi$ is real linear and $f(x)=\phi(x)-i\phi(ix)$, then $f$ is clearly linear over $\R$, and
\[f(ix)=\phi(ix)-i\phi(-x)=i(\phi(x)-i\phi(ix))=if(x),\]
so $f$ is linear over $\C$. Finally, if $X$ is normed, then
\[|\phi(x)|=|\Re(f(x))|\leq|f(x),\]
so $\|\phi\|\leq\|f\|$. On the other hand, if $f(x)\neq 0$, let $\alpha=\sgn(f(x))$, then
\[|f(x)|=\alpha f(x)=f(\alpha x)=\Re(f(\alpha x))=\phi(\alpha x).\]
This implies $\|f\|\leq\|\phi\|$.
\end{proof}
A \textbf{maximal subspace} $M$ of a vector space $X$ is a proper subspace not properly contained in any proper subspace of $X$. A \textbf{linear variety} (affine subspace, linear manifold) is a translate $x+M$ of a subspace $M$. A \textbf{hyperplane} is a translate of a maximal subspace.
\begin{proposition}\label{linear functional and hyperplane}
Let $M$ be a subspace of a vector space $X$ over $\K$. Then:
\begin{itemize}
\item[(a)] $M$ is maximal iff $\dim X/M=1$;
\item[(b)] $M$ is maximal iff $M=N(f)$, the null space of some nontrivial linear functional $f$ on $X$;
\item[(c)] a linear variety $H$ is a hyperplane iff $H=f^{-1}(a)$ for some
nontrivial linear functional $f$ on $X$ and scalar $a$.
\end{itemize}
\end{proposition}
\begin{proof}
If $\dim X/M\geq 2$, there are $x,y\in X$ such that $x+M$ and $y+M$ are linearly independent in $X/M$. Moreover, we have the proper inclusions $M\subsetneq\langle M,x\rangle\subsetneq\langle M,x,y\rangle$ and $M$ is not maximal. Conversely, if $M$ is not maximal, there is a proper subspace $N$ of $X$ which contains $M$ properly; hence $\dim X/M>\dim N/M>1$.\par
Let $f$ be a linear functional and let $M =N(f)$. Then $\dim X/M=1$, hence $M$ is maximal by (a). Conversely, if $M$ is maximal and $x\notin M$, then $M\oplus\K x=X$. Defining $f(y)=f(m+ax)=a$ for each $y\in X$, $f$ is a linear functional with null space $M$.\par
If $H$ is a hyperplane, then $H=x+M$ for some $x\in X$ and maximal subspace $M\sub X$. Since $M$ is maximal, there is a nontrivial linear functional $f$ on $X$ such that $M=N(f)$ by (b). Let $f(x)=a$. Then $f(y)=f(x)$ iff $f(y-x)=0$, iff $y\in H$. So $H=f^{-1}(a)$.\par
Conversely, suppose $H=f^{-1}(a)$ for some nontrivial linear functional $f$ and scalar $a$. By Proposition~\ref{linear functional prop}(d), for any $x\in H$, $H-x$ is the null space of $f$ by Proposition~\ref{linear functional prop}(d); $H-x$ is maximal by (b).
\end{proof}
If $X$ is a complex vector space, we say that $M\sub X$ is an \textbf{$\R$-subspace} if it is closed under addition and multiplication by real scalars. An \textbf{$\R$-variety} is a translate of an $\R$-subspace; an \textbf{$\R$-hyperplane}, a translate of a maximal $\R$-subspace. As we have observed, for any complex linear functional $f$ on $X$, $f(x)=\phi(x)-i\phi(ix)$ where $\phi=\Re(f)$ is a real linear functional on $X$. By Proposition~\ref{linear functional and hyperplane}(b), the null space $N(\phi)$ of $\phi$ is a maximal subspace of the real linear space $X$---in other words, a maximal $\R$-subspace. As follows from the proposition below, $N(f)=N(\phi)\cap iN(\phi)$.
\begin{proposition}\label{linear functional real and complex hyperplane}
Let $X$ be a complex vector space. Then
\begin{itemize}
\item[(a)] a subspace $M$ of $X$ is maximal iff there is a maximal $\R$-subspace $N$ such that $M=N\cap iN$.
\item[(b)] a linear variety $H$ is a hyperplane iff there is a maximal $\R$-subspace $M$ and $x,y\in X$ such that $H=(x+M)\cap(y+iM)$.
\end{itemize}
\end{proposition}
\begin{proof}
If $M$ is a maximal subspace of $X$, then there is a nontrivial linear functional $f$ on $X$ such that $M=N(f)$. Moreover, with $\phi=\Re(f)$, $f(x)=\phi(x)-i\phi(ix)$ for every $x\in X$. Hence $x\in M$ iff $\phi(x)=0$ and $\phi(ix)=0$ iff $x\in N(\phi)\cap iN(\phi)$.\par
Conversely, if $N$ is a maximal $\R$-subspace, there is a real linear functional $\phi$ whose null space is $N$. With $f(x)=\phi(x)-i\phi(ix)$, we have $N\cap iN=N(f)$.\par
If $H$ is a hyperplane, then $H=x+M$ for some $x\in X$ and maximal subspace $M$. There exists a maximal $\R$-subspace $N$ such that $M=N\cap iN$ by (a). Thus $H=x+N\cap iN=(x+N)\cap (x+iN)$.\par
Conversely, if $N$ is a maximal $\R$-subspace and $x,y\in X$, we want to show that $(x+N)\cap (Y+iN)=H$ is a hyperplane. To do this, it suffices to show that there is a complex linear functional $f$ on $X$ and $a\in\C$ such that $H=f^{-1}(a)$. Since $N$ is a maximal $\R$-subspace, there is a real linear functional $\phi$ on $X$ Ruch that $N=N(\phi)$. Let $f$ be the associated complex linear functional $f(w)=\phi(w)-i\phi(iw)$. With $a=\phi(x)-i\phi(iy)$, then
\[f(w)=a\iff \phi(w)=\phi(x),\phi(iw)=\phi(iy)\iff w\in x+N,w\in y+iN.\]
Thus the claim follows.
\end{proof}
\section{Dominated extension}
\begin{proposition}
Let $M$ be a subspace of a real vector space $X$, $x_0\notin M$, $p$ a sublinear functional defined on $X$ and let $f$ be a linear functional defined on $M$ with $f\leq p$ on $M$. Then there exists a linear functional $F$ defined on $M\oplus\R x_0$ such that $F\leq p$.
\end{proposition}
\begin{proof}
Every vector $x$ in $M\oplus\R x_0$ has the form $x=m+tx_0$. Every linear extension is uniquely determined by our choice of the number $\alpha=F(x_0)$. We have to pick $\alpha$ in such a way that $F(x)\leq p(x)$ for all $x\in X$. Thus, we have to ensure the following inequality for all $t\in\R$:
\[f(m)+\alpha t\leq p(m+tx_0).\]
For $t>0$ this inequality is equivalent to
\[\alpha\leq t^{-1}p(m+tx_0)-f(t^{-1}m),\]
and for $t=-s<0$ is equivalent to
\[\alpha\geq -s^{-1}p(m-sx_0)+f(s^{-1}m).\]
We show that there exists a number $\alpha$ satisfying both inequalities for all $m$ and $t$. To this end, we prove the inequality
\[\sup_{m\in M,s>0}[-s^{-1}p(sm-sx_0)+s^{-1}f(m)]\leq \inf_{m\in M,t>0}[t^{-1}p(tm+tx_0)-t^{-1}f(m)].\]
In fact, we will show that
\begin{align*}
-s^{-1}p(sm_1-sx_0)+s^{-1}f(m_1)\leq t^{-1}p(tm_2+tx_0)-t^{-1}f(m_2),\quad \forall m_1,m_2\in M,s,t>0,
\end{align*}
which can be written as
\[f(m_1)+f(m_2)\leq t^{-1}p(tm_2+tx_0)+s^{-1}p(sm_1-sx_0)=p(m_2+x_0)+p(m_1-x_0).\]
This inequality then follows immediately from our hypothesis and the property of $p$:
\begin{align*}
f(m_1)+f(m_2)=f(m_1+m_2)\leq p(m_2+x_0+m_1-x_0)\leq p(m_2+x_0)+p(m_1-x_0).
\end{align*}
Thus the claim is proved.
\end{proof}
The dominated extension theorem below enables us to prove continuous extension theorems because a linear functional $f$ on a topological vector space $X$ is continuous iff there is a continuous seminorm $p$ on $X$ such that $|f|\leq p$ (Proposition~\ref{LCS as image continuity iff}). The key to obtaining the complex version from the real one is the one-to-one correspondence between real and complex linear functionals.
\begin{theorem}[\textbf{Dominated Extension}]\label{Hahn-Banach dominated extension}
Let $M$ be a subspace of a vector space $X$ over $\K$, let $p$ be a sublinear functional defined on $X$ and let $f$ be a linear functional on $M$.
\begin{itemize}
\item[(a)] If $\K=\R$ and $f\leq p$ on $M$ then there exists a linear functional $F$ defined on $X$ which extends $f$ such that $F\leq p$. If $p$ is a seminorm then $|F|\leq p$.
\item[(b)] If $\K=\C$, $p$ is a seminorm and $|f|\leq p$ on $M$ then there is a linear functional $F$ on $X$ which extends $f$ and such that $|F|\leq p$.
\end{itemize}
\end{theorem}
\begin{proof}
The general case for real vector spaces follows by an application of Zorn's lemma. For the complex case, suppose that $p$ is a seminorm on $X$ aud $|f|\leq p$ on $M$. Let $\phi=\Re(f)$. By (a) there is a real linear extension $\Phi$ of $\phi$ to $X$ such that $\Phi(x)\leq p(x)$ for all $x\in X$. Let $F(x)=\Phi(x)-i\Phi(ix)$ as in Proposition~\ref{linear functional real and complex}. Then $F$ is a complex linear extension of $f$, and as in the proof of Proposition~\ref{linear functional real and complex}, if $\alpha=\overline{\sgn F(x)}$ we have 
\[|F(x)|=\alpha F(x)=F(\alpha x)=\Re(F(\alpha x))=\Phi(\alpha x)\leq p(\alpha x)=p(x).\]
This finishes the proof. 
\end{proof}
\begin{corollary}\label{LCS continuous functional extension}
Let $M$ be a subspace of a locally convex space $X$ and let $f$ be a continuous linear functional on $M$. Then $f$ has a continuous linear extension $F$ on $X$.
\end{corollary}
\begin{proof}
As a subspace of a locally convex space, $M$ is locally convex in its relative topology. Moreover, if $\mathscr{P}$ is the family of continuous seminorms on $X$ then $\mathscr{P}|_M$ generates the relative topology on $M$. Thus, if $f$ is a continuous linear form on $M$, there exists $p\in\mathscr{P}$ such that $|f|\leq p$ on $M$ (Proposition~\ref{LCS as image continuity iff}). By Theorem~\ref{Hahn-Banach dominated extension} there exists a linear extension $F$ of $f$ such that $|F|\leq p$. Continuity of $F$ follows from Proposition~\ref{LCS as image continuity iff}.
\end{proof}
As an immediate consequence of the dominated version, we have the following result.
\begin{proposition}\label{TVS sublinear induced linear functional}
Let $p$ be a sublinear functional on a real vector space $X$. For any $w\in X$, there exists a linear functional $F$ on $X$ such that:
\begin{itemize}
\item[(a)] $F(w)=p(w)$.
\item[(b)] $-p(-x)\leq F(x)\leq p(x)$ for all $x\in X$.
\item[(c)] If $p$ is a seminorm then $|F|\leq p$.
\end{itemize}
In either case, if $p$ is continuous then $F$ is continuous
\end{proposition}
\begin{proof}
Let $M=\R w$ and consider $f:M\to\R,aw\mapsto ap(w)$. Clearly $f$ is a linear functional; we show that $f\leq p$ on $M$. For $a\geq 0$, $f(aw)=ap(w)=p(aw)$. Note that $0=p(w-w)\leq p(w)+p(-w)$; hence $p(w)\geq -p(-w)$ and for $a<0$ we have
\[f(aw)=ap(w)\leq -ap(-w)=p(aw).\]
By Theorem~\ref{Hahn-Banach dominated extension}, there exists a linear functional $F$ on $X$ extending $f$ such that $F\leq p$ everywhere. Since $-F(x)=F(-x)\leq p(-x)$, it follows that $-F(-x)=F(x)\geq -p(-x)$. As to (c), if $p$ is a seminorm, $p(-x)=p(x)$ and the result follows from (b). The assertion about continuity follows by
proving continuity at $0$.
\end{proof}
In Example~\ref{TVS uncomplemented eg} we showed that the subspace $c_0$ (null sequences) of $\ell^\infty$ did not have a topological complement. We can now prove an affirmative result about complements.
\begin{proposition}
Let $X$ be a Hausdorff topological vector space and let $M$ be a finite-dimensional (or codimensional) subspace. Then $M$ has a topological complement.
\end{proposition}
\begin{proof}
If $M$ is finite-codimensional, then its algebraic complement $N$ is finite dimensional. Hence we can only consider the finite-dimensional case.   By Proposition~\ref{TVS topological complement iff} it suffices to show that there is a continuous projection $P$ from $X$ onto $M$. Let $\{x_1,\dots,x_n\}$ be a basis for $M$. For any scalars $a_1,\dots,a_n$ and each $1\leq k\leq n$, let $f_k(\sum_ia_ix_i)=a_k$. Since $M$ is finite-dimensional, the $f_k$'s are continuous by Proposition~\ref{TVS finite dim prop}(f) and therefore may be extended to continuous linear functionals $F_k$ on $X$. Then the map $P:X\to M,x\mapsto\sum_iF_i(x)x_i$ is continuous. Since, for each $k$, $P(x_k)=\sum_if_i(x_k)x_i=x_k$, it follows that $P^2=P$.
\end{proof}
\section{Consequences for normed spaces}
\begin{proposition}\label{Hahn-Banach norm-preserving extension}
Let $X$ be a normed space. Then if $f$ is a continuous linear functional defined on a linear subspace $M$ of $X$, there exists a continuous linear extension $F$ of $f$ such that $\|F\|=\|f\|$.
\end{proposition}
\begin{proof}
Let the linear functional $f$ be continuous on $M$. As such, $|f|\leq\|f\|\|\cdot\|$. Theorem~\ref{Hahn-Banach dominated extension} yields a continuous linear extension $F$ of $f$ such that $|F|\leq\|f\|\|\cdot\|$ on $X$. This immediately implies that $\|F\|\leq\|f\|$, whereas the fact that the unit ball of $M$ is a subset of the unit ball of $X$ implies that $\|F\|\geq\|f\|$, thus they equal.
\end{proof}
Some immediate implications of Proposition~\ref{Hahn-Banach norm-preserving extension} are collected in the following proposition.
\begin{proposition}\label{NVS linear functional prop}
Let $X$ be a normed vector space.
\begin{itemize}
\item[(a)] For any $x\in X$, there is a continuous linear functional $f$ such that $f(x)=\|x\|$ and $\|f\|=1$. Therefore $x=0$ iff every $f\in X^*$ vanishes on $x$.
\item[(b)] For any $x\in X$, we have $\|x\|=\sup\{|f(x)|:\|f\|=1\}$.
\item[(c)] Suppose $M$ is a subspace of $X$ and $x\notin\widebar{M}$. Then $\delta:=d(w,M)>0$ and there exists $f\in X^*$ with $\|f\|=1$ that vanishes on $M$ and $f(w)=\delta$. Hence $w\in\widebar{M}$ iff every continuous linear functional that vanishes on $M$ vanishes on $w$.
\item[(d)] Given a finite set $\{x_1,\dots,x_n\}$ of linearly independent vectors, there exists $\{f_1,\dots,f_n\}$ such that $f_i(x_j)=\delta_{ij}$.
\end{itemize}
\end{proposition}
\begin{proof}
To prove (c), define $f$ on $M+\K x$ by $f(m+ax)=a\delta$. Then $f(x)=\delta$, $f|_{M}=0$ and for $a\neq 0$,
\[|f(m+ax)|=|a|\delta=|a|d(x,M)\leq a\|a^{-1}m+x\|=\|m+ax\|.\]
Thus Proposition~\ref{Hahn-Banach norm-preserving extension} can be applied, with $p(x)=\|x\|$ and $M+\K x$. Note that (a) is the special case of (c) with $Y=\{0\}$, and (b) follows immediately.\par
To prove (d), let $M$ denote the linear span of $\{x_1,\dots,x_n\}$. For each $i=1,\dots,n$, there exist linear functionals $f_i$ defined on $M$ such that $f_i(x_j)=\delta_{ij}$. As $M$ is finite-dimensional, each $f_i$ is continuous by Proposition~\ref{TVS finite dim prop}(f). It only remains to extend each $f_i$ to $X$ by Proposition~\ref{Hahn-Banach norm-preserving extension}.
\end{proof}
We show next that there are extensions of continuous linear functionals of strictly greater norm on any subspace of any normed space.
\begin{proposition}\label{NVS linear functional increase norm}
If $f$ is a continuous linear functional defined on the closed proper subspace $M$ of the normed space $X$ then there are continuous linear extensions $F$ of $f$ with $\|F\|>\|f\|$.
\end{proposition}
\begin{proof}
Choose a unit vector $x\notin M$, let $\delta>0$ be the distance from $x$ to $M$ and choose a scalar $b>\|f\|$. Define $g$ on $M\oplus\K x$ by taking $g(m+ax)=f(m)+ab$. Clearly $g$ is linear and $g|_M=f$. As to its continuity, suppose $(m_n)$ and $(a_n)$ are sequences from $M$ and $\K$, respectively, such that $m_n+a_nx\to 0$. By the proof of Proposition~\ref{NVS linear functional prop}, $|a_n\delta|\leq\|m_n+a_nx\|$ for every $n$ and therefore $a_n\to 0$; this implies $m_n\to 0$, so $g(m_n+a_nx)=f(m_n)+a_nb\to 0$ and $g$ is continuous. Since $g(x)=b>\|f\|$, it follows that $\|g\|\geq\|f\|$. Finally, extend $g$ to $X^*$ by Proposition~\ref{Hahn-Banach norm-preserving extension}.
\end{proof}
\section{Minimal sublinear functionals}
We denote the class of sublinear functionals on a vector space $X$ by $X^{\#}$. $X^{\#}$ is not a linear space but it is closed under the pointwise operations of addition and multiplication by positive scalars. We order $X^{\#}$ by taking, for $p,q\in X^{\#}$, $p\leq q$ to mean that $p(x)\leq q(x)$ for every $x$ in $X$.
\begin{proposition}\label{sublinear is linear iff p(-x)+p(x)=0}
If $p$ is a sublinear functional on a real vector space $X$ then the following are equivalent:
\begin{itemize}
\item[(\rmnum{1})] $p$ is linear.
\item[(\rmnum{2})] $p(x)+p(-x)=0$ for every $x\in X$.
\item[(\rmnum{3})] $p(x)+p(-x)\leq 0$ for every $x\in X$.
\end{itemize}
\end{proposition}
\begin{proof}
If $p$ is linear clearly (\rmnum{2}) and (\rmnum{3}) holds. Since $0=p(0)\leq p(x)+p(-x)$, we only need to prove $(\rmnum{2})\Rightarrow(\rmnum{1})$. If (\rmnum{2}) holds then $p(x)=-p(-x)$ and it follows that, for any scalar $a$, $p(ax)=ap(x)$. For all $x,y\in X$,
\[p(x)=p(x+y-y)\leq p(x+y)+p(-y)=p(x+y)-p(y)\]
which implies that $p(x)+p(y)=p(x+y)$.
\end{proof}
The following result has many important applications; in particular, it creates a sublinear functional beneath a given sublinear functional.
\begin{proposition}\label{sublinear functional beneath}
If $p$ is a sublinear functional on a real vector space $X$ then the \textbf{auxiliary functional} $q(x)=\inf\{p(x+tw)-tp(w):t\geq 0,w\in X\}$ is a sublinear functional such that $q\leq p$.
\end{proposition}
\begin{proof}
Given $x,w\in X$ and $t\geq 0$, $p(tw)\leq p(x+tw)+p(-x)$. Thus $-p(-x)\leq p(x+tw)-p(tw)$ and we can define the auxiliary functional (associated with $p$)
\[q(x)=\inf\{p(x+tw)-tp(w):t\geq 0,w\in X\}.\]
By letting $t=0$, it is clear that $q\leq p$. We now show that $q$ is sublinear. If $a=0$ then $p(ax+tw)-tp(w)=p(tw)-tp(w)=0$ for all $t\in 0$ and $w\in X$ so $q(0x)=q(0)=0$. For $a>0$,
\begin{equation*}\small
\begin{aligned}
q(ax)&=\inf\{p(ax+tw)-tp(w):t\geq 0,w\in X\}=\inf\{p(ax+a(t/a)w)-a(t/a)p(w):t\geq 0,w\in X\}\\
&=a\inf\{p(x+(t/a)w)-(t/a)p(w):t\geq 0,w\in X\}=aq(x).
\end{aligned}
\end{equation*}
For $x,y\in X$ and $s,t\geq 0$ but not both $0$, let $w=(sx+ty)/(s+t)$ so that $sx+ty=(s+t)w$. Then
\begin{equation*}\small
\begin{aligned}
q(x+y)&\leq p(x+y+(s+t)w)-(s+t)p(w)=p(x+sw+y+tw)-sp(w)-tp(w)\\
&\leq p(x+sw)-sp(w)+p(y+tw)-tp(w).
\end{aligned}
\end{equation*}
which implies that $q(x+y)\leq q(x)+q(y)$.
\end{proof}
\begin{proposition}\label{sublinear is linear iff minimal}
A sublinear functional $p$ on a real vector space $X$ is a minimal element of $X^{\#}$ iff $p$ is linear.
\end{proposition}
\begin{proof}
Suppose that $p,q\in X^{\#}$, $q\leq p$ and $p$ is linear. Since $q$ is sublinear, $0=q(x-x)\leq q(x)+q(-x)$ so $-q(-x)\leq q(x)$. Since $q(-x)\leq p(-x)=-p(x)$, it follows that $p(x)\leq -q(-x)\leq q(x)$. Therefore $p=q$ and $p$ is minimal.\par
Conversely, suppose that $p$ is a minimal element of $X^{\#}$ and let $q$ be the auxiliary sublinear functional. Since $q\leq p$, the minimality of $p$ implies that $q=p$. If we let $t=1$ and $x=-w$ then, by the definition of $q$,
\[p(-w)=q(-w)\leq p(-w+w)-p(w)= -p(w).\]
Hence $p(-w)+p(w)\leq 0$ and the linearity of $p$ follows from Proposition~\ref{sublinear is linear iff p(-x)+p(x)=0}.
\end{proof}
\begin{corollary}\label{sublinear functional linear beneath}
For any sublinear functional $p$ on a real vector space $X$ there is a linear functional $f$ on $X$ such that $f\leq p$.
\end{corollary}
\begin{proof}
Let $p\in X^{\#}$. Clearly, the class $\mathscr{L}=\{q\in X^{\#}:q\leq p\}$ is nonempty. We now show that every chain in $\mathscr{L}$ has an upper bound so that it will have a minimal element $f$ by Zorn's lemma; $f$ must be linear by Proposition~\ref{sublinear is linear iff minimal}.\par
Let $\mathscr{Q}$ be a totally ordered subset of $\mathscr{L}$. We claim that for any $x\in X$, $\{q(x):q\in\mathscr{Q}\}$ is bounded below. If not, then for each $n\in\N$ there exists $q_n\in\mathscr{P}$ such that $q_n(x)\leq -n$. Since $\mathscr{Q}$ is totally ordered, $\widetilde{q}_n=\min(q_1,\dots,q_n)\in\mathscr{P}$ for each $n\in\N$. Thus $\{\widetilde{q}_n\}$ is a decreasing sequence of sublinear functionals with the property that $\widetilde{q}_n(x)\leq -n$ for every $n$. Hence
\[0=\widetilde{q}(x-x)\leq \widetilde{q}_n(x)+\widetilde{q}_n(-x)\leq -n+\widetilde{q}(-x).\]
Hence $\widetilde{q}_n(-x)\geq n$ for each $n$. As this contradicts the fact that $q_1$ is real-valued, $q*(x)=\inf\{q(x):q\in\mathscr{Q}\}$ is a real number for every $x\in X$. For any $q\in\mathscr{Q}$, $q*\leq q\leq p$ i.e, $q^*\leq q$. Hence it only remains to show that $q*$ is sublinear to prove that $p*$ is a lower bound for $\mathscr{Q}$. By standard properties of infima, $q*(0)=0$ and $q^*(tx)=tq^*(x)$ for all $t\geq 0$. Let $q_1,q_2\in\mathscr{Q}$. Since $\mathscr{Q}$ is totally ordered, we may assume that $q_1\leq q_2$; hence, for any $x,y\in X$,
\[q^*(x+y)\leq q_1(x+y)\leq q_1(x)+q_1(y)\leq q_1(x)+q_2(y)\]
and similarly $q^*(x+y)\leq q_2(x)+q_1(y)$. Therefore $q^*(x+y)\leq\inf_{q_1,q_2\in\mathscr{Q}}[q_1(x)+q_2(y)]=q^*(x)+q^*(y)$.
\end{proof}
We can now reprove the dominated version of the Hahn-Banach theorem.
It suffices to establish the real case1 the complex version following exactly as in Theorem~\ref{Hahn-Banach dominated extension}.
\begin{theorem}
Let $X$ be a real vector space, $p$ a sublinear functional on $X$, $M$ a subspace of $X$ and $f:M\to\R$ linear functional on $M$ such that $f\leq p$. Then there is a linear functional $F$ on $X$ which extends $f$ and such that $F\leq p$.
\end{theorem}
\begin{proof}
For any $m\in M$ and $x\in X$, we have $-p(-x)+f(-m)\leq -p(-x)+p(-m)$. Thus
\[-p(-x)\leq -p(-x)+p(-m)-f(-m)\leq p(x-m)-f(-m)=p(x-m)+f(m).\]
and the map
\[q(x)=\inf\{p(x-m)+f(m):m\in M\}\]
is therefore real-valued. Two things are immediate: $q\leq f\leq p$ on $M$ (let $m=x$) and $q\leq p$ on $X$ (let $m=0$). We now show that $q$ is sublinear.\par
For each $m$ in $M$, $f(-m)\leq p(-m)$; hence $f(m)+p(-m)\geq 0$ for every $m$ in $M$ so $q(0)\geq 0$. By letting $m=0$ in $p(0-m)+f(m)$, it follows  that $q(0)=0$. For $a>0$,
\begin{equation*}\small
\begin{aligned}
q(ax)&=\inf\{p(ax-m)+f(m):m\in M\}=\inf\{ap(x-m/a)+af(m/a):m\in M\}=aq(x).
\end{aligned}
\end{equation*}
Given $x,y\in X$,
\begin{equation*}\small
\begin{aligned}
q(x+y)&=\inf\{p(x+y-m)+f(m):m\in M\}=\inf\{p(x+y-(m+n))+f(m)+f(n):m,n\in M\}\\
&\leq\inf\{p(x-m)+f(m):m\in M\}+\inf\{p(y-n)+f(n):n\in M\}=q(x)+q(y).
\end{aligned}
\end{equation*}
By Proposition~\ref{sublinear functional linear beneath} there exists a linear functional $F\leq q\leq p$ on $X$. Since $F\leq f$ on $M$, it follows that $F=f$ on $M$ since $f$ is a minimal element of $M^{\#}$.
\end{proof}
\section{Positive linear functionals}
We use the Hahn-Banach theorem to prove an extension theorem for positive linear functionals on ordered vector spaces. Recall that a partial order is a transitive, anti-symmetric, reflexive relation.
\begin{definition}[\textbf{Ordered Vector Space}]
An ordered vector space is a pair $(X,\preceq)$, where $X$ is a real vector space and $\preceq$ is a partial order on $X$ that satisfies the following two axioms for all $x,y,z\in X$ and all $\lambda\in\R$.
\begin{itemize}
\item[(O1)] If $0\preceq x$ and $0\leq\lambda$, then $0\preceq\lambda x$. 
\item[(O2)] If $x\preceq y$, then $x+z\preceq y+z$.
\end{itemize}
In this situation the set $P:=\{x\in X:0\preceq x\}$ is called the \textbf{positive cone}. A linear functional $f:X\to\R$ is called positive if $f(x)\geq 0$ for all $x\in P$.
\end{definition}
\begin{proposition}\label{ordered vector space char by cone}
Let $(X,\preceq)$ be a ordered vector space, then the positive cone $P$ satisfies the following conditions:
\begin{itemize}
\item[(P1)] If $x\in P$ and $\lambda\geq 0$, then $\lambda x\in P$.
\item[(P2)] If $x,y\in P$, then $x+y\in P$.
\item[(P3)] If $x\in P$ and $-x\in P$, then $x=0$.
\end{itemize}
Conversely, given a subset $P$ satisfying the conditions above, the relation $\preceq$ defined by
\[x\preceq y\iff y-x\in P\]
makes $X$ into a ordered vector space.
\end{proposition}
\begin{proof}
The properties (P1), (P2) and (P3) are clear from (O1) and (O2). Conversely, if $P$ satisfies these conditions, then we define
\[x\preceq y\iff y-x\in P.\]
It follows from (P3) that $\preceq$ is partial order on $X$, and the conditions (O1), (O2) follow from (P1) and (P2). This proves the claim.
\end{proof}
\begin{theorem}[\textbf{Hahn-Banach for Positive Linear Functionals}]\label{Hahn-Banach theorem for positive linear functional}
Let $(X,\preceq)$ be an ordered vector space and let $P\sub X$ be the positive cone. Let $M\sub X$ be a linear subspace satisfying the following condition
\begin{itemize}
\item[(O3)] For each $x\in X$ there exists a $m\in M$ such that $x\preceq m$.
\end{itemize}
Let $f:M\to\R$ be a positive linear functional, i.e. $f(m)\geq 0$ for all $m\in M\cap P$. Then there is a positive linear functional $F:X\to\R$ that extends $f$.
\end{theorem}
\begin{proof}
Fix an element $x\in X$. Then the set $\{m\in M:x\preceq m\}$ is nonempty by hypothesis. It follows also that there exists $m_0\in M$ such that $-x\preceq -m_0$, so that $m_0\preceq x$ by (O2). Since $f$ is positive, it follows that the restriction of $f$ to $\{m\in M:x\preceq m\}$ is bounded below. We define
\[p(x)=\inf\{f(m):m\in M,m\preceq x\}.\]
This function is clealy sublinear and satisfies $p(m)=f(m)$ for all $m\in M$. Now by the Hahn-Banach Theorem, there exists a linear functional $F:X\to\R$ such that $F|_{M}=f$ and $F(x)\leq p(x)$ for all $x\in X$. If $x\in P$, then $-x\prec 0\in M$, hence $F(-x)\leq p(-x)\leq f(0)=0$, and so $F(x)\geq 0$. This finishes the proof.
\end{proof}
\begin{example}
The assumption $(O3)$ cannot be removed in Theorem~\ref{Hahn-Banach theorem for positive linear functional}. The space $X:=BC(\R)$ of bounded continuous real valued functions on $\R$ is an ordered vector space with $f\preceq g$ if and only if $f\leq g$. The subspace $M:=C_c(\R)$ of compactly supported continuous functions does not satisfy (O3) and the positive linear functional
\[\phi:C_c(\R)\to\R,\quad f\mapsto\int_{\R}f(x)dx\]
does not extend to a positive linear functional on $BC(\R)$. In fact, let $\Phi:BC(\R)\to\R$ be such an extension, then
\[\Phi(f-\|f\|_\infty)=\Phi(f)-\|f\|_\infty\Phi(1)\leq 0\]
Thus $\Phi$ is a bounded linear operator on $BC(\R)$. However, the operator $f\mapsto\int_{\R}f$ is not bounded.
\end{example}
\section{Geometric form}
The geometric form of the Hahn-Banach theorem generalizes the notion that if a line $L$ does not pierce a sphere $B$ in $\R^3$ then $L$ is contained in a plane disjoint from $B$. In the geometric form, we replace "line" by "linear variety", "ball" by "open convex set", and "plane" by "hyperplane". This version is equivalent to the Hahn-Banach extension theorem.\par
Before proving the geometric version, we establish a connection between domination by a sublinear functional and separation and that any open convex set $G$ in a topological vector space is expressible in the form $B_p(w)=\{x\in X:p(x-w)<1\}$ for some sublinear functional $p$.\par
Let $p$ be a sublinear functional on the real or complex vector space $X$ and let $B_p=\{x\in X:p(x)<1\}$. It is easy to verify that:
\begin{itemize}
\item $B_p$ is convex.
\item $rB_p=\{x\in X:p(x)<r\}$.
\item For $w\in X$, $w+B_p=B_p(w):=\{x\in X:p(x-w)<1\}$.
\end{itemize}
\begin{proposition}\label{sublinear functional domination prop}
Let $f$ be a nontrivial linear functional on a real vector space $X$, let $p$ be a positive sublinear functional. Then
\begin{itemize}
\item[(a)] $f\leq p$ iff $f^{-1}(1)\cap B_p=\emp$.
\item[(b)] If $X$ is a topological vector space, $p$ is continuous and $f\leq p$ then $f$ is continuous. 
\end{itemize}
\end{proposition}
\begin{proof}
Suppose that $f\leq p$ and let $H=f^{-1}(1)$. Then if $x\in B_p$, $f(x)<1$, so $x\notin H$, i.e., $f^{-1}(1)\cap B_p=\emp$. Conversely, suppose there exists $x\neq 0$ such that $f(x)>p(x)\geq 0$ and let $a=1/f(x)$. Then $p(ax)<f(ax)=1$ and $ax\in f^{-1}(1)\cap B_p$.\par
Suppose $p$ is continuous and $f\leq p$. Since $B_p$ is an open convex neighborhood of $0$, it contains a disked subneighborhood $D$ of $0$; let $q$ be the gauge of $D$. Since $D\sub B_p$, $p\leq q$. Since $f\leq p$, for every $x\in X$, $f(-x)\leq p(-x)\leq q(-x)=q(x)$, i.e., $|f|\leq q$. Since $D$ is a neighborhood of $0$, $q$ is continuous by Proposition~\ref{seminorm continuous iff}; hence $f$ is continuous by Proposition~\ref{LCS as image continuity iff}.
\end{proof}
We already noted some connections between seminorms and absorbent disks. If $p$ is seminorm then $B_p$ and $\widebar{B}_p$ are absorbent disks. If $D$ is an absorbent disk in a topological vector space then its gauge
\[p_D(x):=\inf\{a>0:x\in aD\}\]
is a seminorm and $B_{p_D}\sub D\sub\widebar{B}_{p_D}$. Also, a seminorm $p$ on a topological vector space is continuous iff $B_p$ is open. If we take an absorbent convex set $D$ instead of an absorbent disk---i.e., if we drop "balanced"---and consider the gauge function $p_D$, the only difference is that $p_D$ is positive homogeneous instead of absolutely homogeneous; in other words, the gauge $p_D$ of an absorbent convex set is a positive sublinear functional.
\begin{proposition}\label{TVS absorbent convex inclusion}
If $D$ is an absorbent convex set in the topological vector space $X$, then we have
\[\Int D\sub B_{p_D}\sub D\sub\widebar{B}_{p_D}\sub\widebar{D}.\]
\end{proposition}
\begin{proof}
Let $x\in\Int D$. Since $(1+1/n)x\to x$, $x\in (1+1/n)^{-1}D$ eventually. Therefore $p_D(x)<1$ and $\Int D\sub p_D$. If $p_D(x)<1$, there exists $a>1$ such that $ax\in D$. Since $D$ is convex and $0\in D$, the line segment $[0,ax]\sub D$, which implies that $x\in D$.\par
To prove that $\widebar{B}_{p_D}\sub\widebar{D}$, first consider $x\in X$ such that $p_D(x)<1$. There must exist $a>1$ such that $ax\in D$. Since $D$ is convex and $0\in D$, this implies $x\in D\sub\widebar{D}$. If $p_D(x)=1$ then for any $0<a<1$, $ax\in B_{p_D}\sub\widebar{D}$. By the continuity of scalar multiplication, for any neighborhood $V$ of $x$, there is some open ball $B_r(1)$, $0<r<1$, such that $B_r(1)x\sub V$. It follows that $V$ contains points of $D$. Hence $x\in\widebar{D}$.
\end{proof}
\begin{proposition}\label{TVS absorbent convex guage prop}
Let $p_D$ denote the gauge of an absorbent convex set $D$ in the topological vector space $X$. Then:
\begin{itemize}
\item[(a)] If $D$ is open then $D=B_{p_D}$.
\item[(b)] $p_D$ is continuous if and only if $D$ is a neighborhood of $0$.
\item[(c)] if $p_D$ is continuous then $B_{p_D}=\Int D$ and $\widebar{B}_{p_D}=\widebar{D}$.
\end{itemize}
\end{proposition}
\begin{proof}
If $D$ is open then $D=\Int D$ and the fact that $D=B_{p_D}$ follows immediately from Proposition~\ref{TVS absorbent convex inclusion}. If $p_D$ is continuous, it is clear that $B_{p_D}$ is an open neighborhood of $0$. Conversely, suppose that $D$ is a neighborhood of $0$. Since $D\sub\widebar{B}_{p_D}$, $\widebar{B}_{p_D}$ is a neighborhood of $0$ and the continuity of $p$ follows from Proposition~\ref{seminorm continuous iff}(c).\par
If $p_D$ is continuous then $B_{p_D}$ is open and $\widebar{B}_{p_D}$ is closed. It follows from Proposition~\ref{TVS absorbent disk inclusion} that $B_{p_D}=\Int D$ and $\widebar{D}=\widebar{B}_{p_D}$.
\end{proof}
We already know that there exists nonpositive sublinear functionals, so the guages of absorbent convex sets do not exhaust sublinear functionals. Nevertheless, we have the following results.
\begin{proposition}\label{TVS sublinear functional prop}
If $X$ is a topological vector space and $p$ is a sublinear functional on $X$ then:
\begin{itemize}
\item[(a)] $p(x)-p(y)\leq p(x-y)$.
\item[(b)] $p$ is uniformly continuous iff $p$ is continuous at $0$;
\item[(c)] if $p$ is positive then $p$ is continuous iff $B_p$ is open;
\item[(d)] the open convex subsets of $X$ are those of the form $w+B_p=B_p(w)$ for some $w\in X$ and continuous positive sublinear functional $p$. 
\end{itemize}
\end{proposition}
\begin{proof}
Suppose $p$ is continuous at $0$ and $x_\alpha\to x$. Then $x_\alpha-x\to 0$ and $x-x_\alpha\to 0$. Thus, given $r>0$, for sufficiently large $\alpha$ we have $p(x-x_\alpha)<r$ and $p(x_\alpha-x)<r$. By (a) it follows that $|p(x)-p(x_\alpha)|<r$, so $p$ is continuous.\par
Suppose that $p$ is positive, $B_p$ is open and $x_\alpha\to x$. For any $r>0$, $rB_p$ is open so $x_\alpha\to rB_p$ eventually, i.e., $p(x_\alpha)<r$ eventually. Thus $p(x_\alpha)\to 0$ and $p$ is continuous at $0$; $p$ is therefore continuous by (b).\par
If $G$ is an open convex set and $w\in G$ then $D=G-w$ is an open convex neighborhood of $0$. Let $p_D$ be the gauge of $D$. By Proposition~\ref{TVS absorbent convex inclusion}, $G-w=D=B_{p_D}$.
\end{proof}
The result below was essentially first proved by Mazur for normed spaces. Bourbaki subsequently called it the geometric form of the Hahn-Banach theorem.
\begin{theorem}[\textbf{Geometric Form}]\label{Hahn-Banach geometric form}
Let $G$ be an open convex subset of a topological vector space $X$ and let $M$ be a linear variety of $X$. If $M\cap G=0$ then there is a closed hyperplane $H\sups M$ such that $H\cap G=0$.
\end{theorem}
\begin{proof}
With notation as above, first note that we may translate things so that $M$ is a subspace and prove the existence of a maximal subspace $H\sups M$ which does not meet $G$. Second, it suffices to prove the theorem in the real case. If that has been done and $\K=\C$, view $X$ as a real vector space and $M$ an $\R$-subspace. The result for the real case then yields a maximal $\R$-subspace $H\sups M$ which does not meet $G$. Then $iH\sups iM=M$ and $H\cap iH$ is a maximal complex subspace (Proposition~\ref{linear functional real and complex hyperplane}) which does not meet $G$.\par
Suppose that $\K=\R$ and that $M$ is a subspace of $X$. Since $G$ is an open convex set, by Proposition~\ref{TVS sublinear functional prop}(d) there exists a continuous positive sublinear functional $p$ such that $G=w+B_p$ for some $w\in X$. Since $G\cap M=0$, $p(m-w)>1$ for each $m\in M$. Define $f$ on $M\oplus\R w$ by taking $f(m+aw)=-a$. We show that $f\leq p$ on $M\oplus\R w$. For $a\geq 0$ this is clear since $p$ is positive. For $a<0$, since $p(y-w)\geq 1$ for $y\in M$, we have
\[f(m+aw)=-a\leq-ap(-m/a-w).\]
By the Hahn-Banach extension theorem, $f$ has an extension $F$ such that $F\leq p$ on $X$. By Proposition~\ref{sublinear functional domination prop}(b), $F$ is continuous. Since $f$ vanishes on $M$, the null space $N(F)$ of $F$ is a closed maximal subspace of $X$ which contains $M$. To see that $N(F)\cap G=0$, let $x\in N(f)$. Then $F(x)=0$; since $F\leq p$ we then get
\[p(x-w)\geq F(x-w)=F(x)-F(w)=-f(w)=1,\]
so $x\notin G$.
\end{proof}
It is common to describe separation of sets in topological spaces by means of real-valued continuous functions-as in completely regular spaces, for example. The following propositions effect separations of this type.
\begin{proposition}[\textbf{Separating Open Convex and Subspace}]\label{TVS separation of open convex with subspace}
Let $M$ be a subspace of a topological vector space $X$ and $G$ a nonempty open convex subset which does not meet $M$. Then there is a continuous linear functional $f$ on $X$ such that $f=0$ on $M$ and $f>0$ (or $\Re(f)>0$) on $G$.
\end{proposition}
\begin{proof}
We know that there is a closed maximal subspace $H\sups M$ which does not meet $G$ by Theorem~\ref{Hahn-Banach geometric form}. We may assume that $H$ is the null space $N(f)$ of some continuous linear functional $f$. Can $f$ change sign on $G$? If there are vectors $x$ and $y$ in $G$ such that $f(x)=a>0$ and $f(y)=b<0$, let $c=-b/a$ or $-a/b$, whichever makes $c\geq 1$. Now, with $c=-a/b$, we have
\[w:=\frac{x}{1+c}+\frac{cy}{1+c}\in G,\quad f(w)=0,\]
which contradicts the fact that $H\cap G=0$. It only remains to choose $f$ or $-f$.\par
In the complex case, view $X$ as a real vector space. By the real case there is a real linear functional $\phi$ such that $\phi>0$ on $G$ and $\phi=0$ on $M$. Let $f$ be the continuous linear functional defined by taking $f(x)=\phi(x)-i\phi(ix)$ at each $x\in X$ and let $H=N(\phi)$. Then $N(f)=H\cap iH\sups M$.
\end{proof}
\begin{proposition}[\textbf{Separating Point and Subspace}]\label{LCS separation of point with subspace}
Let $X$ be a locally convex space, $M\sub X$ a closed subspace and $x\notin M$. Then there is a continuous linear functional $f$ on $X$ which vanishes on $M$ but not on $x$.
\end{proposition}
\begin{proof}
Since $X$ is locally convex and $x$ is not in the dosed set $M$, there is an open convex neighborhood $G$ of $x$ which does not meet $M$. By the previous result, there is a continuous linear functional $f$ which vanishes on $M$ but not on $G$, hence not on $x$.
\end{proof}
\begin{proposition}[\textbf{Abundency of Continuous Linear Functionals}]\label{LCS abundency of dual}
Let $X$ be a locally convex space and $x\notin\widebar{\{0\}}$. Then there is a continuous linear functional $f$ such that $f(x)\neq 0$.
\end{proposition}
As a consequence of Proposition~\ref{LCS abundency of dual}, if $X$ is a locally convex Hausdorff space and $x\notin\{0\}=\widebar{\{0\}}$, there exists a continuous linear functional $f$ which does not vanish on $x$---the dual $X^*$ of a locally convex Hausdorff space $X$ is total, in other words. This and a few other corollaries like it are listed below.
\begin{proposition}\label{LCHS linear functional prop}
Let $X$ be a locally convex Hausdorff space and $X^*$ its continuous dual. Then:
\begin{itemize}
\item[(a)] $X^*$ separates the points of $X$ in the sense that if $x\neq y$ then there exists $f\in X^*$ such that $f(x)\neq f(y)$; if $x$ and $y$ are linearly independent, there exists $f\in X^*$ such that $f(x)=0$ and $f(y)=1$.
\item[(b)] If $f(x)=0$ for all $f\in X^*$, then $x=0$.
\item[(c)] If $M$ is a subspace of $X$ and $w$ a vector such that, for any $f\in X^*$, $f(M)=0$ implies $f(w)=0$ then $w\in\widebar{M}$.
\end{itemize}
\end{proposition}
\begin{proposition}\label{TVS nontrivial dual iff proper convex nbhd}
A topological vector space $X$ has a nontrivial dual iff there is a proper convex neighborhood of $0$.
\end{proposition}
\begin{proof}
Let $V$ be a proper convex neighborhood of $0$ in $X$ and let $p_V$ be its gauge. For all $r>0$ and $x\in rV$, $p_V(x)\leq r$ so $p_V$ is continuous at $0$ and therefore continuous by Proposition~\ref{TVS sublinear functional prop}(b). Since $V$ is proper there exists $w\notin V$ so that $p_V(w)>0$. By Proposition~\ref{TVS sublinear induced linear functional} there is a continuous linear functional $f$ on $X$ such that $f(w)=p_V(w)>0$, so $f$ is nontrivial. Conversely, if $f$ is a nontrivial continuous linear functional $f$ on $X$, then $V=\{x\in X:f(x)<1\}$ is a proper convex neighborhood of $0$.
\end{proof}
\section{Separation of convex sets}
In Theorem~\ref{Hahn-Banach geometric form} we showed that an open convex set $G$ and a subspace $M$ could be separated by a continuous linear functional $f$ in the sense that $f=0$ on $M$ but is positive on $G$. Thus $\sup f(M)\leq\inf f(G)$. Separating convex sets in this manner is the theme of this part. We consider three kinds:
\begin{definition}
Let $A$ and $B$ be disjoint convex subsets of a vector space $X$ and let $f$ be a real nontrivial linear functional on $X$. For $c\in\R$ and $H=f^{-1}(c)$ we say that $A$ and $B$ are:
\begin{itemize}
\item \textbf{separated by the hyperplane $\bm{H}$} if for all $x\in A$ and $y\in B$, $f(x)\leq c\leq f(y)$; equivalently, $A$ and $B$ are separated by a hyperplane iff $\sup f(A)\leq\inf f(B)$, iff $\{0\}$ is separated from $B-A$ by a hyperplane.
\item \textbf{strictly separated by the hyperplane $\bm{H}$} if for all $x\in A$ and $y\in B$, $f(x)<c<f(y)$; equivalently $\{0\}$ is strictly separated from $B-A$ by a hyperplane;
\item \textbf{strongly separated by the hyperplane $\bm{H}$} if there is some $\eps>0$ such that for all $x\in A$ and $y\in B$, $f(x)\leq c-\eps<c+\eps\leq f(y$; equivalently, $A$ and $B$ are strongly separated by a hyperplane iff $\sup f(A)\leq\inf f(B)$, iff $\{0\}$ is strongly separated from $B-A$ by a hyperplane.
\end{itemize}
\end{definition}
Let us interpret these things geometrically now. With notation above, the hyperplane $H=f^{-1}(c)$ splits $X$ into a lower, a middle and an upper convex part:
\[L=\{x\in X:f(x)\leq c\},\quad H=\{x\in X:f(x)=c\},\quad U=\{x\in X:f(x)\geq c\}.\]
$L$ and $U$ are called the \textbf{half spaces determined by $\bm{H}$}. When defined by strict inequalities, $L$ and $U$ are called the strict half spaces determined by $H$. If a set $E$ is wholly in $L$ or $U$, we say that $E$ \textbf{lies to one side of $\bm{H}$}. With respect to these conventions, $A$ and $B$ are separated if they lie in the distinct half spaces determined by $H$, and strictly separated if they lie in the strict half spaces determined by $H$. The geometric interpretation of strong separation is that $A$ and $B$ are on opposite sides of hyperplanes (namely, $f^{-1}(c-\eps)$ and $f^{-1}(c+\eps)$) "parallel" to $H$, lying a positive distance to either side of it. In the first case, $A$ and $B$ separated, $A$ and $B$ can actually meet, while if they are strictly separated, they cannot-although possibly $\sup f(A)=\inf f(B)$. They are strongly separated iff $\sup f(A)<\inf f(B)$.\par
As we show next, a convex set does not meet a hyperplane iff it lies strictly to one side of it.
\begin{proposition}
Let $X$ be a real vector space. A convex subset $G$ of $X$ is strictly to one side of a hyperplane $H$ iff $G\cap H=0$. If $H=f^{-1}(c)$ for some linear functional $f$ on $X$ and real number $c$, this means that $G\cap H=\emp$ implies that $f(G)\sub(-\infty,c)$ or $f(G)\sub(c,+\infty)$.
\end{proposition}
\begin{proof}
We use the notation of the statement. The necessity of the condition is obvious. Conversely suppose that $H=f^{-1}(c)$ for some linear functional $f$ on the real vector space $X$ and real number $c$. If the convex set $G\sub X$ is not strictly on one side of $H$, there exist $x,y\in G$ such that $f(x)=a\leq c\leq b=f(y)$. Since $c\in [a,b]$, there exist $\alpha,\beta\geq 0$, $\alpha+\beta=1$, such that $c=\alpha a+\beta b$. Since $c=f (ax+\beta y)$ and $G$ is convex, $ax+\beta y\in G\cap H$.
\end{proof}
So far everything has been purely algebraic. Our main interest, of course, centers on what happens in topological vector spaces, in situations where $f$ is a continuous linear functional ($H$ is a closed hyperplane) and the disjoint convex sets $A$ and $B$ have additional topological properties such as openness or one compact and the other closed. As our first entrance into these waters, we note the following fact.
\begin{proposition}\label{TVS closed hyperplane separation}
Let $X$ be a real topological vector space and let $H=f^{-1}(c)$ for some nontrivial continuous linear functional $f$ on $X$ and real number $c$. If $G$ has nonempty interior and lies to one side of $H$ then $\widebar{G}$ is also on one side of $H$ and $\Int G$ is strictly to one side of $H$.
\end{proposition}
\begin{proof}
Suppose $f$ is a nontrivial continuous linear functional on the real topological vector space $X$, that $G\sub X$ has nonempty interior and $f(G)\sub(-\infty,c]$ for some $c\in\R$. Then $f(\widebar{G})\sub\widebar{f(G)}\sub(-\infty,c]$. Since $f$ is nontrivial, $f$ is an open map. Thus $\Int f(G)=f(\Int G)\sub(-\infty,c)$.
\end{proof}
Having established some preliminary results, we can now prove our first separation theorems for convex sets.
\begin{theorem}[\textbf{Separating Open Convex Sets}]\label{TVS separation of convex sets}
Let $A$ and $B$ nonempty disjoint convex subsets of a real topological vector space $X$. If $A$ is open then $A$ and $B$ are separated by a closed hyperplane; if $B$ is open as well then $A$ and $B$ are strictly separated by a closed hyperplane.
\end{theorem}
\begin{proof}
The algebraic difference $A-B=\bigcup_{x\in B}(A-x)$ is open, convex, and does not meet $\{0\}$. Therefore, by Theorem~\ref{Hahn-Banach geometric form}(a), there is a continuous linear functional $f$ on $X$ which is positive on $A-B$. Hence $f(x)>f(y)$ for any $x\in A$ and $y\in B$ and $f(A)$ is bounded below. Let $a=\inf f(A)$. Then for any $x\in A$ and $y\in B$, $f(x)\geq a\geq f(y)$. Since $f$ is nontrivial, $f$ is an open map. Since $A$ is open and $f(A)\sub[a,+\infty)$, $\Int f(A)=f(\Int A)=f(A)\sub(a,+\infty)$. If $B$ is open as well then $f(B)\sub (-\infty,a)$, by the same argument.
\end{proof}
\begin{theorem}[\textbf{Strong Separation}]\label{TVS strong separation of convex sets}
Let $A$ and $B$ be disjoint nonempty convex subsets of a real topological vector space $X$.
\begin{itemize}
\item[(a)] $A$ and $B$ are strongly separated by a closed hyperplane iff there is an open convex neighborhood $V$ of $0$ such that $(A+V)\cap B=0$, or, equivalently, $V\cap (B-A)=0$.
\item[(b)] If $X$ is locally convex then $A$ and $B$ are strongly separated by a closed hyperplane iff $0\notin\widebar{B-A}$. 
\end{itemize}
\end{theorem}
\begin{proof}
Let $V$ be an open convex neighborhood of $0$ such that $(A+V)\cap B=0$. Since $A+V$ is open and convex and disjoint from $B$, there is a closed hyperplane $f^{-1}(c)$ for some continuous real linear functional $f$ and $c\in\R$ which separates $(A+V)$ and $B$ by Theorem~\ref{TVS separation of convex sets}. Since $f$ is nontrivial, it is an open map by Proposition~\ref{TVS finite dim prop}. Hence there is some positive number $\eps$ such that the open interval $(-\eps,\eps)\sub f(V)$ and we may choose $v\in V$ such that $f(v)<0$. Since $f^{-1}(c)$ separates $(A+V)$ and $B$, it follows that, for all $x\in A$ and $y\in B$, $f(x)+f(v)\geq f(y)$ (using $-f$, if necessary). Hence $\inf f(A)>\sup f(B)$.\par
Conversely, suppose that $A$ and $B$ are strongly separated by a closed hyperplane $f^{-1}(c)$ as above so that there is some $\eps>0$ such that for all $x\in A$ and $y\in B$,
\[f(x)\leq c-\eps<c+\eps\leq f(y).\]
Let $V=f^{-1}((-\eps,\eps))$. Clearly, $V$ is an open convex neighborhood of $0$ and if $(A+V)\cap B\neq 0$, there are clements $x\in A$, $y\in B$, and $v\in V$ such that $f(y)=f(x)+f(v)$. This implies that $|f(y)-f(x)|=|f(v)|<\eps$, which is a contradiction. We conclude that $(A+V)\cap B=\emp$. The second claim follows from (a).
\end{proof}
\begin{theorem}\label{LCS separation of closed and compact convex sets}
Let $A$ and $B$ be nonempty disjoint convex subsets of the locally convex space $X$.
\begin{itemize}
\item[(a)] If $A$ is closed and $B$ is compact then they are strongly separated by a closed hyperplane, i.e., there exists a continuous linear functional $f$ on $X$ such that $\sup f(A)<\inf f(B)$.
\item[(b)] If $A$ is closed and $w\notin A$, there is a continuous linear functional $f$ on $X$ such that $f(w)>\sup f(A)$.
\end{itemize}
\end{theorem}
\begin{proof}
Once (a) is established, (b) follows by taking $B=\{w\}$. Therefore we prove only (a). Suppose that $A$ is closed and $B$ is compact. The compact set $B$ is contained in the open set $A^c$ so, by Proposition~\ref{topological group neighborhood of compact} and the local convexity of $X$, there is a convex neighborhood $V$ of $0$ such that $B+V\sub A^c$. The desired result now follows from Theorem~\ref{TVS strong separation of convex sets}.
\end{proof}
In $\R^2$ a square is the intersection of four closed half planes, closed half spaces actually; triangles are the intersection of three closed half planes. A circle is the intersection of the closed half planes determined by its tangents. The generalization of these notions to general convex subsets of locally convex spaces is contained in the following proposition.
\begin{proposition}[\textbf{Structure of Subspaces and Convex Sets}]\label{LCS intersection of hyperplane and half plane}
Let $X$ be a locally convex space. Then:
\begin{itemize}
\item[(a)] any closed linear variety $M\sub X$ is the intersection of all closed hyperplanes which contain it.
\item[(b)] viewing $X$ as a real space, any nonempty closed convex set $K$ is the intersection of all closed half spaces that contain it.
\end{itemize}
\end{proposition}
\begin{proof}
Since we can translate things, there is no loss of generality in assuming that $M$ is a subspace. Let $L$ denote the intersection of all closed maximal subspaces $H$ that contain $M$. Clearly $M\sub L$. To reverse the inclusion, suppose that $x\notin M$. By Proposition~\ref{LCS separation of point with subspace} there is a continuous linear functional $f$ on $X$ which is $1$ on $x$ and $0$ on $M$. Thus $M\sub f^{-1}(0)$ and $x\notin f^{-1}(0)$; therefore, $x\notin L$. It follows that $L\sub M$.\par
Let $L$ be the intersection of the closed half spaces that contain $K$. Clearly $K\sub L$. If $x\notin K$, by Theorem~\ref{LCS separation of closed and compact convex sets}(b), there is a continuous linear functional $f$ on $X$ such that $a=f(x)>\sup f(K)=b$. Thus $K$ is a subset of the half space $f^{-1}((-\infty,b])$ but $x\notin f^{-1}((-\infty,b])$. Therefore $x\notin L$.
\end{proof}
\begin{proposition}\label{LCS not in disk functional}
Let $X$ be a locally convex space, $V$ a balanced, convex neighborhood of $0$. For any $x\notin V$, there is a continuous linear functional $f$ on $X$ such that $\sup|f(V)|\leq|f(x)|$.
\end{proposition}
\begin{proof}
Let $V$ be a disk such that $x\notin V$ and let $p$ be the gauge of $V$ so that $p(x)\geq 1$ and $p\leq 1$ on $V$. On the linear space $\K x$ define $g(ax)=ap(x)$ for $a\in\K$. Since $g=p$ on $\K x$, $g$ may be extended to a continuous linear functional $f$ on $X$ such that $|f|\leq p$. Hence $\sup|f(V)|\leq\sup p(V)=1\leq p(x)=f(x)$.
\end{proof}
\newpage
\chapter{Duality}
\section{Paired spaces}
A \textbf{bilinear functional} $\langle\cdot,\cdot\rangle:X\times Y\to\K,(x,y)\mapsto\langle x,y\rangle$, is a map which is linear in either argument when the other is fixed. We usually omit explicit reference to the bilinear functional $\langle\cdot,\cdot\rangle$ and just refer to $(X,Y)$ as a \textbf{pairing} or \textbf{pair} or say that $X$ and $Y$ are \textbf{paired spaces}. The order is unimportant: We also refer to $(Y,X)$ as a pair with respect to $\langle y,x\rangle'=\langle x,y\rangle$ i.e., the same bilinear functional. If for each nonzero $x\in X$ there exists $y\in Y$ such that $\langle x,y\rangle\neq 0$ then $Y$ is said to \textbf{distinguish points} of $X$; the analogous meaning is attached to $X$ \textbf{distinguishes points} of $Y$. If each vector space distinguishes the points of the other, then we call $(X, Y)$ a \textbf{dual pair}.\par
If $X$ is a linear space and $X^{\star}$ its algebraic dual---the vector space of all linear functionals on $X$---then the natural pairing of $X$ and $X^{\star}$ is that arising from the natural or canonical bilinear functional on $X\times X^{\star}:(x,x^{\star})\mapsto\langle x,x^{\star}\rangle$. The expression "natural map" is also used to indicate this bilinear functional acting on the product of $X$ and any subspace of $X^{\star}$. If $X$ is a topological vector space, then its topological (continuous) dual $X^*$, the space of all continuous linear functionals on $X$, is a conspicuous subspace of $X^{\star}$ to consider natural pairings of $X$ with. If $X$ and $Y$ are paired spaces and $y\in Y$, then the map
\[y^{\star}:X\to\K,\quad x\mapsto\langle x,y\rangle\]
is obviously a linear functional on $X$; that is, $y^{\star}\in X^{\star}$. The map $y\mapsto y^{\star}$ is generally not one-to-one. For it to be injective, it is clearly necessary and sufficient that $X$ distinguishes points of $Y$; in this case, we will usually not distinguish between $y$ and the linear functional $y^{\star}$ and refer to $y$ itself as a member of $X^{\star}$.
\begin{example}
Let $X$ be a vector space.
\begin{itemize}
\item[(a)] Under the natural pairing $(X,X^{\star})$ is a dual pair.
\item[(b)] If $M$ is a subspace of $X^{\star}$, then $(X,M)$ is a dual pair with respect to the natural pairing iff $M$ is a total subspace.
\item[(c)] If $X$ is a topological vector space and $f\neq 0$ then there exists $x\in X$ such that $f(x)\neq 0$, i.e., $X$ distinguishes points of $X^{\star}$.
\item[(d)] If $X$ is a locally convex Hausdorff space, then $X^*$ distinguishes points of $X$ by Proposition~\ref{LCHS linear functional prop} so $(X,X^*)$ is a dual pair.
\item[(e)] If $(X,\langle\cdot,\cdot\rangle)$ is an inner product space, then $(X,X)$ does not form a paired system with respect to $\langle\cdot,\cdot\rangle$ since the inner product is sesquilinear, not bilinear. 
\end{itemize}
\end{example}
In many of the results to follow, having a dual pair is not necessary: In some, validity remains intact for any pairing; in others, only something like "$Y$ distinguishes points of $X$" is needed. We have not made a consistent attempt in the sequel to always provide minimal hypotheses, however.
\section{Weak topologies}
In an inner product space $(X,\langle\cdot\,,\cdot\rangle)$ to say that $x_n\to x$ weakly means that $\langle x_n,y\rangle\to\langle x,y\rangle$ for all $y\in X$. Riesz and Hilbert made extensive and effective use of weak convergence from the early 1900s on. As a result of the invention of the Lebesgue integral there came many new types of convergence. There was nothing sacrosanct about norm convergence for the pioneers; if anything, weak convergence was the preferred form. Banach introduced weak convergence of vectors in normed spaces---namely $x_n\to x$ weakly if $f(x_n)\to f(x)$ for all $f\in X^*$---and the analogous notion for convergence of a sequence $\{f_n\}$ from $X^*$, what we call weak$^*$ convergence today. We first considered weak topologies on a topological vector space $X$.
\begin{definition}
If $X$ and $Y$ are paired vector spaces, then the map $p_y=|\langle\cdot\,,y\rangle|$ determines a seminorm on $X$ for each $y\in Y$. The weakest topology $\sigma(X,Y)$ for $X$ for which the seminorms $\{p_y:y\in Y\}$ are continuous is called the \textbf{weak topology on $\bm{X}$ for the pair $\bm{(X,Y)}$}; the space of the second argument determines the topology on the first. An analogous meaning is attached to $\sigma(Y,X)$.
\end{definition}
\begin{proposition}\label{weak topo Hausdorff iff}
For a pair $(X,Y)$, the following are equivalent:
\begin{itemize}
\item[(\rmnum{1})] $X$ distinguishes points of $Y$.
\item[(\rmnum{2})] The map $Y\to X^{\star},y\mapsto y^{\star}$ is injective.
\item[(\rmnum{3})] $\sigma(Y,X)$ is Hausdorff.
\end{itemize}
\end{proposition}
\begin{proof}
It is clear that $(\rmnum{1})\Leftrightarrow(\rmnum{2})$. If $X$ distinguishes the points of $Y$, then for each nonzero $y\in Y$, there is some $x\in X$ such that $p_x(y)=|\langle x,y\rangle|\neq 0$. Thus $\sigma(Y,X)$ is Hausdorff by Proposition~\ref{seminorm topo Hausdorff iff}. The converse can be done similarly.
\end{proof}
\begin{corollary}\label{weak star topo is Hausdorff}
For any topological vector space $X$, $X$ distinguish points in $X^*$, so the topology $\sigma(X^*,X)$, i.e., the weak$^*$ topology, is always Hausdorff.
\end{corollary}
A neighborhood base at $0$ for $\sigma(X,Y)$ is given by finite intersections of sets $\{x\in X:|(x,y)|<r\}$ with $r>0$, $y\in Y$, i.e., by sets such as
\[V_{0,r}(y_1,\dots,y_n)=\{x\in x:|\langle x,y_i\rangle|<r,i=1,\dots,n\}\]
where $n\in\N$ and $y_1,\dots,y_n\in Y$. A typical neighborhood of $z\in X$, namely $z+V_{0,r}(y_1,\dots,y_n)$ is
\[V_{z,r}(y_1,\dots,y_n)=\{x\in x:|\langle x-z,y_i\rangle|<r,i=1,\dots,n\}.\]

For a topological vector space $X$, the weak topology $\sigma(X,X^*)$ is seen to be coarser than the original topology on $X$. Hence convergence implies weakened convergence. Although we tend to think of neighborhoods as "small", weak neighborhoods of $0$ are generally of considerable girth. As to their topological obesity, weak neighborhoods of $0$ in infinite-dimensional locally convex Hausdorff spaces are unbounded (Example~\ref{TVS no bounded nbhd eg}(b)). As to their algebraic breadth, for infinite-dimensional $X$, for any $f\in X^*$, $V_{0,1}(f)$ contains $f^{-1}(0)$, a subspace of codimcnsion $1$. The weak topology $\sigma(X^*,X)$ is called the \textbf{weak$^*$ topology} for $X^*$.
\begin{example}
If $X$ is a topological vector space, it is easy to characterize weak convergence of a net $(x_\alpha)$: $x_\alpha\to x$ in $\sigma(X,X^*)$ iff for each $x^*\in X^*$ we have $\langle x_\alpha,x^*\rangle\to \langle x,x^*\rangle$. This follows from the facts that the seminorms $\{|\langle\cdot\,,x^*\rangle|:x^*\in X^*\}$ generate $\sigma(X,X^*)$ and (Proposition~\ref{LCS Cauchy net under seminorm}(a)) a net $(w_\alpha)$ in a locally convex space converges to $w$ iff $p(w_\alpha-w)\to 0$ for each seminorm $p$ from a family of seminorms which generates the topology on $X$. Similarly, we can show a net $(x_\alpha)$ is Cauchy in $\sigma(X,X^*)$ iff for every $x^*\in X^*$, the net $(\langle x_\alpha,x^*\rangle)$ is Cauchy.\par
The weak topology with its comparatively huge neighborhoods is so coarse that it makes convergence much easier. If $(X,\langle\cdot\,,\cdot\rangle)$ is a Hilbert space, for example, then all continuous linear functionals on $X$ are of the form $x\mapsto\langle x,y\rangle$. If $(e_n)$ is a sequence of orthonormal vectors, then $(e_n)$ converges weakly to $0$ but does not converge to $0$ in the norm topology: By Bessel's inequality, $\sum_n\langle e_n,y\rangle$ converges for any $y\in X$. So it follows that $\langle e_n,y\rangle\to 0$ for any $y\in X$; since $\|e_n\|=1$ for every $n$, $x_n\not\to 0$ in the norm topology. Thus, in infinite-dimensional Hilbert spaces, the weakened topology is strictly coarser than the original topology.
\end{example}
\begin{example}
As we will show, the dual of the Banach space $(c_0,\|\cdot\|_\infty)$ of null sequences is $\ell^1$. Consider the sequence $x_n=(1,\dots,1,0,\dots)$ which is $0$ after the $n$-th entry. For any $f\in X^*$ with $f(x)=\sum_na_nb_n$, $(b_n)\in\ell^1$, we have $f(x_n)=\sum_{k=1}^{n}b_k$ hence $(x_n)$ is a weak Cauchy sequence without a weak limit. Thus $c_0$ with its weak topology is not complete.
\end{example}
If $X$ and $Y$ are paired spaces and $K\sub X$ is compact with respect to $\sigma(X,Y)$, we say $K$ is \textbf{$\bm{\sigma(X,Y)}$-compact} or the less precise $K$ is \textbf{weakly compact}. Similarly, if $f$ is a scalar-valued function on $X$ which is continuous when $X$ carries $\sigma(X,Y)$, we say that $f$ is \textbf{weakly continuous} or the safer $f$ is \textbf{$\bm{\sigma(X,Y)}$-continuous}. We denote the $\sigma(X,Y)$-closure of a set $S$ by $\mathrm{cl}_{\sigma(X,Y)}S$.
\begin{example}
Let $(X,Y)$ be a pair and consider the canonical map $y\mapsto y^{\star}$ of $Y$ into the algebraic dual $X^{\star}$ of $X$. Then $\sigma(X,Y)$ is the initial topology determined by the linear maps $\{y^{\star}:y\in Y\}$. It is straightforward to verify that this is the same as the initial topology determined by the linear map
\begin{align}\label{weak topo induced from product}
A:X\to\prod_{Y\in Y}\K,\quad x\mapsto(\langle x,y\rangle)_{y\in Y}.
\end{align}
Note that $A$ is injective if and only if $Y$ distinguishes points of $X$; in this case $A$ is a relatively open map and $(X,\sigma(X, Y))$ is linearly homeomorphic to $R(A)$.
\end{example}
Certain properties of $\sigma(X,Y)$ are immediate because it is a seminorm topology. For example: a subset $B\sub X$ is weakly bounded iff $\langle B,y\rangle$ is a bounded set of scalars for each $y\in Y$. We can say more about weak boundedness, however, namely: Weak boundedness and weak total boundedness are the same thing as now show.
\begin{proposition}\label{weak topo bounded iff totally bounded}
If $(X,Y)$ is a pair and $Y$ distinguishes points of $X$, then $B\sub X$ is $\sigma(X,Y)$-bounded iff $B$ is $\sigma(X,Y)$-totally bounded.
\end{proposition}
\begin{proof}
Since totally bounded sets are generally bounded (Proposition~\ref{TVS totally bounded is bounded}), we need only prove that weakly bounded implies weakly totally bounded. To this end, suppose that $B$ is $\sigma(X,Y)$-bounded. According to Proposition~\ref{LCS bounded set iff}, $B$ is $\sigma(X,Y)$-bounded iff $\langle B,y\rangle$ is a bounded set of scalars for each $y\in Y$. By the Heine-Borel theorem however, a set of scalars is bounded iff it is relatively compact. Thus, by Tychonoff's theorem, $\prod_{y\in Y}\widebar{\langle B,y\rangle}$ is compact. With $A$ as in $(\ref{weak topo induced from product})$ above, this implies that $A(B)\sub\prod_y\langle B,y\rangle$ is relatively compact, hence totally bounded. Since $Y$ distinguishes points of $X$, $A$ is injective and it follows that $B=A^{-1}(A(B))$. Since $A$ is a linear homeomorphism of $(X,\sigma(X,Y))$ onto $R(A)$, the linear map $A^{-1}$ is uniformly continuous. Since the uniformly continuous image of a totally bounded set is totally bounded, $B$ is $\sigma(X,Y)$-totally bounded.
\end{proof}
\begin{proposition}\label{weak topo X^star is complete}
If $X$ is any vector space and $X^{\star}$ its algebraic dual, then $(X^{\star},\sigma(X^{\star},X))$ is complete.
\end{proposition}
\begin{proof}
If $(f_\alpha)$ is a Cauchy net in $(X^*,\sigma(X^*,X))$ then, for any $x\in X$, $(f_\alpha(x))$ is a Cauchy net of scalars by Proposition~\ref{LCS Cauchy net under seminorm}(b). Define the function $f$ at each $x\in X$ to be $\lim f_\alpha(x)$. The linearity of $f$ follows immediately from the linearity of each $f_\alpha$ and the continuity of scalar operations.
\end{proof}
We noted in Proposition~\ref{uniform space precompact iff totally bounded} that a closed, totally bounded subset of a complete topological group is compact. Hence, Proposition~\ref{weak topo bounded iff totally bounded} and \ref{weak topo X^star is complete} show that weak algebraic duals $(X^{\star},\sigma(X^{\star},X))$ resemble finite-dimensional spaces in that their closed and bounded subsets are compact. Generally locally convex Hausdorff spaces with this property are called \textbf{semi-Montel spaces}.\par
What are the $\sigma(X,Y)$-continuous linear functionals on $X$? Certainly, each $y\in Y$ is such a functional, but are there others? As we show next, there are no.
\begin{theorem}[\textbf{Weak Representation Theorem}]\label{weak representation theorem}
For any pair $(X,Y)$, the set $Y$ comprises all $\sigma(X,Y)$-continuous linear functionals on $X$:
\begin{itemize}
\item[(a)] If $f$ is a weakly continuous linear functional on $X$, then there exists $y\in Y$ such that $f(x)=\langle x,y\rangle$ for every $x\in X$; moreover $y$ is unique if and only if $X$ distinguishes points of $Y$.
\item[(b)] Let $X^\bot=\{y\in Y:y^{\star}=0\}$, then $(X,\sigma(X,Y))^*$ can be identified with $Y/X^\bot$.
\end{itemize}
\end{theorem}
\begin{proof}
Continuity of a linear functional $f$ on a topological vector space $X$ is equivalent to $|f|\leq p$ for some continuous seminorm $p$ on $X$ (Corollary~\ref{TVS linear functional iff seminorm}). This means for a $\sigma(X,Y)$-continuous linear functional $f$, there exist $y_1,\dots,y_n\in Y$ such that $|f(x)|\leq\max_i|\langle x,y_i\rangle|$ for each $x\in X$. For each $i=1,\dots,n$, let $p_{y_i}(x)=|\langle x,y_i\rangle|$, and set $p=\max_ip_i$. Since
\[N(p)=\bigcap_{i=1}^{n}N(p_{y_i})=\bigcap_{i=1}^{n}N(y_i^*)\sub N(f),\]
it follows that $f$ may be expressed as a linear combination $f=\sum_{i=1}^{n}a_iy_i^*=y^{\star}$ (Proposition~\ref{linear functional prop}). The uniqueness assertion and (b) are clear.
\end{proof}
\section{Polars and polar topologies}
We begin our approach to polar topologies by singling out sets of functions in $Y$ which are small on certain subsets of $X$ and developing some of their properties. For each statement about $Y$, there is, of course, a corresponding dual statement about $X$.
\begin{definition}
Let $(X,Y)$ be a pair. If $E\sub X$ then the \textbf{polar of $\bm{E}$ in $\bm{Y}$} is given by
\[E^{\circ}_Y=\{y\in Y:\sup_{x\in E}|\langle x,y\rangle|\leq 1\}.\]
If $E\sub Y$, then the \textbf{polar of $\bm{E}$ in $\bm{X}$} is given by
\[E^{\circ}_X=\{x\in X:\sup_{y\in E}|\langle x,y\rangle|\leq 1\}.\]
When there is no ambiguity, we simple write $E^\circ$ for the polar of $E$.
\end{definition}
\begin{remark}\label{absolute polar}
Some authors use the term absolute polar for what we call polar and use "polar" of $E\sub X$ for 
\[E^r:=\{y\in Y:\sup_{X\in E}\Re(x,y)\leq 1\}.\]
Clearly, $E^\circ\sub E^{r}$. But if $E$ is balanced, then $E^r=E^\circ$ by the following argument: If $\Re(\langle x,y\rangle)\leq 1$ for all $x\in E$ then $\Re(\langle ax,y\rangle)\leq 1$ for all $|a|\leq 1$---in particular for $a=e^{i\theta}$ for all $\theta$---which implies that $|\langle x,y\rangle|\leq 1$.
\end{remark}
\begin{example}[\textbf{Polars of the closed unit ball}]\label{polar of closed unit ball}
Let $X$ be a normed space with closed unit ball $B_X$ and unit sphere $S_X$. Similarly, let $B_{X^*}$ and $S_{X^*}$ be the closed unit ball and unit sphere in $X^*$. Then
\[(B_X)^{\circ}_{X^*}=(S_{X})_{X^*}^\circ=\{f\in X^*:\sup_{\|x\|\leq 1}|f(x)|\leq 1\}=\{f\in X^*:\sup_{\|x\|=1}|f(x)|\leq 1\}=B_{X^*}.\]
Similarly,
\[(B_{X^*})^\circ_X=(S_{X^*})^\circ_X=\{x\in X:\sup_{\|f\|\leq 1}|f(x)|\leq 1\}=\{x\in X:\sup_{\|f\|=1}|f(x)|\leq 1\}=B_{X}.\]
\end{example}
\begin{example}[\textbf{Polar of a Subspace}]\label{polar of subspace}
If $(X,Y)$ is a pair and $M$ is a subspace of $X$, then
\[M^\circ=\{y\in Y:\langle M,y\rangle=0\}=M^{\bot}.\]
For if $|(m,y)|\leq 1$ for each $m\in M$ then for any $r>0$ we have $|\langle m/r,y\rangle|\leq 1$, i.e., $|\langle m,y\rangle|\leq r$; and therefore $\langle m,y\rangle=0$ for each $m\in M$.
\end{example}
\begin{example}
If $X$ is a complex topological vector space, it can also be viewed as a real one. The resulted weakened topology on $X$ is the same, regardless of which perspective is adopted. Let the "real" and "complex", duals of $X$ be denoted $X^*(\R)$ and $X^*(\C)$, respectively. For $f\in X^*(\C)$, it follows from Proposition~\ref{linear functional real and complex} that $f(x)=g(x)-ig(ix)$, where $g$ denotes the real linear functional $\Re(f)$. Since $f$ is continuous, so is $g$ by Proposition~\ref{TVS linear functional continuous iff kernel closed}(b). Thus $g(x)$ and $ig(ix)$ are continuous, i.e., they belong to $X^*(\R)$. The equality of $\sigma(X,X^*(\C))$ and $\sigma(X,X^*(\R))$ follows from the observation that
\[\frac{1}{\sqrt{2}}\{g(x),g(ix)\}^{\circ}\sub\{f\}^\circ\sub\{g(x),g(ix)\}^{\circ}.\]
Thus we may deal with complex or real spaces solely in our discussion.
\end{example}
We will use the subbase theorem Proposition~\ref{TVS generating topology} to define polar topologies. Therefore, we want absorbent sets. The following result characterizes sets with absorbent polars.
\begin{proposition}\label{polar absorbent iff set bounded}
Let $(X,Y)$ be a pair and $B\sub X$. Then $B^\circ$ is an absorbent subset of $Y$ if and only if $B$ is $\sigma(X,Y)$-bounded.
\end{proposition}
\begin{proof}
By Proposition~\ref{LCS bounded set iff}, $B\sub X$ is $\sigma(X,Y)$-bounded iff $\langle B,y\rangle$ is a bounded set of scalars for each $y\in Y$. For any $y\in Y$, $\langle B,y\rangle$ is bounded iff $|\langle B,ay\rangle|\leq 1$ for sufficiently small values of $|a|$. In other words, $\langle B,y\rangle$ is bounded iff $ay\in B^{\circ}$ for sufficient small values of $|a|$-i.e., iff $E^\circ$ absorbs $y$.
\end{proof}
The elementary properties of polars are summarized in the following proposition.
\begin{proposition}\label{polar prop}
Let $(X,Y)$ be a pair and let $A$ and $B$ be subset of $X$.
\begin{itemize}
\item[(a)] $A^\circ$ is a $\sigma(Y,X)$-closed disk.
\item[(b)] If $A\sub B$ then $B^\circ\sub A^\circ$.
\item[(c)] For $a\neq 0$, $(aA)^\circ=a^{-1}A^\circ=|a|^{-1}A^\circ$.
\item[(d)] $A\sub A^{\circ\circ}=(A^\circ)^\circ$, and $A^\circ=A^{\circ\circ\circ}$; $A^{\circ\circ}$ is called the \textbf{bipolar} of $A$.
\end{itemize}
\end{proposition}
\begin{proof}
To prove that $A^\circ$ is absolutely convex, let $a_1,\dots,a_n$ be scalars such that $\sum_i|a_i|\leq 1$, and let $y_1,\dots,y_n\in A^\circ$. For any $x\in A$,
\[|\langle x,\sum_ia_iy_i\rangle|\leq\sum_i|a_i||\langle x,y_i\rangle|\leq\sum_i|a_i|\leq 1.\]
To see that $A^\circ$ is $\sigma(Y,X)$-closed, note that for any $x$ in $A$ the set $\{y\in Y:|\langle x,y\rangle\leq 1\}$ is $\sigma(Y,X)$-closed since $\sigma(Y,X)$ is the inverse image topology determined by the maps $\{\langle x,\cdot\rangle:x\in X\}$. It only remains to observe that
\[A^\circ=\bigcap_{x\in A}\{y\in Y:|\langle x,y\rangle|\leq 1\}.\]
Part (b) and (c) are immediate from definition. To prove (d), the inclusion $A\sub A^{\circ\circ} $ follows directly from the definition. Thus
\[A^{\circ\circ\circ}=(A^\circ)^{\circ\circ}\sups A^\circ,\quad A^{\circ\circ\circ}=(A^{\circ\circ})^\circ\sub A^\circ.\]
This implies (d).
\end{proof}
The results of Proposition~\ref{polar prop} show that a set can be swollen in certain ways without affecting its polar. This simplifies certain things about polar topologies.
\begin{proposition}\label{polar of hull and closure}
Let $(X,Y)$ be a pair and let $E$ be a subset of $X$. Then
\[E^\circ=(\bal(E))^\circ=(\conv(E))^\circ=(\mathrm{cl}_{\sigma(X,Y)}E)^\circ=(\mathrm{cl}_{\sigma(X,Y)}\convbal(E))^\circ.\]
\end{proposition}
\begin{proof}
All claims follow from Proposition~\ref{polar prop} and their proofs are virtually identical. For example, note that $E\sub\mathrm{cl}_{\sigma(X,Y)}\convbal(E)\sub E^{\circ\circ}$, so
\[E^{\circ}=E^{\circ\circ\circ}\sub (\mathrm{cl}_{\sigma(X,Y)}\convbal(E))^\circ\sub E^\circ.\]
This proves the claim.
\end{proof}
The preceding results about polars follow directly from definitions. The following central characterization of bipolars relies on the Hahn-Banach theorem.
\begin{theorem}[\textbf{Bipolar Theorem}]
Let $(X,Y)$ be a pair and let $E$ be a subset of $X$. Then
\[E^{\circ\circ}=\mathrm{cl}_{\sigma(X,Y)}\convbal(E).\]
\end{theorem}
\begin{proof}
Set $\widetilde{E}=\mathrm{cl}_{\sigma(X,Y)}\convbal(E)$. Proposition~\ref{polar prop} imply that $E\sub\widetilde{E}\sub E^{\circ\circ}$. By Proposition~\ref{LCS separation of closed and compact convex sets}, if $w\notin\widetilde{E}$, there exists a $\sigma(X,Y)$-continuous real linear functional $f$ on $X$ such that $\sup_{x\in\widetilde{E}}f(x)=a<f(w)$. Since $\widetilde{E}$ is balanced, it contains $0$, and therefore $a>0$. Hence, we may replace $f$ by $f/a$ to get $\sup_{x\in\widetilde{E}}f(x)=1<f(w)$.\par
Now let $F$ be the complex linear functional $F(x)=f(x)-if(ix)$. By Theorem~\ref{weak representation theorem} there is some $y\in Y$ such that $F(x)=\langle x,y\rangle$ for all $x\in X$. Since $\widetilde{E}$ is balanced and $\Re(\langle x,y\rangle)=F(x)\leq 1$ for all $x\in\widetilde{E}$, it follows that $\langle x,y\rangle\leq 1$ for all $x\in\widetilde{E}$ (Remark~\ref{absolute polar}); hence $y\in E^\circ$. Since $|F(w)|\geq|f(w)|>1$, it follows that $w\notin E^{\circ\circ}$, whence $E^{\circ\circ}\sub\widetilde{E}$.
\end{proof}
We characterize polars of unions and intersections next.
\begin{proposition}\label{polar union and intersection}
Let $(X,Y)$ be a pair, let $\{E_i:i\in I\}$ be a collection of subsets of $X$. Then:
\begin{itemize}
\item[(a)] $(\bigcup_{i\in I}E_i)^{\circ}=\bigcap_{i\in I}E_i^\circ$;
\item[(b)] If each $E_i$ is a $\sigma(X,Y)$-closed disk, then
\[\Big(\bigcap_{i\in I}E_i\Big)^\circ=\mathrm{cl}_{\sigma(X,Y)}\convbal\Big(\bigcup_{i\in I}E_i^\circ\Big).\] 
\end{itemize}
\end{proposition}
\begin{proof}
Part (a) follows from the observation that $y\in(\bigcup_iE_i)^\circ$ iff $|\langle x,y\rangle|\leq 1$ for all $x\in\bigcup_iE_i$, iff $y\in\bigcap_iE_i^\circ$. As for (b) note that the bipolar theorem implies that $E_i=E_i^{\circ\circ}$ for each $i\in I$. By (a), $\bigcap_iE_i=\bigcap_iE_i^{\circ\circ}=(\bigcup_iE_i^\circ)^\circ$, which implies that
\[\Big(\bigcap_iE_i\Big)^\circ=\Big(\bigcup_iE_i^\circ\Big)^{\circ\circ}=\mathrm{cl}_{\sigma(X,Y)}\convbal\Big(\bigcup_{i\in I}E_i^\circ\Big)\]
so the claim follows.
\end{proof}
Suppose $X$ is a set, $G$ a topological group and $\mathcal{S}$ a collection of subsets of $X$. Recall that the $\mathcal{S}$-topology $\mathcal{T}_\mathcal{S}$ on the group $G^X$ of maps of $X$ into $G$ has as subbasic neighborhoods of $0$ the sets
\[N(S,V)=\{f\in G^X:f(S)\sub V\}\]
where $S\in\mathcal{S}$ and $V$ is a neighborhood of $0$. Given a pair $(X,Y)$, we view $X$ as a collection of functions on $Y$, more exactly we view $X$ as a subgroup of $\K^Y$. We then consider a collection $\mathcal{S}$ of subsets $S$ of $Y$ and topologize $X$ by means of the sets $N(S,1)=S^\circ$.\par
Now if $\mathcal{S}$ is a collection of $\sigma(Y,X)$-bounded subsets of $Y$, $\mathcal{S}^\circ=\{S^\circ:S\in\mathcal{S}\}$ determines a topology $\mathcal{T}_\mathcal{S^\circ}$ for $X$. By Proposition~\ref{polar absorbent iff set bounded} and \ref{polar prop}, each $S^\circ$ is an absorbent disk in $X$. Hence, by the subbase theorem, the collection of positive multiples of finite intersections of sets from $S^\circ$ is a base at $0$ for a locally convex topology $\mathcal{T}_{\mathcal{S}^\circ}$ for $X$ that we call the \textbf{polar topology determined by $\bm{\mathcal{S}}$}.\par
The complication of having to consider finite intersections is frequently unnecessary in practice. In many cases, $\mathcal{S}$ is "directed" in the sense that for any $A,B\in\mathcal{S}$, there exists $C\in\mathcal{S}$ such that $A\cup B\sub C$; therefore $(A\cup B)^\circ=A^\circ\cap B^\circ\sups C^\circ$; hence positive multiples of the sets $S^\circ$ themselves form a base at $0$ for $\mathcal{T}_{\mathcal{S}^\circ}$. Since polar topologics are locally convex, we know that they are generated by seminorms. What seminorms? We show next that $\mathcal{T}_{\mathcal{S}^\circ}$ is generated by the gauges $p_{S^\circ}$ of the sets $S^\circ$.
\begin{proposition}
Let $(X,Y)$ be a pair and $\mathcal{S}$ be a collection of $\sigma(Y,X)$-bounded subsets of $Y$. Then the  topology $\mathcal{T}_{\mathcal{S}^\circ}$ on $X$ is generated by the seminorms $\{p_{S^\circ}:S\in\mathcal{S}\}$.
\end{proposition}
\begin{proof}
Since $\mathcal{S}$ is a collection of $\sigma(Y,X)$-bounded sets, $\langle x,S\rangle$ is a bounded set for any $x\in X$ and $S\in\mathcal{S}$. Thus, we can consider the function $p_S:X\to\R$ defined by
\[p_S(x)=\sup_{y\in S}|\langle x,y\rangle|.\]
It is routine to verify that each such $p_S$ is a seminorm on $X$. If $p_S(x)=r>0$, then $x/r\in S^\circ$ or $x\in rS^\circ$. If $t<r$, then $x/t\notin S^\circ$. In other words, $p_S(x)=p_{S^\circ}(x)=\inf\{a>0:x\in aS^\circ\}$. The polar topology determined by $\mathcal{S}$ has as a subbase at $0$ positive multiples of the sets
\[S^\circ=\{x\in X:p_S(x)=\sup_{y\in S}|\langle x,y\rangle|\leq 1\}=\widebar{B}_{p_{S^\circ}},\]
thus the claim follows.
\end{proof}
Since polar topologies are $\mathcal{S}$-topologies, several things follow immediately. For example, by Proposition~\ref{S-topo Hausdorff if}, an $\mathcal{S}$-topology is Hausdorff if $\mathcal{S}$ covers $Y$. For polar topologies in dual pairs, we have
\begin{proposition}\label{polar topo Hausdorff iff}
Let $(X,Y)$ be a pair and assume that $Y$ distinguishes points of $X$. A collection $\mathcal{S}$ of $\sigma(Y,X)$-bounded subsets of $Y$ determines a Hausdorff polar topology on $X$ iff the linear span $M$ of $\bigcup\mathcal{S}$ is $\sigma(Y,X)$-dense in $Y$.
\end{proposition}
\begin{proof}
To demonstrate sufficiency, we consider an element $x\in X$ and suppose that $p_S(x)=\sup_{y\in S}|\langle x,y\rangle|=0$ for every $S$ in $\mathcal{S}$. Since $x$ vanishes on each $S\in\mathcal{S}$, it vanishes on the linear span $M$ of $\bigcup\mathcal{S}$ as well. Since $x$ is $\sigma(Y,X)$-continuous, it must also vanish on $\mathrm{cl}_{\sigma(Y,X)}M=X$. Since $Y$ distinguishes points of $X$, this means that $x$ must be $0$. By Proposition~\ref{seminorm topo Hausdorff iff}, it follows that $\mathcal{T}_{\mathcal{S}^\circ}$ is Hausdorff.\par
Conversely, suppose that $\mathrm{cl}_{\sigma(Y,X)}M\neq Y$. By the Hahn-Banach theorem there is a nonzero $\sigma(Y,X)$-continuous linear functional $x$ on $Y$ such that $x(\mathrm{cl}_{\sigma(Y,X)}M)=\{0\}$. By the weak representation theorem we may assume that $x\in X$. Thus $x\neq 0$ and $p_S(y)=0$ for each $S\in\mathcal{S}$; hence $\mathcal{T}_{\mathcal{S}^\circ}$ is not Hausdorff (Proposition~\ref{seminorm topo Hausdorff iff}).
\end{proof}
Generally, if $\mathcal{S}$ is replaced by the collection $\mathcal{S}'$ of all subsets of finite unions of sets in $\mathcal{S}$, the $\mathcal{S}$-topology is unaffected (Proposition~\ref{S-topo operation on generating set}). For polar topologies we have the following stronger result.
\begin{proposition}\label{polar topo operation on generating set}
If $(X,Y)$ is a pair and $\mathcal{S}$ a collection of $\sigma(Y,X)$-bounded subsets of $Y$ then the polar topology determined by $\mathcal{S}$ on $X$ is not altered if $\mathcal{S}$ is replaced by any of the following collections of subsets of $Y$:
\begin{itemize}
\item[(a)] subsets of finite unions of sets in $\mathcal{S}$;
\item[(b)] scalar multiples of sets in $\mathcal{S}$;
\item[(c)] balanced hulls or convex hulls of sets in $\mathcal{S}$;
\item[(d)] $\sigma(Y,X)$-closures of sets in $\mathcal{S}$;
\item[(e)] $\sigma(Y,X)$-closures of the disked hulls of sets in $\mathcal{S}$.
\end{itemize}
\end{proposition}
\begin{proof}
Let $\mathfrak{U}_\mathcal{S}(0)$ denotes the filter of neighborhoods of $0$ in the polar topology on $X$ determined by $\mathcal{S}$.\par
If $\mathcal{S}_1$ is the collection of subsets of finite unions of sets of $\mathcal{S}$, then the polar topology $\mathcal{T}_{\mathcal{S}_1^\circ}$ is finer than $\mathcal{T}_{\mathcal{S}^\circ}$. Conversely, suppose that $S$ is a subset of $\bigcup_{j=1}^{n}S_j$, where $S_j\in\mathcal{S}$. Then $(\bigcup_{j=1}^{n}S_j)^\circ=\bigcap_{j=1}^{n}S_j^\circ$ by Proposition~\ref{polar prop}, from which it follows that $S^\circ$ is a $\mathcal{T}_{\mathcal{S}^\circ}$-neighborhood of $0$. Other parts can be proved similarly using Proposition~\ref{polar prop}.
\end{proof}
\begin{example}[\textbf{Weak Topology $\sigma(X,Y)$}]
If $\mathcal{S}$ denotes the collection of one-point subsets of $Y$, then $\mathcal{T}_{\mathcal{S}^\circ}=\sigma(X,Y)$. Thus $\sigma(X,Y)$ is a topology of pointwise convergence. In view of Proposition~\ref{polar topo operation on generating set}, $\mathcal{S}$ may be expanded to the collection of $\sigma(Y,X)$-closed absolutely convex hulls of finite subsets of $Y$---all sets of the form $\{\sum_ia_iy_i:\sum_i|a_i|\leq 1\}$ for finite collections of vectors $\{y_i\}\sub Y$---without affecting $\mathcal{T}_{\mathcal{S}^\circ}$. The basic $\sigma(X,Y)$-neighborhoods of $0$ are of the form $V_{0,r}(y_1,\dots,y_n)=\{x\in X:|\langle x,y_i\rangle|\leq r\text{ for }i=1,\dots,n\}$ where $r>0$ and $y_1,\dots,y_n\in Y$.
\end{example}
\begin{example}[\textbf{Mackey Topology $\tau(X,Y)$}]\label{Mackey topo def}
Now consider the collection $\mathcal{S}$ of all $\sigma(Y,X)$-compact disks of $Y$. The polar topology $\tau(X,Y)$ determined by $\mathcal{S}$ on $X$ is called the \textbf{Mackey topology}. Note that Proposition~\ref{polar topo operation on generating set} notwithstanding, $\tau(X,Y)$ is \textit{not} generated  the class of $\sigma(Y,X)$-compact subsets of $Y$ since the balanced convex hull of a $\sigma(Y,X)$-compact set need not be $\sigma(Y,X)$-compact.
\end{example}
\begin{example}[\textbf{Strong topology $\beta(X,Y)$}]\label{strong topo def}
The strangest polar topology for $X$ is that determined by the class of all $\sigma(Y,X)$-bounded subsets of $Y$. This topology, $\beta(X,Y)$, the topology of uniform convergence on weakly bounded subsets of $Y$, is called the \textbf{strong (strongest polar) topology}. Generally, since positive multiples and finite unions of bounded sets are bounded, the sets $\{B^\circ:B\text{ is $\sigma(Y,X)$-bounded}\}$ form a base at $0$ for $\sigma(X,Y)$. Since a weakly compact set must be weakly bounded, the Mackey topology $\tau(X,Y)$ of Example~\ref{Mackey topo def} is coarser than $\beta(X,Y)$. Given a topological vector space $X$, $(X^*,\beta(X^*,X))$ is referred to as the \textbf{strong dual} of $X$ and $\beta(X^*,X)$-bounded subsets of $X^*$ are called \textbf{strongly bounded}.
\end{example}
\section{Alaoglu's theorem}
\begin{theorem}[\textbf{Alaoglu's Theorem}]
Let $X$ be a topological vector space. If $U$ is a neighborhood of $0$ in $X$, then its polar $U^\circ$ is $\sigma(X^*,X)$-compact.
\end{theorem}
\begin{proof}
Let $U$ be a neighborhood of $0$ in $X$ and let $X^{\star}$ denote the algebraic dual of $X$. Since $U$ is absorbent and $U\sub(U^{\circ}_{X^{\star}})^{\circ}_X$, it follows that $(U^{\circ}_{X^{\star}})^{\circ}_X$ is absorbent. This implies that $((U^{\circ}_{X^{\star}})^{\circ}_X)^\circ_{X^{\star}}=U^{\circ}_{X^{\star}}$ is $\sigma(X^{\star},X)$-bounded, hence $\sigma(X^{\star},X)$-totally bounded. Since $U^{\circ}_{X^{\star}}$ is $\sigma(X^{\star},X)$-closed and $(X^{\star},\sigma(X^{\star},X))$ is complete, it follows that $U^{\circ}_{X^{\star}}$ is $\sigma(X^{\star},X)$-compact.\par
Now we show that $U^\circ_{X^{\star}}=U^\circ_{X^*}$. It it clear that $U^\circ_{X^*}\sub U^\circ_{X^{\star}}$. On the other hand, if $y\in U^{\circ}_{X^{\star}}$, then $y$ is bounded on a neighborhood $0$ in $X$, so it is continuous by Proposition~\ref{TVS bounded on a nbhd is continuous}.  Thus $U^\circ_{X^{\star}}=U^\circ_{X^*}$. Since $\sigma(X^*,X)$ is the subspace topology induced by $\sigma(X^{\star},X)$ on $X^*$, it follows that $U^\circ_{X^*}$ is $\sigma(X^*,X)$-compact.
\end{proof}
\begin{corollary}\label{NVS weak topo closed unit ball compact}
Let $X$ be a normed vector space. Then the closed unit ball in $X^*$ is $\sigma(X^*,X)$-compact.
\end{corollary}
\begin{proof}
Let $B_X$ be the closed unit ball in $X$ and $B_{X^*}$ that in $X^*$. Then $B_{X^*}=B_X^\circ$, so the claim follows from Alaoglu's theorem.
\end{proof}
\begin{corollary}\label{NVS weak topo HB prop}
Let $X$ be a normed vector space. Then the weak topology on $X^*$ has the Heine-Borel property, namely, norm-bounded closed subsets are weakly compact.
\end{corollary}
One may wonder when the weak topology can be metrized. In general this never happens, unless the space $X$ is finite dimensional. In spite of this, we can metrize weakly compact subsets in $X^*$.
\begin{proposition}\label{NVS countable total metrizability of weakly compact}
Let $K\sub X$ be a $\sigma(X,X^*)$-compact subset of a normed space $(X,\|\cdot\|)$. If there exists a countable total subset $B=\{f_n:\|f_n\|=1\}$ of $X^*$, then $\sigma(X,X^*)\cap K$ is metrizable.
\end{proposition}
\begin{proof}
Let $K\sub X$ be $\sigma(X,X^*)$-compact. The metrizability of $K$ is clear if $X$ is finite dimensional so suppose that $X$ is infinite-dimensional and that $B=\{f_n:\|f_n\|=1\}$ is a denumerable total subset of $X^*$. For $x\in X$, let
\[p(x)=\sum_{n=1}^{\infty}2^{-n}|f_n(x)|.\]
Because $\|f_n\|=1$ for all $n\in\N$ and $B$ is total, it follows that $p$ is a norm and $p(\cdot)\leq\|\cdot\|$. Since $K$ is $\sigma(X,X^*)$-compact, $K$ is $\sigma(X,X^*)$-bounded, hence norm-bounded. Assuming, with no loss of generality, that $\|x\|\leq 1$ for all $x\in K$, it follows that $|f_n(x)|\leq 1$ for all $f_n\in B$ and $x\in K$. For $r>0$, choose $N$ big enough so that $\sum_{n>N}2^{-n}<r$, so that $K\cap r\{f_1,\dots,f_{N}\}^\circ\sub 2rB_p$. Therefore, as we may assume that $0\in K$, every $p$-neighborhood of $0$ contains a $\sigma(X,X^*)\cap K$-neighborhood of $0$. This implies that the identity map $I:(K,\sigma(X,X^*)\cap K)\to(K,p)$ is continuous. But since $K$ is $\sigma(X,X^*)$-compact and $(K,p)$ is Hausdorff, $I$ is a homeomorphism and $\sigma(X,X^*)\cap K$ is the topology on $K$ induced by $p$.
\end{proof}
\begin{proposition}\label{NVS dual unit ball weakly metrizable iff}
Let $X$ be a normed space. Then the relative weak$^*$ topology of $B_{X^*}$ is induced by a metric if and only if $X$ is separable.
\end{proposition}
\begin{proof}
We now that $B_{X^*}$ is weakly compact by Alaoglu's theorem, so one dierction follows from Proposition~\ref{NVS countable total metrizability of weakly compact}. For the other, suppose that the relative weak$^*$ topology of $B_{X^*}$ is induced by some metric $d$. Let $U_n$ be the $d$-open balls of radius $1/n$. Then for each $n$, there exists a finite subset $A_n$ such that
\[U_n\sups B_{X^*}\cap A_n^\circ\sups B_{X^*}\cap A_n^\bot.\]
It then follows that $\{0\}=\bigcap_nA_n^\bot=(\bigcup_nA_n)^\bot$, and thus $\mathrm{cl}_{\sigma(X,X^*)}(\bigcup_nA_n)=X$.
\end{proof}
\begin{example}
Let $X$ be a normed space. Then the closed unit ball $B_{X^*}$ is waekly compact in $X^*$. If $\sigma(X^*,X)$ coincides with the norm topology on $X^*$ then $X^*$ is locally compact, hence is finite-dimensional. Thus the weak$^*$ topology coincides with the norm topology iff $X$ is finite-dimensional.
\end{example}
\begin{example}
Recall that $(\ell^1)^*=\ell^\infty$. Since $\ell^1$ is separable, Proposition~\ref{NVS dual unit ball weakly metrizable iff} implies that $(B_{\ell^\infty},\sigma(\ell^\infty,\ell^1))$ is metrizable. Since $B_{\ell^\infty}$ is weakly compact by Alaoglu's Theorem, $(B_{\ell^\infty},\sigma(\ell^\infty,\ell^1))$ is a complete metric space and the Baire Category Theorem is applicable. Let $\{x_n\}$ be a sequence of elements in $\ell^1$ such that $x_n\to 0$ weakly and let $\eps>0$. For each positive integer $n$ let
\[F_n=\{\phi\in B_{\ell^\infty}:|\langle x_m,\phi\rangle|\leq\eps/3\text{ for }m\geq n\}\]
It is easy to see that $F_n$ is weakly closed in $B_{\ell^\infty}$ and, because $x_n\to 0$ weakly, $\bigcup_nF_n=B_{\ell^\infty}$. By the theorem of Baire, there is an $F_N$ with non-empty weak-star interior. An equivalent metric on $(B_{\ell^\infty},\sigma(\ell^\infty,\ell^1))$ is given by
\[d(\phi,\psi)=\sum_{j=1}^{\infty}2^{-j}|\phi(j)-\psi(j)|.\]
Since $F_N$ has a nonempty weak-star interior, there is a $\phi\in F_N$ and a $\delta>0$ such that $B_{\delta}(\phi)\sub F_N$. Let $J>0$ be such that $2^{-J+1}<\delta$. Fix $n\geq N$ and define $\psi_n\in\ell^\infty$ by
\[\psi_n(j)=\begin{cases}
\phi(j)&1\leq j\leq J,\\
\sgn(x_n(j))&j>J.
\end{cases}\]
Thus $\psi_n(j)x_n(j)=|x_n(j)|$ for $j>J$. It is easy to see that $\psi_n\in B_{\ell^\infty}$. Also, $d(\phi,\psi_n)=\sum_{j=J+1}^{\infty}|\phi(j)-\psi_n(j)|\leq 2\cdot 2^{-J}<\delta$. So $\psi_n\in F_n$ and hence $|\langle x_n,\psi_n\rangle|<\eps/3$ for $n\geq N$. This implies
\[|\langle x_n,\psi_n\rangle|=\Big|\sum_{j=1}^{J}\phi(j)x_n(j)+\sum_{j=J+1}^{\infty}|x_n(j)|\Big|<\eps/3.\]
Since $x_n\to 0$ weakly, there is an $N_1>N$ such that for $n>N_1$, $\sum_{j=1}^{J}|x_n(j)|<\eps/3$ (using the coefficient functional). Combining this we get
\begin{align*}
\|x_n\|&=\sum_{j=1}^{\infty}|x_n(j)|<\eps/3+\Big|\sum_{j=J+1}^{\infty}|x_n(j)|+\sum_{j=1}^{J}\phi(j)x_n(j)\Big|+\Big|\sum_{j=1}^{J}\phi(j)x_n(j)\Big|\\
&<2\eps/3+\sum_{j=1}^{J}|x_n(j)|<\eps.
\end{align*}
This implies $x_n\to 0$ in the norm topology of $\ell^1$. Therefore, if the weak topology of $\ell^1$ is metrizable, then the weak and norm topologies on $\ell^1$ agree. However, this is not the case, by Example~\ref{weak topo metrizable iff finite dim}.
\end{example}
\begin{example}
On the other hand, in $\ell^p$ for $1<p<\infty$, a sequence $\{x_n\}$ in $\ell^p$ converges to $x\in\ell^p$ weakly if and only if $\sup_n\|x_n\|_p<\infty$ and $\lim_nx_n(j)=x(j)$ for each $j$. To prove this, we first note that, since the evaluation functionals are continuous, if $x_n\to x$ weakly in $\ell^p$ then the conditions above are satisfied. Conversely, let $\{x_n\}$ be a sequence in $\ell^p$ such that $\sup_n\|x_n\|_p\leq M$ and $\lim_nx_n(j)=x(j)$ for each $j$. Choose $y\in\ell^q=(\ell^p)^*$, and by H\"older's inequality and Minkowski's inequality, we have
\begin{align*}
\langle x_n-x,y\rangle&\leq\sum_{j=1}^{N}|x_n(j)-x(j)||y(j)|+\Big(\sum_{j=N+1}^{\infty}|x_n(j)-x(j)|^p\Big)^{1/p}\Big(\sum_{j=N+1}^{\infty}|y(j)|^q\Big)^{1/q}\\
&\leq\sum_{j=1}^{N}|x_n(j)-x(j)||y(j)|+(M+\|x\|_p)\Big(\sum_{j=N+1}^{\infty}|y(j)|^q\Big)^{1/q}
\end{align*}
and therefore
\[\limsup_n\langle x_n-x,y\rangle\leq(M+\|x\|_p)\Big(\sum_{j=N+1}^{\infty}|y(j)|^q\Big)^{1/q}\]
for each $N$. Since $y\in\ell^q$ and $1<q<\infty$, we then get $\langle x_n-x,y\rangle\to 0$, whence $x_n$ converge to $x$ weakly in $\ell^p$.
\end{example}
Hahn introduced the canonical embedding of $X$ in $X^{\star\star}$ of the next definition.
\begin{definition}
Let $X$ be a vector space. Given any $x\in X$, $\langle x,\cdot\rangle\in X^{\star\star}$. We call the map
\[J:X\to X^{\star\star},\quad x\mapsto\langle x,\cdot\rangle\]
the \textbf{canonical embedding} of $X$ in $X^{\star\star}$.
\end{definition}
It is clear that $J$ is linear. Since $X^{\star}$ distinguishes the points of $X$, $Jx=0$ implies that $Jx(x^{\star})=\langle x,x^{\star}\rangle=0$ for all $x^{\star}\in X^{\star}$ and therefore that $x=0$; hence $J$ is injective. Because of this, we usually identify $X$ and $J(X)$. If $X$ is a normed space, it follows immediately from Proposition~\ref{NVS linear functional prop} that the canonical embedding $J$ is a linear isometry.
\begin{proposition}
Let $X$ be a normed vector space. Then the spaces $X$, $X^*$, and $X^{**}$ are finite-dimensional iff one of them is finite-dimensional. In particular, finite dimensional normed vector spaces are reflexive.
\end{proposition}
\begin{proof}
If $\dim X=n$, let $\{x_1,\dots,x_n\}$ be a basis for the normed space $X$. For $i,j=1,\dots,n$, define the linear functional $f_i(x_j)=\delta_{ij}$. The $f_i$ are continuous because $X$ is finite-dimensional. To see that the $f_i$'s form a basis for $X^*$, consider $f\in X^*$ and let $a_i=f(x_i)$ so that $f=\sum_ia_if_i$. The $f_i$ are linearly independent because if $\sum_ia_if_i=0$ then, for all $0=\sum_ia_if_i(x_j)=a_j$. The same argument shows that $\dim X^{**}=n$. Conversely, suppose $\dim X^{**}=n$. Since $J$ is a linear isometry, $\dim X=\dim J(X)\leq n$. Hence, $X$ is finite-dimensional, and therefore $\dim X=\dim X^{**}=n$.
\end{proof}
How big is the normed space $X$ in its bidual $X^{**}$? As Goldstine's theorem below shows, it is $\sigma(X^{**},X^*)$-dense in $X^{**}$.
\begin{theorem}[\textbf{Goldstine}]\label{NVS weak topo dense in bidual}
If $X$ is a normed space then $X$ is $\sigma(X^{**},X^*)$-dense in $X^{**}$.
\end{theorem}
\begin{proof}
Let $B_X$, $B_{X^*}$ and $B_{X^{**}}$ denote the closed unit balls of the normed space $X$, $X^*$ and $X^{**}$, respectively. It suffices to show that $B_X$ is $\sigma(X^{**},X^*)$-dense in $B_{X^{**}}$. We may view $X$ as a subset of $X^{**}$. For the dual pair $(X^{**},X^*)$ it follows from the bipolar theorem that
\[((B_X)^{\circ}_{X^*})^{\circ}_{X^{**}}=\mathrm{cl}_{\sigma(X^{**},X^*)}B_X.\]
But the polar of $B_X$ in $X^*$ is
\begin{equation*}
\begin{aligned}
(B_X)^{\circ}_{X^*}&=\{y\in X^*:|\langle y,x\rangle_{(X^*,X^{**})}|\leq 1\text{for all $x\in B_X$}\}\\
&=\{y\in X^*:|\langle x,y\rangle_{(X,X^*)}|\leq 1\text{for all $x\in B_X$}\}=(B_X)^{\circ}_{X}=B_{X^*}
\end{aligned}
\end{equation*}
so $((B_X)^{\circ}_{X^*})^{\circ}_{X^{**}}=(B_{X^*})^\circ_{X^{**}}=B_{X^{**}}$ and this proves the claim.
\end{proof}
Now we turn to a representation theorem for normed vector spaces and locally convex Hausdorff spaces.
\begin{proposition}[\textbf{Representation Theorem}]\label{LCHS NVS representation}
\mbox{}
\begin{itemize}
\item[(a)] Each normed space $X$ is linearly isometric to a subspace of $C(T,\K)$ for some compact set $T$, with the latter endowed with the supremum norm.
\item[(b)] Each locally convex Hausdorff space is linearly homeomorphic to a subspace of $C(T,\K)$ for some locally compact topological space $T$, with the latter endowed with the compact open topology.
\end{itemize}
\end{proposition}
\begin{proof}
Let $X$ be a normed space and consider the closed unit ball $B_{X^*}$ of $X^*$. We know $B_{X^*}$ is $\sigma(X^*,X)$-compact by the Alaoglu theorem. With respect to $(X,X^*)$ in the natural pairing, consider a slightly modified canonical embedding, the linear isomorphism
\[\kappa:X\to C(B_{X^*},\K),\quad x\mapsto\langle x,\cdot\rangle.\]
By Proposition~\ref{NVS linear functional prop} we have $\|x\|=\sup_{\|f\|\leq 1}|f(x)|=\|\kappa(x)\|_\infty$. Hence $\kappa$ is linear isometry. Since $\kappa$ maps Cauchy sequences into Cauchy sequences, if $X$ is a Banach space, then it is linearly isometric to a closed subspace of $C(B_{X^*},\K)$.\par
Since any locally convex Hausdorff space $X$ is linearly homeomorphic to a subspace of a product $\prod_sX_s$ of Banach spaces $(X_s,\|\cdot\|_s)$ (Proposition~\ref{LCS subspace of product}), it suffices to demonstrate the theorem for such products. For each $s\in S$, let $B_s$ be the closed unit ball in the normed space $X_s'$. Each $B_s$ is compact in the $\sigma(X_s',X_s)$-topology $\mathcal{T}_s$. The sets $\bigcup\{\mathcal{T}_s:s\in S\}$ form a base of open sets for a topology on $T=\bigcup_{s\in S}B_s$. Since each $B_s$ is compact, it follows that $T$ is locally compact. If $E$ is a subset of $T$ which intersects infinitely many of the disjoint sets $B_s$ then $E$ cannot be compact. Consequently, any compact subset $K$ of $T$ must be contained in finitely many of the $B_s$ and therefore the compact-open topology on $C(T,\K)$ is determined by the seminonns $p_K(f)=\sup_{y\in K}|f(y)|$ where $K$ is a finite union of the $B_s$.\par
Next, consider the map $A:\prod_sX_s\to C(T,\K)$ defined as follows: Let $x=(x_s)$ and $t\in\bigcup_sB_s$. Since the sets $B_s$ are disjoint, there is a unique $s\in S$ such that $t=u_s\in B_s$ and we define $Ax(t)=\langle x_s,u_s\rangle$. Since each $x_s$ is $\mathcal{T}_s$-continuous on $B_s$ and a convergent net in $T$ must eventually belong to some $B_s$, $Ax$ is continuous. $A$ is clearly a linear isomorphism. For $K=\bigcup_{i=1}^{n}B_{s_i}$ and $x\in X$,
\begin{equation*}\small
\begin{aligned}
p_K(Ax)=\sup_{y\in K}|Ax(y)|=\max_{i}\{\sup_{y\in B_{s_i}}|Ax(y)|\}=\max_{i}\{\sup_{y\in B_{s_i}}|\langle x_s,y\rangle|\}=\max_i\{\|x\|_{s_1},\dots,\|x\|_{s_n}\}.
\end{aligned}
\end{equation*}
Hence, for any $r>0$, $p_K(Ax)<r$ iff $\|x_{s_i}\|<r$ for $i=1,\dots,n$, so
\[A^{-1}(rB_{p_K})=\{x\in\prod_sX_s:\|x_{s_i}\|<r\text{ for }i=1,\dots,n\}\]
and $A$ is a linear homeomorphism.
\end{proof}
\section{Application: the Stone-\v{C}ech compactification}
\begin{definition}
A \textbf{compactification} of a topological space $X$ is a compact Hausdorff space which contains a dense homeomorphic copy of $X$.
\end{definition}
Recall that a topological space $X$ is called \textbf{completely regular} if for every closed subset $C$ of $X$ and every point $x\in C^c$, there is a continuous function $f:X\to [0,1]$ such that $f(x)=0$ and $f(C)=\{1\}$. We now prove that a completely regular Hausdorff space $X$ possesses a compactification $\beta X$ with the property that each bounded continuous map of $X$ into $\R$ has a continuous extension to $\beta X$.\par
\begin{lemma}
Let $X$ be completely regular and $Y\sub X$. Then $Y$ is completely regular.
\end{lemma}
\begin{proof}
Suppose that $F\sub Y$ is closed in $Y$ and $x\in Y\setminus F$. Then there exists a closed set $\widetilde{F}$ in $X$ such that $\widetilde{F}\cap Y=F$, and so $x\notin\widetilde{F}$. By complete regularity of $X$ there is a continuous $f:X\to [0,1]$ such that $f(x)=0$ and $f(\widetilde{F})=\{1\}$. Now let $g=f|_{Y}$, the restriction of $f$ to $Y$, then $g$ satisfies the requirements.
\end{proof}
If $x\in X$, the evaluation map $\delta_x:BC(X)\to\K$ is defined by $\delta_x(f)=f(x)$ for every $f\in BC(X)$. It is easy to see that $\delta_x\in BC(X)^*$ and $\|\delta_x\|=1$. Let $\delta:X\to BC(X)^*$ be defined by $\delta(x)=\delta_x$. If $\{x_\alpha\}$ is a net in $X$ and $x_\alpha\to x$, then $f(x_\alpha)\to f(x)$ for every $f\in BC(X)$. This says that $\delta_{x_\alpha}\to\delta_x$ weakly. Hence $\delta:X\to BC(X)^*$ is weak$^*$ continuous. Is $\delta$ a homeomorphism of $X$ onto $\delta(X)$?
\begin{proposition}\label{evaluation on BC(X) is homeomorphism iff completely regular}
The map $\delta:X\to\delta(X)$ is a homeomorphism with the weak$^*$ topology if and only if $X$ is completely regular and Hausdorff.
\end{proposition}
\begin{proof}
Assume $X$ is completely regular and Hausdorff. If $x_1\neq x_2$, then there is an $f$ in $BC(X)$ such that $f(x_1)=1$ and $f(x_2)=0$; thus $\delta_{x_1}(f)\neq\delta_{x_2}(f)$. Hence $\delta$ is injective. To show that $\delta$ is an open map, let $U$ be an open subset of $X$ and let $x_0\in U$. Since $X$ is completely regular, there is an $f\in BC(X)$ such that $f(x_0)=1$ and $f=0$ on $X\setminus U$. Let $V=\{\mu\in BC(X)^*:\langle f,\mu\rangle>0\}$. Then $V$ is weak$^*$ open in $BC(X)^*$ and $V\cap\delta(x)=\{\delta_x:f(x)>0\}$. So if $V\cap\delta(X)$ is weak$^*$ open in $\delta(X)$ and $\delta_{x_0}\in V\cap\delta(X)\sub\Delta(U)$. Since $x_0$ was arbitrary, $\delta(U)$ is open in $\delta(X)$. Therefore $\delta$ is a homeomorphism.\par
Now assume that $\delta$ is a homeomorphism onto its image. Since $B_{BC(X)^*}$, with the weak$^*$ topology, is a compact Hausdorff space, it is completely regular. Since $\delta(X)$ is homeomorphic to a subspace of $B_{BC(X)^*}$, it is completely regular and Hausdorff. Thus $X$ is completely regular.
\end{proof}
\begin{theorem}[\textbf{Stone-\v{C}ech Compactification}]
Let $X$ be a completely regular Hausdorff space. Then there is a compact Hausdorff space $\beta X$ such that:
\begin{itemize}
\item[(a)] there is a embedding $\delta:X\to\beta X$ with dense image;
\item[(b)] if $f\in BC(X)$, then there is a continuous map $f^\beta:\beta X\to\K$ such thatf $f^\beta\circ\delta=f$;
\item[(c)] if $K$ is a compact space having these properties, then $K$ is
homeomorphic to $\beta X$.
\end{itemize}
The compact Hausdorff space $\beta X$ is called the \textbf{Stone-Cech compactiflcation} of $X$.
\end{theorem}
\begin{proof}
Let $\delta:X\to BC(X)^*$ be the map defined by $\delta(x)=\delta_x$ and let $\beta X$ be the weak-star closure of $\delta(X)$ in $BC(X)^*$. By Alaoglu's Theorem and the fact that $\|\delta_x\|=1$ for all $x$, $\beta X$ is compact. By the preceding proposition, (a) holds.\par
Fix $f\in BC(X)$ and define $f^\beta:\beta  X\to \K$ by $f^\beta(\tau)=\langle f,\tau\rangle$ for every $\tau\in\beta X\sub BC(X)^*$. Clearly $f^\beta$ is continuous and $f^\beta\circ\delta(x)=\langle f,\delta(x)\rangle=f(x)$. So $f^\beta\circ \delta=f$ and (b) holds.\par
To show that $\beta X$ is unique, assume that $K$ is a compact space and $\varphi:X\to K$ is an embedding with dense image such that for $f\in BC(X)$, there is an $\widetilde{f}\in C(K)$ such that $f=\widetilde{f}\circ\varphi$. Define $\Phi:\delta(X)\to K$ by $\Phi(\delta(x))=\varphi(x)$. The ideal is to extend $\Phi$ to a homeomorphism of $\beta X$ onto $K$. If $\tau_0\in\beta X$, then that there is a net $\{x_\alpha\}$ in $X$ such that $\delta_{x_\alpha}\to\tau_0$ in $\beta X$. Now $\{\varphi(x_\alpha)\}$ is a net in $K$ and since $K$ is compact, it has a cluster point $w_0\in K$. For $F\in C(K)$, let $f=F\circ\varphi$; so $f\in BC(X)$. Also,
\[F(\varphi(x_\alpha))=f(x_\alpha)=\langle f,\delta_{x_\alpha}\rangle\to\langle f,\tau_0\rangle=f^\beta(\tau_0).\]
Hence $F(w_0)=f^\beta(\tau_0)$ for any $F$ in $C(K)$. This implies that $w_0$ is the unique cluster point of $\{\varphi(x_\alpha)\}$; thus $\varphi(x_\alpha)\to w_0$. We define $\Phi(\tau_0)=w_0$. It is easy to see this definition does not depend on the choice of $(x_\alpha)$, so we get a map $\Phi:\beta(X)\to K$ so that if $f\in BC(X)$, then $f^\beta=
\widetilde{f}\circ\Phi$.\par
To show that $\Phi:\beta X\to K$ is continuous, let $\{\tau_\alpha\}$ be a net in $\beta X$ such that $\tau_\alpha\to\tau$. If $F\in C(K)$, let $f=F\circ\varphi$, so $f\in BC(X)$ and $\widetilde{f}=F$. Also, $f^\beta(\tau_\alpha)\to f^\beta(\tau)$. But $F(\varphi(\tau_\alpha))=f^\beta(\tau_\alpha)\to f^\beta(\tau)=F(\Phi(\tau))$. It follows that $\Phi(\tau_\alpha)\to\tau$, so $\Phi$ is continuous.\par
If $\tau_1\neq\tau_2$ in $\beta X$, then there exists $f\in BC(X)$ such that $\langle f,\tau_1\rangle\neq\langle f,\tau_2\rangle$. That is, $f^\beta(\tau_1)\neq f^\beta(\tau_2)$. Let $\widetilde{f}\in C(K)$ be such that $f=\widetilde{f}\circ\varphi$, then $\widetilde{f}(\Phi(\tau_1))\neq\widetilde{f}(\Phi(\tau_2))$. Thus $\Phi(\tau_1)\neq\Phi(\tau_2)$ and $\Phi$ is injective. Since $\Phi(\beta X)\sups\varphi(\delta(X))=\varphi(X)$, $\Phi(\beta X)$ is dense in $K$. But $\Phi(\beta X)$ is compact, so $\Phi$ is bijective. Since $\beta X$ and $K$ are compact Hausdorff, it is then a homeomorphism.
\end{proof}
\begin{example}
For $X=\N$ with the discrete topology, $\widetilde{X}=BC(X,\K)$ is just $\ell^\infty$ and $\N$ is homeomorphic to the evaluation maps $\{n':n\in\N\}$, so $\beta\N=\mathrm{cl}_{\sigma((\ell^\infty)^*,\ell^\infty)}\N$. That is about as concrete as we can get about $\beta\N$.
\end{example}
\begin{corollary}\label{completely regulat dual of BC(X) char}
If $X$ is completely regular Hausdorff and $\mu\in M(\beta X)$, define $L_\mu:BC(X)\to\K$ by
\[L_\mu(f)=\int_{\beta X}f^\beta\,d\mu\]
for eachf in $BC(X)$. Then the map $\mu\mapsto L_\mu$ is an isometric isomorphism of $M(\beta X)$ onto $BC(X)^*$.
\end{corollary}
\begin{proof}
Define $\beta:BC(X)\to C(\beta X)$ by $f\mapsto F^\beta$. It is easy to see that $\beta$ is linear. Considering $X$ as a subset of $\beta X$, the fact that $X$ is dense implies that $\beta$ is an isometry. If $g\in C(\beta X)$ and $f=g|_X$, then $g=f^\beta$; hence $\beta$ is surjective.\par
If $\mu\in M(\beta X)=C(\beta X)^*$, it is easy to check that $L_\mu\in BC(X)^*$ and $\|L_\mu\|=\|\mu\|$, since $\beta$ is an isometry. Conversely, if $L\in BC(X)^*$, then $L\circ\beta^{-1}\in C(\beta X)^*$ and $\|L\|=\|L\circ\beta^{-1}\|$. Hence there is a $\mu\in M(\beta X)$ such that $L(\beta^{-1}g)=\int g\,d\mu$ for every $g\in C(\beta X)$. Since $\beta^{-1}g=g|_X$, it follows that $L=L_\mu$.
\end{proof}
\section{Equicontinuity}
Ascoli's theorem shows a close connection between equicontinuity of a set of continuous functions and compactness in the compact-open topology. A similar sort of consanguinity exists between equicontinuous subsets of $X^*$ and $\sigma(X^*,X)$-compactness. The principal result of this part is that every locally convex topology for a vector space $X$ is a polar topology generated by the equicontinuous subsets of $X^*$.\par
If $H$ is a collection of maps and $U$ some subset of their common domain, then $H(U)$ denotes the set $\bigcup_{h\in H}h(U)$. Similarly, if $V$ is some subsets of their codomain, then $H^{-1}(V)$ denotes the set $\bigcap_{h\in H}h^{-1}(V)$. A collection $H$ of linear maps from the topological vector space $X$ into the topological space $Y$ is \textbf{equicontinuous} if for each neighborhood $V$ of $0$ in $Y$, there is a neighborhood $U$ of $0$ in $X$ such that $H(U)\sub V$. In other words, $H$ is equicontinuous if it is equicontinuous at $0$.
\begin{proposition}\label{equicontinuous weak closure}
If $X$ is a topological vector space and $H$ is an equicontinuous subset of its dual $X^*$, then the $\sigma(X^*,X)$-closure of $H$ is also an equicontinuous subset of $X^*$.
\end{proposition}
\begin{proof}
Let $H\sub X^*$ be equicontinuous. Given $r>0$ and $x\in X$, there is a neighborhood $V$ of $x$ such that $|h(x)-h(y)|<r$ for each $h\in H$ and $y\in V$. We show that this condition holds for each $f\in\mathrm{cl}_{\sigma(X^*,X)}H$ as well. To this end let $f\in\mathrm{cl}_{\sigma(X^*,X)}H$ and let $y\in V$. Since $f$ is in the $\sigma(X^*,X)$-closure of $H$, given $r'>0$, there is an $h\in H$ such that $|h(x)-f(x)|<r'$ and $|h(y)-f(y)|<r'$. We then have
\[|f(x)-f(y)|\leq|f(x)-h(x)|+|h(x)-h(y)|+|h(y)-f(y)|\leq 2r'+r.\]
It follows that $\mathrm{cl}_{\sigma(X^*,X)}H$ is equicontinuous.
\end{proof}
\begin{proposition}\label{equicontinuous convbal hull}
Let $\mathcal{L}(X,Y)$ denote the linear space of all continuous linear maps of the topological vector space $X$ into the locally convex space $Y$. Then the disk hull $\convbal(H)$ of an equicontinuous subset $H$ of $\mathcal{L}(X,Y)$ is equicontinuous. Since subsets of equicontinuous sets are equicontinuous it follows that if $H$ is equicontinuous, then so are $\bal(H)$ and $\conv(H)$.
\end{proposition}
\begin{proof}
Let $V$ be a balanced convex neighborhood of $0$ in $Y$ and choose a neighborhood $U$ of $0$ in $X$ such that $H(U)\sub V$. Then
\[\convbal(H)(U)=\{\sum_ia_ih_i(U):h_i\in H,\sum_i|a_i|\leq 1\}\sub\convbal(H(U))\sub\convbal(V)=V.\]
Thus $\convbal(H)$ is equicontinuous.
\end{proof}
In Proposition~\ref{LCS as image continuity iff} we showed that for a linear map $A$ taking a topological vector space $X$ into a locally convex space $Y$ to be continuous, it is necessary and sufficient that for each continuous serninorm $q$ on $Y$, there be a continuous seminorm $p$ on $X$ such that $q\circ A\leq p$. The following result yields a similar equicontinuity criterion for a family of linear maps.
\begin{proposition}\label{LCS equicontinuous iff}
Let $H$ be a family of linear maps mapping the locally convex space $X$ into the locally convex space $Y$. Then $H$ is equicontinuous iff for each continuous seminorm $q$ on $Y$, there is a continuous seminorm $p$ on $X$ such that $q\circ h\leq p$ for each $h\in H$.
\end{proposition}
\begin{proof}
In the notation above, suppose that, the condition holds. To prove equicontinuity it suffices to consider neighborhoods of $0$ in $Y$ of the form $B_q$ where $q$ is a continuous seminorm on $Y$. By the condition, there exists a continuous seminorm $p$ on $X$ such that for each $h\in H$, $q\circ h\leq p$. Hence $H(B_p)\sub B_q$, and so $H$ is equicontinuous.\par
Conversely, suppose that $H$ is an equicontinuous subset of $\mathcal{L}(X,Y)$. If $q$ is a continuous seminorm on $Y$, then the equicontinuity of $H$ implies that there must be some continuous seminorm $p$ on $X$ such that $H(B_p)\sub B_q$. Since $h(B_p)\sub B_q$ is equivalent to $q\circ h\leq p$ by Proposition~\ref{seminorm prop}(a) and (c), the claim follows.
\end{proof}
Now we characterize equicontinuous subsets of $X^*$ as the polars of neighborhoods of $0$ in $X$.
\begin{proposition}\label{equicontinuous iff polar is nbhd}
Let $X$ be a topological vector space and let $X^*$ be its dual. A subset $H$ of $X^*$ is equicontinuous iff (a) or (b) hold.
\begin{itemize}
\item[(a)] $H$ is contained in the polar of some neighborhood of $0$ in $X$.
\item[(b)] $H^\circ$ is a neighborhood of $0$ in $X$.
\end{itemize}
\end{proposition}
\begin{proof}
If $H\sub X^*$ is equicontinuous, then there is some neighborhood $V$ of $0$ in $X$ such that $H(V)\sub D$, the closed unit disk of $\K$ which implies that $H\sub V^\circ$. Conversely suppose $H\sub V^\circ$ for some neighborhood $V$ of $0$ in $X$. For $r>0$, then $|h(rv)|\leq r$ for all $h\in H$ and $v\in V$, i.e., $H(rV)\sub rD$.\par
We prove $(a)\Leftrightarrow(b)$. Let $V$ be a neighborhood of $0$ in $X$. If $H\sub V^\circ$, then $H^\circ\sups V^{\circ\circ}\sups V$, so $H^\circ$ is a neighborhood of $0$ in $X$. Conversely, if $H^\circ$ is a neighborhood of $0$ in $X$, then $H\sub H^{\circ\circ}$.
\end{proof}
Ascoli's theorem shows a close connection between compactness in the compact-open topology and equicontinuity. The following result bears some resemblance.
\begin{theorem}\label{weak topo equicontinuous is precompact}
If $X$ is a topological vector space and $H$ is an equicontinuous subset of $X^*$, then $H$ is relatively $\sigma(X^*,X)$-compact.
\end{theorem}
\begin{proof}
Since $H$ is equicontinuous, $H$ is contained in the polar of some neighborhood $V$ of $0$ in $X$ by Proposition~\ref{equicontinuous iff polar is nbhd}(a). By the Alaoglu theorem, $V^\circ$ is $\sigma(X^*,X)$-compact. Thus $H$ is relatively $\sigma(X^*,X)$-compoct.
\end{proof}
As a consequence of Theorem~\ref{weak topo equicontinuous is precompact}, an equicontinuous set of linear functionals is weak$^*$ bounded; we denote by $\eps(X,X^*)$ the polar topology determined by the equicontinuous subsets of $X^*$. We put this to immediate use in the following proposition, a central result in the theory of locally convex spaces.
\begin{proposition}\label{LCS topo is eps(X,X^*)}
If $\mathcal{T}$ is a locally convex topology for the linear space $X$ then $\mathcal{T}=\eps(X,X^*)$, the polar topology determined by the equicontinuous subsets of $X^*$.
\end{proposition}
\begin{proof}
Let $(X,\mathcal{T})$ be a locally convex space. Consider the neighborhood base $\mathcal{B}$ at $0$ for $(X,\mathcal{T})$ of closed disks. By the weak representation theorem, it follows that the dual of $(X,\sigma(X,X^*))$ is $X^*$. Hence the closed half spaces of $(X,\sigma(X,X^*))$ are the same as those of $(X,\mathcal{T})$. Since any $B\in\mathcal{B}$ is the intersection of all closed half spaces that contain it (Proposition~\ref{LCS intersection of hyperplane and half plane}), $\mathcal{B}$ is $\sigma(X,X^*)$-closed. By the bipolar theorem, $B=B^{\circ\circ}$ for each $B\in\mathcal{B}$.\par
If $H\sub X^*$ is equicontinuous, then $H^\circ$ is a $\mathcal{T}$-neighborhood of $0$ by Proposition~\ref{LCS equicontinuous iff}(b); consequently $\sigma(X,X^*)\sub\mathcal{T}$. On the other hand, $B^\circ$ is equicontinuous for any $B\in\mathcal{B}$ (Proposition~\ref{LCS equicontinuous iff}(a)), therefore $B=B^{\circ\circ}\in\eps(X,X^*)$ and $\mathcal{T}\sub\eps(X,X^*)$.
\end{proof}
\section{Topologies for pairs and permanence properties}
By weakening a topology on a given space, its space of continuous functions generally diminishes. Yet if $(X,\mathcal{T})$ is a locally convex space and $\mathcal{T}$ is weakened to $\sigma(X,X^*)$, the class $X^*$ of continuous linear functionals on $X$ remains the same. Similarly, the stronger the topology on a given set, the more continuous functions there are. And just as $\mathcal{T}$ may be weakened so mew hat without altering $X^*$, $\mathcal{T}$ may also be strengthened without changing $X^*$: indeed, $\mathcal{T}$ may be strengthened up to $\tau(X,X^*)$, the Mackey topology, or topology of uniform converge on the $\sigma(X^*,X)$-compact disks of $X^*$, without affecting $X^*$. This profound result concerning the allowable variability in $\mathcal{T}$ without affecting $X^*$ is called the \textbf{Mackey-Arens theorem} and is the main result of this part. It is central in duality theory and its consequences reverberate throughout the sequel.
\begin{definition}
Let $(X,Y)$ be a pair in which $X$ distinguishes points of $Y$. A locally convex topology $\mathcal{T}$ for $X$ is a \textbf{topology of the pair} (is \textbf{compatible with the pairing}) if $Y=(X,\mathcal{T})^*$.
\end{definition}
The weak topology $\sigma(X,Y)$ for a pair $(X,Y)$ in which $X$ distinguishes points of $Y$ is a topology of the pair as shown by the weak representation theorem. If the normed space $X$ is not reflexive then the norm topology on $X^*$ is not a topology of the pair $(X^*,X)$.\par
The following result says that the continuous linear functionals on a topological vector space are precisely the linear functionals that are bounded on neighborhoods of $0$.
\begin{proposition}\label{TVS dual char}
Let $X$ be a topological vector space, let $X^{\star}$ be its algebraic dual and consider the pair $(X,X^{\star})$. If $\mathcal{B}$ is a base at $0$ in $X$, then the dual $X^*=\bigcup_{B\in\mathcal{B}}B^\circ_{X^{\star}}$.
\end{proposition}
\begin{proof}
Let $\mathcal{B}$ be a base at $0$ in $X$. Any $f\in\bigcup_{B\in\mathcal{B}}B^\circ_{X^{\star}}$ is bounded on a neighborhood of $0$, hence continuous by Proposition~\ref{TVS bounded on a nbhd is continuous}. Conversely, if $f\in X^*$, then $|f|$ must be smaller than $1$ on some $B\in\mathcal{B}$ by continuity.
\end{proof}
Now let $(X,Y)$ be a pair and assume that $X$ distinguishes points of $Y$. In this case $(Y,\sigma(Y,X))$ can be embedded to $(X^{\star},\sigma(X^{\star},X))$ via the map $y\mapsto y^{\star}$. The following result is our starting point. 
\begin{proposition}[\textbf{Dual for the Mackey Topology}]\label{dual of Mackey topo}
Let $(X,Y)$ be a pair in which $X$ distinguishes points of $Y$. Then $(X,\tau(X,Y))^*=Y$.
\end{proposition}
\begin{proof}
Since a base for $\sigma(X,Y)$ is given by the sets $\mathcal{B}^\circ$, where $\mathcal{B}$ consists of $\sigma(Y,X)$-compact disk, it follows from Proposition~\ref{TVS dual char} that $(X,\tau(X,Y))^*=\bigcup_{B\in\mathcal{B}}(B^\circ_{X})^\circ_{X^{\star}}$. Since $X$ distinguishes points of $Y$, we can embedd $(Y,\sigma(Y,X))$ into $(X^{\star},\sigma(X^{\star},X))$. Then each $B\in\mathcal{B}$ is disked and compact for $\sigma(X^{\star},X)$, and the bipolar theorem implies that $B=(B^{\circ}_X)^{\circ}_{X^{\star}}$. Therefore we have
\[(X,\tau(X,Y))^*=\bigcup_{B\in\mathcal{B}}(B^\circ_{X})^\circ_{X^{\star}}=\bigcup\mathcal{B}=Y,\]
from which the claim follows.
\end{proof}
\begin{theorem}[\textbf{Topologies of the Pair}]\label{dual topo char}
Let $(X,Y)$ be a pair in which $X$ distinguishes points of $Y$. A locally convex topology $\mathcal{T}$ on $X$ is a topology of the pair iff $\mathcal{T}$ is a polar topology determined by a collection $\mathcal{S}$ of $\sigma(Y,X)$-compact disks of $Y$ which cover $Y$.
\end{theorem}
\begin{proof}
If $\mathcal{T}$ is a topology of the pair $(X,Y)$ on $X$ then $Y=(X,\mathcal{T})^*$. Consequently $\mathcal{T}=\eps(X,Y)$, the polar topology determined by the class $\mathcal{E}$ of equicontinuous subsets of $Y$. Since singletons are equicontinuous, $\mathcal{E}$ covers $Y$. As follows from Proposition~\ref{polar topo operation on generating set}, $\eps(X,Y)$ is unchanged if $\mathcal{E}$ is replaced by the collection $\mathcal{S}$ of $\sigma(Y,X)$-closures of disked hulls of sets in $\mathcal{E}$. Such sets are disks by Proposition~\ref{TVS bal conv absor int and closure}. The disked hull of an equicontinuous set is equicontinuous by Proposition~\ref{equicontinuous convbal hull} and the closure of an equicontinuous set is equicontinuous by Proposition~\ref{equicontinuous weak closure}, so the sets of $\mathcal{S}$ are $\sigma(Y,X)$-compact by Theorem~\ref{weak topo equicontinuous is precompact}.\par
Conversely, suppose that $\mathcal{S}'$ is a cover of $Y$ consisting of $\sigma(Y,X)$-compact disks. Since the topology $\mathcal{T}$ is coarser than $\tau(X,Y)$ and finer than $\sigma(X,Y)$, if follows from Proposition~\ref{dual of Mackey topo} that
\[Y=(X,\tau(X,Y))^*\sub(X,\mathcal{T})^*\sub (X,\sigma(X,Y))^*=Y\]
which proves the claim.
\end{proof}
An immediate consequence of Theorem~\ref{dual topo char} is the following result, which is also called the Mackey-Arens theorem.
\begin{theorem}[\textbf{Bound on Topologies of the Pair}]
Let $(X,Y)$ be a pair in which $X$ distinguishes points of $Y$. A locally convex topology $\mathcal{T}$ is a topology of the pair $(X,Y)$ iff $\sigma(X,Y)\sub\mathcal{T}\sub\tau(X,Y)$.
\end{theorem}
As another application of Theorem~\ref{dual topo char}, we obtain a characterization of the dual of the space $C(X,\K,c)$ of continuous $\K$-valued functions on the topological space $X$ with compact-open topology. For $x\in X$ consider the associated evaluation map (continuous linear functional)
\[x^*:C(X,\K,c)\mapsto\K,\quad f\mapsto f(x).\]
We characterize $C(X,\K,c)^*$ in terms of evaluation maps from which it follows that when $X$ is compact, $\convbal(X^*)$ is $\sigma(X^{\star},X)$-dense in $C(X,\K,c)^*$.
\begin{proposition}\label{dual of C(X,K,c)}
Let $X$ be a topological space and $\mathcal{K}$ be the collection of compact subsets in $X$. Let $\widetilde{X}=C(X,\K,c)$, then
\[\widetilde{X}'=\langle\bigcup_{K\in\mathcal{K}}\mathrm{cl}_{\sigma(\widetilde{X}^*,\widetilde{X})}\convbal(K')\rangle.\]
If $X$ is compact, then this simplifies to $\widetilde{X}'=\mathrm{cl}_{\sigma(\widetilde{X}^*,\widetilde{X})}\convbal(X^*)$.
\end{proposition}
\begin{proof}
Let $H$ be the space on the right. Since $x^*\in H$ for each $x\in X$, $(\widetilde{X},H)$ is a dual pair. To show that $H=\widetilde{X}'$, we show that the compact-open topology is the polar topology generated by a collection of $\sigma(H,\widetilde{X})$-compact disks of $H$, namely the sets $\{\mathrm{cl}_{\sigma(\widetilde{X}^*,\widetilde{X})}\convbal(K'):K\in\mathcal{K}\}$. To see that each $K^*:=\mathrm{cl}_{\sigma(\widetilde{X}^*,\widetilde{X})}\convbal(K')$ is compact, we recall that
\begin{itemize}
\item $(\widetilde{X}^*,\sigma(\widetilde{X}^*,\widetilde{X}))$ is complete.
\item A closed and totally bounded subset of a complete topological group is compact.
\item A set is $\sigma(\widetilde{X}^*,\widetilde{X})$-totally bounded iff it is $\sigma(\widetilde{X}^*,\widetilde{X})$-bounded.
\end{itemize}
As $\sigma(\widetilde{X}^*,\widetilde{X})$-compactness implies $\sigma(H,\widetilde{X})$-compactness, it suffices to show that each $K^*$ is $\sigma(\widetilde{X}^*,\widetilde{X})$. To do this, we use Proposition~\ref{LCS bounded set iff} and show that $\sup_{x\in K^*}|\langle f,x\rangle|<+\infty$ for each $f$ in $\widetilde{X}$. To this end let $x_1,\dots,x_n\in K$ and $a_1,\dots,a_n\in\K$ be such that $\sum_i|a_i|\leq 1$. Then $\sum_ia_ix_i'\in K^*$ and for any $f\in\widetilde{X}$,
\[\Big|\langle f,\sum_{i=1}^{n}a_ix_i'\rangle\Big|=\Big|\sum_{i=1}^{n}a_if(x_i)\Big|\leq\Big(\sum_i|a_i|\Big)p_K(f)\leq p_K(f)=\sup_{x\in K}|f(x)|.\]Finally, we show that
\[(K^*)^\circ=\widebar{B}_{p_K}=\{f\in\widetilde{X}:p_K(f)\leq 1\}.\]
By the argument above, $|\phi(f)|\leq p_K$ for each $f\in\widetilde{X}$ and each $\phi\in K^*$; hence $\widebar{B}_{p_K}\sub(K^*)^\circ$. To obtain the reverse inclusion, consider any $f\in(K^*)^\circ$ and any $x\in K$. Then $|f(x)|\leq 1$ and it follows that $p_K(f)\leq 1$ .
\end{proof}
The result of Proposition~\ref{LCS intersection of hyperplane and half plane}(b) states that any closed convex set in a real locally convex space $X$ is the intersection of all the closed half spaces that contain it, a half space being a set of the form $L=f^{-1}((-\infty,c])$ where $f$ is a linear functional and $c$ a real number. A half space $L$ is closed iff $f$ is continuous. Thus, if $\mathcal{T}_1$ and $\mathcal{T}_2$ are topologies of the pair $(X,X^*)$, since $(X,\mathcal{T}_1)^*=(X,\mathcal{T}_2)^*$, the class of closed convex sets is the same in either topology---we emphasize the convex here since because it is not generally true that closed implies weakly closed. We record this for future reference.
\begin{proposition}[\textbf{Permanence of Closed Convex Sets}]\label{dual topo permanence of closed convex}
Let $(X,Y)$ be a pair. Then
\begin{itemize}
\item[(a)] the class of closed convex subsets of $X$ is the same with respect to any topology of the pair;
\item[(b)] the closure of a convex subset of $X$ is the same in any topology of the pair.
\item[(c)] For any topology $\mathcal{T}$ of the pair and any $\mathcal{T}$-closed disk $B$ of $X$, $B=B^{\circ\circ}$ by the bipolar theorem.
\end{itemize}
\end{proposition}
The very definition of topology of a pair is one which ensures a certain kind of permanence, namely, of the space of continuous linear functionals. Proposition~\ref{dual topo permanence of closed convex} is an example of another kind of immutability. Our main result in this part asserts that, for any locally compact Hausdorff space $X$, the bounded sets of $X$ are the same in any topology of the dual pair $(X,X^*)$.
\begin{definition}
A \textbf{barrel} is a closed absorbent disk.
\end{definition}
Certainly, there is no dearth of barrels in a topological vector space since the closure of the disked hull of any absorbent set is a barrel; in particular, the closed absolute disked of any neighborhood of $0$ is a barrel. In locally convex spaces there is a base of barrels at $0$ (Proposition~\ref{LCS nbhd of disked sets}). We note that if $X$ is a locally convex spaces, it is not necessary to distinguish between "barrel and "weak barrel" by virtue of Proposition~\ref{dual topo permanence of closed convex}. Since dealing with polars forces consideration of weak closures, this observation simplifies (as well as makes possible) many results.\par
Barrels are important because of there connection with boundedness in polar topologis. Before showing this, we first give a characterization for barrels in a locally convex space.
\begin{proposition}\label{dual topo barrel iff polar of weak bounded}
Let $\mathcal{T}$ be a topology of the pair $(X,Y)$. Then $B$ is a barrel in $(X,\mathcal{T})$ iff $B$ is the polar of a $\sigma(Y,X)$-bounded subset of $Y$.
\end{proposition}
\begin{proof}
Let $\mathcal{T}$ be a topology of the pair $(X,Y)$. If $B$ is a barrel in $X$ then, since $B$ is absorbent, $B^\circ$ is $\sigma(Y,X)$-bounded. By the bipolar theorem, $B=B^{\circ\circ}$. Conversely, if $H\sub X$ is $\sigma(Y,X)$-bounded then $H^\circ$ is absorbent by Proposition~\ref{polar absorbent iff set bounded}, a $\sigma(X,Y)$-closed disk by Proposition~\ref{polar prop}(a).
\end{proof}
\begin{proposition}\label{polar topo bounded iff}
Let $\mathcal{T}$ be a topology of the pair $(X,Y)$ and $\mathcal{S}$ be a collection of $\sigma(X,Y)$-bounded sets that covers $X$. Then the following are quivalent:
\begin{itemize}
\item[(\rmnum{1})] $H\sub Y$ is $\mathcal{T}_{\mathcal{S}^\circ}$-bounded.
\item[(\rmnum{2})] $H^\circ$ absorb every element in $\mathcal{S}$.
\item[(\rmnum{3})] $H\sub B^\circ$ for some barrel $B$ in $X$ that absorbs every set in $\mathcal{S}$.
\end{itemize}
\end{proposition}
\begin{proof}
Let $S\in\mathcal{S}$. Then $H^\circ$ absorbs $B$ iff for some $r>0$, $rS\sub H^\circ$ and this is equivalent to $H\sub H^{\circ\circ}\sub(1/r)S^\circ$. As the sets $S^\circ$ are a base at $0$ for $\mathcal{T}_{\mathcal{S}^\circ}$, it follows that $H$ is $\mathcal{T}_{\mathcal{S}^\circ}$-bounded iff $H^\circ$ absorbs all sets in $\mathcal{S}$.\par
Let $B$ be a barrel in $X$ such that $B$ absorbs every set in $\mathcal{S}$. Then it follows from (\rmnum{2}) that $B^\circ$ is $\mathcal{T}_{\mathcal{S}^\circ}$-bounded and so is any subset of $B^\circ$.\par
Conversely, suppose that $H$ is a $\mathcal{T}_{\mathcal{S}^\circ}$-bounded set in $Y$. Since $\mathcal{S}$ covers $X$, it follows that $H$ is $\sigma(Y,X)$-bounded, so $H^\circ$ is a $\sigma(X,Y)$-barrel by Proposition~\ref{dual topo barrel iff polar of weak bounded}, hence a barrel in the original topology on $X$ by Proposition~\ref{dual topo permanence of closed convex}. Since $H$ is $\mathcal{T}_{\mathcal{S}^\circ}$-bounded, $H$ must be absorbed by the polar of any set in $\mathcal{S}$, which implies $H^\circ$ absorbs every set in $\mathcal{S}$. It only remains to observe that $H\sub H^{\circ\circ}$.
\end{proof}
Our next two results have to do with the absorbent properties of barrels.
\begin{proposition}\label{barrel absorb convex compact}
If $B$ is a barrel in the topological vector space $X$, then $B$ absorbs each convex compact subset $K$ of $X$.
\end{proposition}
\begin{proof}
Suppose that there is some $x\in K$ such that
\begin{align}\label{barrel absorb convex compact-1}
K\cap(x+V)\sub nB\text{ for some neighborhood $V$ of $0$ and $n\in\N$}.
\end{align}
In other words, suppose that $K$ contains a relative neighborhood of $x$ which is absorbed by $B$. Translated to the origin, the condition becomes $(K-x)\cap V\sub nB-x$. We contend that the existence of such an $x$, $V$, and $n$ suffices to prove the claim. To see this, we argue as follows: $K$ is compact, so $K-x$ is bounded and there must be some $a\geq 1$ such that $(K-x)\sub aV$. Since $K$ is convex and $0\in K-x$, it follows that for any $y\in K-x$,
\[(1/a)y=(1/a)y+(1-1/a)0\in K-x\]
i.e., $K-x\sub a(K-x)$. Since $(K-x)\sub aV$, we then get $K-x\sub a(K-x)\cap aV\sub a(nB-x)$, which implies that
\[K\sub a(nB-x)+x=anB+(1-a)x.\]
As $B$ is absorbent, there exists $r>0$ such that $(1-a)x\in rB$. Since $B$ is convex, $K\sub anB+rB=(an+r)B$ by Proposition~\ref{TVS convex set prop} and $B$ absorbs $K$.\par
We now show that if $(\ref{barrel absorb convex compact-1})$ is not satisfied, then $B$ is not absorbent. Assume that $K\cap(x+V)\nsubseteq nB$ for any $x\in K$, any $n\in\N$ and $V\in\mathfrak{U}(0)$. For any open neighborhood $V_0$ of $0$ and any $x_0\in K$, there exists $x_1\in K$ and an open neighborhood $V_1$ of $0$ such that $x_1+\widebar{V}_1\sub(x_0+V_0)\cap B^c$. Similarly, there exists $x_2\in K$ and an open neighborhood $V_2$ of $0$ such that $x_2+\widebar{V}_2\sub(x_1+V_1)\cap(2B)^c$. In this way we get a sequence $\{K\cap(x_n+\widebar{V}_n)\}$ of closed nonempty subsets of $K$. Since $K$ is compact, the sequence must have a nonempty intersection. If $y$ is a member of this intersection, however, then $y\notin nB$ for any positive integer $n$---in other words, $B$ is not absorbent, which is contradictory. Hence the condition must be satisfied and the proof is complete.
\end{proof}
Next, we prove Mackey's theorem that the bounded sets of a locally convex space $X$ are the same in any topology of the dual pair $(X,X^*)$. Consequently, we need not distinguish between bounded and weakly bounded subsets of a locally convex space.
\begin{theorem}[\textbf{Mackey}]\label{LCS bounded set permanence}
For any locally convex space $(X,\mathcal{T})$ the bounded sets are the same in any topology of the pair $(X,X^*)$.
\end{theorem}
\begin{proof}
Let $(X,\mathcal{T})$ be a locally convex space. If a set is $\mathcal{T}$-bounded, then it is bounded in any topology which is coarser than $\mathcal{T}$. Since $\sigma(X,X^*)\sub\mathcal{T}\sub\tau(X,X^*)$ by the Mackey-Arens theorem, it suffices to show that any $\sigma(X,X^*)$-bounded subset $B$ of $X$ is $\tau(X,X^*)$-bounded.\par
To this end, let $V$ be a closed disked $\tau(X,X^*)$-neighborhood of $0$. Since $V$ is also a $\mathcal{T}$-neighborhood of $0$, $V^\circ$ is $\sigma(X^*,X)$-compact by the Alaoglu theorem. Since $B$ is $\sigma(X,X^*)$-bounded, $B^\circ$ is a $\sigma(X^*,X)$-barrel in $X^*$ by the dual form of Proposition~\ref{dual topo barrel iff polar of weak bounded}. As such, $B^\circ$ must absorb $V^\circ$ by Proposition~\ref{barrel absorb convex compact}. Consequently, $V=V^{\circ\circ}$ absorbs $B^{\circ\circ}\sups B$ by Proposition~\ref{polar prop}(b) and the proof is complete.
\end{proof}
We showed that discontinuous linear functionals exist on any infinite-dimensional pseudometrizable linear space (Lemma~\ref{TVS metrizable infinite dim noncontinuous dual}). The next result provides instances in which a linear map that takes bounded sets into bounded sets is discontinuous.
\begin{example}[\textbf{Bounded But Not Continuous}]
We have already observed that the weak topology $\sigma(X,X^*)$ on an infinite-dimensional Hilbert space $X$ is strictly weaker than the norm topology: Any orthonormal sequence $\{e_n\}$ of vectors converges weakly to $0$ but does not converge to $0$ in the norm topology. For such a space, the identity map from $(X,\sigma(X,X^*))$ to $(X,\|\cdot\|)$ is discontinuous, something that remains true for any topological vector space $X$ where $\sigma(X,X^*)$ is strictly weaker than the original topology. The identity map takes bounded sets into bounded sets by Theorem~\ref{LCS bounded set permanence}.
\end{example}
\begin{example}[\textbf{Dual of Normed Spaces Carry $\beta(X^*,X)$}]\label{NVS dual carry beta(X^*,X)}
Let $X$ be a normed space and let $X^*$ carry its norm topology $\mathcal{T}_n$. Let $B_X$ and $B_{X^*}$ denote the closed unit balls of $X$ and $X^*$, respectively. By Example~\ref{polar of closed unit ball}, $B_{X^*}=B_{X}$. The norm topology is a topology of the dual pair $(X,X^*)$ so the norm-bounded set $B_X$ is $\sigma(X,X^*)$-bounded by Theorem~\ref{LCS bounded set permanence}. As such, $B_X^\circ\in\beta(X^*,X)$ and $\mathcal{T}_n\sub\beta(X^*,X)$.\par
Conversely, if $B$ is a $\sigma(X,X^*)$-bounded subset of $X$, then it is norm bounded by Theorem~\ref{LCS bounded set permanence}. Therefore, there exists $r>0$ such that $B\sub rB_X$. Hence $B^\circ\sups(rB_X)^\circ=(1/r)B_{X^*}$ and $B^\circ$ is seen to be a norm neighborhood of $X^*$; thus $\beta(X^*,X)\sub\mathcal{T}_n$. A normed space $X$ need not carry $\beta(X,X^*)$, however.
\end{example}
\begin{example}[\textbf{Metrizable LCS Carry $\tau(X,X^*)$}]
Let $(X,\mathcal{T})$ be a pseudometrizable locally convex space. By the Mackey-Arens theorem, we know that $\mathcal{T}\sub\tau(X,X^*)$. To see that each $\tau(X,X^*)$-neighborhood $U$ of $0$ is a $\mathcal{T}$-neighborhood of $0$, let $\{V_n\}$ be a decreasing base of balanced $\mathcal{T}$-neighborhoods of $0$ in $X$. If $U$ is not a $\mathcal{T}$-neighborhood of $0$, then for each $n\in\N$, there exists $x_n\in(1/n)V_n\setminus U$. Since $nx_n\to 0$, $\{nx_n\}$ is $\mathcal{T}$-bounded. By Theorem~\ref{LCS bounded set permanence}, $\{nx_n\}$ is $\tau(X,X^*)$-bounded as well and therefore should be absorbed by $U$. Hence there should be $k\in\N$ such that $\{nx_n:n\in\N\}\sub kU$ which implies that
$kx_k\in kU$ or that $x_k\in U$, which contradicts the way the $x_n$ were chosen.
\end{example}
\section{Weak$^*$ closed sets}
\begin{definition}
Let $X$ be a locally convex Hausdorff space. The \textbf{equicontinuous weak$^*$ topology} $e\sigma(X^*,X)$ is the finest topology on $X^*$ that induces $\sigma(X^*,X)\cap E$ on each equicontinuous subset $E\sub X^*$. Thus $e\sigma(X^*,X)$ consists of all $B\sub X^*$ such that $B\cap E\in\sigma(X^*,X)\cap E$ for each equicontinuous set $E\sub X^*$.
\end{definition}
Since a translate $f+E$ of an equicontinuous set $E$ is equicontinuous, the $e\sigma(X^*,X)$-neighborhoods of $f\in X^*$ are of the form $f+V$ where $V$ is an $e\sigma(X^*,X)$-neighborhood of $0$. Generally, however, $e\sigma(X^*,X)$ is not a linear topology. Komura [1962] exhibited a locally convex Hausdorff space $X$ and an $e\sigma(X^*,X)$-neighborhood $U$ of $0$ such that $V+V\not\sub U$ for any $e\sigma(X^*,X)$-neighborhood $V$ of $0$. Hence addition is not $e\sigma(X^*,X)$-continuous in this case. However, we have the following result.
\begin{proposition}\label{LCHS esigma topo bal absorbent nbhd}
Let $X$ be a locally convex Hausdorff space. Then $e\sigma(X^*,X)$ has a base of balanced absorbent neighborhoods of $0$.
\end{proposition}
\begin{proof}
Let $(X,\mathcal{T})$ be an LCHS. We show first that if $W$ is an $e\sigma(X^*,X)$-neighborhood of $0$, the balanced core $\balcore(W)$ of $W$ is also an $e\sigma(X^*,X)$-neighborhood of $0$. For any equicontinuous subset $E\sub X^*$, the balanced core $\balcore(E)$ is equicontinuous since it is contained in $E$. Since $W$ is an $e\sigma(X^*,X)$-neighborhood of $0$, $W\cap\balcore(E)$ contains a $\sigma(X^*,X)\cap\balcore(E)$ neighborhood of $0$, say $U:=\{x_1,\dots,x_n\}^\circ\cap\balcore(E)$, where $x_1,\dots,x_n\in X$. Since $U$ is a balanced subset of $W$, it is contained in $\balcore(W)$. Hence $\balcore(W)$ is an $e\sigma(X^*,X)$-neighborhood of $0$. Since $\balcore(W)\sub W$, the first result follows.\par
To show that there is a base of absorbent $e\sigma(X^*,X)$-neighborhoods of $0$, let $W$ be a balanced $e\sigma(X^*,X)$-neighborhood of $0$ and let $f\in X^*$. The balanced hull $E=\bal(\{f\})$ is an equicontinuous subset of $X^*$ to which $f$ belongs (Proposition~\ref{equicontinuous convbal hull}). Since $W$ is an $e\sigma(X^*,X)$-neighborhood of $0$, there exist $x_1,\dots,x_n\in X$ such that $\{x_1,\dots,x_n\}^\circ\cap E\sub W$. Since $\{x_1,\dots,x_n\}^\circ$ is absorbent, there exists $r\in(0,1]$ such that $af\in\{x_1,\dots,x_n\}^\circ$ for all $|a|\leq r$. Since $E$ is balanced, $af\in\{x_1,\dots,x_n\}^\circ\cap E\sub W$ for $|a|\leq r$ and $W$ is absorbent.
\end{proof}
\begin{proposition}\label{LCHS esigma closed iff}
Let $X$ be a locally convex Hausdorff space. A subset $F\sub X^*$ is $e\sigma(X^*,X)$-closed iff $F\cap D$ is $\sigma(X^*,X)$-closed for every $\sigma(X^*,X)$-closed equicontinuous disk $D\sub X^*$.
\end{proposition}
\begin{proof}
If $F$ is $e\sigma(X^*,X)$-closed, then by the definition of $e\sigma(X^*,X)$, $F\cap E$ is $\sigma(X^*,X)$-closed for every $\sigma(X^*,X)$-closed equicontinuous subset $E$ of $X^*$, hence for every $\sigma(X^*,X)$-closed equicontinuous disk $D$. Conversely, suppose that $F\cap D$ is $\sigma(X^*,X)$-closed for every equicontinuous $\sigma(X^*,X)$-closed disk $D\sub X^*$. Then if $H\sub X^*$ is equicontinuous, $H^{\circ\circ}=\mathrm{cl}_{\sigma(X^*,X)}H$ is equicontinuous by Propositions~\ref{equicontinuous weak closure}and \ref{equicontinuous convbal hull}, so is an equicontinuous $\sigma(X^*,X)$-closed disk. By hypothesis, $F\cap M^{\circ\circ}$ is $\sigma(X^*,X)$-closed . Since $M\sub M^{\circ\circ}$, $F\cap M$ is closed in $M$ for the topology $\sigma(X^*,X)\cap M$. As $M$ is an arbitrary equicontinuous subset of $X^*$, $F$ is $e\sigma(X^*,X)$-closed.
\end{proof}
Recall that precompact sets are bounded, hence weakly bounded. Thus polars of precompact sets generate a vector topology on $X^*$ which we discuss now.
\begin{definition}
Let $X$ be a locally convex Hausdorff space and $X^*$ its dual. Let $\mathcal{P}$ be the precompact subsets of $X$. The collection $\mathcal{P}^\circ$ forms a base of neighborhoods of $0$ in $X^*$ for a vector topology $p(X^*,X)$ called the \textbf{topology of precompact convergence} on $X^*$.
\end{definition}
Since finite sets are precompact, $\sigma(X^*,X)\sub p(X^*,X)$ and $p(X^*,X)$ is Hausdorff. Now we show that $p(X^*,X)\sub e\sigma(X^*,X)$.
\begin{proposition}\label{LCHS precompact convergence on equicontinuous}
Let $X$ be a locally convex Hausdorff space. If $E\sub X^*$ is equicontinuous then $p(X^*,X)\cap E=\sigma(X^*,X)\cap E$. Hence $p(X^*,X)\sub e\sigma(X^*,X)$.
\end{proposition}
\begin{proof}
Since, as observed above, $\sigma(X^*,X)\sub p(X^*,X)$, if $E\sub X^*$ is equicontinuous then $\sigma(X^*,X)\cap E\sub p(X^*,X)\cap E$. Now let $(f_0+P')\cap E$ be a $p(X^*,X)$-neighborhood of $f_0\in E$ where $P$ is precompact. Since $E$ is equicontinuous, $E-f_0$ is equicontinuous, so there exists a neighborhood $W$ of $0$ in $X$ such that $\sup_{f\in E}|(f-f_0)(W)|\leq 1/2$. Since $P$ is precompact, it is totally bounded, so there exist $x_1,\dots,x_n\in X$ such that $P\sub\bigcup_{i=1}^{n}(x_i+W)$. Suppose that $f\in (f_0+(1/2)\{x_1,\dots,x_n\}^\circ)\cap E\in\sigma(X^*,X)\cap E$. Then $(f-f_0)\in(1/2)\{x_1,\dots,x_n\}^\circ$. For any $x\in P\sub\bigcup_{i=1}^{n}(x_i+W)$, there exists an index $i$ and $w\in W$ such that $x=x_i+w$. Then
\[|(f-f_0)(x)|\leq|(f-f_0)(x_i)|+|(f-f_0)(w)|\leq 1,\]
so $f-f_0\in P^\circ$ and $(f_0+(1/2)\{x_1,\dots,x_n\}^\circ)\cap E\sub(f_0+P^\circ)\cap E$. This proves the claim.
\end{proof}
We show next that when $X$ is metrizable, $e\sigma(X^*,X)$ is a vector topology; in fact, we have $e\sigma(X^*,X)=p(X^*,X)$ in this case. First, we establish the following technical lemma.
\begin{lemma}\label{LCHS descending nbhd esigma nbhd lemma}
Let $(X,\mathcal{T})$ be a metrizable locally convex space and let $W\sub X^*$ be an $e\sigma(X^*,X)$-open neighborhood of $0$. Let $\{V_n\}$ be a descending base of $\mathcal{T}$-closed, disked neighborhoods of $0$ with $V_1=X$. Then for each $n\in\N$ there exists a finite set $F_n\sub V_n$ such that, with $A_n=\bigcup_{i=1}^{n-1}F_i$ ($A_1=\{0\}$), we have $A_n^\circ\cap V_n^\circ\sub W$.
\end{lemma}
\begin{proof}
Since $V_1=X$, $V_1^\circ=\{0\}$. Thus $A_1^\circ\cap B_1^\circ=\{0\}\sub W$. Now suppose $F_1,\dots,F_{n-1}$ have been found, so that $A_n^\circ\cap V_n^\circ\sub W$, or equivalently $A_n^\circ\cap V_n^\circ\cap W^c=\emp$. Suppose that there is no finite set $F_n$ satisfying the desired condition. Then, for any finite set $F_n\sub V_n$, we have
\begin{align}\label{LCHS descending nbhd esigma nbhd lemma-1}
(A_{n+1})^\circ\cap V_{n+1}^\circ\cap W^c=(A_n\cup F_n)^\circ\cap V_{n+1}\cap W^c=A_n^\circ\cap F_n^\circ\cap V_{n+1}^\circ\cap W^c\neq\emp.
\end{align}
Since $V_{n+1}^\circ$ is $\sigma(X^*,X)$-compact by Alaoglu's theorem, $V_{n+1}^\circ\cap W^c$ is $\sigma(X^*,X)$-compact. By $(\ref{LCHS descending nbhd esigma nbhd lemma-1})$, the family of $\sigma(X^*,X)$-closed subsets $A_n^\circ\cap F^\circ\cap V_{n+1}\cap W^c$, where $F$ is a finite subset of $V_n$, of the $\sigma(X^*,X)$-compact set $V_{n+1}\cap W^c$ has finite intersection property, hence has a nonempty intersection. In particular, for the family of singletons,
\[\bigcap_{x\in V_n}A_n^\circ\cap\{x\}^\circ\cap V_{n+1}^\circ\cap W^c\neq\emp.\]
But note that
\begin{align*}
\bigcap_{x\in V_n}A_n^\circ\cap\{x\}^\circ\cap V_{n+1}^\circ\cap W^c&=A_n^\circ\cap\Big(\bigcap_{x\in V_n}\{x\}^\circ\Big)\cap V_{n+1}^\circ\cap W^c\\
&=A_n^\circ\cap V_n^\circ\cap V_{n+1}^\circ\cap W^c\sub A_n^\circ\cap V_n^\circ\cap W^c=\emp.
\end{align*}
This is a contradiction, so the claim is proved.
\end{proof}
\begin{theorem}[\textbf{Banach-Dieudonn\'e Theorem}]
If $X$ is a metrizable locally convex space then $e\sigma(X^*,X)=p(X^*,X)$, i.e., the strongest topology on $X^*$ which induces the $\sigma(X^*,X)$-topology on each equicontinuous subset of $X^*$ is the topology of uniform convergence on precompact subsets of $X$.
\end{theorem}
\begin{proof}
Since $p(X^*,X)\sub e\sigma(X^*,X)$ and $e\sigma(X^*,X)$-neighborhoods of $f\in X^*$ are translates of $e\sigma(X^*,X)$-neighborhoods of $0$, it suffices to show that if $W$ is an $e\sigma(X^*,X)$-open neighborhood of 0, there exists a precompact set $P$ such that $P^\circ\sub W$. Let $F_n$, $A_n$, and $V_n$ be as in Lemma~\ref{LCHS descending nbhd esigma nbhd lemma} and let $P=\bigcup_nF_n$. Since $F_n\sub V_n$ and $\{V_n\}$ is a descending base of neighborhoods of $0$ in $X$, it is easy to see $P$ is compact. Since $A_n\sub P$ and $A_n^\circ\cap V_n^\circ\sub W$, it follows that $P^\circ\cap V_n^\circ\sub A_n^\circ\cap V_n^\circ\sub W$ for all $n\in\N$. Thus $P^\circ\cap(\bigcup_nV_n^\circ)=P^\circ\sub W$, which proves the theorem.
\end{proof}
\begin{corollary}\label{LCHS Frechet space p=esigma=tau}
If $X$ is a Fr\'echet space then $p(X^*,X)=e\sigma(X^*,X)=\tau(X^*,X)$.
\end{corollary}
\begin{proof}
The topology $p(X^*,X)$ is generated by polars of sets $\mathrm{cl}_{\sigma(X^*,X)}\convbal(P)$, where $P\sub X$ is precompact. By Proposition~\ref{dual topo permanence of closed convex}, we have $\mathrm{cl}_{\sigma(X^*,X)}\convbal(P)=\mathrm{cl}_{\mathcal{T}}\convbal(P)$. Since $(X,\mathcal{T})$ is complete, each $\mathrm{cl}_{\mathcal{T}}\convbal(P)$ is compact (Theorem~\ref{TVS hull of totally bounded or compact sets}) which implies that $p(X^*,X)\sub\tau(X,X^*)$. Since every compact set is precompact, it follows that $p(X^*,X)=\tau(X^*,X)$. By Banach-Dieudonn\'e Theorem, we get $p(X^*,X)=e\sigma(X^*,X)=\tau(X^*,X)$.
\end{proof}
\begin{theorem}[\textbf{Krein-Smulian Theorem}]
Let $X$ be a Fr\'echet space. A convex set $F\sub X^*$ is $\sigma(X^*,X)$-closed iff $F$ is $e\sigma(X^*,X)$-closed.
\end{theorem}
\begin{proof}
This follows from Corollary~\ref{LCHS Frechet space p=esigma=tau} and Proposition~\ref{dual topo permanence of closed convex}, since $\tau(X^*,X)$ is a topology for the pairs.
\end{proof}
We now use Krein-Smulian Theorem to show that $\sigma(X^*,X)$-closedness of a subspace of a normed space is equivalent to $\sigma(X^*,X)$-closedness of its unit ball. This result is also called Krein-Smulian Theorem.
\begin{theorem}
Let $X$ be a normed space, let $B_{X^*}$ denote the unit ball of $X^*$ and let $M$ be a subspace of $X^*$. Then $M$ is $\sigma(X^*,X)$-closed iff $M\cap B_{X^*}$ is $\sigma(X^*,X)$-closed.
\end{theorem}
\begin{proof}
Since equicontinuous sets are $\sigma(X^*,X)$-bounded (Theorem~\ref{weak topo equicontinuous is precompact}), they are norm-bounded by Mackey's theorem. Hence every $\sigma(X^*,X)$-closed equicontinuous disk $D$ of $X^*$ is contained in $nB_{X^*}$ for some $n\in\N$. Thus $D\cap M=(D\cap nB_{X^*})\cap M=D\cap n(B_{X^*}\cap M)$. Thus if $M\cap B_{X^*}$ is $\sigma(X^*,X)$-closed then $D\cap M$ is $\sigma(X^*,X)$-closed. Thus, $M$ is $e\sigma(X^*,X)$-closed, and $\sigma(X^*,X)$-closed by Krein-Smulian Theorem.\par
Conversely, suppose that $M$ is $\sigma(X^*,X)$-closed, that $f\in\mathrm{cl}_{\sigma(X^*,X)}(M\cap B_{X^*})$, and that $(f_\alpha)$ is a net in $M\cap B_{X^*}$ that converges weakly to $f$. Since $M$ is $\sigma(X^*,X)$-closed, $f\in M$ and therefore for every $x\in B_X$, $f_\alpha(x)\to f(x)$. For $x\in B_X$ and every $\alpha$, we have $|f_\alpha(x)|\leq\|f\|\|x\|\leq 1$, which implies that $|f(x)|\leq 1$ and therefore that $\|f\|\leq 1$; thus $f\in M\cap B_{X^*}$.
\end{proof}
\section{Orthogonals and adjoints}
The notion of "orthogonal" of this part provides a sort of generalization of the geometric notion of orthogonality. Orthogonals facilitate conversion of certain statements in $X$ to assertions in $X^*$ and generally play a role in duality theory.
\begin{definition}
Let $X$ and $Y$ be paired spaces. For $S\sub X$ the set
\[S^\bot=\{y\in Y:\langle x,y\rangle=0\text{ for each $s\in S$}\}\]
is the \textbf{orthogonal} (or \textbf{annihilator}) of $S$. Analogous conventions hold for subsets of $Y$.
\end{definition}
It is immediate that $S^\circ$ is a subspace no matter what $S$ is. If $S$ is a subspace, however, then $S^\circ=S^\bot$ by Example~\ref{polar of subspace}.
\begin{proposition}\label{pair annihilator prop}
Let $X$ and $Y$ be paired linear space and let $H$ and $S$ be subsets of $X$.
\begin{itemize}
\item[(a)] If $S\sub H$ then $H^\bot\sub S^\bot$.
\item[(b)] $S^\bot=\langle S\rangle^\bot=(\mathrm{cl}_{\sigma(X,Y)}\langle S\rangle)^\bot$.
\item[(c)] $S\sub S^{\bot\bot}=(S^\bot)^\bot$ and $S^\bot=S^{\bot\bot\bot}$.
\item[(d)] $S^\bot$ is a $\sigma(Y,X)$-closed subspace.
\item[(e)] $S^{\bot\bot}=\mathrm{cl}_{\sigma(X,Y)}\langle S\rangle$; in particular, if $M$ is a $\sigma(X,Y)$-closed subspace, then $M=M^{\bot\bot}$.
\item[(f)] If $\{S_i:i\in I\}$ is a family of subsets then $(\bigcup_iS_i)^\bot=\bigcap_iS_i^\bot$.
\item[(g)] If $\{M_i:i\in I\}$ is a family of $\sigma(X,Y)$-closed subspaces, then $(\bigcap_iM_i)^\bot=\mathrm{cl}_{\sigma(Y,X)}\langle\bigcup_iM_i^\bot\rangle$. 
\end{itemize}
\end{proposition}
\begin{proof}
Since $S\sub\langle S\rangle\sub\mathrm{cl}_{\sigma(X,Y)}\langle S\rangle$, it follows that $\mathrm{cl}_{\sigma(X,Y)}\langle S\rangle^\bot\sub\langle S\rangle^\bot\sub S^\bot$. Let $y\in S^\bot$ and $x\in\mathrm{cl}_{\sigma(X,Y)}\langle S\rangle$. Choose a net $(x_\alpha)$ from $\langle S\rangle$ which converges weakly to $x$. Since $\langle x_\alpha,y\rangle=0$ for every $\alpha$ and $\langle\cdot,y\rangle$ is $\sigma(X,Y)$-continuous, it follows that $\langle x,y\rangle=\lim_\alpha\langle x_\alpha,y\rangle=0$.\par
For any $x\in S$ and $y\in S^\bot$, we have $\langle x,y\rangle=0$. Hence $S\sub S^{\bot\bot}$. By (a) this implies that $S^{\bot\bot\bot}\sub S^{\bot}$; equality follows from the fact that $S^\bot\sub(S^\bot)^{\bot\bot}$. Since $S^\bot=\langle S\rangle^\bot=\langle S\rangle^\circ$, (d) follows from Proposition~\ref{polar prop}.\par
By (b) and (c), $S^{\bot\bot}=(\mathrm{cl}_\sigma\langle S\rangle)^{\bot\bot}\sups\mathrm{cl}_\sigma\langle S\rangle$. If this inclusion is proper, there is an $x\in S^{\bot\bot}\setminus\mathrm{cl}_\sigma\langle S\rangle$. By Proposition~\ref{LCS separation of point with subspace}, there exists a $\sigma(X,Y)$-continuous linear functional $f$ on $X$ such that $f(x)=1$ that vanishes on $\mathrm{cl}_\sigma\langle S\rangle$. By the weak representation theorem, there exists $y\in Y$ such that $f(x)=\langle x,y\rangle$. Thus $y\in(\mathrm{cl}_\sigma\langle S\rangle)^\bot=S^\bot$ but $y\notin(S^{\bot\bot})^{\bot}$, which contradicts (c).\par
Part (f) is clear from definition. By (e), $M_i=M_i^{\bot\bot}$ for each $i$, therefore
\begin{align*}
\bigcap_iM_i=\bigcap_iM_i^{\bot\bot}=\Big(\bigcup_iM_i^\bot\Big)^{\bot}=\Big(\mathrm{cl}_{\sigma(Y,X)}\Big\langle\bigcup_iM_i^\bot\Big\rangle\Big)^\bot.
\end{align*}
Hence by (e), we get (g).
\end{proof}
From now on, $(X,X^*)$ and $(Y.Y^*)$ are assumed to be dual pairs.. We regard $(X^*,X)$ and $(Y^*,Y)$ as paired spaces as well: if $\langle x,x^*\rangle$ is the bilinear functional on $(X,X^*)$, we take $\langle x,x^*\rangle=\langle x,x^*\rangle$ as the bilinear functional on $(X^*,X)$.\par
In solving finite systems of linear equations, it is possible to achieve a certain symmetry in phrasing through the use of adjoints; i.e., through the
use of the conjugate transpose of the matrix of coefficients. Generalizations include the Fredholm alternative theorem for compact maps $A$ mapping a normed space into itself.\par
Let $X^{\star}$ and $Y^{\star}$ denote the algebraic duals of $X$ and $Y$, and let $A:X\to Y$ be a linear map. We define the \textbf{algebraic adjoint} $A^{\star}:Y^{\star}\to X^{\star}$ as follows: For $y^{\star}\in Y^{\star}$, consider the map $A^{\star}y^{\star}:X\to\K,x\mapsto\langle Ax,\rangle$. Evidently $A^{\star}y^{\star}\in X^{\star}$. Thus, the defining equation for $A*$ is
\[\langle Ax,y^{\star}\rangle=\langle x,A^{\star}y^{\star}\rangle.\]
\begin{example}
Let $B=\{e_1,\dots,e_n\}$ be a basis for $\R^n$. The dual basis $\{e^1,\dots,e^n\}$ for $(\R^n)^*=\R^n$, is $e^j(e_i)=\delta_{ij}$. Associate with each linear map $A:\R^n\to\R^n$ the matrix $(a_{ij})$ where $Ae_j=\sum_{i}a_{ij}e_i$ for each $j$. The matrix associated with $A^{\star}$ with respect to the dual basis is the matrix transpose $(a_{ji})$.
\end{example}
As we have already noted, the continuous duals $X^*$ and $Y^*$ of topological spaces $X$ and $Y$ are embedded in $X^{\star}$ and $Y^{\star}$, respectively. Moreover, by the weak representation theorem, $X^*$ and $Y^*$ are the repective duals of $(X,\sigma(X,X^*))$ and $(Y,\sigma(Y,Y^*))$. Although we are certainly at liberty to restrict $A^{\star}$ to $Y^*$, $A*|_{Y^*}=A^*$ need not map $Y^*$ into $X^*$. A complete answer to the question "When does $A^*$ map $Y^*$ into $X^*$?" is given in the next proposition.
\begin{definition}
Let $(X,X^*)$ and $(Y,Y^*)$ denote dual pairs. In saying that linear map $A:X\to Y$ is \textbf{weakly continuous}, we mean that $A:(X,\sigma(X,X^*))\to(Y,\sigma(Y,Y^*))$ is continuous. Similarly, we say a linear map $B:Y^*\to X^*$ is weakly continuous if $B:(Y^*,\sigma(Y^*,Y))\to(X^*,\sigma(X^*,X))$ is continuous.
\end{definition}
\begin{proposition}\label{ajoint in continuous dual iff weak continuous}
Let $(X,X^*)$ and $(Y,Y^*)$ denote dual pairs and let $A:X\to Y$ be linear. Then $R(A^*)\sub X^*$ iff $A$ is weakly continuous.
\end{proposition}
\begin{proof}
Suppose $R(A^*)\sub X^*$ and that $(x_\alpha)$ is a net from $X$ that is $\sigma(X,X^*)$-convergent to $0$. This means that for any $x^*\in X^*$ we have $\langle x_\alpha,x^*\rangle\to 0$. In particular, for any $y^*\in Y^*$, $\langle x_\alpha,A^*y^*\rangle\to 0$ which is equivalent to $\langle Ax_\alpha,y\rangle\to 0$; this implies that $Ax_\alpha$ is $\sigma(Y,Y^*)$-convergent to $0$, i.e., that $A$ is weakly continuous.\par
Conversely, suppose that $A$ is weakly continuous. To show that $R(A^*)\sub X^*$, suppose that the net $(x_\alpha)$ from $X$ is $\sigma(X,X^*)$-convergent to $0$. Since $A$ is weakly continuous, $(Ax_\alpha)$ is $\sigma(Y,Y^*)$-convergent to $0$. Hence, for any $y^*\in Y^*$ we have $(A^*y^*)(x_\alpha)=\langle Ax_\alpha,y^*\rangle\to 0$. In other words, $A^*y^*$ is continuous on $(X,\sigma(X,X^*))$. By the weak representation theorem, $A^*y^*\in X^*$.
\end{proof}
Because of Proposition~\ref{ajoint in continuous dual iff weak continuous}, we always assume that the linear map is weakly continuous whenever we speak of adjoints.
\begin{definition}
Let $(X,X^*)$ and $(Y,Y^*)$ be dual pairs and let $A:X\to Y$ be a weakly continuous linear map. The restriction $A^*:Y^*\to X^*$ of the algebraic adjoint $A^{\star}$ to $Y^*$ is called the \textbf{adjoint} (or \textbf{transpose}) of $A$.
\end{definition}
\begin{proposition}\label{TVS adjoint prop}
Let $A:X\to Y$, $B:Y\to Z$ be a weakly continuous linear maps.
\begin{itemize}
\item[(a)] $(AB)^*=B^*A^*$ and $(\alpha A+\beta B)^*=\alpha A^*+\beta B^*$.
\item[(b)] If $A$ is invertible, then $A^*$ is invertible and $(A^*)^{-1}=(A^{-1})^*$.
\item[(c)] If $X$ and $Y$ are normed then $\|A\|=\|A^*\|$.
\end{itemize}
\end{proposition}
\begin{proof}
Let $B:X\to Y$ and $A:Y\to Z$ be weak continuous. To see that $(AB)^*=B'A^*$, let $x\in X$ and $z^*\in Z^*$, we have
\begin{align*}
\langle x,(AB)^*z^*\rangle=\langle ABx,z^*\rangle=\langle Bx,A^*z\rangle=\langle x,B^*A^*z^*\rangle.
\end{align*}
Since this holds for arbitrary $x$ and $z^*$, we get $(AB)^*=B^*A^*$. The second claim is easy to prove, and (b) follows from (a).\par
Now assume that $X$ and $Y$ are normed. For $A\in\mathcal{L}(X,Y)$, $y^*\in Y^*$ and $x\in X$,
\[|(A^*y^*)(x)|=|y^*(Ax)|\leq\|y^*\|\|Ax\|\leq\|y^*\|\|A\|\|x\|\]
which implies $\|A^*y^*\|\leq\|A\|\|y^*\|$ and therefore $\|A^*\|\leq\|A\|$. Since $A^{**}=A$, this gives $\|A^*\|=\|A\|$. 
\end{proof}
The ajoint of a linear map $A:X\to Y$ is a linear map from $X^*$ to $Y^*$. Since we regard $(X^*,X)$ and $(Y^*,Y)$ as paired spaces as well, if $A^*$ is weak$^*$-continuous, we can further consider the biadjoint $A^{**}:X^{**}\to Y^{**}$.
\begin{proposition}\label{adjoint involutive}
Let $(X,X^*)$ and $(Y,Y^*)$ be duai pairs and let the linear map $A:X\to Y$ be weakly continuous. Then $A^*$ is weakly$^*$ continuous and $A^{**}|_{X}=A$.
\end{proposition}
\begin{proof}
Suppose that $(y^*_\alpha)$ is $\sigma(Y^*,Y)$-convergent to $0$. Then $\langle x,A^*y^*_\alpha\rangle=\langle Ax,y^*_\alpha\rangle\to 0$ for each $x\in X$, which means $A^*y^*_\alpha\to 0$ weakly. Thus $A^*$ is weak$^*$ continuous. We define $A^{**}:X\to Y$ by the condition
\[\langle y^*,A^{**}x\rangle=\langle A^*y^*,x\rangle=\langle x,A^*y^*\rangle=\langle Ax,y^*\rangle.\]
Since $(Y,Y^*)$ is a dual pair, it follows that $A^{**}x=Ax$ and therefore that $A^{**}|_X=A$.
\end{proof}
\begin{proposition}[\textbf{Polars and Adjoints}]\label{polar and adjoint}
Let $(X,X^*)$ and $(Y,Y^*)$ be dual pairs and let the linear map $A:X\to Y$ be weakly continuous. For any subsets $E\sub X$, $G\sub Y$,
\begin{itemize}
\item[(a)] $A(E)^\circ=(A^*)^{-1}(E^\circ)$.
\item[(b)] $A(E)\sub G$ implies $A^*(G^\circ)\sub E^\circ$.
\item[(c)] if $E$ and $G$ are weakly closed disks then $A(E)\sub G$ iff $A^*(G^\circ)\sub E^\circ$. 
\item[(d)] $N(A^*)=R(A)^\bot$ and $N(A)=R(A^*)^\bot$.
\item[(e)] $A^*$ is injective iff $R(A)$ is weakly dense, and $A$ is injective iff $R(A^*)$ is weakly dense. 
\end{itemize}
\end{proposition}
\begin{proof}
Let $\D=\{a\in\K:|a|\leq 1\}$. Then we observe that
\[y\in A(E)^\circ\iff |\langle A(E),y\rangle|=|\langle E,A^*y\rangle|\sub \D\iff A^*y\in E^\circ\iff y\in (A^*)^{-1}(E^\circ).\]
This prove (a). By Proposition~\ref{polar prop}(b), $A(E)\sub E$ implies $A(E)^\circ\sups G^\circ$. By (a) we have $A(E)^\circ=(A^*)^{-1}(E^\circ)$, so $A^*(G^\circ)\sub E^\circ$.\par
Now let $E$ and $G$ be weakly closed disks. If $A^*(G^\circ)\sub E^\circ$, then
\[E=E^{\circ\circ}\sub A^*(G^\circ)^\circ=(A^{**})^{-1}(G^{\circ\circ})=A^{-1}(G).\]
Thus $A(E)\sub G$. This proves (c). By the definition of $A^*$, since $(X,X^*)$ is a dual pair,
\[x^*\in N(A^*)\iff\langle x,A^*x^*\rangle=\langle Ax,x^*\rangle=0,\ \forall x\in X\iff x^*\in R(A)^\bot,\]
so (d) follows. By Proposition~\ref{pair annihilator prop}(b), $R(A)^\bot=(\mathrm{cl}_{\sigma(Y,Y^*)}R(A))^\bot$. Thus, by (d), if $R(A)$ is weakly dense in $Y$, then $N(A^*)=Y^\bot=0$. Conversely, if $N(A^*)=0$, then $N(A^*)^\bot=Y=R(A)^{\bot\bot}=\mathrm{cl}_{\sigma(Y,Y^*)}R(A)$ by Proposition~\ref{pair annihilator prop}(e). This proves the first part of (e), the second part can be proved similarly.
\end{proof}
In the following, $A$ denotes a linear map of a locally convex Hausdorff space $X$ into a locally convex Hausdorff space $Y$. We obtain various relations about and between continuity of $A$ and continuity of $A^*$.
\begin{definition}
Let $(X, X^*)$ and $(Y,Y^*)$ be dual pairs. The linear map $A:X\to Y$ is \textbf{strongly continuous} if $A:(X,\beta(X,X^*))\to(Y,\beta(Y,Y^*))$ is continuous, and is \textbf{Mackey continuous} if $A:(X,\tau(X,X^*))\to(Y,\tau(Y,Y^*))$ is continuous.
\end{definition}
In our first result, we characterize continuity of $A$ by means of its adjoint $A^*$.
\begin{proposition}\label{LCHS weak continuous is continuous iff}
Let $A$ be a weakly continuous linear map of locally convex Hausdorff spaces. Then $A:X\to Y$ is continuous iff $A^*$ maps equicontinuous subsets of $Y^*$ into equicontinuous subsets of $X^*$.
\end{proposition}
\begin{proof}
Suppose that $A$ is continuous and that $E$ is an equicontinuous subset of $Y^*$. To prove that $A^*(E)$ is an equicontinuous subset of $X^*$, we use the criterion of Proposition~\ref{equicontinuous iff polar is nbhd}(a) and show that $A^*(E)\sub V^\circ$ for some neighborhood $V$ of $0$ in $X$. Since $E$ is equicontinuous, $E^\circ$ is a neighborhood of $0$ in $Y$ by Proposition~\ref{equicontinuous iff polar is nbhd}. Since $A$ is continuous, $V=A^{-1}(E^\circ)$ is a neighborhood of $0$ in $X$. By the dual form of Proposition~\ref{polar and adjoint}(a), $A^*(E)^\circ=A^{-1}(E^\circ)=V$. Thus $A^*(E)\sub A^*(E)^{\circ\circ}\sub V^\circ$.\par
Conversely, suppose that $A^*$ preserves equicontinuity. By Proposition~\ref{equicontinuous iff polar is nbhd}, polars of equicontinuous subsets $V\sub Y^*$ form a base at $0$ in $Y$. By Proposition~\ref{polar and adjoint}(a), $A^{-1}(V^\circ)=A^*(V)^\circ$. Hence $A^{-1}(V^\circ)$ is a neighborhood of $0$ in $X$ by Proposition~\ref{equicontinuous iff polar is nbhd}.
\end{proof}
\begin{theorem}\label{LCHS continuous and weak continuous}
Let $A:X\to Y$ be a linear map between locally convex Hausdorff spaces. Then:
\begin{itemize}
\item[(a)] if $A$ is weakly continuous, then $A$ and $A^*$ are Mackey continuous and strongly continuous;
\item[(b)] if $A$ is continuous, then $A$ is weakly continuous but not conversely.
\end{itemize}
\end{theorem}
\begin{proof}
Suppose that $A$ is weakly continuous. Then $A^*$ is also weakly continuous, so we only need to prove (a) for $A$. Let $V$ be a $\sigma(Y^*,Y)$-compact disk in $Y^*$, so that $V^\circ$ is a basic $\tau(Y,Y^*)$-neighborhood of $0$ in $Y$. By the weak$^*$ continuity and linearity of $A^*$, $A^*(V)$ is a $\sigma(X^*,X)$-compact disk of $X^*$. Hence by Proposition~\ref{polar and adjoint}, $A^*(V)^\circ=(A^{**})^{-1}(V^\circ)=A^{-1}(V^\circ)$ is a basic $\tau(X,X^*)$-neighborhood of $0$ in $X$.\par
Now let $V$ be a $\sigma(Y^*,Y)$-bounded subset of $Y^*$ so $V^\circ$ is a basic $\beta(Y,Y^*)$-neighborhood of $0$ in $Y$. Since $A^*$ is weak$^*$ continuous and continuity implies boundedness (Proposition~\ref{TVS homogeneous image of bounded is bounded}), $A^*(V)$ is a $\sigma(X^*,X)$-bounded subset of $X^*$. By Proposition~\ref{polar and adjoint}, $A^*(V)^\circ=(A^{**})^{-1}(V^\circ)=A^{-1}(V^\circ)$ is a basic $\beta(X,X^*)$-neighborhood of $0$ in $X$.\par
The weak topology $\sigma(Y,Y^*)$ is generated by the seminorms $\{q_{y^*}:y^*\in Y^*\}$. If $A$ is continuous, then for each $y^*\in Y^*$, the map $p_{y^*}(x)=|\langle Ax,y^*\rangle|$ is a continuous seminorm on $X$. It follows from Proposition~\ref{LCS as image continuity iff} that $A$ is weakly continuous.
\end{proof}
\section{Subspaces and quotients}
If $M$ is a subspace of a linear space $X$, then, after making some identifications, $X/M$ is seen to be complementary to $M$ in the sense that $X=M\oplus X/M$. This lies at the base of the duality exhibited below.\par
All pairings of TVS $X$ are the natural pairings with their continuous duals $X^*$. If $M$ is a subspace of a TVS $X$, $X/M$ carries the quotient topology. The following results are purely algebraic.
\begin{proposition}[\textbf{Dual of Subspaces and Quotients}]\label{dual of subspace and quotient}
Let $M$ be a subspace of the TVS $X$. Then:
\begin{itemize}
\item[(a)] $(X/M)^*$ is linearly isomorphic to $M^\bot$.
\item[(b)] If $X$ is locally convex, then $M^*$ is linearly isomorphic to $X^*/M^\bot$.
\end{itemize}
\end{proposition}
\begin{proof}
Given $f\in M^\bot$, we define $\bar{f}:X/M\to\K$ by taking $\bar{f}(x+M)=f(x)$. The map $\bar{f}$ is well-defined and clealy linear. The open sets $U\sub X/M$ of the quotient topology are those such that $\pi^{-1}(U)$ is open in $X$. Since $\pi$ is continuous and $f=\bar{f}\circ\pi$, the continuity of $f$ implies the continuity of $\bar{f}$. The map
\[B:M^\bot\mapsto(X/M)^*,\quad f\mapsto\bar{f}\]
is linear and injective. It is surjective since, for any $\bar{f}\in(X/M)^*$, $\bar{f}\circ\pi\in M^\bot$. It follows that that $M^\bot$ and $(X/M)^*$ are linearly isomorphic.\par
Consider the linear map
\[A:X^*\to M^*,\quad f\mapsto f|_M.\]
If $X$ is locally convex, then any linear functional which is continuous on $M$ may be extended to a continuous linear functional on $X$ by Corollary~\ref{LCS continuous functional extension}. Thus $A$ is surjective and it follows that $M^*$ is linearly isomorphic to $X^*/A^{-1}(0)=X^*/M^\bot$.
\end{proof}
Now we consider the weak topology on subspaces and quotients.
\begin{proposition}[\textbf{Weak Topology on Subspacess}]\label{weak topo subspace}
\mbox{}
\begin{itemize}
\item[(a)] If $M$ is a subspace of a locally convex space $X$, then $\sigma(M,M^*)$ is the relative topology induced by $\sigma(X,X^*)$ on $M$. Also $\sigma(M,M^*)=\sigma(M,X^*|_M)$.
\item[(b)] If $X$ and $Y$ are paired spaces and $M$ is a subspace of $X$, then $M$ and $Y/M^\bot$ are paired spaces and $\sigma(M,Y/M^\bot)$ is the relative topology induced by $\sigma(X,Y)$ on $M$. If $(X,Y)$ is a dual pair, then so is $(M,Y/M^\bot)$.
\end{itemize}
\end{proposition}
\begin{proof}
A base at $0$ for $\sigma(M,M^*)$ is given by polars $H^\circ$ of finite subsets $H$ of $M^*$, polars being computed in $M$. Since $X$ is locally convex, any $f\in M^*$ may be extended to an $f^*\in X^*$. Hence $M^*=X^*|_M$ and $\sigma(M,M^*)=\sigma(M,X^*)=\sigma(M,X^*|_M)$. To see that $\sigma(M,M^*)$ is the relative topology induced by $\sigma(X,X^*)$ on $M$, note that a relative $\sigma(X,X^*)$-basic neighborhood of $0$ is, for a finite subset $H$ of $X^*$,
\[M\cap H^\circ=M\cap (H|_M)^\circ=(H|_M)^\circ\]
which is a basic $\sigma(M,M^*)$-neighborhood of $0$.\par
Let $X^*=(X,\sigma(X,Y))^*$. Since $(X,\sigma(X,Y))$ is a locally convex space, $X^*/M^\bot\cong M^*$ by Proposition~\ref{dual of subspace and quotient}(b) where $M^\bot$ is computed in $X^*$. By the weak representation theorem, $X^*$ may be identified with $Y/X^\bot$, $X^\bot$ being computed in $Y$. Therefore
\[M^*\cong X^*/M^\bot\cong(Y/X^\bot)/(M^\bot/X^\bot)\cong Y/M^\bot.\]
Thus $M$ and $Y/M^\bot$ are paired spaces with respect to the bilinear functional:
\[M\times Y/M^\bot\to\K,\quad (m,y+M^\bot)\mapsto \langle m,y\rangle\]
which is well-defined because we are identifying functions $y\in Y$ which vanish on $M$. It is a dual pairing if the original pairing is.\par
By (a), it follows that $\sigma(M,M^*)=\sigma(M,Y/M^\bot)=\sigma(M,X^*)=\sigma(M,Y/X^\bot)$ and that $\sigma(M,Y/X^\bot)$ is the relative topology induced by $\sigma(X,Y/X^\bot)$ on $M$. By the nature of the pairing of $X$ and $Y/X^\bot$, $\sigma(X,Y/X^\bot)$ is seen to coincide with $\sigma(X,Y)$.
\end{proof}
Given a subspace $M$ of a topological vector space, here are potentially three topologies for $X/M$: the quotient topology $\mathcal{T}_q$ induced by the original topology on $X$, the weak topology $\sigma(X/M,(X/M)^*)$ (when $X/M$ carries $\mathcal{T}_q$), and the quotient topology induced on $X/M$ by $\sigma(X,X^*)$. The following proposition shows that the latter two topologies are the same.
\begin{proposition}[\textbf{Weak Topology on Quotients}]\label{weak topo quotient}
\mbox{}
\begin{itemize}
\item[(a)] If $M$ is a subspace of a topological vector space $X$, then the weak topology $\sigma(X/M,(X/M)^*)$ for $X/M$ is tho quotient topology on $X/M$ induced by $\sigma(X,X^*)$.
\item[(b)] If $X$ and $Y$ are paired space and $M$ is a subspace of $X$, then $X/M$ and $M^\bot$ are paired spaces and $\sigma(X/M,M^\bot)$ is the quotient topology on $X/M$ induced by $\sigma(X,Y)$.
\end{itemize}
\end{proposition}
\begin{proof}
A neighborhood base at $0$ for $\sigma(X/M,(X/M)^*)$ consists of polars in $X/M$ of finite subsets $H$ of $(X/M)^*$. Thus a typical weak basic neighborhood of $0$ in $X/M$ is of the form
\[\widebar{U}=\{\bar{x}\in X/M:|\langle\bar{x},\bar{f}\rangle|\leq 1,\bar{f}\in H\}.\]
Associated with each $f\in X^*$ is an $\bar{f}\in(X/M)^*$ via $f=\bar{f}\circ\pi$. Then
\[U:=\pi^{-1}(\widebar{U})=\{x\in X:|\langle x,f\rangle|\leq 1,f=\bar{f}\circ\pi,\bar{f}\in H\}=H^\circ\in\sigma(X,X^*).\]
Since sets such as $U$ form a base of balanced neighborhoods of $0$ for $\sigma(X,X^*)$, the family of all such $\pi(U)$ is a baee at $0$ for the quotient topology on $X/M$ induced by $\sigma(X,X^*)$. It follows that $\sigma(X/M,(X/M)^*)$ coincides with the quotient topology induced on $X/M$ by $\sigma(X,X^*)$.\par
Now let $X$ carry $\sigma(X,Y)$. Then by Proposition~\ref{dual of subspace and quotient}(a), we can identify $(X/M)^*$ with $M^\bot$. From (a), it follows that $\sigma(X/M,M^\bot)$ is the quotient topology on $X/M$ induced by $\sigma(X,X^*)=\sigma(X,Y/X^\bot)=\sigma(X,Y)$ on $X/M$. The bilinear functional pairing $X/M$ and $M^\bot$ is the map $(x+M,y)\mapsto\langle x,y\rangle$; it is a dual pairing if the original pairing is.
\end{proof}
Equicontinuous subsets of the dual $M^*$ of a closed subspace $M$ of a locally convex Hausdorff space $X$ and of $(X/M)^*$ are completely determined by the equicontinuous subsets of $X^*$, as we show next.
\begin{proposition}
Let $M$ be a subspace of a locally convex Hausdorff space $X$.
\begin{itemize}
\item[(a)] A subset $E$ of the dual $M^*$ of $M$ is equicontinuous on $M$ iff there is an equicontinuous subset $E^*$ of $X^*$ such that $E^*|_M=E$.
\item[(b)] If $M$ is closed, then the equicontinuous subsets of $(X/M)^*$ are in one-to-one correspondence with the equicontinuous subsets of of $X^*$ which lie in $M^\bot$.
\end{itemize}
\end{proposition}
\begin{proof}
Consider the inclusion map $I:M\to X$. Since $I$ is continuous, it is weakly continuous so we can consider its adjoint $I^*:X^*\to M^*$. For $m\in M$ and $m^*\in M^*$,
\[\langle m,I^*m^*\rangle=\langle Im,m^*\rangle=\langle m,m^*\rangle.\]
Thus $I^*$ is the map $m^*\mapsto m^*|_M$. Since $I$ is continuous, $I^*$ maps the equicontinuous subset $E^*$ of $X^*$ into the equicontinuous set $E^*|_M\sub M^*$ (Proposition~\ref{LCHS weak continuous is continuous iff}).
\[\begin{tikzcd}
M^*\ar[d,no head]&X^*\ar[l,swap,"I^*"]\ar[d,no head]\\
M\ar[r,"I"]&X
\end{tikzcd}\]

Conversely, if $E$ is an equicontinuous subset of $M^*$, then by Proposition~\ref{equicontinuous iff polar is nbhd} there is a closed balanced convex neighborhood $V$ of $0$ in $X$ such that $E\sub (M\cap V)^\circ_{M^*}$. Let $p_V$ denote the gauge of $V$ in $X$. For any $m\in M$ and $r>p_V(m)$ we have $m/r\in V\cap M$, hence $|f(m)/p_V(m)|\leq 1$ for any $f\in E$, and so $|f|\leq p_V$ on $M$. By Hahn-Banach theorem, $f$ can be extended to $F\in X^*$ with $|F|\leq p_V$. Let $E^*=\{F\in X^*:F|_M\in E\}$. Then for any $F\in E^*$ and $x\in V$ we have $|F(x)|\leq p_V(x)\leq 1$, thus $E^*\sub V^\circ$. It follows that $E^*$ is equicontinuous.\par
Let $\pi:X\to X/M$ denote the quotient map and identify $M^\bot$ and $(X/M)^*$. For $x\in X$ and $f\in M^\bot$, we have
\[f(x)=\langle\pi x,f\rangle=\langle x,\pi^*f\rangle.\]
Thus $\pi^*$ is seen to be the inclusion of $M^\bot$ into $X^*$.
\[\begin{tikzcd}
X^*\ar[d,no head]&M^\bot\cong(X/M)^*\ar[l,swap,"\pi^*"]\ar[d,no head]\\
X\ar[r,"\pi"]&X/M
\end{tikzcd}\]
Since $\pi$ is continuous, $\pi^*$ maps equicontinuous subsets of $M^\bot$ into equicontinuous subsets of $X^*$ which lie in $M^\bot$ (Proposition~\ref{LCHS weak continuous is continuous iff}). Conversely, if $E$ is an equicontinuous subset of $X^*$ lying in $M^\bot$, there is a neighborhood $V$ of $0$ in $X$ such that $E\sub V^\circ\cap M^\bot$ (Proposition~\ref{equicontinuous iff polar is nbhd}). Hence
\[(\pi^*)^{-1}(E)=E\sub(\pi^*)^{-1}(V^\circ)=\pi(V)^\circ.\]
Since $\pi(V)$ is a neighborhood in $X/M$, it follows by Proposition~\ref{equicontinuous iff polar is nbhd} that $E$ is an equicontinuous subset of $M^\bot$. 
\end{proof}
\section{Products and direct sums}
Let $\{(X_s,Y_s):s\in Y\}$ denote a family of dual pairs and assume that $X_s$ and $Y_s$ are all locally convex spaces. Let $X=\prod_sX_s$, $Y=\bigoplus_sY_s$. A dual pair $(X,Y)$ is then given by
\[\langle(x_s),(y_s)\rangle=\sum_s\langle x_s,y_s\rangle.\]
We may identify $X_s$ and $Y_s$ are subspaces of $X$ and $Y$, respectively. We further note that if $\pi_s:X\to X_s$ is the projection and $I_s:Y_s\to Y$ the injection, then
\[\langle \pi_sx,y_s\rangle=\langle x,I_sy\rangle.\]
Thus by Proposition~\ref{ajoint in continuous dual iff weak continuous}, since $\pi_s$ and $I_s$ are continuous with the usual topology, they are weakly continuous.\par
Now we want to consider polar topologies on the duality $(X,Y)$. If $\mathcal{S}_s$ is a family of weakly bounded subsets of $X_s$, then it is immediate that each product $\mathcal{S}=\prod_s\mathcal{S}_s$ is a family of $\sigma(X,Y)$-bounded subset of $X$, and $\mathcal{S}$ covers $X$ if each $\mathcal{S}_s$ covers $X_s$. Dually, let $\mathcal{G}_s$ be a family of weakly bounded subsets of $Y_s$; then $\mathcal{G}=\bigoplus_s\mathcal{G}_s$ is a family of $\sigma(Y,X)$-bounded subsets and $\mathcal{G}$ covers $Y$ if each $\mathcal{G}_s$ covers $Y_s$.
\begin{proposition}\label{polar topo for product and direct sum}
The product of $\mathcal{T}_{\mathcal{G}_s^\circ}$ is identical with the topology $\mathcal{T}_{\mathcal{G}^\circ}$ on $Y$; dualy, the locally convex direct sum of $\mathcal{T}_{\mathcal{S}_s^\circ}$ is identical with the topology $\mathcal{T}_{\mathcal{S}^\circ}$ on $X$.
\end{proposition}
\begin{proof}
If $G=\bigoplus_{s\in H}G_{s}$ where $G_{s}\in\mathcal{G}_{s}$ and $H\sub S$ contains $n$ elements, then by the definition of the dual pair $(X,Y)$, we have
\[G^\circ\sub\prod_{s\in H}(G_{s})^\circ\times\prod_{s\notin H}X_s\sub nG^\circ,\]
which proves the first assertion.\par
Dually, let $S=\prod_sS_s$ and assume each $S_s$ to be weakly closed and disked. It is evident that the disked hull $\convbal(\bigcup_sS_s^\circ)$ is contained in $S^\circ$. Conversely, if $y=(y_s)\in S^\circ$, then $\sum_s|\langle x_s,y_s\rangle|\leq 1$ for all $x=(x_\alpha)\in S$. Letting $\lambda_s=\sup\{|\langle x_s,y_s\rangle|:x\in S\}$, it follows that $\lambda_s=0$ except for finitely many $s\in S$, and $\sum_s\lambda_s\leq 1$. Now $y_s\in \lambda_sS_s^\circ$; hence $y=\sum_sy_s\in\convbal(\bigcup_sS_s^\circ)$, which shows that $S^\circ=\convbal(\bigcup_sS_s^\circ)$. Since the sets $\convbal(\bigcup_sS_s^\circ)$ form a neighborhood base at $0$ for the locally convex direct sum of the topologies $\mathcal{T}_{\mathcal{S}_s^\circ}$, this topology is identical with the topology $\mathcal{T}_{\mathcal{S}^\circ}$ on $Y$.
\end{proof}
We now apply Proposition~\ref{polar topo for product and direct sum} to the polar topologies we have already defined.
\begin{proposition}\label{weak Macky strong topo for product and direct sum}
Let $\{(X_s,Y_s):s\in Y\}$ denote a family of dual pairs and let $X=\prod_sX_s$, $Y=\bigoplus_sY_s$. Then
\begin{itemize}
\item[(a)] $\sigma(X,Y)=\prod_s\sigma(X_s,Y_s)$ and $\sigma(Y,X)=\bigoplus_s\sigma(Y_s,X_s)$ iff the number of spaces $X_s\neq 0$ is finite.
\item[(b)] $\tau(X,Y)=\prod_s\tau(X_s,Y_s)$ and $\tau(Y,X)=\bigoplus_s\tau(Y_s,X_s)$.
\item[(c)] $\beta(X,Y)=\prod_s\tau(X_s,Y_s)$ and $\beta(Y,X)=\bigoplus_s\tau(Y_s,X_s)$.
\end{itemize}
\end{proposition}
\begin{proof}
Note that part (c) is clear from Proposition~\ref{LCS bounded set permanence} and Proposition~\ref{polar topo for product and direct sum}.\par
If $\mathcal{S}_s$ denotes the family of all finite subsets of $Y_s$, it is evident that $\mathcal{S}=\bigoplus_s\mathcal{S}_s$ is fundamental for the family of all finite subsets of $Y$. Thus $\sigma(X,Y)=\prod_s\sigma(X_s,Y_s)$ by Proposition~\ref{polar topo for product and direct sum}. On the other hand, this is not ture for arbitrary product, and is easily seen to be true if and only if $X_s=\{0\}$ for all but finitely many indices. Thus the second part of (a) follows.\par
If $\mathcal{S}_s$ denotes the family of all weakly compact disks of $X_s$, then $\mathcal{S}=\prod_s\mathcal{S}_s$ is a fundamental subfamily of the family of all weakly compact disks of $X$; in fact, if $D$ is such a disk in $X$, then for each $s\in S$, $\pi_s(D)$ is a weakly compact disk in $X_s$, since each $\pi_s$ is weakly continuous. Tychonoff theorem and part (a) guarantee that $\prod_s\pi_s(D)$ is weakly compact and we have $D\sub\prod_s\pi_s(D)$. Thus the topology $\mathcal{T}_{\mathcal{S}}=\bigoplus_s\tau(Y_s,X_s)$ on $Y$ is $\tau(Y,X)$.\par
Let $\mathcal{G}_s$ denotes the family of all weakly compact disks of $Y_s$. We show that $\mathcal{G}=\bigoplus_s\mathcal{G}_s$ is a fundamental system of weakly compact disks in $Y$. If $D$ is such a set, $D$ is $\sigma(Y,X)$-bounded and hence $\tau(Y,X)$-bounded. Thus by the preceeding argument and , $D$ is contained in $\bigoplus_{s\in H}P_s(D)$, where $H\sub S$ is a suitable finite subset of $S$ and $P_s$ denotes the projection of $Y$ onto $Y_s$. Since $P_s$ is weakly continuous, $P_s(D)$ is weakly compact, which is the desired conclusion.
\end{proof}
\begin{proposition}\label{dual space of product and direct sum}
Let $\{X_s\}$ be a family of topological vector spaces.
\begin{itemize}
\item[(a)] $(\prod_sX_s)^*$ is linearly isomorphic to $\bigoplus_sX_s^*$.
\item[(b)] $(\bigoplus_sX_s)^*$ is linearly isomorphic to $\prod_sX_s^*$.
\end{itemize}
\end{proposition}
\begin{proof}
Let $f\in(\prod_sX_s)^*$. Then the restriction $f_s$ of $f$ on $X_s$ is continuous, so we have a map $I:(\prod_sX_s)^*\to\bigoplus_sX_s^*$. Let $f\in(\prod_sX_s)^*$. Then there exist a neighborhood $U=\prod_sU_s$ of $0$ such that $|f(x)|\leq 1$ for $x\in U$. Since only finitely many components $U_s$ are not $X_s$, it follows that $f|_{X_s}\neq 0$ for finitely many $s\in S$. Thus the map $I$ has its image in $\bigoplus_sX_s^*$ and (a) is proved. Part (b) is clear.
\end{proof}
\chapter{Barreled spaces}
\section{$\mathcal{S}$-topology for $\mathcal{L}(X,Y)$}
Let $X$ and $Y$ be topological spaces. We use $\mathcal{L}(X,Y)$ to denote the linear space of continuous linear maps from $X$ to $Y$. If $X$ and $Y$ are normed spaces, $\mathcal{L}(X,Y)$ is a normed space by taking, for $A\in\mathcal{L}(X,Y)$,
\[\|A\|=\sup_{\|x\|=1}\|Ax\|.\]
We investigate some other ways to topologize $\mathcal{L}(X,Y)$ in this part.\par
Let $X$ be a set and $G$ a topological abelian group with neighborhood filter $\mathfrak{U}(0)$ at $0$. If $\mathcal{S}$ is any collection of subsets in $X$, the $\mathcal{S}$-topology on $G^X$ is generated by the subsets
\[N(S,V)=\{f\in G^X:f(S)\sub V\},\quad S\in\mathcal{S},V\in\mathfrak{U}(0).\]
By Proposition~\ref{S-topo operation on generating set}, we know that $\mathcal{T}_\mathcal{S}$ is unaffected if $\mathcal{S}$ is enlarged to subsets of finite unions of sets in $\mathcal{S}$ in which case the $N(S,V)$ are a base at $0$, rather than just a subbase.\par
Now consider topological vector spaces $X$ and $Y$, and let $\mathfrak{U}(0)$ denote the filter of neighborhoods of $0$ in $Y$. View $\mathcal{L}(X,Y)$ as a subspace of $Y^X$, we can restrict the $\mathcal{S}$-topology to $\mathcal{L}(X,Y)$, which is generated by the sets
\[N(S,V)=\{A\in\mathcal{L}(X,Y):A(S)\sub V\},\quad S\in\mathcal{S},V\in\mathfrak{U}_Y(0).\]
Clearly, for any nonzero scalar $a$ we have $aN(S,V)=N(a^{-1}S,V)=N(S,aV)$. Since $U\sub V$ implies that $N(S,U)\sub N(S,V)$, it follows that if $V$ is balanced, so is $N(S,V)$.\par
When does this becomes a vector topology on $\mathcal{L}(X,Y)$? If $\mathcal{S}$ is a collection of subsets of $X$ which is closed with respect to the formation of subsets of finite unions and the $V$'s are the balanced neighborhoods of $0$ in $Y$, then the $N(S,V)$ form a filter base of balanced sets for a group topology $\mathcal{T}_\mathcal{S}$ on $\mathcal{L}(X,Y)$. By Proposition~\ref{TVS generating topology}, it is evident that if the $N(S,V)$ are absorbent, then $\mathcal{T}_\mathcal{S}$ is a vector topology. The following proposition is an analogue of Proposition~\ref{polar absorbent iff set bounded}.
\begin{proposition}\label{S-topo on L(X,Y) absorbent nbhd iff}
Let $X$ and $Y$ be TVS. Let $S\sub X$ and $V\in\mathfrak{U}_Y(0)$.
\begin{itemize}
\item[(a)] $N(S,V)$ is absorbent in $\mathcal{L}(X,Y)$ iff $V$ absorb $A(S)$ for any $A\in\mathcal{L}(X,Y)$.
\item[(b)] If $S$ is bounded, then $N(S,V)$ is absorbent in $\mathcal{L}(X,Y)$ for every neighborhood $V$ of $0$.
\end{itemize}
\end{proposition}
\begin{proof}
By definition, $N(S,V)$ is absorbent in $\mathcal{L}(X,Y)$ iff, given any $A\in\mathcal{L}(X,Y)$, there exists $r>0$ such that $A\in aN(S,V)=N(S,aV)$ for all $|a|\geq r$ or, quivalently, $A(S)\sub aV$ for $|a|\geq r$, i.e., $V$ absorbs $A(S)$. Thus (a) follows.\par
If $S$ is bounded then so is $A(S)$ for any $A\in\mathcal{L}(X,Y)$ (Proposition~\ref{TVS homogeneous image of bounded is bounded}), and is absorbed by any neighborhood of $0$ in $Y$. Thus $N(S,V)$ is absorbent for any $V$.
\end{proof}
\begin{theorem}[\textbf{$\mathcal{S}$-Topology on $\mathcal{L}(X,Y)$}]\label{S-topo on L(X,Y) prop}
Let $X$ and $Y$ be topological vector spaces, let $\mathcal{B}_Y(0)$ denote a base of balanced neighborhoods of $0$ in $Y$ and let $\mathcal{S}$ be a collection of bounded subsets of $X$ which is closed with respect to the formation of subsets of finite unions. The sets $N(S,V)$ form a base at $0$ for a vector topology $\mathcal{T}_\mathcal{S}$ called the \textbf{$\mathcal{S}$-topology on $\mathcal{L}(X,Y)$}. It is:
\begin{itemize}
\item[(a)] Hausdorff if the linear span of $\bigcup\mathcal{S}$ is dense in $X$ and $Y$ is Hausdorff;
\item[(b)] locally convex if $Y$ is. Moreover, if $\mathscr{P}$ is a family of continuous seminorms which generates the topology on $Y$ then the seminorms $p_S(A)=\sup p(A(S))$ generates $\mathcal{T}_\mathcal{S}$. 
\end{itemize}
\end{theorem}
\begin{proof}
Suppose that $Y$ is Hausdorff and that the linear span of $\bigcup\mathcal{S}$ is dense in $X$. If $A$ is a nonzero element of $\mathcal{L}(X,Y)$ then, because of the density of $\bigcup\mathcal{S}$, there must be some $x\in\bigcup\mathcal{S}$ such that $Ax\neq 0$. Since $Y$ is Hausdorff, there must be some neighborhood $V$ of $0$ in $Y$ such that $Ax\notin V$. Thus, if $x\in S\in\mathcal{S}$, $A\notin N(S,V)$. It follows that $\mathcal{T}_\mathcal{S}$ is Hausdorff.\par
The local convexity statement follows from the observation that if $\mathcal{B}_Y(0)$ is a base of neighborhoods of $0$ in $Y$ then $\{N(S,B):S\in\mathcal{S},B\in\mathcal{B}_Y(0)\}$ is a neighborhood base at $0$ for $\mathcal{T}_\mathcal{S}$; and if $B$ is convex, so is $N(S,B)$.\par
As for the seminorm assertion, if $\mathscr{P}$ is a base of continuous seminorms for $Y$, then $\{\widebar{B}_p:p\in\mathscr{P}\}$ is a base at $0$ for $Y$. Hence $\{N(S,\widebar{B}_p):S\in\mathcal{S},p\in\mathscr{P}\}$ is a base of neighborhoods of $0$ for $\mathcal{T}_\mathcal{S}$. Since, for any $p\in\mathscr{P}$, $S\in\mathcal{S}$, and $A\in\mathcal{L}(X,Y)$,
\[p_S(A)\leq 1\iff \sup p(A(S))\leq 1\iff A(S)\sub\widebar{B}_p\]
it follows that $\widebar{B}_{p_S}=N(S,\widebar{B}_p)$.
\end{proof}
As mentioned above, an $\mathcal{S}$-topology is unchanged if $\mathcal{S}$ is replaced by subsets of finite unions of sets in $\mathcal{S}$. Some other modifications which do not alter $\mathcal{T}_\mathcal{S}$ are listed below.
\begin{proposition}\label{S-topo on L(X,Y) operation on S}
Let $X$ and $Y$ be TVS, $\mathcal{S}$ a collection of hounded subsets of $X$. Let $\mathcal{T}_\mathcal{S}$ be the $\mathcal{S}$-topology on $\mathcal{L}(X,Y)$. Then $\mathcal{S}$ may be replaced by any of the following collections $\mathcal{S}'$ without affecting $\mathcal{T}_\mathcal{S}$:
\begin{itemize}
\item[(a)] subsets of finite unions of sets in $\mathcal{S}$;
\item[(b)] scalar multiples of sets in $\mathcal{S}$;
\item[(c)] finite sums of sets in $\mathcal{S}$;
\item[(d)] balanced hulls or closures of sets in $\mathcal{S}$;
\item[(e)] if $X$ and $Y$ are locally convex, closed disked hulls of sets in $\mathcal{S}$.
\end{itemize}
\end{proposition}
\begin{proof}
Part (a) follows from Proposition~\ref{S-topo operation on generating set}. For (b), clearly $\mathcal{T}_\mathcal{S}\sub\mathcal{T}_{\mathcal{S}'}$. Conversely, by Proposition~\ref{TVS bounded set union sum product}(c), a scalar multiple of a bounded set is bounded and, for $a\neq 0$ and $V$ a neighborhood of $0$ in $Y$, $N(aS,V)=N(S,a^{-1}V)$, a $\mathcal{T}_\mathcal{S}$-neighborhood of $0$.\par
Sums of hounded sets are bounded by Proposition~\ref{TVS bounded set union sum product}(c). Given a neighborhood $V$ of $0$ in $Y$, choose a neighborhood $U$ of $0$ in $Y$ such that $U+U\sub V$. For $S_1,S_2\in\mathcal{S}$, note that $N(S_1,U)\cap N(S_2,U)\sub N(S_1+S_2,V)$. Thus (c) follows.\par
The balanced hull and the closure of a bounded set $S$ are bounded by Proposition~\ref{TVS bounded set union sum product}. Moreover, the balanced hull of $S$ is given by $\bigcup\{aS:|a|\leq 1\}$. If $V$ is balanced and $A(S)\sub V$, then for any $|a|\leq 1$ we have $A(aS)=aA(S)\sub aV\sub V$. Thus $N(S,V)\sub N(\bal(S),V)$. If $V$ is closed and $A\in N(S,V)$, then $A(\widebar{S})\sub\widebar{A(S)}\sub V$ by continuity, so $N(S,V)\sub(\widebar{S},V)$.\par
Suppose $X$ and $Y$ are locally convex. The closed balanced convex hull of a bounded subset of a locally convex space is bounded by Proposition~\ref{TVS bounded set union sum product} and \ref{TVS bounded set closure bal}. Since $\widebar{\convbal}(E)=\widebar{\convbal(E)}$, in view of (d), we only have to show that if the sets of $S$ are replaced by their convex hulls, the $\mathcal{S}$-topology does not change. To this end, let $V$ be a convex neighborhood of $0$ in $Y$ and note that if $A\in N(S,V)$ then $A (\conv(S))\sub V$ or $A\in N(\conv(S),V)$.
\end{proof}
If $\mathcal{S}$ is the class of all finite subsets of $X$, the resulting $\mathcal{S}$-topology for $\mathcal{L}(X,Y)$ is called the \textbf{strong operator topology} or the \textbf{topology of pointwise converyence}. If $\mathcal{S}$ is the class of compact subsets or bounded subsets of $X$, the resulting $\mathcal{S}$-topologies for $\mathcal{L}(X,Y)$ are called, respectively, the \textbf{topology of uniform convergence on compact sets} (or the \textbf{topology of compact convergence}) and the \textbf{topology of uniform convergence on bounded sets} (or the \textbf{topology of bound or bounded convergence} or \textbf{bounded topology}).
\begin{example}[\textbf{Bounded Topology for Normed Spaces}]\label{NVS bounded topo on L(X,Y)}
Suppose that $X$ and $Y$ are normed spaces and that $\mathcal{L}(X,Y)$ is topologized by the uniform norm. Then the topology $\mathcal{T}_u$ induced by the uniform norm is the topology $\mathcal{T}_b$ of uniform convergence on bounded sets. To see this, let $B_X$ and $B_Y$ denote the closed unit balls in $X$ and $Y$, respectively and let $r$ be positive. Since
\[\{A\in\mathcal{L}(X,Y):\|A\|\leq r\}=N(B_X,rB_Y)\]
it follows that $\mathcal{T}_u\sub\mathcal{T}_b$. Let $B$ be a bounded subset of $X$ and consider $N(B,rB_Y)$. Since $B$ is bounded, $B\sub aB_X$ for some $a>0$. Thus $N(B,rB_Y)\sups N(aB_X,rB_Y)=N(B_X,a^{-1}rB_Y)$ and every $\mathcal{T}_b$-neighborhood of $0$ is a uniform neighborhood of $0$.
\end{example}
The following result concerning strong convergence is trivial but extremely useful:
\begin{proposition}\label{bounded operator converge on dense}
Let $X$ and $Y$ be normed vector spaces, $\{T_n\}$ be in $\mathcal{L}(X,Y)$ with $\sup_n\|T_n\|<+\infty$, and $T\in\mathcal{L}(X,Y)$. If $\|T_nx-Tx\|\to 0$ for all $x$ in a dense subset $D$ of $X$, then $T_n\to T$ strongly.
\end{proposition}
\begin{proof}
Let $C$ be an upper bound of $\|T_n\|$ and $\|T\|$. Given $x\in X$ and $\eps>0$, choose $x^*\in D$ such that $\|x-x^*\|<\eps/3$. For $n$ so large that $\|T_nx^*-Tx^*\|<\eps/3$, we have
\[\|T_nx-Tx\|\leq\|T_nx-T_nx^*\|+\|T_nx^*-Tx^*\|+\|Tx^*-Tx\|\leq 2C\|x-x^*\|+\eps/3<\eps,\]
so that $T_nx\to Tx$.
\end{proof}
\begin{example}
If $X$ is a normed space, there is no distinction between norm bounded and $\sigma(X,X^*)$-bounded by Mackey's Theorem on bounded sets (Theorem~\ref{LCS bounded set permanence}). Recall that the strong topology $\beta(X,X^*)$ for $X$ is that determined by the polars of all $\sigma(X^*,X)$-bounded subsets of $X^*$. Consequently, $\beta(X^*,X)=\mathcal{T}_b$ on $X^*=\mathcal{L}(X,\K)$. By Example~\ref{NVS bounded topo on L(X,Y)}, we also have $\mathcal{T}_b=\mathcal{T}_u$, thus $\beta(X^*,X)$ is the norm topology of $X^*$, something already noted in Example~\ref{NVS dual carry beta(X^*,X)}.
\end{example}
\begin{example}
Let $X$ be a topological vector space. The polar topologies are similar to $\mathcal{S}$-topologies, but generally not the same. For polar topologies for $X^*$, $\mathcal{S}$ consists of $\sigma(X,X^*)$-bounded subsets of $X$, not the stronger notion of bounded subsets of $X$. Polar topologies are locally convex; $\mathcal{S}$-topologies need not be. However, if $X$ is locally convex, then by Mackey's theorem on bounded sets, these two definition coincide.
\end{example}
For later use, we characterize bounded subsets in $\mathcal{S}$-topologies.
\begin{proposition}\label{S-topo on L(X,Y) bounded iff}
Let $X$ and $Y$ be topological vector spaces and let $\mathcal{S}$ be a collection of bounded subsets of $X$. A subset $H$ of $\mathcal{L}(X,Y)$ is bounded in the $\mathcal{S}$-topology $\mathcal{T}_\mathcal{S}$ iff for each neighborhood $V$ of $0$ in $Y$, $H^{-1}(V)$ absorbs each $S\in\mathcal{S}$ or, equivalently, $H(S)$ is a bounded subset of $Y$ for each $S\in\mathcal{S}$. In particular, equicontinuous subsets of $\mathcal{L}(X,Y)$ are bounded in any $\mathcal{S}$-topology.
\end{proposition}
\begin{proof}
A subset $H\sub\mathcal{L}(X,Y)$ is $\mathcal{T}_\mathcal{S}$-bounded iff for any $S\in\mathcal{S}$ and any neighborhood $V$ of $0$ in $Y$, there exists $r>0$ such that $|a|\geq r$ implies that $H\sub aN(S,V)=N(S,aV)$ or $H(S)\sub aV$, i.e., that $H(S)$ is bounded. This is equivalent to $S\sub rH^{-1}(V)$.\par
If $H$ is equicontinuous then, for any neighborhood $V$ of $0$ in $Y$, there is a neighborhood $U$ of $0$ in $X$ such that $H(U)\sub V$ or $U\sub H^{-1}(V)$. Consequently, for each neighborhood $V$ of $0$ in $Y$, $H^{-1}(V)$ is a neighborhood of $0$ in $X$ and so absorbs any bounded subset of $X$, hence each $S\in\mathcal{S}$. It follows from the discussion above that $H$ is $\mathcal{T}_\mathcal{S}$-bounded.
\end{proof}
\section{Barreled spaces}
Recall that a barrel is a closed absorbent disk.
\begin{definition}
A locally convex space $X$ is \textbf{barreled} if each barrel in $X$ is a neighborhood of $0$.
\end{definition}
In Proposition~\ref{S-topo on L(X,Y) bounded iff} we observed that, for any topological vector spaces $X$ and $Y$, an equicontinuous subset $H\sub\mathcal{L}(X,Y)$ is bounded in any $\mathcal{S}$-topology for $\mathcal{L}(X,Y)$, in particular, $H$ is bounded in the pointwise convergence topology, which means that $H(x)$ is bounded for each $x\in X$ (Proposition~\ref{S-topo on L(X,Y) bounded iff}). Thus, equicontinuity implies pointwisc boundedness. A result in the converse direction is the Banach-Steinhaus theorem or principle of uniform boundedness: If $X$ is barreled and $Y$ is a locally convex space, then pointwise boundedness in $\mathcal{L}(X,Y)$ implies equicontinuity. A point of this section is to show that the widest class of locally convex spaces $X$ for which pointwise boundedness implies equicontinuity in $\mathcal{L}(X,Y)$, for any topological vector space $Y$, is the class of barreled spaces.\par
Any Baire locally convex space is barreled but there are plenty of barreled incomplete normed spaces as well as normed spaces that are not barreled. The finest locally convex topology $\mathcal{T}_{lc}$ (Example~\ref{LCS finest LC topo nonmetrizable}) for a linear space $X$ has the filterbase $\mathcal{B}$ of all absorbent disks as a base at $0$. Thus, if $B$ is a $\mathcal{T}_{lc}$ barrel, $B\in\mathcal{B}$, so any linear space with the finest locally convex topology is barreled. This suggests that stronger topologies are more likely to be barreled than weaker ones.
\begin{example}\label{C([0,1]) nonbarreled}
Consider the space $X=(C([0,1]),\|\cdot\|_1)$ of continuous $\K$-valued functions on the closed interval $[0,1]$ normed by $\|\cdot\|_1$. We exhibit a barrel $B$ in $X$ which is not a neighborhood of $0$. Let
\[B=\{f\in X:\|f\|_\infty\leq 1\}.\]
That $B$ is absorbent, balanced, and convex is clear. To see that $B$ is closed, let $g\in\widebar{B}$ and $(g_n)$ be a sequence of points from $B$ that is $\|\cdot\|_1$-convergent to $g$. Since $\|g_n-g\|_1\to 0$, $(g_n)$ must possess a subsequence $(g_{n_k})$ that converges to $g$ almost everywhere. Since each $g_{n_k}\in B$, we have $\|g\|_\infty\leq 1$, so $g\in B$. Thus $B$ is a barrel.\par
Is $B$ a neighborhood of $0$ in $X$? If so, then $B$ would have to contain a ball $B_r(0)$ of functions of small integral. Yet no matter how small a function's integral is, its peak can still be arbitrarily large. Since membership in $B$ involves only a peak restriction, it cannot contain any such $B_r(0)$ and therefore cannot be a neighborhood of $0$ in $X$.
\end{example}
\begin{example}\label{c_00 nonbarreled}
Let $c_{00}$ be the linear space of all sequences that are eventually zero, endowed with $\|\cdot\|_\infty$. Let $\Lambda_n:c_{00}\to\K$ be the linear functional such that $\Lambda_n(x)=x_n$, and consider the set
\[B=\{x\in c_{00}:|x_n|\leq n^{-1}\text{ for all $n$}\}=\bigcap_n\Lambda_n^{-1}([-n^{-1},n^{-1}]).\]
Then $B$ is a barrel. Can $B$ contain a neighborhood $V=B_r(0)$ of $0$? For $n$ such that $r/2>n^{-1}$ the sequence whose $n$-th entry is $r/2$, $0$ elsewhere, belongs to $V$ but not $B$. Hence $B$ cannot contain any such neighborhood $V$ and is therefore not a neighborhood of $0$.
\end{example}
Given a topological vector space $X$, for a subset $H$ of $X^*$ to be $\sigma(X^*,X)$-bounded, each seminorm $p_x(f)=|f(x)|$ must be bounded on $H$ for each $x\in X$; in other words, for each $x$ in $X$, $\langle x,H\rangle$ is bounded or $H$ is pointwise bounded. The following theorem says that pointwise boundedness ($\sigma(X^*,X)$-boundedness) of a set of linear functionals on a barreled space implies equicontinuity and characterizes barreled spaces in the class of locally convex spaces.
\begin{theorem}[\textbf{Banach-Steinhaus Theorem for Linear Functionals}]\label{Banach-Steinhaus for linear functional}
A locally convex space $X$ is barreled iff $\sigma(X^*,X)$-bounded subsets of $X^*$ are equicontinuous.
\end{theorem}
\begin{proof}
Suppose that $(X,\mathcal{T})$ is barreled and that $H$ is a $\sigma(X^*,X)$-bounded subset of $X^*$. Since $H$ is $\sigma(X^*,X)$-bounded, it is contained in the polar $B^\circ$ of some barrel $B$ in $X$ (Proposition~\ref{polar topo bounded iff}). Since $X$ is barreled, $B$ is a neighborhood of $0$; since $H\sub B^\circ$, equicontinuity of $H$ follows from Proposition~\ref{equicontinuous iff polar is nbhd}(a).\par
Conversely, suppose that $\sigma(X^*,X)$-boundedness implies equicontinuity and let $B$ be a barrel in the locally convex space $X$. By Proposition~\ref{dual topo barrel iff polar of weak bounded}, $B=H^\circ$ for some $\sigma(X^*,X)$-bounded subset $H\sub X^*$. By hypothesis $H$ is equicontinuous, so $H^\circ$ is a neighborhood of $0$ (Proposition~\ref{equicontinuous iff polar is nbhd}), it follows that $B$ is a neighborhood of $0$ in $X$, thus $X$ is barreled.
\end{proof}
Since equicontinuous subsets in $X^*$ are relatively $\sigma(X^*,X)$-compact by Theorem~\ref{weak topo equicontinuous is precompact}, we have the following corollary.
\begin{corollary}
Let $X$ be a barreled space. Then every $\sigma(X^*,X)$-bounded subset in $X^*$ is relatively $\sigma(X^*,X)$-compact.
\end{corollary}
We turn to another characterization for barreled spaces. Recall that an equicontinuous set is bounded in any $\mathcal{S}$-topology.
\begin{proposition}\label{LCHS equicontinuous is strongly bounded}
Let $X$ be a locally convex space. If $H$ is an equicontinuous subset of $X^*$, then $H$ is $\beta(X^*,X)$-bounded.
\end{proposition}
\begin{proof}
It follows from Proposition~\ref{S-topo on L(X,Y) bounded iff} that an equicontinuous subset of $X^*$ is bounded in the topology $\mathcal{T}_b$ of uniform convergence on bounded subsets of $X$. By Mackey's theorem on bounded sets, the bounded subsets of $X$ are the $\sigma(X,X^*)$-bounded sets so $\mathcal{T}_b=\beta(X^*,X)$.
\end{proof}
\begin{theorem}\label{LCS barreled iff carry strong topology}
A locally convex space $(X,\mathcal{T})$ is barreled iff $\mathcal{T}=\beta(X,X^*)$.
\end{theorem}
\begin{proof}
Suppose that $(X,\mathcal{T})$ is barreled. By Theorem~\ref{Banach-Steinhaus for linear functional} we have $\beta(X,X^*)=\eps(X,X^*)$. Since $X$ is locally convex, $\mathcal{T}=\eps(X,X^*)$ by Proposition~\ref{LCS topo is eps(X,X^*)}. Thus, $\mathcal{T}$ coincides with the topology $\beta(X,X^*)$ of uniform convergence on $\sigma(X^*,X)$-bounded subsets of $X^*$.\par
Conversely, suppose that $\mathcal{T}=\beta(X,X^*)$. If $E\sub X^*$ is $\sigma(X^*,X)$-bounded, then $E^\circ$ is a $\beta(X,X^*)$-neighborhood of $0$ by definition. Since $E\sub E^{\circ\circ}$, $E$ is a subset of the polar of a neighborhood of $0$, so is equicontinuous by Proposition~\ref{equicontinuous iff polar is nbhd}. Thus $X$ is barreled by Theorem~\ref{Banach-Steinhaus for linear functional}.
\end{proof}
Finally, we have a further characterization for barreled spaces. For this, recall that a function $f:X\to\R$ on a topological space $X$ is lower semicontinuous iff $f^{-1}((a,+\infty))$ is open for any $a\in\R$.
\begin{proposition}\label{TVS barrel is closed ball of lsc seminorm}
Let $X$ be a topological vector space. Then the barrels in $X$ are of the form $\widebar{B}_p$ where $p$ is a lower sernicontinuous serninorm.
\end{proposition}
\begin{proof}
Suppose that $B$ is a barrel in $X$ and let $p$ be its gauge. Since $B$ is closed, $B=\widebar{B}_p$ by Proposition~\ref{TVS absorbent disk inclusion}. The lower semicontinuity of $p$ follows from the fact that for each $a>0$, 
\[p^{-1}((-\infty,a])=p^{-1}([0,a])=a\widebar{B}_p=aB.\]
On the other hand, if $p$ is a lower semicontinuous seminorm on $X$, it is clear that $\widebar{B}_p=p^{-1}([0,1])$ is a barrel.
\end{proof}
\begin{theorem}\label{barreled iff lsc seminorm is continuous}
The locally convex space $X$ is barreled iff each lower semicontinuous seminorm on $X$ is continuous.
\end{theorem}
\begin{proof}
By Proposition~\ref{TVS barrel is closed ball of lsc seminorm}, any barrel in $X$ is of the form $\widebar{B}_p$ for some lower semicontinuous seminorm $p$. Since a seminorm $p$ is continuous iff $\widebar{B}_p$ is a neighborhood (Proposition~\ref{seminorm continuous iff}), the claim follows.
\end{proof}
\section{Baire topological vector spaces}
We are still left to show that Banach spaces are barreled, so that Theorem~\ref{Banach-Steinhaus for linear functional} can be viewed as a generalization of the usual Banach-Steinhaus theorem. In fact, we have more:
\begin{proposition}\label{LCS nonmeager is barreled}
Any nonmeager locally convex space is barreled.
\end{proposition}
\begin{proof}
Let $B$ be a barrel in the nonmeager locally convex space $X$. Since $B$ is absorbent, $X=\bigcup_{n\in\N}nB$. Since $X$ is nonmeager, there exists $n\in\N$ such that $nB$ has nonempty interior; hence so does $B$. As $B$ is a disk, however, so is $\Int B$ (Proposition~\ref{TVS bal conv absor int and closure}); therefore $0\in\Int B$ and $B$ is a neighborhood of $0$.
\end{proof}
Note that not all barreled spaces are Baire spaces, however. An example is given below.
\begin{example}\label{LCS finest LC topo barreled nonBaire}
Let $X$ be an infinite-dimmensional vector space endowed with the finest locally convex topology $\mathcal{T}_{lc}$, the inverse image topology determined by the class of all seminorms on $X$ or, equivalently, the topology that has the filterbase of all absorbent disks as a base at $0$. As noted before, $(X,\mathcal{T}_{lc})$ is barreled. A few more observations about $\mathcal{T}_{lc}$ are now in order.
\begin{itemize}
\item[(a)] If $x\neq 0$, the map $ax\mapsto a$ defines a nontrivial linear functional on $\K x$. By extending $\{x\}$ to a Hamel base for $X$ and defining $f$ arbitrarily on the rest of the basis vectors, we get a linear functional $f$ is defined on all of $X$. The seminorm $|f|$ then does not vanish on $x$ and therefore $\mathcal{T}_{lc}$ is a Hausdorff topology by Proposition~\ref{seminorm topo Hausdorff iff}.
\item[(b)] Given any linear functional $f$ on $X$, $|f|$ is a seminorm on $X$, so it continuous. Thus any linear functional on $\mathcal{T}_{lc}$ is continuous.
\item[(c)] By a similar argument as (a), if $x$ is not in the subspace $M$, there exists a linear functional $f$ on $X$ which vanishes on $M$ but not on $x$; $f$ is continuous by (b). By Proposition~\ref{LCHS linear functional prop}, $x\in\widebar{M}$ iff every continuous linear functional which vanishes on $M$ vanishes on $x$. Thus any subspace in $X$ is closed.
\end{itemize}
To see that $(X,\mathcal{T}_{lc})$ is meager, let $A=\{x_n\}$ be a denumerable subset of a Hamel base $B$ for $X$. For each $n\in\N$, let $M_n$ be the linear span of $B\setminus A$ and $\{x_1,\dots,x_n\}$. As each $M_n$ is a closed proper subspace, each is a nowhere dense subset of $X$. Since $X=\bigcup_nM_n$, $X$ is meager.
\end{example}
\begin{example}[\textbf{Infinite-Dimensional Baire Space}]
Let $X$ be a Hausdorff topological vector space and suppose that $\{x_n\}$ is a Hamel base for $X$. For each $n$ the linear span $M_n$ of $\{x_1,\dots,x_n\}$ is closed by Proposition~\ref{TVS finite dim prop}(d). Since each $M_n$ is nowhere dense and $X=\bigcup_nM_n$, this means that $X$ is meager. Hence any infinite-dimensional Baire space has uncountable dimension. In particular, by the Baire category theorem, any infinite-dimensional complete pseudometrizable topological vector space is of uncountable dimension.
\end{example}
Suppose $X$ is an infinite-dimensional normed space. As such, $X^*$ is infinitedimensional a.s well and therefore the $\sigma(X,X^*)$-neighborhoods of $0$ are unbounded; hence $\sigma(X,X^*)$ is not normable. But when is $\sigma(X,X^*)$ metrizable? The non-metrizability of the weak topology for infinite-dimensional normed spaces was discovered by Wehausen.
\begin{example}[\textbf{Weak Topology is Metrizable iff Finite-Dimensional}]\label{weak topo metrizable iff finite dim}
If $X$ is a finite dimensional Hausdorff topological vector space then $X^*=X^{\star}$ distinguishes points in $X$, so $\sigma(X,X^*)$ is Hausdorff and $(X,\sigma(X,X^*))$ is linearly homeomorphic to $\K^n$; thus $\sigma(X,X^*)$ is metrizable.\par
If $X$ is an infinite-dimensional normed space, then $X^*$ is a Banach space, hence a Baire space, and therefore the dimension of $X^*$ is uncountable. We show next that $\sigma(X,X^*)$ is not first countable, hence not metrizable. To see this, suppose that $\{V_n\}$ is a denumerable base of $\sigma(X,X^*)$-neighborhoods of $0$. Each $V_n$ is the polar of a finite subset $F_n$ of $X^*$. Since $F=\bigcup_nF_n$ is denumerable and $X^*$ is of nondenumerable dimension, the linear span of $F$ is a proper subset of $X^*$. Hence, there is some $g\in X^*$ which is linearly independent of $F$. Thus, $\bigcap_{f\in F_n}N(f)\not\sub N(g)$ for each $n$. This means there must be some $x\in\bigcap_{f\in F_n}N(f)$ such that $g(x)>1$. Thus $x\in V_n$ but $x\notin\{g\}^\circ$, so $V_n\not\sub\{g\}^\circ$ for any $n$ and $\{V_n\}$ cannot be a base at $0$ for $\sigma(X,X^*)$.
\end{example}
Recall that nonmeager topological spaces need not to be Baire. However, things are different for topological vector spaces.
\begin{proposition}\label{TVS Baire iff nonmeager}
A topological vector space $X$ is a Baire space iff it is nonmeager.
\end{proposition}
\begin{proof}
Since a Baire space is nonmeager, we need only show that a nonmeager topological vector space is Baire. To this end, let $X$ be a topological vector space which is not a Baire space. As such, by Theorem~\ref{Baire space iff}, it must have a meager neighborhood $V$ of $0$. Since $V$ is absorbent, we then have $X=\bigcup_nnV$ and so $X$ is meager.
\end{proof}
In fact, we have a stronger result about Baire topological vector space. For this, we need the following lemma.
\begin{lemma}\label{interior lemma}
Let $A$ and $B$ be subsets of a topological space. If $B$ is closed and $\Int A=0$, then $\Int(A\cup B)=\Int B$.
\end{lemma}
\begin{proof}
Let $x\in\Int(A\cup B)$. Let $U$ be an open neighborhood of $x$ such that $U\sub A\cup B$. Then $U\cap B^c\sub A$. But $\Int A=\emp$ and $U\cap B^c$ is open, so $U\cap B^c=\emp$, i.e., $U\sub B$.
\end{proof}
\begin{proposition}\label{TVS Baire iff not covered by multiple of nowhere dense}
A topological vector space $X$ is Baire iff there is no nowhere dense set $B$ for which $X=\bigcup_nnB$.
\end{proposition}
\begin{proof}
If $X$ is Baire, it is nonmeager so there cannot be a nowhere dense set $B$ for which $X=\bigcup_nnB$. Conversely, assume that $X$ is not a Baire space. As such, it must be meager by Theorem~\ref{TVS Baire iff nonmeager}, so let $\{B_n\}$ be a sequence of nowhere dense sets which covers $X$. Since not all the $B_n$ can be empty, the topology on $X$ cannot be the indiscrete topology $\{X,\emp\}$, so there exists a proper balanced neighborhood $U$ of $0$. Choose a closed balanced neighborhood $V$ of $0$ such that $V+V\sub U$, and consider the set $B=\bigcup_{n}n^{-1}(V\cap B_n)$. Note that for any $n\in\N$, $V\cap B_n\sub nB$; so for any $k\in\N$ we have
\[kV=k\bigcup_{n=1}^{\infty}(V\cap B_n)\sub k\bigcup_{n=1}^{\infty}nB\sub\bigcup_{n=1}^{\infty}nB.\]
Since $V$ is absorbent and balanced, we have $X=\bigcup_kkV$, thus $X=\bigcup_nnB$. We claim that $B$ is nowhere dense. To see this, suppose that $y\in\Int\widebar{B}$ and let $W$ be an open neighborhood of $x$ such that $y\in W\sub\widebar{B}$. Since $B=\bigcup_nn^{-1}(V\cap B_n)$, for any $k\in\N$ we have
\[\widebar{B}=\widebar{\bigcup_{n<k}n^{-1}(V\cap B_n)}\cup\widebar{\bigcup_{n\geq k}n^{-1}(V\cap B_n)}.\]
Since each $V\cap B_n$ is nowhere dense, $\bigcup_{n<k}n^{-1}(V\cap B_n)$ is nowhere dense; therefore, by Lemma~\ref{interior lemma},
\[W\sub\widebar{\bigcup_{n\geq k}n^{-1}(V\cap B_n)}\sub\widebar{k^{-1}V}=k^{-1}V.\]
Let $x\in X$ and choose $r>0$ so that $y\pm rx\in W$. Then we have
\[2rx=(y+rx)-(y-rx)\in W-W\sub k^{-1}(V-V)\sub k^{-1}U.\]
Since $k$ is arbitrary and $U$ is balanced, this implies $x\in U$, which is a contradiction since $U\neq X$.
\end{proof}
\begin{proposition}\label{TVS Baire absorbent is nonmeager}
If $E$ is an absorbent subset of the Baire topological vector space $X$, then $E$ is nonmeager.
\end{proposition}
\begin{proof}
Suppose that $E$ is absorbent subset of $X$ and that $(F_n)$ is a sequence of closed sets such that $E\sub\bigcup_nF_n$. For each positive integer $m$, $mE\sub\bigcup_{n}mF_n$, so $X=\bigcup_mmE=\bigcup_m\bigcup_nmF_n$. Since $X$ is nonmeager, there must be integers $n$ and $m$ such that $\Int mF_n=\emp$, which implies that $\Int F_n=\emp$ and $E$ is seen to be nonmeager.
\end{proof}
\begin{proposition}\label{TVS Baire iff no nowhere dense balanced absorbent}
A topological vector space $X$ with a indiscrete topology is a Baire space iff there are no nowhere dense balanced absorbent sets in $X$.
\end{proposition}
\begin{proof}
By Proposition~\ref{TVS Baire iff nonmeager}, if $X$ is a Baire space there cannot be a nowhere dense subset $B$ such that $X=\bigcup_nnB$, hence no nowhere dense balanced absorbent set. We argue the converse for complex vector spaces only and just mention what happens in the real case which is much easier to prove.\par
If $X$ is not a Baire space, it is meager by Proposition~\ref{TVS Baire iff nonmeager}, so let $\{E_n\}$ be a sequence of closed nowhere dense sets which covers $X$. Let $U$ be a closed balanced proper neighborhood of $0$, and choose a closed balanced neighborhood $V$ of $0$ such that $V+V\sub U$. For each positive integer $n$, let $A_n=V\cap E_n$ so that $\{A_n\}$ is a sequence of closed rare sets which covers $V$. We combine and rotate the $A_n$ as follows: For each $n\in\N$, let
\[B_n=\bigcup_{k=0}^{n-1}e^{\frac{2k\pi i}{n}}(A_1\cup\cdots\cup A_n).\]
(In the real case, take $B_n=(A_1\cup\cdots\cup A_n)\cup(-1)(A_1\cup\cdots\cup A_n)$.) Using $1/n$ as an astringent, as in the proof of Proposition~\ref{TVS Baire iff not covered by multiple of nowhere dense}, we use contractions of $B_n$ to form the rare absorbent set $B=\bigcup_nn^{-1}B_n$. The balanced core of $B$ is the set we ultimately seek. The argument of Proposotion~\ref{TVS Baire iff not covered by multiple of nowhere dense} mutatis mutandis yields the nowhere denseness of $B$. The remainder of the argument is devoted to proving that $B$ is absorbent.\par
For any $y\in X$, $V\cap\C y$ is a complete pesudometrizable space, hence is Baire. Since the sets $\{A_n\cap\C y:n\in\N\}$ cover $V\cap\C y$, one of them, $A_N\cap\C y$ say, must have nonempty interior in the relative topology on $V\cap\C y$, i.e., there must exist $a\in\C$ and $r>0$ such that $B_r(a)y\sub A_N$. Since $e^{it}$ is uniformly continuous on $[0,2\pi]$, there exists an integer $p\geq N$ such that
\begin{align}\label{TVS Baire iff no nowhere dense balanced absorbent-1}
|e^{is}-e^{it}||a|\leq r/2\for|s-t|<2\pi/p.
\end{align}
Suppose $n\geq p$. Then given $\theta\in[0,2\pi]$, there exists an integer $k$ with $0\leq k\leq n-1$ such that $|\theta-2k\pi/n|\leq 2\pi/n\leq 2\pi/p$, so that
\begin{align}\label{TVS Baire iff no nowhere dense balanced absorbent-2}
|e^{i\theta}-e^{\frac{2k\pi i}{n}}||a|\leq r/2.
\end{align}
Since $B_n\sups\bigcup_{k=0}^{n-1}e^{\frac{2k\pi i}{n}}A_N$, $(\ref{TVS Baire iff no nowhere dense balanced absorbent-2})$ and $(\ref{TVS Baire iff no nowhere dense balanced absorbent-3})$ implies
\begin{align}\label{TVS Baire iff no nowhere dense balanced absorbent-3}
by\in B_n\for||b|-|a||\leq r/2.
\end{align}
Now choose $n\geq p$ so that $|a|/(n+1)\leq r/2$. For any $d\in\C$ such that $|d|\leq |a|/n$, there exists $m\geq n$ such that $|a|/(m+1)\leq|d|\leq|d|/m$. Since
\[\frac{|a|-r/2}{m}\leq\frac{|a|-|a|/(n+1)}{m}\leq\frac{|a|-|a|/(m+1)}{m}=\frac{|a|}{m+1}\leq |d|\leq\frac{|a|}{m}\leq\frac{|a|+r/2}{m},\]
it follows that $|m|d|-|a||\leq r/2$ and therefore that $d y\in (1/m)B_m$. So for any $d$ such that $0\leq|d|\leq |a|/n$ (since $0$ is clearly in $B$), $d y\in B$ and $B$ is seen to be absorbent. The balanced core of $B$ is the desired nowhere dense balanced absorbent set (Corollary~\ref{TVS balanced hull and core of closed open}).
\end{proof}
\section{Banach-Steinhaus theorem}
\begin{theorem}[\textbf{Banach-Steinhaus Theorem for LCS}]\label{Banach-Steinhaus for LCS}
Let $X$ be a barreled space. For any locally convex space $Y$, if $H\sub\mathcal{L}(X,Y)$ is pointwise bounded, then $H$ is equicontinuous.
\end{theorem}
\begin{proof}
Let $H\sub\mathcal{L}(X,Y)$ be pointwise bounded. Given a neighborhood $V$ of $0$ in $Y$, we must find a neighborhood $U$ of $0$ in $X$ such that $H(U)\sub V$. Since $Y$ is locally convex, it has a base of barrels at $0$ (Proposition~\ref{LCS nbhd of disked sets}), so we may assume that $V$ is a barrel. By continuity and linearity, each $A^{-1}(V)$ is a closed disk. Since these properties are stable under intersection, $U=H^{-1}(V)$ is a closed disk as well. If we can show that $U$ is absorbent, $U$ will be a barrel, hence a neighborhood of $0$ in $X$, and the equicontinuity of $H$ will follow. But the absorbency of $U$ means, given $x\in X$, for all sufficiently small scalars $a$ we have $ax\in U=H^{-1}(V)$, or equivalently, $aH(x)\sub V$. Since $H$ is pointwise bounded, the condition is satisfied and the theorem follows.
\end{proof}
A Banach space $X$ is nonmeager by the Baire category theorem, hence barreled. Therefore if $Y$ is any normed space and $H\sub\mathcal{L}(X,Y)$ is pointwise bounded, then it is equicontinuous by Theorem~\ref{Banach-Steinhaus for LCS}. Therefore it is bounded in the topology of uniform convergence on bounded sets (Proposition~\ref{S-topo on L(X,Y) bounded iff}). For normed spaces, this means that $H$ is bounded in the norm topology on $\mathcal{L}(X,Y)$, i.e., $\sup_{A\in H}\|A\|<+\infty$. In summary:
\begin{theorem}[\textbf{Banach-Steinhaus Theorem for Normed Spaces}]\label{Banach-Steinhaus for NVS}
If $H$ is a collection of continuous linear maps of a Banach space $X$ into a normed linear space $Y$ such that at every $x\in X$, $\sup_{A\in H}\|Ax\|<+\infty$, then $\sup_{A\in H}\|A\|<+\infty$.
\end{theorem}
The following converse shows that in the class of locally convex spaces the principle of uniform boundedness will not stretch beyond the barreled spaces.
\begin{proposition}
Let $X$ be a locally convex space and let $Y$ be a nonzero topological vector space not carrying the indiscrete topology. If pointwise boundedness implies equicontinuity in $\mathcal{L}(X,Y)$ then $X$ is barreled.
\end{proposition}
\begin{proof}
To prove that $X$ is barreled, we show that each $\sigma(X^*,X)$-bounded subset $B$ of $X^*$ is equicontinuous. To that end, let $y$ be a vector in $Y$ not in $\widebar{\{0\}}$. For each $f\in B$, consider the map $yf:X\to Y,x\mapsto f(x)y$. Clearly $yf\in\mathcal{L}(X,Y)$. Since $B$ is $\sigma(X^*,X)$-bounded, $yB=\{yf:f\in B\}$ is a pointwise bounded subset of $\mathcal{L}(X,Y)$, hence equicontinuous by hypothesis. To see that $B$ is equicontinuous, let $W$ he a balanced neighborhood of $0$ in $Y$ to which $y$ does not belong. Since $yB$ is equicontinuous, there is a neighborhood $V$ of $0$ in $X$ such that $B(V)y\sub W$. We claim that $B\sub V^\circ$, which implies that $B$ is equicontinuous by Proposition~\ref{equicontinuous iff polar is nbhd}. Since $B(V)y\sub W$, if there exists $a\in B(V)$ such that $|a|>1$, then $ay\in W$. Since $W$ is balanced, this implies $y\in W$, contradicting the way in which $W$ was chosen. Consequently $B(V)\sub\{a\in\K:|a|\leq 1\}$ and the claim follows.
\end{proof}
\begin{proposition}
Let $X$ be barreled, $Y$ a locally convex space and $Y^X$ denote the space of all maps of $X$ into $Y$. If $(A_\alpha)$ is a pointwise bounded net from $\mathcal{L}(X,Y)$ that converges to $A\in Y^X$ pointwise, then $A\in\mathcal{L}(X,Y)$ and $A_\alpha\to A$ in the compact-open topology.
\end{proposition}
\begin{proof}
That such a limit $A$ is linear is straightforward to show. Since $(A_\alpha)$ is pointwise bounded, it is equicontinuous by the Banach-Steinhaus theorem. The pointwise closure $\mathrm{cl}_p\{A_\alpha\}$ of the equicontinuous subset $\{A_\alpha\}\sub\mathcal{L}(X,Y)$ is equicontinuous by an argument similar to that of Lemma~\ref{equicontinuous pointwise closure}. Since $A\in\mathrm{cl}_p\widebar{\{A_\alpha\}}$, A is continuous. Since the point-open and compact-open topologies coincide on equicontinuous subsets of $\mathcal{L}(X,Y)$, $A_\alpha\to A$ in the compact-open topology.
\end{proof}
Now we provide some variants of the Banach-Steinhaus theorem that do not involve convexity; the idea is to substitute nonmeager for barreled.
\begin{theorem}[\textbf{Banach-Steinhaus Theorem Without Convexity}]\label{Banach-Steinhaus without convexity}
Let $X$ and $Y$ be topological vector spaces and $H\sub\mathcal{L}(X,Y)$. Then:
\begin{itemize}
\item[(a)] If $H$ is pointwise bounded on a nonmeager set $B\sub X$ then $H$ is equicontinuous and $H$ is poiutwise bounded on $X$. In particular, if $X$ is a Baire space and $H$ is pointwise bounded on $X$, then $H$ is equicontinuous.
\item[(b)] If $X$ is nonmeager and the pointwise bounded net $(A_\alpha)$ of $\mathcal{L}(X,Y)$ converges pointwise to the map $A$, then $A$ is a continuous linear map.
\end{itemize}
\end{theorem}
\begin{proof}
Suppose that $H$ is pointwise bounded on a nonmeager set $B\sub X$. Given a neighborhood $V$ of $0$ in $Y$, we produce a neighborhood $U$ of $0$ in $X$ such that $H(U)\sub V$. Choose a closed balanced neighborhood $W$ of $0$ such that $W+W\sub V$. For any $x\in B$, $H(x)$ is bounded so there exists $n\in\N$ such that $H(x)\sub nW$. Let $C=H^{-1}(W)$, then $B\sub\bigcup_nnC$. Since $C$ is closed, the nonmeagerness of $B$ implies that some $nC$---hence $C$ itself---has a nonempty interior. Suppose that $x\in\Int C$ and that $U$ is a neighborhood of $0$ such that $x+U\sub C$. For any $A\in H$, we then have $Ax+A(U)\sub A(C)$, and so 
\[A(U)\sub A(C)-Ax\sub W-W\sub V,\]
which implies $H(U)\sub V$. Thus $H$ is equicontinuous.\par
Since $H$ is equicontinuous, $H$ is bounded in any $\mathcal{S}$-topology by Proposition~\ref{S-topo on L(X,Y) bounded iff}. Hence $H$ is bounded in the topology of pointwise convergence and therefore $H(x)$ is bounded at each $x\in X$.\par
Let $X$ be nonmeager and $(A_\alpha)$ be as in (b). Obviously, such a limit $A$ is linear. Since $\{A_\alpha\}$ converges, it is bounded at each point of $X$, so is equicontinuous by (a). Thus, given a closed neighborhood $V$ of $0$ in $Y$, there is a neighborhood $U$ of $0$ in $X$ such that $A_\alpha(U)\sub V$ for each index $\alpha$, which implies that $A(U)\sub\widebar{V}=V$ and establishes the continuity of $A$.
\end{proof}
\begin{proposition}\label{Banach-Steinhaus pointwise bounded on convex compact}
Let $X$ be a Hausdorff topological vector space, $Y$ be a topological vector space, and $H\sub\mathcal{L}(X,Y)$.  If $H$ is pointwise bounded on a compact subset $K$ of $X$, then $H(K)$ is bounded.
\end{proposition}
\begin{proof}
Since $K$ is compact Hausdorff, it is Baire. By Theorem~\ref{Banach-Steinhaus without convexity} $H$ is equicontinuous, hence is bounded in compact convergence topology. Then $H(K)$ is bounded by Proposition~\ref{S-topo on L(X,Y) bounded iff}.
\end{proof}
To conclude, we provide some applications for the Banach-Steinhaus theorem.
\begin{proposition}\label{TVS bilinear jointly continuous}
Suppose $B:X\times Y\to Z$ is bilinear and jointly continuous, $X$ is a Baire space, and $Y$ and $Z$ are topological vector spaces. Then $B$ is continuous.
\end{proposition}
\begin{proof}
Assume that $x_\alpha\to x_0$ and $y_\alpha\to y_0$. Let $U$ and $V$ be neighborhoods of $0$ in $Z$ such that $U+U\sub V$. Define $A_\alpha(x)=B(x,y_\alpha)$. Since $B$ is continuous as a function of $y$, we have $\lim_\alpha A_\alpha(x)=B(x,y)$. Thus $\{A_\alpha\}$ is a pointwise bounded subset of $Z$. Since each $A_\alpha$ is a continuous linear mapping of the Baire space $X$, the Banach-Steinhaus theorem implies that $\{A_\alpha\}$ is equicontinuous. Hence there is a neighborhood $V$ of $0$ in $X$ such that $A_\alpha(V)\sub U$. Note that
\[B(x_\alpha,y_\alpha)-B(x_0,y_0)=A_\alpha(x_\alpha-x_0)+B(x_0,y_\alpha-y_0).\]
Since $x_\alpha\to x_0$ and $y_\alpha\to y_0$, there exist an index $\beta$ such that $x_\alpha\in x_0+V$ for $\alpha\succeq\beta$, so that $A_\alpha(x_\alpha-x_0)\in U$, and $B(x_0,y_\alpha-y_0)\in U$. Hence
\[B(x_\alpha,y_\alpha)-B(x_0,y_0)\in U+U\sub V.\]
This gives the claim.
\end{proof}
\begin{proposition}[\textbf{Principle of Condensation of Singularities}]
Let $X$ be a Banach space, $(Y_n)$ a sequence of normed vector spaces, and $H_n$ an unbounded family in $\mathcal{L}(X,Y_n)$ for each $n$. Then the set
\[R=\{x\in X:\sup_{A\in H_n}\|Ax\|=+\infty\text{ for all $n$}\}\]
is residual, and thus dense in $X$.
\end{proposition}
\begin{proof}
If we define $X_{n,m}=\{x\in X:\sup_{A\in H_n}\|Ax\|\leq m\}$, then the complement of $R$ is given by the countable union $\bigcup X_{n,m}$. Since each $H_n$ is unbounded, the set $X_{n,m}$ is meager by Theorem~\ref{Banach-Steinhaus without convexity}. Therefore, the complement of $R$ is also meager, so $R$ is residual. The fact that $R$ is dense comes from Theorem~\ref{Baire space iff}.
\end{proof}
\section{Infrabarreled spaces}
We introduce infrabarreled spaces here. A subset $D$ of a topological vector space is \textbf{bornivorous} or a \textbf{bornivore} if it absorbs bounded sets. A locally convex space is \textbf{infrabarreled} (or \textbf{quasi-barreled}) if every bornivorous barrel is a neighborhood of $0$. Obviously, barreledness implies infrabarreledness. Normed spaces need not be barreled---$(C([0,1]),\|\cdot\|_1)$ is not barreled, for example---but they are infrabarrcled as we will show.
\begin{proposition}\label{NVS is infrabarreled}
In a normed space $X$, every bornivore is a neighborhood of $0$; consequently, any normed space is infrabarreled.
\end{proposition}
\begin{proof}
It suffices to note that $X$ has a bounded neighborhood of $0$.
\end{proof}
We have already seen that $\mathcal{T}_{\mathcal{S}^\circ}$-boundedness in a locally convex space $X$ is equivalent to being a subset of the polar of a barrel that absorbs $\mathcal{S}$. For $\beta(X^*,X)$-boundedness, we have the following result.
\begin{proposition}\label{LCS strongly bounded iff in polar of bornivorous barrel}
Let $X$ be a locally convex space. A subset $H$ of $X^*$ is $\beta(X^*,X)$-bounded iff $H$ is contained in the polar of a bornivorous barrel $B$ in $X$.
\end{proposition}
\begin{proof}
By Proposition~\ref{polar topo bounded iff}, $H$ is $\beta(X^*,X)$-bounded iff $H$ is contained in the polar of a barrel $B$ in $X$ that absorbs every $\sigma(X,X^*)$-bounded sets. By Theorem~\ref{LCS bounded set permanence} $\sigma(X,X^*)$-bounded sets are bounded in the origin topology. Thus the claim follows.
\end{proof}
A locally convex space $X$ is barreled iff $\sigma(X^*,X)$-boundedness implies equicontinuity. For infrabarreled spaces, we have the following analogue:
\begin{theorem}\label{Banach-Steinhaus for infrabarreled space}
A locally convex space $X$ is infrabarreled iff each $\beta(X^*,X)$-bounded subset of $X^*$ is equicontinuous.
\end{theorem}
\begin{proof}
Suppose that $X$ is infrabarreled and that $H$ is $\beta(X^*,X)$-bounded subset of $X^*$. By Proposition~\ref{LCS strongly bounded iff in polar of bornivorous barrel}, there is a bornivorous barrel $B$ in $X$ such that $H\sub B^\circ$. Since $X$ is infrabarreled, $B$ is a neighborhood of $0$ and $H$ is equicontinuous by Proposition~\ref{equicontinuous iff polar is nbhd}.\par
Conversely, suppose that $\beta(X^*,X)$-boundedness implies equicontinuity and let $B$ be a bornivorous barrel in $X$. Since $B$ is bornivorous, $B^\circ$ is $\beta(X^*,X)$-bounded by Proposition~\ref{LCS strongly bounded iff in polar of bornivorous barrel}, hence equicontinuous. The polar of an equicontinuous set is a neighborhood of $0$ by Proposition~\ref{equicontinuous iff polar is nbhd} so $B^{\circ\circ}$ is a neighborhood of $0$. It only remains to observe that $B^{\circ\circ}=B$ by Proposition~\ref{dual topo permanence of closed convex}.
\end{proof}
A relatively compact subset of a TVS is totally bounded, hence hounded; hence a relatively $\sigma(X^*,X)$-compact set is $\sigma(X^*,X)$-bounded. For a locally convex space $X$, the bounded sets are the same in any topology of a dual pair; hence relatively $\sigma(X^*,X)$-compact sets are $\tau(X^*,X)$-bounded. We show next that relatively $\sigma(X^*,X)$-compact convex sets are even $\beta(X^*,X)$-bounded.
\begin{proposition}\label{weak precompact convex is strongly bounded}
Every $\sigma(X^*,X)$-relatively compact convex subset of $X^*$ is $\beta(X^*,X)$-bounded.
\end{proposition}
\begin{proof}
Let $H$ be a $\sigma(X^*,X)$-relatively compact convex subset of $X^*$. Then $\mathrm{cl}_{\sigma(X^*,X)}H$ is convex by Proposition~\ref{TVS convex set prop} and $E:=\convbal(\mathrm{cl}_{\sigma(X^*,X)}H)$ is $\sigma(X^*,X)$-compact by an argument like Proposition~\ref{TVS convex hull of finite union of compact}. Thus $E^\circ$ is a basic neighborhod of $\tau(X,X^*)$ so it absorbs every $\tau(X,X^*)$-bounded set, hence every $\sigma(X,X^*)$-bounded set. By Proposition~\ref{polar topo bounded iff}, $E$ is then $\beta(X^*,X)$-bounded, and so is $H$.
\end{proof}
We showed in Proposition~\ref{LCS barreled iff carry strong topology} that barreled spaces $X$ carry $\beta(X,X^*)$. A similar result holds for infrabarreled spaces.
\begin{proposition}\label{LCS infrabarreled carry tau(X,X^*)}
Let $(X,\mathcal{T})$ be a locally convex space. If $X$ is infrabarreled, then $\mathcal{T}=\tau(X,X^*)$.
\end{proposition}
\begin{proof}
By Mackey-Arens theorem, $\mathcal{T}\sub\tau(X,X^*)$. Conversely, if $B$ is a $\sigma(X^*,X)$-compact disk of $X^*$, it is $\beta(X^*,X)$-bounded by Proposition~\ref{weak precompact convex is strongly bounded}. Since $X$ is infrabarreled, $B$ is equicontinuous by Theorem~\ref{Banach-Steinhaus for infrabarreled space}. Since $\mathcal{T}$ is the polar topology generated by the equicontinuous subsets of $X^*$, we get $B^\circ\in\mathcal{T}$.
\end{proof}
\begin{example}\label{NVS infinite dim weak topo neq Mackey topo}
Let $(X,\mathcal{T})$ be an infinite-dimensional normed space. Then $\sigma(X,X^*)$ is not metrizable by Example~\ref{weak topo metrizable iff finite dim}. Hence $\sigma(X,X^*)$ is strictly weaker than the norm topology. Since $\sigma(X,X^*)\sub\mathcal{T}\sub\tau(X,X^*)$, it follows that $\sigma(X,X^*)$ is strictly weaker than $\tau(X,X^*)$ as well. Therefore, $(X,\sigma(X,X^*))$ is not infrabarreled by Proposition~\ref{LCS infrabarreled carry tau(X,X^*)}, hence not barreled.
\end{example}
\section{Permenance properties}
\begin{proposition}
If the topological vector space $X$ has a dense barreled subspace $M$ then $X$ is barreled.
\end{proposition}
\begin{proof}
Let $M$ be a dense barreled subspace of the TVS $X$. Since $M$ is barreled it is locally convex; since closures of convex sets are convex, it follows that $X$ is locally convex. If $B$ is a barrel in $X$, then $B\cap M$ is a barrel in $M$, therefore a neighborhood of $0$ in $M$. Then $\widebar{B\cap M}$ is a neighborhood of $0$ in $X$. Since $\widebar{(B\cap M)}\sub B$, $B$ is a neighborhood of $0$ in $X$.
\end{proof}
\begin{proposition}\label{LCS barreled space permanence}
\mbox{}
\begin{itemize}
\item[(a)] A inductive limit of (infra)barreled spaces is (infra)barreled.
\item[(b)] Quotients and locally convex direct sums of (infra)barreled spaces are (infra)barreled.
\item[(c)] Any product of (infra)barreled spaces is (infra)barreled.
\end{itemize}
\end{proposition}
\begin{proof}
Let $(X,\mathcal{T})$ be the inductive limit of linear maps $\{A_s:X_s\to X\}$. A base at $0$ for the final topology $\mathcal{T}$ is given by the filterbase of all absorbent disks $B\sub X$ such that each $A_s^{-1}(B)$ is a neighborhood of $0$ in $X_s$. If $B$ is a barrel in $(X,\mathcal{T})$ then each $A_s^{-1}(B)$ is absorbent balanced, and convex; $A_s^{-1}(B)$ is closed because $A_s$ is continuous. Thus each $A_s^{-1}(B)$ is a barrel in $X_s$ and therefore a neighborhood of $0$ in $X_s$. It follows that $B$ is a $\mathcal{T}$-neighborhood of $0$ and proves part (a).\par
If $B$ is a bornivorous barrel in $X$, then so is each $A_s^{-1}(B)$. Since each $X_s$ is infrabarreled, each $A_s^{-1}(B)$ is a neighborhood of $0$ in $X_s$.\par
Since $(X,\mathcal{T})$ is barreled iff $\mathcal{T}=\beta(X,X^*)$ and infrabarreled iff $\mathcal{T}=\tau(X,X^*)$, part (c) is a consequence of Proposition~\ref{weak Macky strong topo for product and direct sum}.
\end{proof}
\begin{example}[\textbf{Nonbarreled Dense Subspace}]\label{barreled space contain nonbarreled dense subspace}
Endow $\R^\N$ with the product topology and let $M$ be the subspace of all bounded sequences. As a power of complete spaces, $\R^\N$ is complete. Let $\pi_n$ denote the projection of $\R^\N$ onto $\R$. Since the product topology is determined by the countable family of seminorms $|\pi_n|$, it is metrizable by Proposition~\ref{LCS metrizable iff}. Therefore $\R^\N$ is nonmeager by the Baire category theorem. As a product of locally convex spaces, it is locally convex, hence barreled (Proposition~\ref{LCS barreled space permanence}). Moreover, since any basic open set in $\R^\N$ meets $M$, $M$ is dense in $\R^\N$; $M$ is not barreled because $B=\{x\in\R^\N:\sup_n|x_n|\leq 1\}$ is a barrel in $M$ but not a neighborhood of $0$.
\end{example}
Example~\ref{barreled space contain nonbarreled dense subspace} shows that barreled spaces may have subspaces that are not barreled. The fact that Fr\'echet spaces are generally barreled allows us to generate a class of such spaces-namely, a metric completion of a nonbarreled metrizable locally convex space. Arbitrary subspaces of barreled spaces need not be barreled, but what about closed subspaces? Here, too, the answer is generally negative. Each locally convex Hausdorff space $X$ is linearly homeomorphic to a subspace of a product $Y$ of Banach spaces (Proposition~\ref{LCS subspace of product}). Since Banach spaces are barreled, $Y$ is barreled. Thus if there is a complete locally convex Hausdorff space which is not barreled, we have an example of a closed, nonbarreled subspace of a barreled space. Such an example follows.
\begin{example}[\textbf{Complete Nonbarreled Space}]
Let $T$ be a nondenumerable set and let $X=\{x\in\R^T:\text{$x(t)=0$ for almost all $t$}\}$ endowed with the relative topology induced by the box topology on $\R^T$. This topology makes $X$ into a complete locally convex space (Example~\ref{TVS box product}). It is straightforward to verify that $B=\{x\in X:\sup_{t\in T}|x(t)|\leq 1\}$ is a barrel. But $B$ is not a neighborhood of $0$ by the following argument. Given the basic box neighborhood of $0$, $U=\prod_{t\in T}[-r_t,r_t]$ where each $r_t$ is a positive number, there is a finite subset $H\sub T$ such that $\sum_{t\in H}r_t>1$---if not, then for all $n\geq 2$, the set $A_n=\{t\in T:r_t>n^{-1}\}$ must have fewer than $n$ elements which implies that $\{r_t:t\in T\}=\bigcup_nA_n$ is denumerable. Hence, let $H$ be a finite subset such that $\sum_tr_t>1$. The function $x$ such that $x(t)=r_t$ for $t\in H$ and $x(t)=0$ otherwise is an element of $U\cap X$ but not of $B$. Since $B$ cannot contain any $U\cap X$, $B$ is not a neighborhood of $0$.
\end{example}
Having mentioned some negative results about subspaces of barreled spaces, we now turn toward the positive side. The results below show that sufficiently large subspaces of barreled spaces-large in the sense that they are of small codimension-are barreled. It is easy to verify that the projection $B\cap M$ of a barrel $B$ in a TVS $X$ onto a subspace $M$ is a barrel. The key to showing that a finite-codimensional subspace of a barreled or infrabarreled space is barreled or infrabarreled, respectively, is the following result that asserts that all barrels in subspaces are of this type.
\begin{proposition}\label{LCS barrel in subspace finite-codim}
Let $M$ be a finite-codimensional subspace of the locally convex space $X$. If $B$ is a
\begin{itemize}
\item[(a)] barrel in $M$, there is a barrel $B'$ in $X$ such that $B=B'\cap M$;
\item[(b)] bornivorous disk in $M$, there is a bornivorous disk $B'$ in $X$ such that $B=B'\cap M$;
\item[(c)] bornivorous barrel in $M$, there is a bornivorous barrel $B'$ in $X$ such that $B=B'\cap M$. 
\end{itemize}
\end{proposition}
\begin{proof}
If we can prove the assertion for subspaces of codimension $1$, it will suffice. By an induction argument, it will then hold for subspaces of finite codimension. So assume that $M$ is of codimension $1$ and let $B$ be a barrel in $M$. Let $W$ be the linear span $\langle\mathrm{cl}_XB\rangle$ of the closure of $B$ in $X$. Since $B$ is absorbent in $M$, $M\sub\langle\mathrm{cl}_XB\rangle=W$. Consider two possibilities for $W$.\par
If $W=X$, any $y\in X$ can be written as $y=\sum_ia_ix_i$ with . Since $B$ is a disk, so is $\mathrm{cl}_XB$ and therefore, for $|c|\geq\sum_i|a_i$, we have $y\in c(\mathrm{cl}_XB)$, i.e., $\mathrm{cl}_XB$ is absorbent, and $\mathrm{cl}_XB$ is a barrel in $X$. Since $B$ is closed in $M$, $(clx B)\cap M=B$ and we may take $B'$ to be $\mathrm{cl}_XB$.\par
If $W\neq X$, then $B$ must be closed in $X$ (not just $M$). Choose $x\in X$ such that $X=W\oplus\K x=M\oplus\K x$. Let $\D=\{a\in\K:|a|\leq 1\}$ and note that $\D x$ is a closed disk in $X$. We now show that $B'=B+\D x$ is the desired barrel. Since $B'\cap M=(B+\D x)\cap M=B$, all that remains to be verified is that $B'$ is a barrel in $X$.\par
First note that $B'$ is closed because it is the sum of a closed and a compact set (Proposition~\ref{topological group compact product closed is closed}). Generally, finite sums of disks are disks, so $B'$ is a disk. To see that $B'$ is absorbent, suppose that $w=u+tx\in X$. Since $B$ is absorbent in $M$, $au\in B$ for scalars $a$ of sufficiently small magnitude; hence $aw=au+atx\in B'$ and the proof of (a) is complete.\par
As in the proof of (a), it suffices to consider subspaces of codimension $1$. Choose $x\in X$ such that $X=M\oplus\K x$ and let $B$ be a bornivorous disk in $M$. Let $P$ be the projection $m+tx\mapsto m$ of $X=M\oplus\K x$ onto $M$. There are two possibilities for $P$: It is locally bounded (maps bounded sets into bounded sets) or it is not. If it is locally bounded, then $P^{-1}(B)$ is a bornivorous disk in $X$, as is straightforward to verify. Moreover, $P^{-1}(B)\cap M=B$, so $B'=P^{-1}(B)=B+\K x$ is the desired bornivorous disk in $X$.\par
Now suppose that $P$ is not locally bounded. By Proposition~\ref{TVS bounded set iff} then there is a bounded sequence $(x_n)$ in $X$ such that $(m_n)=(Px_n)$ is not bounded. If $x_n=m_n+t_nx$, then $(t_n)$ must be unbounded, too---if not, then $(m_n)=(Px_n-t_nx)$ would be bounded. By extracting a subsequence, if necessary, we may suppose that $|t_n|\geq 1$ for every $n$ and that $|t_n|\to+\infty$. Let $H$ be the balanced convex hull of $\{x_n\}\cup\{x\}$ and let $y_n=-Px_n/t_n=-m_n/t_n$. For every $n\in\N$, $y_n\in M$ and
\[y_n-x=-m_n/t_n-x=-x_n/t_n\in(1/t_n)H.\]
Since $\{x_n\}\cup\{x\}$ is bounded, $H$ is bounded by Proposition~\ref{TVS bounded set closure bal} and \ref{LCS convex hull of bounded}.\par
Let $\mathcal{B}$ denote the class of all bounded disks in $X$ containing the bounded set $H$. For a set to be bornivorous in $X$, it suffices for it to absorb the sets in $\mathcal{B}$, for if $J$ absorbs each set in $\mathcal{B}$ and $L$ is any bounded set in $X$, $\convbal(L\cup H)$ is bounded and is in $B$. As such, $J$ must absorb it and therefore $J$ absorbs $L$.\par
Let $D\in\mathcal{B}$. Since $D\cap M$ must be bounded in $M$ and $B$ is bornivorous in $M$, there must be $r_D>0$ such that $2r_D(D\cap M)\sub E$. Let $G=\convbal(\bigcup_{D\in\mathcal{B}}r_DD)$. Since $G$ absorbs each set in $\mathcal{B}$, $G$ is bornivorous. We now show that $G\cap M\sub B$.\par
Each $w\in G\cap M$ is a finite sum of the form $\sum_ia_i(m_i+b_ix)$ where $a_i$ and $b_i$ are scalars such that $\sum_i|a_i|\leq 1$, $\sum_ia_ib_i=0$ (since $x\notin M$), $m_i\in M$ and $m_i+b_ix\in r_{D_i}D_i$. For each $i$, $H\sub D_i$; since $|b_i/t_n|<r_{D_i}$ for sufficiently large $n$,
\[(b_i/t_n)H\sub (b_i/t_n)D_i\sub r_{D_i}D_i.\]
By this inclusion, and since $m_i+b_ix\in r_{D_i}D_i$ for each $i$, for sufficiently large $n$ we have
\[m_i+b_iy_n=m_i+b_ix+b_i(y_n-x)\in r_{D_i}D_i+(b_i/t_n)H\sub (2r_{D_i}D_i)\cap M\sub B.\]
Since $\sum_ia_ib_i=0$, $\sum_ia_i(m_i+b_ix)=\sum_ia_i(m_i+b_iy_n)\in B$, which proves the claim.\par
Let $B'=\convbal(G\cup B)$. Then $B'$ is a disk and bornivorous since $G$ is. Since $G\cap M\sub B$, we have $(G\cup B)\cap M\sub B$ and so $B'\cap M\sub B$. Since obviously $B\sub B'\cap M$, the proof of (b) is complete.\par
Let $B$ be a bornivorous barrel in $M$. As such, $B$ is a closed bornivorous disk in $M$. By (b), there is a bornivorous disk $D$ in $X$ such that $B=D\cap M$. Therefore $B'=\widebar{D}$ is the desired bornivorous barrel in $X$.
\end{proof}
\begin{proposition}
A finite-codimensional subspace of a barreled or infrabarreled space is barreled or infrabarreled, respectively.
\end{proposition}
\begin{proposition}\label{LCS barreled denumerable-codim is barreled}
A subspace $M$ of denumerable codimension of a barreled space $X$ is barreled.
\end{proposition}
\begin{proof}
Since $M$ is of denumerable codimension, there is an increasing sequence $(M_n)$ of subspaces of $X$ such that $M_1=M$, the codimension of each $M_n$ in $M_{n+1}$ is $1$ and $X=\bigcup_nM_n$. Given a barrel $B$ in $M$, by Proposition~\ref{LCS barrel in subspace finite-codim} there is a sequence $(B_n)$ of barrels, starting with $B_1=B$ in $M$, and $B_n$ in $M_n$ such that $B_{n+1}\cap M_n=B_n$ for each $n$. Clearly $V=\bigcup_nB_n$ is an absorbent disk in $X$ so $\widebar{V}$ is a barrel in $X$. Since $X$ is barreled, $\widebar{V}$ a neighborhood of $0$. In the remainder of the argument we show that $\widebar{V}\sub 2V$, from which it follows that $V$ is a neighborhood of $0$. Since $B=V\cap M$, as a routine verification will show, it follows that $B$ is a neighborhood of $0$ in $M$.\par
For $z\notin 2V=\bigcup_n2B_n$, clearly $z\notin 2B_n$ for any $n$. If we view each $M_n$ as a real vector space, there exists a real continuous linear functional $f_n$ on $M_n$ such that $\sup_{x\in 2B_n}f_n(x)\leq f_n(z)$ by Proposition~\ref{LCS not in disk functional}. Clearly, we may suppose that $f_n(z)=2$, so that $\sup_{x\in B_n}f_n(x)\leq 1$ for every $n$; since $B_n$ is balanced, $|\sup_{x\in B_n}f(x)|\leq 1$. By Corollary~\ref{LCS continuous functional extension} we may assume that each $f_n$ has been extended to a member of $X^*$. Thus $f_n\in B_N^\circ$ for each $n\in\N$.\par
We now contend that $\{f_n\}$ is $\sigma(X^*,X)$-bounded on $X=\bigcup_nM_n$. To see this, choose $x\in\bigcup_nM_n$ and $n\in\N$ such that $x\in M_n$. Since $B_n$ is absorbent in $M_n$, it follows that $tx\in B_n$ for some $t>0$. Since the barrels $B_n$ are increasing, it follows that $|f_m(x)|\leq 1/t$ for $m\geq n$. Thus $\{f_n\}$ is pointwise bounded, that is, $\sigma(X^*,X)$-bounded. Having shown that $\{f_n\}$ is $\sigma(X^*,X)$-bounded, there exists a barrel $U$ in $X$ such that $\{f_n\}^\circ\sub U^\circ$. Since $X$ is barreled, $U$ must be a neighborhood of $0$ in $X$; hence $U^\circ$ is $\sigma(X^*,X)$-compact by the Alaoglu theorem. Consequently, $\{f_n\}$ has a $\sigma(X^*,X)$-cluster point $f\in X^*$. Since there exists a subsequence $\{f_{n_k}\}$ converging to $f$ weakly, it follows that $f(z)=\lim_kf_{n_k}(z)=2$. Given $r>0$, suppose that $x\in V=\bigcup_nB_n$. Then there exists $N>0$ such that $x\in B_N$, so $|f_n(x)|\leq 1$ for $n\geq N$. Since $f_n\to f$, it follows that $|f(x)|\leq 1$, which implies $f\in V^\circ$. By the continuity of $f$, it is now clear that $z\notin\widebar{V}$. Therefore, $\widebar{V}\sub 2V$ and the theorem is proved.
\end{proof}
The algebraic closure $\mathrm{acl}A$ of a subset $A$ of a vector space $X$ is the union of $A$ and those points which are linearly accessible from $A$---those $x\in X$ for which there exists $y\in A$ such that the line segment $[y,x)\sub A$. Proposition~\ref{LCS closure of increasing disk} shows that if an increasing sequence $\{D_n\}$ of disks satisfies a certain condition, the topological closure of $\bigcup_nD_n$ coincides with its algebraic closure. We place it here because of the features the proof shares with Proposition~\ref{LCS barreled denumerable-codim is barreled}.
\begin{proposition}\label{TVS algebraic closure char}
If $D$ is a disk in a vector space $X$, then $\mathrm{acl}D=\bigcap_{r>0}(1+r)D$.
\end{proposition}
\begin{proof}
Let $D$ be a disk. Suppose that $x\in\bigcap_{r>0}(1+r)D$. For any $r>0$ and $t=1/(1+r)$, we have $tx\in D$, so $[0,x)\sub D$ and $x\in\mathrm{acl}D$. Conversely, suppose that $r>0$ and that $x\in\mathrm{acl}D$. Let $y\in D$ be such that $[y,x)\sub D$ and choose $t\in[0,r/(1+r)]$. Since $ty+(1-t)x\in D$ and $-y\in D$, it follows that, for any $c\in[0,1]$,
\[c(-y)+(1-c)[ty+(1-t)x]=[-c+(1-c)t]y+(1-c)(1-t)x\in D.\]
Choose $c=t/(1+t)$ then yeilds $(1-t)/(1+t)x\in D$. Since $t\leq r/(1+r)$, we have $1/(1+r)\leq(1-t)/(1+t)$. Since $D$ is a disk, it follows that $[1/(1+r)] x\in D$ and therefore that $x\in (1+r)D$.
\end{proof}
\begin{proposition}\label{LCS closure of increasing disk}
Let $X$ be a locally convex space with neighborhood filter $\mathfrak{U}(0)$ at $0$, and let $\{D_n\}$ be an increasing sequence of disks of some subspace $M$ of $X$ such that for any $f_n\in D_n^\circ$, $\{f_n:n\in\N\}$ is equicontinuous on $M$. Then
\begin{itemize}
\item[(a)] $\widebar{\bigcup_nD_n}=\mathrm{acl}(\bigcup_n\widebar{D}_n)=\bigcap_{r>0}(1+r)(\bigcup_n\widebar{D}_n)$, or equivalently,
\item[(b)] if $\mathcal{B}$ is a Cauchy filterbase in $\bigcup_nD_n$, then, given any $r>0$, $\mathcal{B}+\mathfrak{U}(0)$ induces a Cauchy filterbase on $(1+r)D_n$ for some positive integer $n$.
\end{itemize}
\end{proposition}
\begin{proof}
View the locally convex space $X$ as a real vector space. To say that $y\in\mathrm{acl}C$, the algebraic closure of the set $C$, means that there exists $x\in C$ such that $[x,y)\sub C$. Hence $y\in\widebar{C}$ and $\mathrm{acl}C\sub\widebar{C}$. In particular, $\mathrm{acl}(\bigcup_n\widebar{D}_n)\sub\widebar{\bigcup_n\widebar{D}_n}$. Since $\bigcup_n\widebar{D}_n\sub\widebar{\bigcup_nD_n}$, it follows that $\mathrm{acl}(\bigcup_n\widebar{D}_n)\sub\widebar{\bigcup_nD_n}$.\par
To show the converse, we show that $\widebar{\bigcup_nD_n}\sub(1+r)(\bigcup_nD_n)$ for each $r>0$. We argue by contrapositive. Suppose that for some $r>0$, $x\notin(1+r)(\bigcup_nD_n)$. By Proposition~\ref{LCS separation of closed and compact convex sets}, for each $n$ there exists a real linear functional in $D_n^\circ$ such that $f_n(x)=1+r$. By hypothesis, $\{f_n\}$ is therefore equicontinuous on $M$. If $\widebar{M}=X$, then, as we show next, $\{f_n\}$ is equicontinuous on $X$.\par
Since $\{f_n\}\sub X^*$ is equicontinuous on $M$, given $\eps>0$, there is a neighborhood $V$ of $0$ in $X$ such that $|f_n(v)|<\eps$ for each $n\in\N$ and $v\in V$. By the continuity of $f_n$, $|f_n(v)|\leq\eps$ for each $n\in\N$ and $v\in\widebar{V}$. By the density of $M$, $\widebar{V}$ is a neighborhood of $0$ in $X$ and therefore $\{f_n\}$ is equicontinuous on $X$.\par
Now returning to the original argument. Since $\{f_n\}$ is equicontinuous on $X$, it is weak$^*$ compact by Theorem~\ref{weak topo equicontinuous is precompact} and therefore has a $\sigma(X^*,X)$-cluster point $f$. As in the proof of Proposition~\ref{LCS barreled denumerable-codim is barreled}, we have $f(x)=1+r$ and $f\in(\bigcup_nD_n)^\circ$, which imply $x\notin\widebar{\bigcup_nD_n}$. This proves (a) when $M$ is dense in $X$.\par
If $\widebar{M}\neq X$, the previous argument shows that $\{f_n\}$ is equicontinuous on $\widebar{M}$. By Proposition~\ref{LCS equicontinuous iff} and the fact that the continuous seminorms on $M$ are restrictions to $M$ of continuous seminorms on $X$ (Proposition~\ref{LCS initial topo seminorm}), there is a continuous seminorm $p$ on $X$ such that $|f_n|\leq p$ for each $n\in\N$ on $M$. By Hahn-Banach theorem, we may assume that $|f_n|\leq p$ on all of $X$ for each $n\in\N$. In other words, using Proposition~\ref{LCS equicontinuous iff} again, we may assume that $\{f_n\}$ is equicontinuous on $X$. The same argument shows that $x\notin\widebar{\bigcup_nD_n}$ and completes the proof of (a).\par
We show that $(a)\Rightarrow(b)$. Let $r>0$ be given and let $Y$ be a completion of $M$ so that a Cauchy filter base $\mathcal{B}$ in $\bigcup_nD_n$ converges to some $x\in Y$. It follows that $x\in\mathrm{cl}_Y(\bigcup_nD_n)=\bigcap_{r>0}(1+r)(\bigcup_n\mathrm{cl}_YD_n)$. Hence, for some positive integer $n$, $x\in(1+r)\mathrm{cl}_YD_n$.\par
Let $U\in\mathfrak{U}(0)$ and choose a balanced neighborhood $V$ of $0$ such that $V+V\sub U$. Since $\mathcal{B}\to x$, there exists $B^*\in\mathcal{B}$ such that $B^*\sub x+V$. For any $B\in\mathcal{B}$, there exists $B'\in\mathcal{B}$ such that $B'\sub B^*\cap B$. Hence $B'\sub B^*\sub x+V$. Since $V$ is balanced, this implies that $x\in B'+V$, so $x+V\sub B'+U\sub B+U$. Since $x\in(1+r)\mathrm{cl}_YD_n$, $x+V$ must meet $(1+r)D_n$; hence, so must $B+U$; in other words, each set $B+U$ meets $(1+r)D_n$, so we can consider the trace of $\mathcal{B}+\mathfrak{U}(0)$ on $(1+r)D_n$. Since $\mathcal{B}+\mathfrak{U}(0)$ is a Cauchy filterbase, so is its trace on $(1+r)D_n$.\par
Finally, we prove that $(b)\Rightarrow(a)$. Let $r>0$ be given. Let $x\in\widebar{\bigcup_nD_n}$ and consider the filterbase $\mathcal{B}=\{(x+V)\cap\bigcup_nD_n:V\in\mathfrak{U}(0)\}$ of neighborhoods of $x$ in $\bigcup_nD_n$. Since $\mathcal{B}\to x$, $\mathcal{B}$ is Cauchy. Moreover, the filterbase $\mathcal{B}+\mathfrak{U}(0)$ induces $\mathcal{B}$ again on $\bigcup_nD_n$. So, by hypothesis, $\mathcal{B}$ is a Cauchy filterbase in $(1+r)D_n$ for some positive integer $n$. Hence $x\in(1+r)\widebar{D}_n$ and, as argued in the proof above, it follows that $\widebar{\bigcup_nD_n}$ is the algebraic closure of $\bigcup_nD_n$.
\end{proof}
\chapter{Bornological spaces}
\section{Banach disks}
Let $X$ be a vector space and $D$ be a disk in $X$. Then the linear span $\langle D\rangle$ is given by $\bigcup_nnD$. It is clear that $D$ is absorbent in $\langle D\rangle$, so we may consider the gauge $p_D$ of $D$ in $\langle D\rangle$. The seminormed space $(\langle D\rangle,p_D)$ is denoted $X_D$. If $B$ and $D$ are disks and $B\sub D$, then $p_D\leq p_B$ on $\langle B\rangle$ so the topology $X_B$ gets as a subspace of $X_D$ is weaker than the topology on $X_B$.\par
Having discussed $X_D$ when $D$ is any disk, we consider next the case of most interest, the case when $D$ is a bounded disk. If $D$ is a bounded disk in a topological vector space $X$, then each neighborhood $U$ of $0$ contains some positive multiple $r$ of $D$. Hence $U\cap X_D\sups rD$ and $X_D$'s topology is seen to be finer than the topology $X_D$ inherits as a subspace of $X$. That is, the map $I_D:X_D\to X$ is continuous.
\begin{definition}
Let $X$ and $Y$ be locally convex spaces.
\begin{itemize}
\item If $D$ is a bounded disk in the topological vector space $X$ and $X_D$ is a Banach space then $D$ is called a \textbf{Banach disk}.\par
\item A linear map $A:X\to Y$ is \textbf{infrabounded} if it maps Banach disks into bounded disks.
\end{itemize}
\end{definition}
\begin{proposition}\label{TVS Banach disk if}
Let $D$ be a bounded disk in a Hausdorff topological vector space $X$. Then:
\begin{itemize}
\item[(a)] the gauge $p_D$ is a norm;
\item[(b)] if $D$ is sequentially complete, then $X_n$ is a Banach space, i.e., $D$ is a Banach disk.
\end{itemize}
\end{proposition}
\begin{proof}
Let $D$ and $X$ be as above. Let $x$ be a nonzero element of $X_D$ and choose a neighborhood $U$ of $0$ in $X$ such that $x\notin U$. Since $D$ is bounded, $U\sups rD$ for some $r>0$ so $x\notin rD$. Hence $p_D(x)\geq r>0$.\par
Suppose that $D$ is sequentially complete and let $\{x_n\}$ be a Cauchy sequence in $X_D$. By taking a subsequence, we may assume that $x_{n+1}-x_n\in 2^{-n}D$ for each $n\in\N$. Since $X_D$'s topology is finer than the subspace topology on $X_D$, $\{x_n\}$ is a Cauchy sequence in $X$ as well. Since, for any $n$,
\[x_n=x_1+(x_2-x_1)+\cdots+(x_n-x_{n-1})\in x_1+2^{-1}D+\cdots+2^{-(n-1)}D\sub x_1+D\]
and $D$ (hence $x_1+D$) is sequentially complete, there exists $x\in x_1+D\sub X_D$ such that $x_n\to x$ in the subspace topology. To see that $x_n\to x$ in $X_D$ as well, fix $m>0$. Then for each $n\geq m$,
\[x_n-x_m=(x_n-x_{n-1})+\cdots+(x_{m+1}-x_m)\in(2^{-(n-1)}+\cdots+2^{-m})D\sub 2^{-(m-1)}D\sub D.\]
As $D$ is sequentially closed, it follows that $x-x_m\in 2^{-(m-1)}D$ and therefore $p_D(x-x_n)\to 0$.
\end{proof}
\begin{theorem}[\textbf{Banach-Mackey Theorem}]\label{TVS barrel absorb Banach disk}
Every barrel in a topological vector space absorbs the Banach disks.
\end{theorem}
\begin{proof}
Let $D$ be a Banach disk and $B$ be a barrel. Then $B\cap X_D$ is a barrel in $X_D$. Since $X_D$ is Banach, it is barreled, so $B\cap X_D$ is a neighborhood of $0$ in $X_D$, thus absorb $D$. 
\end{proof}
\section{Bornological spaces}
Let $X$ be a locally convex space. A family $\mathcal{B}$ of bounded disks in $X$ is said to be \textbf{admissible} if it covers $X$ and for any scalars $a$ and $B\in\mathcal{B}$, $aB\in\mathcal{B}$. A linear map $A:X\to Y$ is said to be \textbf{$\mathcal{B}$-bounded} if $A(B)$ is bounded for any $B\in\mathcal{B}$. A subset in $X$ is said to be \textbf{$\mathcal{B}$-bornivorous} if it absorbs all element of $\mathcal{B}$.
\begin{theorem}\label{LCS B-bornilogical iff}
Let $(X,\mathcal{T})$ be a locally convex space and $\mathcal{B}$ be a admissible family of bounded disks in $X$. Then the following are equivalent:
\begin{itemize}
\item[(\rmnum{1})] every $\mathcal{B}$-bornivorous disk is a neighborhood in $X$.
\item[(\rmnum{2})] $X=\rlim\nolimits_{B\in\mathcal{B}}X_B$.
\item[(\rmnum{3})] every $\mathcal{B}$-bounded linear map $A:X\to Y$ is continuous.
\item[(\rmnum{4})] every $\mathcal{B}$-bounded seminorm is continuous.  
\item[(\rmnum{5})] $X$ carries $\tau(X,X^*)$ and every $\mathcal{B}$-bounded linear functional is continuous.
\end{itemize}
\end{theorem}
\begin{proof}
Let $p$ be a seminorm on $X$. Then $p$ is $\mathcal{B}$-bounded iff for any $B\in\mathcal{B}$ there exists $r>0$ such that $|p(B)|\leq 1$, or equivalently, $B_p$ absorbs $B$. Thus (\rmnum{1}) is easily seen to be equivalent to (\rmnum{4}).\par
Suppose (\rmnum{1}) holds and let $\mathcal{T}_i$ be the inductive limit topology. Since the seminorm topology on $X_B$ is finer than the subspace topology, it follows that $\mathcal{T}\sub\mathcal{T}_i$. On the other hand, every $\mathcal{B}$-bornivorous disk is a neighborhood in $X$. If $U$ is a basic $\mathcal{T}_i$-neighborhood of $0$ in $X$, i.e., if $U$ is a disk in $X$ whose intersection with each $X_B$ is a neighborhood of $0$ in $X_B$, then for any $B\in\mathcal{B}$, there exists $r_B>0$ such that $r_BB\sub U\cap X_B\sub U$. Thus $U$ is a bornivorous disk in $X$. Since $X$ is bornological, $U$ is a $\mathcal{T}$-neighborhood of $0$. Hence $\mathcal{T}_i\sub\mathcal{T}$.\par
Since a $\mathcal{B}$-bounded map is continous restricted to $X_B$ for each $B\in\mathcal{B}$ (Proposition~\ref{TVS pseudometrizable linear map continuous iff bounded}), (\rmnum{2}) implies (\rmnum{3}). Conversely, if (\rmnum{3}) holds, let $D$ be a $\mathcal{B}$-bornivorous disk in $X$. Since $\mathcal{B}$ covers $X$, $D$ is seen to be absorbent, so we can consider the gauge $p_D$ and the identity map $I:X\to(X,p_D)$. To see that $I$ is $\mathcal{B}$-bounded, let $B\in\mathcal{B}$. Then since $D$ is $\mathcal{B}$-bornivorous, there exists $r>0$ such that $B\sub rD\sub r\widebar{B}_{p_D}$, and $B$ is seen to be bounded in $(X,p_D)$. By the hypothesis, $I$ is continuous. Since $D\sups B_{p_D}$, $D$ is a neighborhood in $(X,p_D)$, and thus is a neighborhood in $X$.\par
Now assume the condition (\rmnum{3}) and consider the identity map $I:(X,\mathcal{T})\to(X,\tau(X,X^*))$. Clearly we have $\mathcal{T}\sub\tau(X,X^*)$. By Theorem~\ref{LCS bounded set permanence} any $B\in\mathcal{B}$ is $\tau(X,X^*)$-bounded, so $I$ is $\mathcal{B}$-bounded, and is continuous by hypothesis. It then follows that $\mathcal{T}=\tau(X,X^*)$, thus (\rmnum{5}) holds.\par
Conversely, assume (\rmnum{5}). Let $\mathcal{T}_{\mathcal{B}}$ be the topology generated by all $\mathcal{B}$-bornivorous disk in $X$. Then since each neighborhood in $X$ is $\mathcal{B}$-bornivorous, $\mathcal{T}=\tau(X,X^*)\sub\mathcal{T}_{\mathcal{B}}$. To show that $\mathcal{T}_{\mathcal{B}}\sub\tau(X,X^*)$, we show that $(X,\mathcal{T}_{\mathcal{B}})^*=X^*$. It will then follow from the Mackey-Arens theorem that $\mathcal{T}_{\mathcal{B}}\sub\tau(X,X^*)$.\par
Let $p_B$ denote the gauge of the bornivorous disk $B\sub X$. Since $\mathcal{T}_{\mathcal{B}}$ is generated by the saturated family of seminorms $p_B$ where $B$ is a bornivorous disk in $X$, the $\mathcal{T}_{\mathcal{B}}$-continuity of a linear functional $f$ on $X$ means that $|f|\leq p_B$ for some bornivorous disk $B$ (Proposition~\ref{LCS by seminorm continuous iff}). As already noted, $p_B$ is $\mathcal{B}$-bounded which implies that $f$ is a $\mathcal{B}$-bounded linear functional and therefore $\mathcal{T}$-continuous by hypothesis.
\end{proof}
A set $D$ in a topological vector space is \textbf{bornivorous} if $D$ absorbs all bounded sets $B$. A locally convex space is \textbf{bornological} if each bornivorous disk is a neighborhood of $0$. Now apply Theorem~\ref{LCS B-bornilogical iff}, we then get
\begin{proposition}\label{LCS bornological iff}
Let $(X,\mathcal{T})$ be a locally convex space and $\mathcal{B}$ be the family of all bounded disks in $X$. Then the following are equivalent:
\begin{itemize}
\item[(\rmnum{1})] $X$ is bornological.
\item[(\rmnum{2})] $X=\rlim\nolimits_{B\in\mathcal{B}}X_B$.
\item[(\rmnum{3})] every bounded linear map $A:X\to Y$ is continuous.
\item[(\rmnum{4})] every bounded seminorm is continuous.  
\item[(\rmnum{5})] $X$ carries $\tau(X,X^*)$ and every bounded linear functional is continuous.
\end{itemize}
\end{proposition}
Similarly, a subset $D$ of a topological vector space is \textbf{infrabornivorous} if it absorbs all Banach disks. A locally convex space is \textbf{ultrabornological} if each infrabornivorous disk is a neighborhood of $0$, and a linera map $A:X\to Y$ between locally convex spaces is infrabounded if it takes Banach disks to bounded disks.
\begin{proposition}\label{LCS ultrabornological iff}
Let $(X,\mathcal{T})$ be a locally convex space and $\mathcal{B}^*$ be the family of all Banach disks in $X$. Then the following are equivalent:
\begin{itemize}
\item[(\rmnum{1})] $X$ is ultrabornological.
\item[(\rmnum{2})] $X=\rlim\nolimits_{B\in\mathcal{B}^*}X_B$.
\item[(\rmnum{3})] every infrabounded linear map $A:X\to Y$ is continuous.
\item[(\rmnum{4})] every infrabounded seminorm is continuous.  
\item[(\rmnum{5})] $X$ carries $\tau(X,X^*)$ and every infrabounded linear functional is continuous.
\end{itemize}
\end{proposition}
\begin{example}[\textbf{Examples of Bornological Spaces}]\label{bornological space eg}
\mbox{}
\begin{itemize}
\item[(a)] The closed balls $\widebar{B}_r(0)$, are a base of bounded disked neighborhoods of $0$ in any normed space $X$. Consequently, if $D$ is a bornivorous disk, $D$ contains a neighborhood of $0$, $\widebar{B}_r(0)$, for some $r>0$. Hence $D$ is a neighborhood of $0$ and therefore $X$ is bornological. Since we have already encountered non-barreled normed spaces, it follows that bornological does not imply barreled. Clearly, bornological implies infrabarreled, however.
\item[(b)] Boundedness implies continuity on pseudometrizable spaces, as shown in Proposition~\ref{TVS pseudometrizable linear map continuous iff bounded}. Hence a pseudometrizablc locally convex space is bornological.
\item[(c)] If $X$ is an infinite-dimensional normed space, then $\sigma(X,X^*)\subsetneq\tau(X,X^*)$ by Example~\ref{NVS infinite dim weak topo neq Mackey topo}. Consequently, there must be a disked $\sigma(X,X^*)$-neighborhoods $U$ of $0$ which is not a $\tau(X,X^*)$-neighborhoods of $0$. Since the family of bounded sets is the same for any topology of the dual pair, $U$ is bornivorous in $(X,\sigma(X,X^*))$ and therefore $(X,\sigma(X,X^*))$ is not bornological.\par
Another way to argue the point is to observe that the identity map from $(X,\sigma(X,X^*))$ to $(X,\tau(X,X^*))$ is bounded but not continuous and invoke Proposition~\ref{LCS bornological iff}.
\item[(d)] The closed unit ball $B_X$ of a Fr\'echet space $X$ is sequentially complete, hence a Banach disk by Proposition~\ref{TVS Banach disk if}. Thus, if $B$ absorbs all Banach disks, $B$ absorbs $U$ and $B$ is therefore a neighhorhood of $0$. Thus Fr\'echet spaces are ultrabornological.
\item[(e)] Let $X$ be any vector space with its finest locally convex topolog, that having the family of all absorbent disks as a base at $0$. As a bornivorous disk is absorbent, $X$ is bornological.
\item[(f)] Let $(X,\mathcal{T})$ be a locally convex space. The collection of all bornivorous disks of $X$ satisfies the conditions of the basis theorem for a locally convex topology $\mathcal{T}_{bo}$ on $X$. Clearly $(X,\mathcal{T}_{bo})$ is bornological. It is called the \textbf{associated bornological space} and $\mathcal{T}_{bo}$ the \textbf{associated bornological topology}. As any disked $\mathcal{T}$-neighborhood $U$ of $0$ is bornivorous, each such $U$ is a $\mathcal{T}_{bo}$-neighborhood of $0$ so $\mathcal{T}\sub\mathcal{T}_{bo}$. Hence $\mathcal{T}_{bo}$-bounded sets are $\mathcal{T}$-bounded. A $\mathcal{T}$-bounded set $B$, however, is absorbed by exactly the same class of disks in $(X,\mathcal{T}_{bo})$ as in $(X,\mathcal{T})$ so the class of bounded sets is the same for each topology. $\mathcal{T}_{bo}$ is the finest locally convex topology for $X$ with the same bounded sets as the original topology.
\item[(g)] If in (f), instead of taking the bornivorous disks as a base at $0$, we take the infrabornivorous disks, we get an ultrabornological topology $\mathcal{T}_{ubo}$ called the \textbf{associated ultrabornological topology}. $(X,\mathcal{T}_{ubo})$ is called the \textbf{associated ultrabornological space}.
\end{itemize}
\end{example}
\section{Permanence properties}
\begin{lemma}\label{TVS balanced sets absorb linear image iff}
Let $A:X\to Y$ be a linear map between vector spaces. If $B$ is balanced in $X$ and $D$ balanced in $Y$, then $D$ absorbs $A(B)$ iff $A^{-1}(D)$ absorbs $B$.
\end{lemma}
\begin{proof}
If $A^{-1}(D)$ absorbs $B$, then $B\sub rA^{-1}(D)$ for sufficiently large $r>0$. Hence $A(B)\sub rA(A^{-1}(D))\sub rD$. Conversely, if $D$ absorbs $A(B)$, then $A(B)\sub rD$ for sufficiently large $r>0$. Thus $B\sub A^{-1}(A(B))\sub rA^{-1}(D)$.
\end{proof}
\begin{proposition}\label{LCS map B-bounded iff A^-1}
Let $X$ and $Y$ be locally convex and $\mathcal{B}$ be a directed family of bounded disks in $X$. Let $A:X\to Y$ be a linear map. Then $A$ is $\mathcal{B}$-bounded if and only if $A^{-1}$ takes bornivorous disks into $\mathcal{B}$-bornivorous disks.
\end{proposition}
\begin{proof}
Suppose that $A$ is $\mathcal{B}$-bounded. Let $D$ be a bornivorous disk in $Y$ and $B\in\mathcal{B}$. Since $A(B)$ is bounded, $D$ absorbs $A(B)$, so $A^{-1}(D)$ absorbs $B$ and $A^{-1}(D)$ is seen to be $\mathcal{B}$-bornivorous.\par
Now assume that $A^{-1}$ takes bornivorous disks into $\mathcal{B}$-bornivorous disks. Let $B\in\mathcal{B}$ and $U$ any disked neighborhood of $0$ in $Y$. Since $U$ is bornivorous, $A^{-1}(U)$ is $\mathcal{B}$-bornivorous, by hypothesis. Hence $A^{-1}(U)$ absorbs $B$ or, equivalently, $U$ absorbs $A(B)$. Thus $A(B)$ is a bounded disk.
\end{proof}
\begin{proposition}\label{LCS bornological final topo}
Let $X$ be a vector space with the final topology determined by $\{A_s:X_s\to X\}$.
\begin{itemize}
\item[(a)] If each $X_s$ is bornological, then $X$ is bornological.
\item[(b)] If each $X_s$ is ultrabornological and $X$ is Hausdorff, then $X$ is ultrabornological.
\end{itemize}
\end{proposition}
\begin{proof}
Recall tat the final locally convex topology $\mathcal{T}$ for $X$ is the finest locally convex topology for $X$ with respect to which each of the maps $A_s$ is continuous; those absorbent disks $D$ in $X$ for which every $A^{-1}(D)$ is a neighborhood of $0$ in $X_s$ form a base at $0$ for $\mathcal{T}$. Since each $A_s$ is continuous, each is bounded by Proposition~\ref{TVS homogeneous image of bounded is bounded}. Thus, if $D$ is bornivorous disk in $X$, then each $A_s^{-1}(D)$ is a bornivorous disk. Since each $X_s$ is bornological, each $A_s^{-1}(D)$ is a neighborhood of $0$ in $X_s$. Therefore, since $D$ is absorbent, it is a $\mathcal{T}$-neighborhood of $0$ and $X$ is seen to be bornological.\par
Now suppose that each $X_s$ is ultrabornological and let $D$ be an infrabornivorous disk in $X$. To show that $D$ is a $\mathcal{T}$-neighborhood of $0$, we show that each $A_s^{-1}(D)$ is an infrabornivorous disk---hence a neighborhood of $0$---in each of the ultrabornological spaces $X_s$. Equivalently, we must show that $A_s^{-1}(D)$ is compactivorous. To this end, let $B$ be a compact disk in $X_s$. As its continuous image $A_s(B)$ is a compact disk in $X_s$, $D$ absorbs $A_s(B)$. Hence $A_s^{-1}(D)$ absorbs $B$.
\end{proof}
Thus, quotients and locally convex direct sums of bornological spaces are bornological, hence inductive limits of normed and pseudometrizable locally convex spaces are bornological. As LF-spaces are never metrizable, they constitute a class of nonmetrizable bornological spaces.
\begin{proposition}\label{LCHS ultrabornological iff}
A locally convex Hausdorff space $X$ is 
\begin{itemize}
\item[(a)] bornological iff it is the inductive limit of normed spaces.
\item[(b)] ultrabornological iff it is the inductive limit of Banach spaces.
\end{itemize}
\end{proposition}
\begin{proposition}\label{LCHS bornological + sequentially complete}
If the locally convex Hausdorff space $X$ is sequentially complete and bornological, then it is ultrabornological.
\end{proposition}
\begin{proof}
Let $\mathcal{B}$ denote the set of closed bounded disks of $X$. Since $X$ is bornological,$X=\rlim X_B$. Since $X$ is sequentially complete, so is each $B$. Thus each $B$ is a Banach disk by Proposition~\ref{TVS Banach disk if}. As Banach spaces are ultrabornological, the desired result follows from Proposition~\ref{LCS bornological final topo}.
\end{proof}
Recall that an infrabarreled space is a locally convex space in which each bornivorous barrel is a neighborhood of $0$; consequently bornological spaces are infrabarreled. It is also clear that ultrabornological spaces are barreled, hence ultrabornological spaces $X$ carry $\beta(X,X^*)$ by Proposition~\ref{LCS barreled iff carry strong topology}.\par
The result below generalizes the fact the dual $X^*$ of a normed space $X$ is complete in its norm topology.
\begin{proposition}\label{LCS bornological L(X,Y) complete if Y complete}
Let $X$ be a bornological space and $Y$ be a locally convex spaces. Let $\mathcal{L}(X,Y)$ carry the topology of uniform convergence on bounded sets. If $Y$ is complete, then $\mathcal{L}(X,Y)$ is complete.
\end{proposition}
\begin{proof}
We show that any Cauchy net $(A_\alpha)$ in $\mathcal{L}(X,Y)$ converges to its "pointwise" limit $A$. To define $A$, choose $x\in X$ and a bounded set $D$ to which $x$ belongs. Since $(A_\alpha)$ is a Cauchy net in $\mathcal{L}(X,Y)$, given a neighborhood $V\in\mathfrak{U}_Y(0)$, there exists $\gamma$ such that $A_\alpha-A_\beta\in N(D,V)$ for $\alpha,\beta\succeq\gamma$. Thus $(A_\alpha x)$ is a Cauchy net in $Y$. Since $Y$ is complete, it has a limit in $Y$. We define $Ax$ to be the limit of $(A_\alpha x)$; $A$ is obviously linear.\par
It remains to show that $A$ is continuous and that $A_\alpha\to A$ in $\mathcal{L}(X,Y)$. To show that $A$ is continuous, we show that $A$ is bounded and use the bornologicity of $X$. To this end, let $D\sub X$ be bounded. For any neighborhood $V$ of $0$ in $Y$, choose a disked neighborhood $U\in\mathfrak{U}_Y(0)$ such that $U+U\sub V$. Then $A_\alpha-A_\beta\in N(D,U)$ for sufficiently large indices $\alpha$ and $\beta$; thus $A_\alpha x-A_\beta x\in U$ for $x\in D$. Since each $A_\alpha$ is continuous, it is bounded. Fixing $\beta$, for sufficiently large $\alpha$, $r>1$ and $x\in D$,
\[A_\alpha x\in A_\beta x+U\sub rU+U\sub r(U+U)\sub rV.\]
Taking the limit, it follows that $A(D)$ is absorbed by $V$ so $A$ is bounded, hence continous by Proposition~\ref{LCS bornological iff}.\par
Finally, to show that $A_\alpha\to A$ in $\mathcal{L}(X,Y)$, let $D$ be a bounded set in $X$. Since $(A_\alpha)$ is a Cauchy net, given $V\in\mathfrak{U}_Y(0)$, choose $U\in\mathfrak{U}_Y(0)$ such that $U+U\sub V$. Then there exists $\beta$ such that $A_\alpha-A_\beta\in N(D,U)$ for $\alpha\succeq\beta$. Given any $x$ in $D$, we may choose an index $\beta'\succeq\beta$ such that $A_{\beta'}x-A_\beta x\in N(D,U)$. Hence, for $\alpha\succeq\beta'$, $A_\alpha x-Ax\in U+U\sub V$ for each $x$ in $D$, i.e., $A-A_\alpha\in N(D,V)$, which yields the desired convergence statement.
\end{proof}
\chapter{Closed Graph Theorems}
\section{Maps with closed graph}
Let $X$ and $Y$ be topological spaces and $f:X\to Y$ be a map. The set $\Gamma(f)=\{(x,f(x)):x\in X\}$ is called the graph of $f$. We say that $f$ has a \textbf{closed graph} if $\Gamma(f)$ is a closed subset of $X\times Y$ in the product topology. Continuity usually ($X$ Hausdorff) implies closedness but even closed linear maps do not have to be continuous. We discuss some basic descriptions of closed maps and connections between continuity and closedness in this part. We begin with a useful description of closedness.
\begin{proposition}\label{closed graph iff}
Let $f:X\to Y$ be a map. Then $f$ has a closed graph iff for any net $(x_\alpha)$ in $X$,
\[x_\alpha\to x,f(x_\alpha)\to y\Longrightarrow y=f(x).\]
If the condition only holds for sequences, we say that $\Gamma(f)$ is sequentially closed.
\end{proposition}
\begin{proof}
We use the notation of the statement. If $\Gamma(f)$ is closed, $x_\alpha\to x$ and $f(x_\alpha)\to y$, then obviously $(x,y)\in\widebar{\Gamma(f)}=\Gamma(f)$, thus $y=f(x)$. Conversely, suppose that the condition holds and $(x,y)\in\widebar{\Gamma(f)}$. If so, there is a net
$(x_\alpha,f(x_\alpha))$ from $\Gamma(f)$ such that $(x_\alpha,f(x_\alpha))\to(x,y)$. By the continuity of projections, $x_\alpha\to x$ and $f(x_\alpha)\to y$. By the condition, $y=f(x)$ and therefore $(x,y)\in\Gamma(f)$.
\end{proof}
\begin{example}
A continuous map $f$ which does not have a closed graph is the map of $\R$ with its usual topology into $\R_t$ with the trivial topology, which sending $0$ into $1$ and everything else into $0$. Clearly $f$ is continuous. Consider a basic open set $U=(a,b)\times\R$, $a<0$, $b>0$, to which $(0,0)$ belongs. Since $(0,1)\in U$, we have $(0,0)\in\widebar{\Gamma(f)}$. Since $(0,0)\notin\Gamma(f)$, $\Gamma(f)$ is not closed. 
\end{example}
If however the range space $Y$ is Hausdorff, continuous maps must be closed:
\begin{proposition}\label{closed graph continuous map if Hausdorff}
Let $f:X\to Y$ be a continuous map and $Y$ be Hausdorff. Then $f$ has closed graph.
\end{proposition} 
\begin{proof}
Suppose that $x_\alpha\to x$ and $f(x_\alpha)\to y$. If $f$ is continuous, $f(x_\alpha)$ must converge to $f(x)$. Since net limits are unique in a Hausdorff space, $f(x)=y$.
\end{proof}
\begin{example}\label{closed linear map non continuous}
\mbox{}
\begin{itemize}
\item[(a)] The map of $\R$ into $\R$ sending $0$ into $0$ and $x$ into $1/x$ otherwise has a closed graph but is discontinuous.
\item[(b)] Let $\R_d$ denote $\R$ with the discrete topology. Then the identity map $I:\R\to\R_d$ is clearly linear and discontinuous. It is closed, however, for if $x_\alpha\to x$ in $\R$ and $I(x_\alpha)\to y$ in $\R_d$ then $x_\alpha=y$ eventually. Therefore $y=x=I(x)$.
\item[(c)] Let $X=(C([0,1]),\|\cdot\|_\infty)$ be the space of continuous functions of $[0,1]$ to $\R$. Let $C^1\sub C([0,1])$ be the subspace of elements with continuous derivative. Consider the differentiation operator 
\[D:C^1\to C,\quad f\mapsto f'.\]
Clearly, $D$ is linear. The collection $f_n(x)=x^n$ is contained in the unit ball of $C^1$, hut its image $f_n'$ is unbounded. Since $D$ does not map bounded sets into bounded sets, $D$ is discontinuous. To see that $D$ is closed, we use the criterion of Proposition~\ref{closed graph iff} for sequences: Suppose that $f_n\to f$ in $C^1$ and $f_n'\to g$. As convergence in $C^1$ is uniform, $(f_n)$ and $(f_n')$ are uniformly
convergent sequences. By standard theorems of analysis, this implies that $f'=\lim_nf_n'=g$.
\end{itemize}
\end{example}
\begin{proposition}\label{closed graph compact domain is continuous}
Let $f$ map the topological space $X$ into the compact space $Y$. If $f$ has closed graph, then $f$ is continuous.
\end{proposition}
\begin{proof}
We show that $f$ is continuous by showing that, for any closed subset $F\sub Y$, $f^{-1}(F)$ is closed. To this end, let $x_\alpha\to x$ with $x_\alpha\in f^{-1}(F)$. Since $Y$ is compact, $F$ is also compact, so $(f(x_\alpha))$ has a convergent subnet $f(x_{\alpha_\beta})\to y\in F$. Since $x_{\alpha_\beta}\to x$ and $\Gamma(f)$ is closed, $y=f(x)\in F$ or $x\in f^{-1}(F)$.
\end{proof}
\section{Closed linear maps}
In this part, $X$ and $Y$ will denote locally convex Hausdorff spaces. A linear map $A:X\to Y$ is called \textbf{closed} if it has closed graph. Let $A^*$ denote the the linear map of the dual $Y^*$ of $Y$ into the algebraic dual $X^{\star}$ of $X$ defined by taking $\langle x,A^*y^*\rangle=\langle Ax,y^*\rangle$ for all $x\in X$ and $y^*\in Y^*$. We may characterize the subspace $D(A^*)=\{y^*\in Y^*:A^*y^*\in X^*\}$ as follows:
\begin{proposition}\label{LCHS domain of adjoint}
Let $X$ and $Y$ be locally convex Hausdorff spaces and let $A:X\to Y$ be linear. Then
\[D(A^*)=\bigcup\{A(V)^\circ:\text{$V$ is a disked neighborhood of $0$ in $X$}\}.\]
\end{proposition}
\begin{proof}
Let $\mathfrak{U}(0)$ denote the filter of neighborhoods of $0$ in $X$. For $y^*\in D(A^*)$, we must show that there is a disk $V\in\mathfrak{U}(0)$ such that $y^*\in A(V)^\circ$. From the continuity of $A^*y^*$ on $X$, there must be a disk $V$ in $\mathfrak{U}(0)$ such that $\sup_{x\in V}|A^*y^*(x)|\leq 1$. Since $A^*y^*(x)=y^*(Ax)$, we then get $\sup_{x\in V}|y^*(Ax)|\leq 1$, so that $y^*\in A(V)^\circ$. To complete the proof, we only need to reverse the steps.
\end{proof}
\begin{proposition}\label{LCHS Mackey space weak continuous is continuous}
Let $X$ and $Y$ be locally convex Hausdorff spaces and let $A:X\to Y$ be linear. If $X$ carries its Mackey topology $\tau(X,X^*)$ and $D(A^*)=Y^*$, then $A$ is continuous.
\end{proposition}
\begin{proof}
By Proposition~\ref{ajoint in continuous dual iff weak continuous}, $D(A^*)=Y^*$ iff $A$ is weakly continuous. By Proposition~\ref{LCHS continuous and weak continuous}, then, $A$ is Mackey continuous. Since $Y$'s topology is coarser than $\tau(Y,Y^*)$ by the Mackey-Arens theorem, $A$ is continuous.
\end{proof}
A size condition on $D(A^*)$ may also he used to describe closedness of a linear map.
\begin{proposition}\label{LCHS closed iff D(A^*) weakly dense}
For locally convex Hausdorff spaces $X$ and $Y$, the linear map $A:X\to Y$ is closed iff $D(A^*)$ is $\sigma(Y^*,Y)$-dense in $Y^*$.
\end{proposition}
\begin{proof}
Suppose that $D(A^*)$ is $\sigma(Y^*,Y)$-dense in $Y^*$ and that $(x_0,y_0)\in X\times Y\setminus\Gamma(A)$. We show that $(x_0,y_0)\notin\widebar{\Gamma(A)}$. Since $y_0-Ax_0\neq 0$, there exists $y^*\in Y^*$ such that $\langle y_0-Ax_0,y^*\rangle\neq 0$. By the $\sigma(Y^*,Y)$-density of $D(A^*)$ in $Y^*$, we may assume that $y^*\in D(A^*)$. Viewing $(x,Ax)$ as an element of $X\oplus Y$ and $(-A^*y^*,y^*)$ as an element of $(X\oplus Y)^*$, for every $x\in X$,
\[0=\langle x,-A^*y^*+A^*y^*\rangle=\langle x,-A^*y^*\rangle+\langle Ax,y^*\rangle=\langle(x,Ax),(-A^*y^*,y^*)\rangle.\]
Hence $(-A^*y^*,y^*)$ vanishes on $\Gamma(A)$ and therefore also on $\widebar{\Gamma(A)}$. To prove that $(x_0,y_0)\notin\widebar{\Gamma(A)}$, we show that $\langle(x_0,y_0),(-A^*y^*,y^*)\rangle\neq 0$. We have
\begin{align*}
\langle(x_0,y_0),(-A^*y^*,y^*)\rangle=\langle x_0,-A^*y^*\rangle+\langle y_0,y^*\rangle=\langle y_0-Ax_0,y^*\rangle\neq 0.
\end{align*}
Conversely, suppose that $A$ is closed. To demonstrate the $\sigma(Y^*,Y)$-density of $D(A^*)$, we show that $D(A^*)^\circ=\{0\}$. To that end, suppose that $z\in D(A^*)^\circ$. If $z\neq 0$, $(0,z)\notin\Gamma(A)$ because $A$ is linear. Since $\Gamma(A)$ is a closed subspace of $X\oplus Y$, there exists a continuous linear functional $(x^*,y^*)\in(X\oplus Y)^*$ which vanishes on $\Gamma(A)$ but $\langle(0,z),(x^*,y^*)\rangle=\langle z,y^*\rangle=2$. Since $(x^*,y^*)$ vanishes on $\Gamma(A)$, for every $x\in X$,
\[0=\langle(x,Ax),(x^*,y^*)\rangle=\langle x,x^*\rangle+\langle Ax,y^*\rangle=\langle x,x^*+Ay^*\rangle.\]
Hence $A^*y^*=-x^*$, which implies that $y^*\in D(A^*)$ and leaves us with the contradictory statements $z\in D(A^*)^\circ$ and $(z,y^*)=2$ for $y^*\in D(A^*)$. Therefore the result follows by the bipolar theorem.
\end{proof}
\begin{proposition}\label{closed linear map N(A) is closed}
Let $A:X\to Y$ be a closed linear map between topological vector spaces and assume that $Y$ is Hausdorff. Then $N(A)$ is closed.
\end{proposition}
\begin{proof}
Since $Y$ is a Hausdorff space, $X\times\{0\}$ is a closed subset of $X\times Y$. Hence, since $\Gamma(A)$ is closed, $N(A)\times\{0\}=\Gamma(A)\cap X\times\{0\}$ is closed in $X\times Y$. Since the map $I:X\to X\times Y,x\mapsto(x,0)$ is a homeomorphism, $N(A)=I^{-1}(N(A)\times\{0\})$ is closed in $X$.
\end{proof}
\section{Closed graph theorem}
If a pair $(X,Y)$ of topological vector spaces is such that a closed linear map from $X$ to $Y$ must be continuous, we say that a "closed graph theorem holds for the pair".\par
The main results of this part are roughly parallel to the generalizations of the Banach-Steinhaus theorem, another early victory for category arguments. Viewing those theorems in the context of "for what conditions on the spaces $(X,Y)$ does a Banach-Steinhaus theorem hold?" we obtained two results---one for when $X$ and $Y$ were each locally convex and one for when they were not necessarily locally convex. In the locally convex case, $X$ had to be barreled; without it, $X$ had to be nonmeager. For closed graph theorems there is a similar bifurcation: assuming constantly that $Y$ is completely pseudometrizable, then "$X$ barreled" or "$X$ nonmeager" activates a closed graph theorem, respectively, in the locally convex and nonlocally convex cases.\par
A notion which plays an important role in proving the closed graph theorems is that of \textbf{almost continuity} of a linear map $A:X\to Y$, namely that $\widebar{A^{-1}(V)}$, rather than $A^{-1}(V)$, is a neighborhood of $0$ for each neighborhood $V$ of $0$ in $Y$. In certain cases ($X$ barreled in the locally convex case, $X$ nonmeager otherwise) linear maps are automatically almost continuous. Our first closed graph theorems show that an almost continuous closed linear map is continuous.
\begin{proposition}\label{LCS barreled linear map is almost continuous}
If $X$ is barreled then every linear map $A$ of $X$ into any locally convex space $Y$ is almost continuous.
\end{proposition}
\begin{proof}
In the notation of the statement, $\widebar{A^{-1}(B)}$ is a barrel in $X$ for any barrel $B$ in $Y$. Since $Y$ is locally convex, there is a base of barrels $B$ at $0$. Consequently, if $X$ is barreled, any linear map is almost continuous.
\end{proof}
\begin{proposition}\label{TVS Baire linear map is almost continuous}
If $X$ is a Baire topological vector space and $Y$ any topological vector spaces, then any linear map $A:X\to Y$ is almost continuous.
\end{proposition}
\begin{proof}
Given a neighborhood $U$ of $0$ in $Y$, choose a neighborhood $V$ of $0$ in $Y$ such that $V-V\sub U$. Since $V$ is a neighborhood of $0$, $A^{-1}(V)$ is absorbent in $X$; hence it must be nonmeager by Proposition~\ref{TVS Baire absorbent is nonmeager}. Since $A^{-1}(V)$ cannot be nowhere dense, $\widebar{A^{-1}(V)}$ is a neighborhood of some point. Hence the algebraic difference $\widebar{A^{-1}(V)}-\widebar{A^{-1}(V)}$ is a neighborhood of $0$ in $X$. Since
\[\widebar{A^{-1}(U)}\sups \widebar{A^{-1}(V-V)}=\widebar{A^{-1}(V)-A^{-1}(V)}\sups \widebar{A^{-1}(V)}-\widebar{A^{-1}(V)}\]
it follows that $\widebar{A^{-1}(U)}$ is a neighborhood of $0$ in $X$.
\end{proof}
We can now prove our first closed graph theorems.
\begin{theorem}[\textbf{Closed Graph Theorem}]\label{closed graph theorem}
Let $X$ and $Y$ be topological vector spaces and assume that $Y$ is complete pseudometrizable. Then any closed almost continuous linear map $A:X\to Y$ is continuous. It follows that
\begin{itemize}
\item[(a)] If $A$ is a closed linear map of the Baire topological vector space $X$ into the complete pseudometrizable topological vector space $Y$, then $A$ is continuous.
\item[(b)] If $A$ is a closed linear map of the barreled space $X$ into the complete pseudornetrizable locally convex space $Y$, then $A$ is continuous. 
\end{itemize}
\end{theorem}
\begin{proof}
Let $\{V_n:n\in\N\}$ be a base of closed balanced neighborhoods of $0$ in $Y$ such that $V_{n+1}+V_{n+1}\sub V_n$ for each $n$. Since $A$ is almost continuous, there is a countable family $\{U_n\}$ of balanced neighborhoods of $0$ in $X$ such that $U_n\sub\widebar{A^{-1}(V_n)}$ and $U_{n+1}+U_{n+1}\sub U_n$. We demonstrate continuity by showing that $U_n\sub A^{-1}(V_{n-1})$ for each $n\geq 2$.\par
To this end, fix such an $n$ and choose $x_n\in U_n$. We show $x_n\in A^{-1}(V_{n-1})$ by showing that there is some $z\in V_{n-1}$ such that $(x_n,z)\in\widebar{\Gamma(A)}=\Gamma(A)$. Note that for each $i\in\N$ we have $U_i\sub\widebar{A^{-1}(V_i)}\sub A^{-1}(V_i)+U_{i+1}$, so it follows by induction that there exist and $x_i\in U_i$ and $y_i\in A^{-1}(V_i)$ such that $x_i=y_i+x_{i+1}$. Applying $A$ and summing both sides for $n\leq i\leq m$,
\[Ax_n-Ax_{m+1}=\sum_{i=n}^{m}Ay_i\in V_n+\cdots+V_m\sub V_{n-1}.\]
The sequence of partial sums of $\sum_{i=n}^{+\infty}Ay_i$ is therefore Cauchy, so it converges to some $z$ in the closed neighborhood $V_{n-1}$. For any neighborhood $U$ of $0$ in $X$, since $\widebar{A^{-1}(V_{m+1})}\sub A^{-1}(V_{m+1})+U$, we get
\[\sum_{i=n}^{m}y_i=x_n-x_{m+1}\in x_n+\widebar{A^{-1}(V_{m+1})}\sub x_n+A^{-1}(V_{m+1})+U,\]
so there exists $w\in A^{-1}(V_{m+1})$ such that $\sum_{i=n}^{m}y_i-w\in x_n+U$. Since
\[\sum_{i=n}^{m}Ay_i-z=-\sum_{j=n+1}^{\infty}Ay_i\in V_{m+1}+V_{n+2}+\cdots\sub V_m,\]
this implies
\[A\Big(\sum_{i=n}^{m}y_i-w\Big)-z=\sum_{i=n}^{m}Ay_i-z-Aw\in V_m+V_{m+1}\sub V_{m-1}.\]
Hence the basic neighborhood $(x_n+U)\times(z+V_{m-1})$ of $(x_n,z)$ contains the point $(\sum_{i=n}^{m}y_i-w,A(\sum_{i=n}^{m}y_i-w))\in\Gamma(A)$. Since this holds for all $m\geq n$, this implies $(x_n,z)\in\Gamma(A)$. It follows that $U_n\sub A^{-1}(V_{n-1})$ and $A$ is seen to be continuous.
\end{proof}
\section{Open mapping theorem}
We know that a map is open if open sets are mapped to open sets. We now define an almost analoge for this. Let $X$ and $Y$ be topological vector spaces. A linear map $A:X\to Y$ is \textbf{almost open} if for any neighborhood $U$ of $0$ in $X$, $\widebar{A(U)}$ is a neighborhood of $0$ in $Y$. Clearly, if a map $f$ is bijective then $f$ is almost open if and only if $f^{-1}$ is almost continuous.
\begin{theorem}\label{open mapping theorem}
If $(X,d)$ is a complete pseudometrizable topological vector space and $Y$ is any Hausdorff topological vector space then any closed almost open linear surjection $A:X:\to Y$ is open.
\end{theorem}
\begin{proof}
Let $N=N(A)$ and consider the induced bijective map $\bar{A}:X/N\to Y$. The idea of the proof is to apply the closed graph theorem to $\bar{A}^{-1}$.
\[\begin{tikzcd}
X\ar[d,swap,"\pi"]\ar[r,"A"]&Y\\
X/N\ar[ru,swap,"\bar{A}"]
\end{tikzcd}\]

Since $A$ is closed, so is $N$ by Proposition~\ref{closed linear map N(A) is closed}. Consequently, $X/N$ is a complete pseudometrizable space with the quotient pseudometric $\bar{d}$ defined by $\bar{d}(\bar{x},\bar{y})=d(x+N,y+N)$. A base at $0$ for the quotient topology on $X/N$ is given by sets of the form $V+N$ where $V$ is a neighborhood of $0$ in $X$. For any such set, $\widebar{\bar{A}(V+N)}=\widebar{A(V)}$ is a neighborhood of $0$ in $Y$ by hypothesis, so $\bar{A}$ is almost open. Since $\bar{A}$ is bijective, $\bar{A}^{-1}$ is almost continuous. We now show that $\bar{A}$ is closed.\par
To this end, the criterion of Proposition~\ref{closed graph iff} may be applied. Suppose that $\bar{x}_n\to\bar{x}$ and $\bar{A}\bar{x}_n\to y$. Then there exists a subsequence $(x_{n_k})$ of $(x_n)$ and a sequence $(z_k)$ from $N$ such that $d(x_{n_k}+z_k,x)<1/k$ for each $k\in\N$. Hence $x_{n_k}+z_k\to x$ while $A(x_{n_k}+z_k)=Ax_{n_k}\to y$. Since $A$ is closed, $y=Ax=\bar{A}\bar{x}$ and therefore $A$ is closed.\par
Since the graph of $A$ is closed, so is the graph of $\bar{A}^{-1}$. Hence, since $\bar{A}$ is almost open, $\bar{A}^{-1}$ is a closed almost continuous linear map of the topological vector space $Y$ into the complete pseudometrizable space $X/N$. The continuity of $\bar{A}^{-1}$ now follows from the closed graph theorem; thus, $\bar{A}$ is open. The openness of $A$ now follows by Proposition~\ref{TVS first isomorphism thm}.
\end{proof}
Two situations that produce almost openness are given below. In conjunction with the open mapping theorem above, these yield two particular open mapping theorems.
\begin{proposition}\label{LCS barreled surjective is almost open}
If $Y$ is barreled then each surjective linear map $A$ of any locally convex space $X$ onto $Y$ is almost open.
\end{proposition}
\begin{proof}
Suppose that $A$ is a linear map of the locally convex space $X$ onto the barreled space $Y$. Since $X$ is locally convex, it must have a base of neighborhoods of $0$ which are barrels. Since $A$ is onto, if $B$ is any barrel in $X$, $\widebar{A}(B)$ is a barrel in $Y$ by Proposition~\ref{TVS balanced and absorbent under linear map}, hence a neighborhood of $0$.
\end{proof}
\begin{proposition}\label{TVS Baire surjective is almost open}
A linear map $A$ of a topological vector space $X$ onto a Baire topological vector space $Y$ is almost open.
\end{proposition}
\begin{proof}
Given a neighborhood $U$ of $0$ in $X$, choose a neighborhood $V$ of $0$ such that $V-V\sub U$. Then $A(V)$ is an absorbent subset of $Y$ because $A$ is onto, so is nonmeager by Proposition~\ref{TVS Baire absorbent is nonmeager}. In particular, $A(V)$ is not nowhere dense. Thus $\widebar{A(V)}-\widebar{A(V)}$ contains a neighborhood of $0$. Since
\[\widebar{A(U)}\sups\widebar{A(V-V)}=\widebar{A(V)-A(V)}\sups\widebar{A(V)}-\widebar{A(V)},\]
it follows that $\widebar{A(U)}$ is a neighborhood of $0$ in $Y$.
\end{proof}
Combining the previous three results, we have:
\begin{theorem}\label{open mapping theorem for closed map}
Let $A$ be a closed linear map of a complete pseudometrizable space $X$ onto a topological vector space $Y$. If
\begin{itemize}
\item[(a)] $Y$ is a Baire space, or
\item[(b)] $X$ is locally convex, $Y$ is barreled and $A$ is surjective
\end{itemize}
then $A$ is open.
\end{theorem}
The following result is a variant of Theorem~\ref{open mapping theorem for closed map}(a). To compensate for slightly relaxed conditions on the codomain $Y$, we strengthen the assumption on $A$.
\begin{theorem}\label{open mapping theorem for continuous map}
Let $A$ be a continuous linear map of a complete pseudometrizable topological vector space $X$ into a Hausdorff topological vector space $Y$. If $R(A)$ is a nonmeager subset of $Y$, then:
\begin{itemize}
\item[(a)] $A$ is surjective,
\item[(b)] $A$ is open,
\item[(c)] $Y$ is a complete pseudometrizable space.
\end{itemize}
\end{theorem}
\begin{proof}
Since $A$ is continuous and $Y$ is Hausdorff, $N=N(A)$ is closed; hence $X/N$ is a complete pseudometrizable space. Let $\bar{A}:X/N\to R(A)$ be the induced bijective map. Since $\bar{A}$ is continuous and $Y$ is Hausdorff, $\bar{A}$ is closed. Since $R(A)=\bar{A}(X/N)$ is a Baire space, $\bar{A}$ is an open map by Theorem~\ref{open mapping theorem for closed map}. Thus $\bar{A}$ is a linear homeomorphism. Since a linear homeomorphism is uniformly continuous, $R(A)$ is complete. In particular, $R(A)$ is closed since $Y$ is Hausdorff. Since $R(A)$ is a nonmeager subspace of $Y$, it cannot be nowhere dense; hence $R(A)=\widebar{R(A)}=Y$. Since $\bar{A}$ is open, it follows that $A$ is open as well. To see that $Y$ is pseudometrizable, note that if $\{U_n\}$ is a countable base of neighborhoods of $0$ in $X$, then $\{A(U_n)\}$ is a base at $0$ in $Y=R(A)$ by the openness and continuity of $A$.
\end{proof}
We deduced the open mapping theorem here from the closed graph theorem by applying the closed graph theorem to $\bar{A}^{-1}$. The exact sarne technique may be used to deduce the closed graph theorem from the
open mapping theorem.
\section{Applications}
Let $(X,\langle\cdot,\cdot\rangle)$ and $(Y,\langle\cdot,\cdot\rangle)$ be inner product spaces and let $A:X\to Y$ be any map (not even linear, necessarily). Given $y\in Y$, there may be an element $A^*y\in X$ such that $\langle Ax,y\rangle=\langle x,A^*y\rangle$ for every $x\in X$. From the nondegeneracy of the inner product, it is clear that such $A^*y$ is unique, if exists. The set of $y\in Y$ for which this is true is denoted $D^*$ and we define the adjoint of $A$ to be the map $A^*:D^*\to X,y\mapsto A^*y$. Even though $A$ may not be linear, $A^*$ is a closed linear map.
\begin{proposition}\label{inner product space adjoint}
Let $(X,\langle\cdot,\cdot\rangle)$ and $(Y,\langle\cdot,\cdot\rangle)$ be inner product spaces and let $A:X\to Y$ be any map. Then the map $A^*:D^*\to X$ is linear and closed.
\end{proposition}
\begin{proof}
Let $u,v\in D^*$ with $A^*u=s$ and $A^*v=t$. For any scalars $a$ and $b$ and $x\in X$, then
\[\langle Ax,au+bv\rangle=\bar{a}\langle Ax,u\rangle+\bar{b}\langle Ax,v\rangle=\bar{a}\langle x,s\rangle+\bar{b}\langle x,t\rangle=\langle x,as+bt\rangle.\]
Therefore, $D^*$ is a subspace and $A^*$ is linear. To verify that $A^*$ is closed, suppose that $y_n\to y$ and $A*y_n=z_n\to z$. For every $n$, $\langle Ax,y_n\rangle=\langle x,z_n\rangle$. Since the inner product is continuous, this implies that $\langle Ax,y\rangle=\langle x,z\rangle$. Thus, $y\in D^*$ and $A*y=z$.
\end{proof}
Now let $H$ be a Hilbert space and let $A:H\to H$ be linear. Define the adjoint $A^*$ of $A$ as above. Then by Riesz's theorem we know that $D^*=H$. If $A$ is self-adjoint, i.e., $A=A^*$, then $A$ must be closed. For a Hilbert space, version (a) or (b) of the closed graph theorem applies and asserts that $A$ is continuous. Thus, self-adjoint maps on a Hilbert space must be continuous.\par
The bounded inverse theorem below has interesting applications to questions of uniqueness of norm topologies, some of which are discussed after the theorem. It shows, in particular, that a continuous linear bijection between Fr\'echet spaces must be a homeomorphism.
\begin{theorem}[\textbf{Inverse Mapping Theorem}]
Let $X$ be a complete pseudometrizable topological vector space and $Y$ a Baire Hausdorff topological vector space. If $A:X\to Y$ is a continuous linear bijection, then $A^{-1}$ is continuous, i.e., $A$ is a linear homeomorphism.
\end{theorem}
\begin{proof}
Since $A$ is continuous and $Y$ Hausdorff, $A$ must have a closed graph. It follows from the open mapping theorem that $A$ is open, i.e., $A^{-1}$ is continuous.
\end{proof}
If $\mathcal{T}$ and $\mathcal{T}'$ are comparable metric vector topologies for a vector space $X$ and $X$ is complete with respect to each of them, then they must be equal. To see this, apply the inverse mapping theorem to the identity map. Other results along these lines are:
\begin{itemize}
\item[(a)] A homomorphism of a Banach algebra into a semisimple (the intersection of its maximal ideals is $\{0\}$) commutative Banach algebra is automatically continuous.
\item[(b)] Any semisimple commutative Banach algebra has a unique norm topology.
\end{itemize}
Thus, for example, the sup norm $\|\cdot\|_\infty$ is the "only" Banach algebra norm for the algebra $(C([0,1]),\|\cdot\|_\infty)$.
\begin{example}\label{C([0,1]) closed subspace}
Let $X$ be the Banach space $(C[0,1],\|\cdot\|_\infty)$ of continuous real-valued functions on $[0,1]$.
\begin{itemize}
\item[(a)] If the subspace $C^1$ of continuously differentiable functions were a closed subspace of $X$, it too would be a Banach space. Consequently, the differential operator $D:C^1\to C$, since it is closed by Example~\ref{closed linear map non continuous}, would be continuous by the closed graph theorem. We have already observed, however, that it is not.
\item[(b)] Suppose that $M\sub X$ is some closed subspace of continuously differentiable functions such that the differential operator $D:M\to X,f\mapsto f'$ is continuous. If $B_M$ is the closed unit ball of $M$, then for each $f\in B_M$,
\[\|Df\|_\infty\leq\|D\|\|f\|_\infty\leq\|D\|.\]
In other words, the functions in $B_M$ have uniformly bounded derivatives which, by the mean value theorem, implies that $B_M$ is equicontinuous. Since $\|f\|_\infty$ for each $f\in B_M$, it follows from Ascoli's theorem that $B_M$ is relatively compact in $X$. Therefore $B_M=\widebar{B}_M\cap M$ is compact in $M$ and $M$ is locally compact. Therefore, by Proposition~\ref{TVS locally compact}, $M$ is finite-dimensional.
\end{itemize}
\end{example}
\chapter{Reflexivity}
In this section we investigate the reflexivity of topological vector spaces. In order to use the duality theory, the space $X$ will at least be locally convex and Hausdorff, so $(X,X^*)$ is a dual pair. $J$ denotes the canonical embedding. Since $(X,X^*)$ is a dual pair, $J$ is seen to be a linear isomorphism and we often identify $X$ and $J(X)\sub X^{**}$. $X^*$ stands for the strong dual $(X^*,\beta(X^*,X))$ of $X$.\par
When topologized, $X^{**}$ is endowed with $\beta(X^{**},X^*)$, the topology determined by sets of the form $B^\circ_{X^{**}}$ where $B$ is a $\sigma(X^*,X^{**})$-bounded subset of $X^*$. As we will show, $\sigma(X^*,X^{**})$-boundedness is the same as $\beta(X^*,X)$-boundedness so $\beta(X^{**},X^*)$ may be described as the topology determined by sets of the form $B^\circ_{X^{**}}$ where $B$ is $\beta(X^*,X)$-bounded.
\section{Reflexive spaces}
For any normed space $X$, the norm topology on $X^*$ is the strong topology $\beta(X^*,X)$, the polar topology generated by all $\sigma(X,X^*)$-bounded subsets of $X$ (Example~\ref{NVS dual carry beta(X^*,X)}). Motivated by that, we define:
\begin{definition}
Let $X$ be a locally convex Hausdorff space. With $X^*=(X^*,\beta(X^*,X))$, we take $X^{**}=(X^*,\beta(X^*,X))^*$ and call it the \textbf{bidual} of $X$. In other words, the bidual is the dual of the strong dual $(X^*,\beta(X^*,X))$. We call $X^{**}$ endowed with $\beta(X^{**},X^*)$ the \textbf{strong bidual} of $X$.
\end{definition}
From now on, without further specification, we always endow $X^*$ and $X^{**}$ the strong topology.\par
Suppose $(X,\mathcal{T})$ is a locally convex Hausdorff space and $x\in X$. Since the map $Jx$ is weakly continuous and $\sigma(X^*,X)\sub\beta(X^*,X)$, it follows that it is strongly continuous, i.e., that $J(X)\sub X^{**}$. If $J$ is surjective we say that $X$ is \textbf{semi-refiexive}. If $J$ is a surjcctive homeomorphism, we say that $X$ is \textbf{reflexive}. It is immediate that reflexivity is preserved by linear homeomorphism.
\begin{example}
The strong dual $(X^*,\beta(X^*,X))$ of any bornological space $X$ is complete by Proposition~\ref{LCS bornological L(X,Y) complete if Y complete}. Thus, if a normed---hence bornological---space $X$ is reflexive, it is a Banach space. If $(X,\mathcal{T})$ is a normed space, $J$ is an isometry; hence an embedding, so to show a normed space is reflexive, it is only necessary to show that $J$ is onto---that $X$ is semireflexive.
\end{example}
\begin{proposition}\label{LCHS bounded in X^*}
Let $X$ be a locally convex Hausdorff space. Then:
\begin{itemize}
\item[(a)] the bounded sets of $X^*$ are the same with respect to $\beta(X^*,X)$, $\sigma(X^*,X^{**})$, and $\tau(X^*,X^{**})$; 
\item[(b)] a subset $D\sub X^*$ is $\beta(X^*,X)$-bounded iff $D^\circ\sub X$ absorbs each $\sigma(X,X^*)$-bounded disk of $X$.
\end{itemize}
\end{proposition}
\begin{proof}
The topologies $\sigma(X^*,X^{**})$, and $\tau(X^*,X^{**})$ are topologies of the pair $(X^*,X^{**})$ as is $\beta(X^*,X)$ by the way $X^{**}$ is defined; hence the bounded sets of all three topologies are the same by Mackey's theorem. The second part follows from Proposition~\ref{polar topo bounded iff}.
\end{proof}
By Proposition~\ref{LCHS bounded in X^*}, $\beta(X^{**},X^*)$ may be described as the polar topology determined by $\beta(X^*,X)$-bounded sets.
\begin{proposition}\label{LCHS topo weak than bidual}
If $(X,\mathcal{T})$ is a locally convex Hausdorff space then $\mathcal{T}\sub\beta(X^{**},X^*)\cap X$.
\end{proposition}
\begin{proof}
Any $\mathcal{T}$-closed, disked neighborhood $U$ of $0$ absorbs $\mathcal{T}$-bounded subsets of $X$, so it absorbs $\sigma(X,X^*)$-bounded disks. Since $U=U^{\circ\circ}$, it follows that $U^\circ\sub X^*$ is $\beta(X^*,X)$-bounded by Proposition~\ref{LCHS bounded in X^*}. Since $\beta(X^*,X)$-bounded is equivalent to $\sigma(X^*,X^{**})$-bounded, it follows that $(U^{\circ}_{X^*})^\circ_{X^{**}}\in\beta(X^{**},X^*)$. By the bipolar theorem,
\[(U^{\circ}_{X^*})^\circ_{X^{**}}\cap X=(U^{\circ}_{X^*})^{\circ}_{X}=\mathrm{cl}_{\sigma(X,X^*)}U=U.\]
Thus $U$ is a $\beta(X^{**},X^*)\cap X$-neighborhood of $0$ and $\mathcal{T}\sub\beta(X^{**},X^*)\cap X$.
\end{proof}
Recall that a locally convex space is infrabarreled if every bornivorous barrel is a neighborhood of $0$.
\begin{theorem}\label{LCHS J embedding iff infrabarreled}
For a locally convex Hausdorff space $(X,\mathcal{T})$, the map $J:X\to X^{**}$ is an embedding iff $X$ is infrabarreled.
\end{theorem}
\begin{proof}
Since we have already shown that $\mathcal{T}\sub\beta(X^{**},X^*)\cap X$, $J$ is generally a relatively open map so it only remains to show that $J$ is continuous iff $X$ is infrabarreled. By Theorem~\ref{Banach-Steinhaus for infrabarreled space}, $X$ is infrabarreled iff every $\beta(X^*,X)$-bounded set is equicontinuous. Since subsets of equicontinuous sets are equicontinuous, it suffices to show that continuity of $J$ is equivalent to the equicontinuity of basic $\beta(X^*,X)$-neighborhoods $V$ of $0$.\par
A basic $\beta(X^*,X)$-neighborhood of $0$ is the polar $U^\circ$ of $\sigma(X,X^*)$-closed, bounded disk $U\sub X$. A basic $\beta(X^{**},X^*)$-neighborhood of $0$ is therefore of the form $(U^\circ_{X^*})^\circ_{X^{**}}$. Hence, $J$ is continuous iff
\[J^{-1}((U^\circ_{X^*})^\circ_{X^{**}})=(U^\circ_{X^*})^\circ_{X^{**}}\cap X=(U^{\circ}_{X^*})^{\circ}_{X}=U\]
is a neighborhood of $0$. By Proposition~\ref{equicontinuous iff polar is nbhd}, this is equivalent to the equicontinuity of $U^\circ$.
\end{proof}
We now characterize semireflexivity by means of a weak Heine-Borel property.
\begin{theorem}\label{LCHS semireflexive iff}
Let $(X,\mathcal{T})$ be a locally convex Hausdorff space. Then the following are equivalent.
\begin{itemize}
\item[(\rmnum{1})] $X$ is semireflexive
\item[(\rmnum{2})] $\beta(X^*,X)=\tau(X^*,X)$.
\item[(\rmnum{3})] Every bounded subset of $X$ is contained in a $\sigma(X,X^*)$-compact set.
\end{itemize}
\end{theorem}
\begin{proof}
Since $X^{**}=(X^*,\beta(X^*,X))^*$, $X$ is semireflexive iff $\beta(X^*,X)$ is a topology for the pair $(X^*,X)$. Since $\tau(X^*,X)\sub\beta(X^*,X)$, this is equivalent to $\tau(X^*,X)\sub\beta(X^*,X)$, by the Mackey-Arens theorem.\par
Since $\tau(X^*,X)\sub\beta(X^*,X)$ in general, condition (\rmnum{2}) is equivalent to that every $\beta(X^*,X)$-neighborhod of $0$ in $X^*$ is a $\tau(X^*,X)$-neighborhod of $0$, or equivalently, the polar of any $\sigma(X,X^*)$-bounded set contains the polar of a $\sigma(X,X^*)$-compact disk. It is clear that this is equivalent to (\rmnum{3}).
\end{proof}
By Proposition~\ref{LCS barreled iff carry strong topology}, $(X^*,\beta(X^*,X))$ is barreled iff $\beta(X^*,X)=\beta(X^*,X^{**})$. If $X$ is semireflexive, however, then $X^{**}=X$; hence $\beta(X^*,X)=\beta(X^*,X^{**})$. Consequently:
\begin{corollary}\label{LCHS semireflexive X^* is barreled}
Let $X$ be a locally convex Hausdorff space. If $X$ is semireflexive then $X^*$ is barreled.
\end{corollary}
\begin{theorem}\label{LCHS reflexive iff}
For any locally convex Hausdorff space $X$, the fol1owing statements are equivalent:
\begin{itemize}
\item[(\rmnum{1})] $X$ is reflexive.
\item[(\rmnum{2})] $X$ is barreled and semireflexive.
\item[(\rmnum{3})] $X$ is infrabarreled and semireflexive.
\end{itemize}
\end{theorem}
\begin{proof}
By Theorem~\ref{LCHS J embedding iff infrabarreled}, it is clear that (\rmnum{1})$\Rightarrow$(\rmnum{3}). Thus we only need to show (\rmnum{3})$\Rightarrow$(\rmnum{2}). But note that since $X$ is semireflexive, $\tau(X^*,X)=\beta(X^*,X)$. Thus $\sigma(X^*,X)$-boundedness is equivalent to $\beta(X^*,X)$-boundedness, and infrabarreledness is equivalent to barreledness in view of Theorem~\ref{Banach-Steinhaus for LCS} and Theorem~\ref{Banach-Steinhaus for infrabarreled space}.
\end{proof}
\begin{example}
Since neither $(C[0, 1],\|\cdot\|_1)$ nor $c_{00}$ is barreled, neither of them is reflexive.
\end{example}
We now prove some permanence properties.
\begin{proposition}[\textbf{Permanence of Reflexivity}]\label{LCHS reflexive permanence}
\mbox{}
\begin{itemize}
\item[(a)] Every closed subspace of a semireflexive locally convex Hausdorff space is semireflexive.
\item[(b)] Products of (semi)reflexive locally convex Hausdorff spaces are (semi)reflexive.
\item[(c)] Locally convex direct sums of (semi)reflexive locally convex Hausdorff spaces are (semi)reflexive.
\item[(d)] Strict inductive limits of (semi)reflexive locally convex Hausdorff spaces are (semi)reflexive.
\item[(e)] A locally convex Hausdorff space $X$ is reflexive iff $X^*$ is reflexive and $X$ is semireflexive.
\end{itemize}
\end{proposition}
\begin{proof}
Assume that $X$ is semireflexive. We use the criterion of Theorem~\ref{LCHS semireflexive iff} and show that any bounded subset $B$ of $M$ is contained in a $\sigma(M,M^*)$-compact set. $B$ being bounded in $M$ implies that $B$ is bounded in $X$. Since $X$ is reflexive, $B\sub K$ for some $\sigma(X,X^*)$-compact set $K\sub X$ by Theorem~\ref{LCHS semireflexive iff}. Clearly $K\cap M$ is $\sigma(X,X^*)\cap M$-compact. By Proposition~\ref{weak topo subspace} we have $\sigma(X,X^*)\cap M=\sigma(M,M^*)$, thus $B$ is a subset of the $\sigma(M,M^*)$-compact set $K\cap M$.\par
Let $\{X_s\}$ be a family of (semi)reflexive locally convex Hausdorff spaces. Let $X=\prod_sX_s$ be their product. Then $X$ is locally convex and Hausdorff. By Proposition~\ref{weak Macky strong topo for product and direct sum} and the hypothesis we have $X^*=\bigoplus_sX_s^*$ and $\beta(X^*,X)=\bigoplus\beta(X_s^*,X_s)=\bigoplus\tau(X_s^*,X_s)=\tau(X^*,X)$. Hence $X$ is semireflexive. If $X_s$ are reflexive, then by Proposition~\ref{LCS barreled space permanence} $X$ is barreled, hence reflexive.\par
Now let $X=\bigoplus_sX_s$ be a locally convex direct sum. Then similarly, $X^*=\prod_sX_s^*$ and $\beta(X^*,X)=\prod\beta(X_s^*,X_s)=\prod\tau(X_s^*,X_s)=\tau(X^*,X)$. Hence $X$ is semireflexive. If $X_s$ are reflexive, then by Proposition~\ref{LCS barreled space permanence} $X$ is barreled, hence reflexive.\par
Now assume that $X=\rlim X_n$ and each $X_n$ is (semi)reflexive. Then $X$ is locally convex Hausdorff and barreled. Let $B$ be a closed bounded set in $X$, then by Proposition~\ref{LCS str inductive limit bounded compact} $B\sub X_n$ for some $n$ and $B$ is bounded in $X_n$. Since $X_n$ is reflexive, $B$ is $\sigma(X_n^*,X_n)$-compact and hence $\sigma(X^*,X)$-compact (by Proposition~\ref{weak topo subspace}). Thus $X$ is reflexive.\par
Suppose the locally convex Hausdorff space $X$ is reflexive. Then $(X^*)^{**}=(X^{**})^{*}=X^*$ so $X^*$ is semireflexive. Furthermore, by Corollary~\ref{LCHS semireflexive X^* is barreled}, $X^*$ is barreled. Thus it is reflexive.\par
Now suppose that $X^*$ is reflexive and $X$ is semireflexive. Then by Corollary~\ref{LCHS semireflexive X^* is barreled}, since $X^*$ is reflexive, $(X^{**},\beta(X^{**},X^*))$ is barreled. Let $B\sub X^*$ be a $\beta(X^*,X)$-bounded subset. Then by Proposition~\ref{LCHS bounded in X^*}, $B$ is $\sigma(X^*,X^{**})$-bounded, hence is equicontinuous by Theorem~\ref{Banach-Steinhaus for LCS}. This implies $X$ is infrabarreled, so it is reflexive, by Theorem~\ref{LCHS reflexive iff}.\par
Finally, assume that $X$ is semireflexive and $X/M$ is a quotient of $X$. Then $(X/M)^*\cong M^\bot$ by Proposition~\ref{dual of subspace and quotient}. 
\end{proof}
\begin{corollary}
Let $X$ be a locally convex Hausdorff space and $M$ be a closed subspace of $X$. Then $X/M$ is (semi)reflexive if $X$ is (semi)reflexive.
\end{corollary}
\begin{proof}
By Proposition~\ref{dual of subspace and quotient}, $(X/M)^*\cong M^\bot$, which is a closed subspace of $(X^*,\sigma(X^*,X))$, hence of $(X^*,\beta(X^*,X))$. Since $X^*$ is semireflexive, $M^\bot$ is semireflexive by Proposition~\ref{LCHS reflexive permanence}. If $X$ is reflexive, then $X$ is barreled, so is $X/M$ by Proposition~\ref{LCS barreled space permanence}. Thus $X/M$ is reflexive.
\end{proof}
\section{Reflexivity of Banach spaces}
For Banach spaces, reflexivity is a three space property.
\begin{proposition}\label{Banach space reflexive subspace quotient}
Let $X$ be a Banach space and $M$ be a closed subspace. Then
\begin{itemize}
\item[(a)] $X$ is reflexive iff $X^*$ is reflexive.
\item[(b)] $X$ is reflexive iff $M$ and $X/M$ are reflexive.
\end{itemize}
\end{proposition}
\begin{proof}
Let $M$ be a closed subspace of a normed space $X$ such that both $M$ and $X/M$ are reflexive. It suffices to prove that $X$ is semireflexive. Let $T:(X/M)^*\to M^\bot$ be the usual isometric isomorphism. Let $x^{**}\in X^{**}$. Since $T^*x^{**}\in(X/M)^{**}$ and $X/M$ is reflexive, there is an $x+M$ in $X/M$ such that $T^*x^{**}=J_{X/M}(x+M)$. It follows that $x^{**}-Jx\in (M^\circ_{X^*})^\circ_{X^{**}}=M$. Thus there is $m\in M$ such that $x^{**}-Jx=Jm$. Thus $x^{**}=J(x+m)$ and the claim follows.
\end{proof}
\begin{theorem}[\textbf{Kakutani}]\label{Banach space reflexive iff closed unit ball weakly compact}
The Banach space $X$ is reflexive iff its closed unit ball $B_X$ is weakly compact.
\end{theorem}
\begin{proof}
If $B_X$ is $\sigma(X,X^*)$-compact then for any $r>0$, $rB_X$ is $\sigma(X,X^*)$-compact. Since every bounded set is contained in $rU$ for some $r>0$, $X$ is semireflexive by Theorem~\ref{LCHS semireflexive iff}. As $X$ is barreled, $X$ is reflexive. Conversely, if $X$ is reflexive, then $B_X\sub K$ for some $\sigma(X,X^*)$-compact set $K$ by Theorem~\ref{LCHS semireflexive iff}. Since $B_X$ is a norm-closed convex set, $B_X$ is $\sigma(X,X^*)$-closed, hence $\sigma(X,X^*)$-compact.
\end{proof}
\begin{example}
Consider the Banach space $\widetilde{X}=C(X,\K,\|\cdot\|_\infty)$ of continuous maps of the compact Hausdorff space $X$ into $\K$. If $X$ is finite, with $n$ elements say , then $\widetilde{X}$ is linearly isometric to $\K^n$. As $\K^n$ is locally compact, it is reflexive by Theorem~\ref{Banach space reflexive iff closed unit ball weakly compact}.\par
Now suppose that $X$ is infinite and that $\{x_n\}\sub X$ is denumerable. As $X$ is compact, $\{x_n\}$ has a limit point $x_0$. If $x_0\in\{x_n\}$, remove it from $\{x_n\}$ and reindex the set. By Urysohn's lemma, for each $n\in\N$ there exists $f_n\in C(X,[0,1])$ such that $f_n(\{x_1,\dots,x_n\})=\{1\}$ and $f_n(x_0)=0$. Thus $\|f\|_\infty=1$ for all $n\in\N$ and $\{f_n\}$ is a bounded subset of $\widetilde{X}$. If $X$ is reflexive, its unit ball is weakly compact by Theorem~\ref{Banach space reflexive iff closed unit ball weakly compact}. Hence, by the Eberlein-Smulian theorem, there exists a weakly convergent subsequence $g_n\to g\in\widetilde{X}$ of the bounded sequence $\{f_n\}$. For each $x\in X$, the evaluation map $x^*:f\mapsto f(x)$ is a continuous linear functional on $\widetilde{X}$ with $\widetilde{X}$ carrying either the norm or $\sigma(\widetilde{X},\widetilde{X}^*)$-topology. Therefore, since $g_n\to g$ in the $\sigma(\widetilde{X},\widetilde{X}^*)$-topology, $g_n(x)\to g(x)$ for all $x\in X$. But $g_n(x_m)=1$ for $n\geq m$ implies that $g(x_m)=1$ for all $m\in\N$. Also, $g_n(x_0)=0$ for all $n\in\N$ implies that $g(x_0)=0$. Since $x_0$ is a limit point of $\{x_n\}$, a subsequence converges to $x_0$ so this contradicts the continuity of $g$. Thus, when $X$ is infinite, $C(X,\K)$ is not reflexive.
\end{example}
\begin{proposition}\label{NVS separable prop}
Let $X$ be a normed vector space. The following hold.
\begin{itemize}
\item[(a)] If $X^*$ is separable, then $X$ is separable.
\item[(b)] If $X$ is reflexive and separable, then $X^*$ is separable.
\end{itemize}
\end{proposition}
\begin{proof}
We prove (a), and (b) follows from (a). Thus assume $X^*$ is separable and choose a dense sequence $\{x_n^*\}$ in $X^*$. Choose a sequence $x_n\in X$ such that
\[\|x_n\|=1,\quad \langle x_n,x_n^*\rangle\geq\frac{1}{2}\|x_n^*\|.\]
Let $Y\sub X$ be the linear subspace of all finite linear combinations of the $x_n$. We prove that $Y$ is dense in $X$. To see this, fix any element $x^*\in Y^\bot$. Then there is a sequence $n_k$ such that $\lim_k\|x^*-x_{n_k}^*\|=0$. This implies
\begin{align*}
\|x_{n_k}^*\|\leq 2|\langle x_{n_k},x_{n_k}^*\rangle|=2|\langle x_{n_k},x_{n_k}^*-x^*\rangle|\leq\|x_{n_k}^*-x^*\|\|x_{n_k}\|=2\|x_{n_k}^*-x^*\|.
\end{align*}
The last term on the right converges to zero as $k$ tends to infinity, and
hence $x^*=\lim_{k}x_{n_k}^*=0$. This shows that $Y^{\bot}=\{0\}$. Hence $Y$ is dense in $X$.
\end{proof}
The following result is a useful criterion when we want to prove a space is not separable.
\begin{proposition}\label{NVS not separable if uncountable family}
Let $X$ be a normed vector space. If there exists a uncountable family $S$ in $X$ and $\delta>0$ such that $\|x-y\|>\delta$ for any $x,y\in S$, then $X$ is not separable.
\end{proposition}
\begin{proof}
Let $\{e_n\}$ be a countable dense set of $X$, and define
\[E_n=\{x\in X:\|x-e_n\|\leq\delta/2\}.\]
Then since $\{e_n\}$ is dense we have $X=\bigcup_nE_n$. Since $S$ is uncountable, there exists $N>0$ such that $E_{N}$ contains two distinct elements $x,y$ of $S$. But then
\[\|x-y\|\leq\|x-e_N\|+\|e_N-y\|\leq\delta.\]
This is a contradiction.
\end{proof}
\section{Norm-attaining funtionals}
The preceding results are useful in proving James's profound theorem that a Banach space $X$ is reflexive iff for every $f\in X^*$ there exists a unit vector $x\in X$ such that $f(x)=\|f\|$.
\begin{definition}
Let $X$ be a Banach space and $f\in X^*$. If there exists a unit vector $x\in X$ such that $f(x)=\|f\|$ then $f$ is called \textbf{norm-attaining} and $x$ a \textbf{maximal element} for $f$.
\end{definition}
The maximal element $x$ can also be defined by requiring that $|f(x)|=\|f\|$. In this case, if $f(x)=re^{i\theta}$, then $v=e^{-it}x$ maximizes $f$. By Theorem~\ref{NVS linear functional prop}, any unit vector $x$ is a maximal element for some $f\in X^*$. For $f=0$, any unit vector is a maximal element for $f$.
\begin{example}
Consider the continuous linear functional $f:c_0\to\K,(a_n)\mapsto\sum_n2^{-n}a_n$. Note that $\|f\|=1$. For any $x=(a_n)$ in the unit ball of $c_0$ there exists $N>0$ such that $|a_n|\leq 1/2$ for $n>N$. Therefore
\begin{align*}
|f(x)|&=\Big|\sum_n2^{-n}a_n\Big|\leq\Big|\sum_{n\leq N}2^{-n}a_n\Big|+\Big|\sum_{n>N}2^{-n}a_n\Big|\\
&\leq\sum_{n\leq N}2^{-n}+\frac{1}{2}\sum_{n>N}2^{-n}=1-2^{-N}+2^{-N+1}<1
\end{align*}
Hence $f$ does not attain its norm.
\end{example}
Suppose that $x$ and $y$ are unit vectors in $X$ which are maximal elements for $f\in X^*$. If, for some scalar $b$, $y=bx$, then $f(x)=f(y)=\|f\|$ immediately implies that $b=1$. Thus distinct maximal elements must be linearly independent. But can there be distinct maximal elements?
\begin{example}\label{c_0 norm not attaining}
Consider the Banach space $C(X,\K,\|\cdot\|_\infty)$ of continuous maps of the compact Hausdorff space $X$ into $\K$. Let $x\in X$ be disjoint from the closed set $K\sub X$. By Urysohn's lemma, there exist $f\in C(X,[0,1])$ such that $f(x)=1$ and $f(K)=\{0\}$ and $g\in C(X,[1/2,1])$ such that $g(x)=1$, $g(K)=\{1/2\}$. Thus $f$ and $g$ are distinct maximal elements for the evaluation map $x^*\in C(X,\K)^*$.
\end{example}
The question of uniqueness of maximal element is significant in approximation theory. For the remainder of this discussion, we investigate only existence. We begin by showing that every continuous linear functional on a reflexive Banach space has maximal elements.
\begin{proposition}\label{Banach reflexive norm is attaining}
If $X$ is a reflexive Banach space then any $f\in X^*$ is norm-attaining.
\end{proposition}
\begin{proof}
If the Banach space $X$ is reflexive, the unit sphere $S_X$ of $X$ is $\sigma(X,X^*)$-compact by Theorem~\ref{Banach space reflexive iff closed unit ball weakly compact}. Let $f\in X^*$, $f\neq 0$. Since $f$ is $\sigma(X,X^*)$-continuous, there exists a nonzero $x\in X$ such that $|f(x)|=\max\{|f(u)|:u\in S_X\}$.
\end{proof}
It follows from Example~\ref{c_0 norm not attaining} that $c_0$ is not reflexive. The converse of Proposition~\ref{Banach reflexive norm is attaining} is the theorem of James mentioned above. We now show thatthe question of whether $\K$ is $\R$ or $\C$ is irrelevant with regard to the existence of maximal elements and therefore, ultimately, to reflexivity.
\begin{proposition}
Let $X$ be a complex Banach space, let $f\in X^*$, and let $\phi=\Re(f)$. Then $x$ is a maximal element for $f$ iff $x$ is a maximal element for $\phi$.
\end{proposition}
\begin{proof}
If $x$ is a maximal element for $f$, then $f(x)=\|f\|=\Re(f(x))=\phi(x)=\|\phi\|$ and $x$ is a maximal element for $\phi$. Conversely, if $x$ is a maximal element for $\phi$, then $\phi(x)=\|\phi\|=\|f\|$; since $\|f\|=\|g\|=\Re(f(x))\leq|f(x)|\leq\|f\|$, it follows that $|f(x)|=\|f\|$ and $x$ is a maximal element for $f$.
\end{proof}